Un proceso de software es un conjunto coherente de actividades que se llevan a cabo para la producción de un software. Los procesos de software son complejos y los intentos por automatizarlos han tenido un éxito limitado debido a su inmensa diversidad. Aunque existen muchos procesos diferentes, algunas etapas son comunes entre ellos \cite{sommerville1992software}, \cite{pressman2005software}: Análisis, Diseño, Implementación y Pruebas.\\

Si se considera que el costo total del desarrollo de un sistema de software es de 100 unidades de costo, la figura~\ref{fig:costos}  muestra cómo se gastan estas en las diferentes actividades del proceso \cite{sommerville1992software}. \\

	 \IUfigNoId[.3]{images/costosProceso.png}{fig:costos}{Distribución de costos}

Actualmente existen herramientas que permiten automatizar diferentes actividades del proceso de software, esto permite algunas mejoras en la calidad y productividad del software. El presente artículo expone las actividades realizadas y los resultados obtenidos de desarrollar una herramienta que permite asistir a la etapa de {\it Pruebas} a través de la generación semiautomática de casos de prueba funcionales basados en la documentación de análisis.