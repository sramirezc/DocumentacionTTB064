%Describir las restricciones
A continuación se muestran los resultados más relevantes que se obtuvieron a lo largo del desarrollo del proyecto. 

\subsubsection*{Prototipo}
El desarrollo del proyecto permitió la construcción de PRISMA, un prototipo de editor de casos de uso que permite construir casos de prueba con base en la información de análisis. Algunas de las
características de la herramienta son:

\begin{itemize}
 \item Permite recabar los elementos que componen un documento de análisis: mensajes, reglas de negocio, actores, términos del glosario, pantallas, casos de uso y entidades.
 \item Permite indicar las relaciones entre los diferentes elementos mencionados en el punto anterior.
 \item Permite generar el documento de análisis en dos formatos distintos: pdf y docx.
 \item Permite modelar el proceso de revisión de los casos de uso entre colaboradores de un proyecto.
 \item Permite asistir a la generación semiautomática de casos de prueba con base en el documento de análisis.
\end{itemize}

\subsubsection*{Gestión de las actividades de los colaboradores}
El diseño de la herramienta soporta tres perfiles de usuario diferentes:
\begin{itemize}
 \item Administrador 
 \item Líder de análisis
 \item Analista
\end{itemize}

Estos tres perfiles permiten que existan usuarios con diferentes responsabilidades y niveles de acceso. El Administrador es el encargado de gestionar los proyectos, así como el personal de la organización. El Líder
de análisis y el Analista participan en el proceso de revisión de los casos de uso siendo cada uno colaborador de los proyectos.\\

\subsubsection*{Fórmula para el cálculo de los casos de prueba generados por el sistema}
Un resultado interesante fue la fórmula que permite calcular el número de casos de prueba que pueden ser generados por PRISMA, el resultado del cálculo como se observa en la figura \ref{fig:formula} 
depende de las reglas de negocio asociadas al caso de uso a probar, así como el tipo y la cantidad de entradas asociadas.

\IUfigNoId[.45]{images/formula.png}{fig:formula}{Fórmula para calcular los casos de prueba generados}