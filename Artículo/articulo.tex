
\documentclass{article}
\usepackage{multicol}
\usepackage{graphicx}
\usepackage[sc]{mathpazo} % Use the Palatino font
\usepackage[T1]{fontenc} % Use 8-bit encoding that has 256 glyphs
\linespread{1.05} % Line spacing - Palatino needs more space between lines
\usepackage{microtype} % Slightly tweak font spacing for aesthetics

\usepackage[spanish]{babel}
\selectlanguage{spanish}
\usepackage[utf8]{inputenc}

\usepackage[hmarginratio=1:1,top=32mm,columnsep=20pt]{geometry} % Document margins
\usepackage[hang, small,labelfont=bf,up,textfont=it,up]{caption} % Custom captions under/above floats in tables or figures
\usepackage{booktabs} % Horizontal rules in tables

\usepackage{lettrine} % The lettrine is the first enlarged letter at the beginning of the text

\usepackage{enumitem} % Customized lists
\setlist[itemize]{noitemsep} % Make itemize lists more compact

\usepackage{abstract} % Allows abstract customization
\renewcommand{\abstractnamefont}{\normalfont\bfseries} % Set the "Abstract" text to bold
\renewcommand{\abstracttextfont}{\normalfont\small\itshape} % Set the abstract itself to small italic text

\usepackage{titlesec} % Allows customization of titles
\renewcommand\thesection{\Roman{section}} % Roman numerals for the sections
\renewcommand\thesubsection{\roman{subsection}} % roman numerals for subsections
\titleformat{\section}[block]{\large\scshape\centering}{\thesection.}{1em}{} % Change the look of the section titles
\titleformat{\subsection}[block]{\large}{\thesubsection.}{1em}{} % Change the look of the section titles

\usepackage{fancyhdr} % Headers and footers
\pagestyle{fancy} % All pages have headers and footers
\fancyhead{} % Blank out the default header
\fancyfoot{} % Blank out the default footer
\fancyhead[C]{\tituloTrabajo} % Custom header text
\fancyfoot[RO,LE]{\thepage} % Custom footer text

\usepackage{titling} % Customizing the title section

\usepackage{hyperref} % For hyperlinks in the PDF


\usepackage{prisma}
\usepackage{cite}
\usepackage{ragged2e}
\pagenumbering{roman}
\usepackage{enumitem}
%----------------------------------------------------------------------------------------
%	Sección del título
%----------------------------------------------------------------------------------------

\setlength{\droptitle}{-4\baselineskip} % Move the title up

\pretitle{\begin{center}\Large\bfseries} % Article title formatting
\posttitle{\end{center}} % Article title closing formatting
\title{\tituloTrabajo} % Article title
\author{
\autores \\
\normalsize \instituto \\ 
\normalsize \telefonos \correos 
}
\date{}
%----------------------------------------------------------------------------------------

\begin{document}
\bibliographystyle{IEEEtran}
% Portada
\maketitle

%----------------------------------------------------------------------------------------
%	Contenido del artículo
%----------------------------------------------------------------------------------------
\begin{multicols}{2}[]
\textit{Resumen} - \textbf{\resumen}
\\

\textit{Palabras clave} - \textbf{\palabrasClave}
\\

\section{Introducción}
El presente documento es un manual de usuario cuya finalidad es explicar cómo utilizar el sistema PRISMA.\\

\section{A quién va dirigido}

El presente documento va dirigido a los Analistas y Líderes de análisis que requieran gestionar los elementos del {\bf Editor de Casos de uso}, así como configurar y descargar pruebas utilizando el {\bf Generador de Pruebas} de PRISMA.\\


\section{Objetivo general de este manual}
\label{sec:ObjetivoGeneralManual}

El objetivo de este manual es proporcionar una guía al usuario para que pueda operar el sistema PRISMA de forma correcta, orientándolo sobre cómo ingresar y 
actualizar información en las distintas secciones: Proyectos, Módulos, Casos de uso, Pantallas, Actores, Reglas de negocio, Mensajes, Términos del glosario, Entidades y Pruebas.\\


\section{Objetivos específicos de este manual}
%Los objetivos específicos son que el usuario sepa y pueda dominar las siguientes funciones:

\begin{itemize}
	\item Describir cómo ingresar al sistema.
	\item Describir cómo gestionar los Proyectos.
	\item Describir cómo gestionar los Módulos.
	\item Describir cómo gestionar los Casos de uso.
	\item Describir cómo gestionar las Pantallas.
	\item Describir cómo gestionar los Actores.
	\item Describir cómo gestionar las Reglas de negocio.
	\item Describir cómo gestionar los Mensajes.
	\item Describir cómo gestionar los Términos del glosario.
	\item Describir cómo gestionar las Entidades.
	\item Describir cómo configurar las Pruebas.
	\item Describir cómo generar las Pruebas.
\end{itemize}


\section{Metodología}
	La construcción del prototipo se dividió en dos incrementos:
	\begin{enumerate}
		\item Editor de casos de uso
		\item Generador de casos de prueba
	\end{enumerate}
	
	 En este capítulo se describen las principales actividades realizadas para cada uno de ellos.
	
	\section{Editor de casos de uso}
	\cfinput{Metodologia/editor}
	
	\section{Generador de casos de prueba}
	\cfinput{Metodologia/generador}
	

	
	
	
	

\section{Resultados}
%Describir las restricciones
En este capítulo se describen los resultados más relevantes que se obtuvieron a lo largo del desarrollo del proyecto. 

\section{Prototipo}
El desarrollo del proyecto permitió la construcción de PRISMA, un prototipo de editor de casos de uso que permite construir casos de prueba con base en la información de análisis. Algunas de las
características de la herramienta son:

\begin{itemize}
 \item Permite recabar los elementos que componen un documento de análisis.
 \item PRISMA permite la interacción de tres perfiles de usuario distintos.
 \item El sistema permite generar el documento de análisis en dos formatos distintos: pdf y docx.
 \item El Editor de casos de uso construye la base de datos de manera adecuada.
 \item PRISMA permite modelar el proceso de revisión de los casos de uso, así como gestionar las actividades de los colaboradores.
\end{itemize}

Algunas de las características se detallan a continuación.

\subsection*{Registro de elementos}
PRISMA permite gestionar los elementos que describen el comportamiento de un sistema y que son recabados en la etapa de análisis: mensajes, reglas de negocio, entidades, actores, términos del glosario, pantallas y casos de uso. Con el registro 
de los elementos mencionados es posible modelar las diversas estructuras de información de múltiples sistemas web. 

\section*{Gestión de las actividades de los colaboradores}
El diseño de la herramienta soporta tres perfiles de usuario diferentes:
\begin{itemize}
 \item Administrador 
 \item Líder de análisis
 \item Analista
\end{itemize}

Estos tres perfiles permiten que existan usuarios con diferentes responsabilidades y niveles de acceso. El Administrador es el encargado de gestionar los proyectos, así como el personal de la organización. El Líder
de análisis y el Analista participan en el proceso de revisión de los casos de uso siendo cada uno colaborador de los proyectos.\\

\section{Documento de análisis}
Se construyó un documento de análisis que contiene las especificaciones del comportamiento de PRISMA. Dentro de este documento se describen los siguientes elementos:

\begin{itemize}
 \item Requerimientos del sistema
 \item Análisis de riesgos
 \item Modelo de negocio
 \item Modelo de comportamiento
 \item Modelo de interacción con el usuario
\end{itemize}

\section{Fórmula para el cálculo de los casos de prueba generados por el sistema}
Un resultado interesante fue la fórmula que permite calcular el número de casos de prueba que pueden ser generados por PRISMA, el resultado del cáluco como se observa en la figura \ref{fig:formula} 
depende de las reglas de negocio asociadas al caso de uso a probar, así como el tipo y la cantidad de entradas asociadas.

\IUfigNoId[.7]{images/formula.png}{fig:formula}{Fórmula para calcular los casos de prueba generados}

\section{Conclusiones}
	Las pruebas de software con frecuencia representan un costo muy elevado en relación al invertido en las demás actividades del proceso de software. Debido a ello, se han realizado diversos intentos por automatizar su proceso, como lo son herramientas de software o metodologías de pruebas.\\

	El desarrollo de la herramienta demostró que es posible automatizar las pruebas de software a través de mecanismos semiautomáticos que reutilizan la información de la documentación de análisis.\\
	
	Debido a que la especificación de casos de uso es realizada por diferentes analistas, con frecuencia se encuentran diferentes redacciones para un mismo objetivo, lo cual complica bastante el entendimiento de la información para una computadora, por lo que analizar los pasos de un caso de uso y traducirlos a componentes de pruebas resulta altamente complejo.\\
		
	Por otro lado, se identificó que tener la información de la documentación de análisis en una base de datos permite mantener la información organizada y centralizada. En general el contar con una base de datos de esta información abre las puertas a otros sistemas para automatizar algún proceso dentro del desarrollo de un software, en nuestro caso, por ejemplo, permitió automatizar la generación de casos de prueba funcionales.

\bibliography{Referencias}
\end{multicols}
%----------------------------------------------------------------------------------------

\pagebreak
\section*{Anexos}

\subsection*{Anexo I. Modelo de trayectorias}
\hypertarget{Anexo 1}{}
La figura ~\ref{fig:estadosTrayectoria} muestra un diagrama que modela los pasos más comunes de una trayectoria. Las transiciones con líneas punteadas
son aquellas que no considera la herramienta.

	 \IUfigNoId[.9]{images/estadosTrayectoria.png}{fig:estadosTrayectoria}{Modelo de trayectorias}


\end{document}
