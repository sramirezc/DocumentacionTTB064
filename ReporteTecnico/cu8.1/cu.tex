\begin{UseCase}{CU 8.1}{Registrar regla de negocio}
	{
		Este caso de uso permite al analista registrar la información de una regla de negocio. 
	}
	
	\UCitem{Actor}{\cdtRef{actor:liderAnalisis}{Líder de análisis}, \cdtRef{actor:analista}{Analista}}
	\UCitem{Propósito}{
		Registrar la información de una regla de negocio.
	}
	\UCitem{Entradas}{
		\begin{UClist}
			\UCli \cdtRef{Elemento:Numero}{Número}: \ioEscribir.
			\UCli \cdtRef{Elemento:Nombre}{Nombre}: \ioEscribir.
			\UCli \cdtRef{Elemento:Descripcion}{Descripción}: \ioEscribir.
 			\UCli \cdtRef{ReglaDeNegocio:Redaccion}{Redacción}: \ioEscribir.
 			\UCli \cdtRef{gls:TipoDeReglaDeNegocio}{Tipo}: \ioSeleccionar.
 			\UCli Para el tipo ``Comparación de atributos'':
 			\begin{itemize}
 			 \item \cdtRef{ValorDelParametroEnReglaDeNegocio}{Entidad 1}: \ioSeleccionar.
 			 \item \cdtRef{ValorDelParametroEnReglaDeNegocio}{Atributo 1}: \ioSeleccionar.
 			 \item \cdtRef{Operador}{Operador}: \ioSeleccionar.
 			 \item \cdtRef{ValorDelParametroEnReglaDeNegocio}{Entidad 2}: \ioSeleccionar.
 			 \item \cdtRef{ValorDelParametroEnReglaDeNegocio}{Atributo 2}: \ioSeleccionar.
 			\end{itemize}
 			
 			\UCli Para el tipo ``Unicidad de parámetros'':
 			\begin{itemize}
 			 \item \cdtRef{ValorDelParametroEnReglaDeNegocio}{Entidad}: \ioEscribir.
 			 \item \cdtRef{ValorDelParametroEnReglaDeNegocio}{Atributo único}: \ioEscribir. 
 			\end{itemize}

 			\UCli Para el tipo ``Formato correcto'':
 			\begin{itemize}
 			 \item \cdtRef{ValorDelParametroEnReglaDeNegocio}{Entidad que contiene el atributo para verificar el formato}: \ioSeleccionar.
 			 \item \cdtRef{ValorDelParametroEnReglaDeNegocio}{Atributo que se verificará con la expresión regular}: \ioSeleccionar.
 			 \item \cdtRef{}{Expresión regular}: \ioEscribir.
 			\end{itemize}

		\end{UClist}
	}
	\UCitem{Salidas}{
		\begin{UClist}
			\UCli \cdtRef{Elemento:Clave}{Clave}: \ioCalcular{\cdtIdRef{RN12}{Identificador de elemento}}.
		\end{UClist}
	}
	
	\UCitem{Mensajes}{
		\begin{UClist}
			\UCli \cdtIdRef{MSG1}{Operación exitosa}: Se muestra en la pantalla \cdtIdRef{IU 8}{Gestionar reglas de negocio} para indicar que el registro fue exitoso.
		\end{UClist}
	}

	\UCitem{Precondiciones}{
		Ninguna
	}
	
	\UCitem{Postcondiciones}{
		Ninguna
	}

	\UCitem{Errores}{
		\begin{UClist}
			\UCli \cdtIdRef{MSG4}{Dato obligatorio}: Se muestra en la pantalla \cdtIdRef{IU 8.1}{Registrar regla de negocio} cuando no se ha ingresado un dato marcado como obligatorio.
			\UCli \cdtIdRef{MSG5}{Dato incorrecto}: Se muestra en la pantalla \cdtIdRef{IU 8.1}{Registrar regla de negocio} cuando el tipo de dato ingresado no cumple con el tipo de dato solicitado en el campo.
			\UCli \cdtIdRef{MSG6}{Longitud inválida}: Se muestra en la pantalla \cdtIdRef{IU 8.1}{Registrar regla de negocio} cuando se ha excedido la longitud de alguno de los campos.
			\UCli \cdtIdRef{MSG7}{Registro repetido}: Se muestra en la pantalla \cdtIdRef{IU 8.1}{Registrar regla de negocio} cuando se registre una regla de negocio con un nombre o número que ya este registrado.
			\UCli \cdtIdRef{MSG23}{Caracteres inválidos}: Se muestra en la pantalla \cdtIdRef{IU 8.1}{Registrar regla de negocio} cuando el nombre de la regla de negocio contiene un caracter no válido.
		\end{UClist}
	}

	\UCitem{Tipo}{
		Secundario, extiende del caso de uso \cdtIdRef{CU 8}{Gestionar reglas de negocio}.
	}
\end{UseCase}
%-------------------------------------------------------%trayectoria Principal-----------------------------------------------
 \begin{UCtrayectoria}
    \UCpaso[\UCactor] Solicita registrar una regla de negocio oprimiendo el botón \cdtButton{Registrar} de la pantalla \cdtIdRef{IU 8}{Gestionar reglas de negocio}.
    \UCpaso[\UCsist] Muestra la pantalla \cdtIdRef{IU 8.1}{Registrar regla de negocio} en la cual se realizará el registro de la regla de negocio. 
    \UCpaso[\UCactor] Ingresa el nombre y la descripción de la regla de negocio. \label{cu8.1:ingresaDatos}
    \UCpaso[\UCactor] Selecciona el tipo de regla de negocio. 
    \UCpaso[\UCsist] Verifica que el tipo de regla de negocio requiera parámetros. \refTray{A}
    \UCpaso[\UCsist] Muestra la pantalla \cdtIdRef{IU 8.1a}{Registrar regla de negocio: Comparación de atributos}, 
	\cdtIdRef{IU 8.1b}{Registrar regla de negocio: Unicidad de parámetros} o
	\cdtIdRef{IU 8.1c}{Registrar regla de negocio: Formato correcto}, según corresponda.
    \UCpaso[\UCactor] El usuario ingresa la información de la regla de negocio en la pantalla \cdtIdRef{IU 8.1a}{Registrar regla de negocio: Comparación de atributos}, 
	\cdtIdRef{IU 8.1b}{Registrar regla de negocio: Unicidad de parámetros} o
	\cdtIdRef{IU 8.1c}{Registrar regla de negocio: Formato correcto}, según corresponda.
    \UCpaso[\UCactor] Ingresa la información solicitada en la pantalla correspondiente. \label{cu8:ingresaPaso}
    \UCpaso[\UCactor] Ingresa la redacción de la regla de negocio. \label{cu8.1:descripcion}
    \UCpaso[\UCactor] Solicita guardar la regla de negocio oprimiendo el botón \cdtButton{Aceptar} de la pantalla \cdtIdRef{IU 8.1}{Registrar regla de negocio}. \refTray{B} 
    \UCpaso[\UCsist] Verifica que el actor ingrese todos los campos obligatorios con base en la regla de negocio  \cdtIdRef{RN8}{Datos obligatorios}. \refTray{C}
    \UCpaso[\UCsist] Verifica que el nombre de la regla de negocio no se encuentre registrado en el sistema con base en la regla de negocio  \cdtIdRef{RN6}{Unicidad de nombres}. \refTray{D}
    \UCpaso[\UCsist] Verifica que los datos requeridos sean proporcionados correctamente como se especifica en la regla de negocio \cdtIdRef{RN7}{Información correcta}. \refTray{E}
    \UCpaso[\UCsist] Verifica que el nombre no contenga caracteres inválidos con base en la regla de negocio \cdtIdRef{RN2}{Nombres de los elementos}. \refTray{F}
    \UCpaso[\UCsist] Verifica que el número del caso de uso no se encuentre registrado en el sistema con base en la regla de negocio  \cdtIdRef{RN1}{Unicidad de números}. \refTray{G}
    
    \UCpaso[\UCsist] Registra la información de la regla de negocio en el sistema.
    \UCpaso[\UCsist] Muestra el mensaje \cdtIdRef{MSG1}{Operación exitosa} en la pantalla \cdtIdRef{IU 8}{Gestionar reglas de negocio}
    para indicar al actor que el registro se ha realizado exitosamente.
 \end{UCtrayectoria}
 
 
 %----------------------------------------------------------%trayectoria A---------------------------------------------------- 
 \begin{UCtrayectoriaA}{A}{La regla de negocio no requiere parámetros.}
	\UCpaso[] Continúa con el paso \ref{cu8.1:descripcion} de la trayectoria principal.
 \end{UCtrayectoriaA}
 %----------------------------------------------------------%trayectoria B---------------------------------------------------- 
 \begin{UCtrayectoriaA}[Fin del caso de uso]{B}{El actor desea cancelar la operación.}
    \UCpaso[\UCactor] Solicita cancelar la operación oprimiendo el botón \cdtButton{Cancelar} de la pantalla \cdtIdRef{IU 8.1}{Registrar regla de negocio}.
    \UCpaso[\UCsist] Muestra la pantalla donde se solicitó la operación.
 \end{UCtrayectoriaA}
  %----------------------------------------------------------%trayectoria C---------------------------------------------------- 
 \begin{UCtrayectoriaA}{C}{El actor no ingresó algún dato marcado como obligatorio.}
    \UCpaso[\UCsist] Muestra el mensaje \cdtIdRef{MSG4}{Dato obligatorio} y señala el campo que presenta el error en la pantalla 
	    \cdtIdRef{IU 8.1}{Registrar regla de negocio}, indicando al actor que el dato es obligatorio.
    \UCpaso[] Continúa con el paso \ref{cu8.1:ingresaDatos} de la trayectoria principal.
 \end{UCtrayectoriaA}
 %----------------------------------------------------------%trayectoria D---------------------------------------------------- 
 \begin{UCtrayectoriaA}{D}{El actor ingresó un nombre de regla de negocio repetido.}
    \UCpaso[\UCsist] Muestra el mensaje \cdtIdRef{MSG7}{Registro repetido} y señala el campo que presenta la duplicidad en la pantalla 
	    \cdtIdRef{IU 8.1}{Registrar regla de negocio}, indicando al actor que existe una regla de negocio con el mismo nombre.
    \UCpaso[] Continúa con el paso \ref{cu8.1:ingresaDatos} de la trayectoria principal.
 \end{UCtrayectoriaA}
 
 %----------------------------------------------------------%trayectoria E----------------------------------------------------  
 \begin{UCtrayectoriaA}{E}{El actor proporciona un dato que excede la longitud máxima.}
    \UCpaso[\UCsist] Muestra el mensaje \cdtIdRef{MSG5}{Se ha excedido la longitud máxima del campo} y señala el campo que excede la 
    longitud en la pantalla \cdtIdRef{IU 8.1}{Registrar regla de negocio}, para indicar que el dato excede el tamaño máximo permitido.
    \UCpaso[] Continúa con el paso \ref{cu8.1:ingresaDatos} de la trayectoria principal.
 \end{UCtrayectoriaA}
 
 %----------------------------------------------------------%trayectoria F---------------------------------------------------- 
 \begin{UCtrayectoriaA}{F}{El actor ingresó un nombre con caracteres inválidos.}
    \UCpaso[\UCsist] Muestra el mensaje \cdtIdRef{MSG23}{Caracteres inválidos} y señala el campo que contiene los caracteres inválidos.
    \UCpaso[] Continúa con el paso \ref{cu8.1:ingresaDatos} de la trayectoria principal.
 \end{UCtrayectoriaA}
 %----------------------------------------------------------%trayectoria G---------------------------------------------------- 
 \begin{UCtrayectoriaA}{G}{El actor ingresó un número de regla de negocio repetido.}
    \UCpaso[\UCsist] Muestra el mensaje \cdtIdRef{MSG7}{Registro repetido} y señala el campo que presenta la duplicidad en la pantalla 
	    \cdtIdRef{IU 8.1}{Registrar regla de negocio}, indicando al actor que existe una regla de negocio con el mismo número.
    \UCpaso[] Continúa con el paso \ref{cu8.1:ingresaDatos} de la trayectoria principal.
 \end{UCtrayectoriaA}
 
