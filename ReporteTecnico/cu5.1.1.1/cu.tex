\begin{UseCase}{CU 5.1.1.1}{Registrar trayectoria}
	{
		Las trayectorias describen los escenarios ideales y alternos de un sistema mediante una serie de pasos.
		Este caso de uso permite al analista registrar la trayectoria principal o una trayectoria alternativa de un caso de uso.
		
	}
	%\UCitem{\DONEUC}{Edición}
	\UCitem{Versión}{1.0}
	\UCccsection{Administración de Requerimientos}	
	\UCitem{Autor}{Natalia Giselle Hernández Sánchez}	
	\UCccitem{Evaluador}{}
	\UCitem{Operación}{Administración}
	\UCccitem{Prioridad}{Alta}
	\UCccitem{Complejidad}{Media}
	\UCccitem{Volatilidad}{Alta}
	\UCccitem{Madurez}{Baja}
	\UCitem{Estatus}{Terminado}
	\UCitem{Fecha del último estatus}{}
	\UCsection{Atributos}
	\UCitem{Actor}{\cdtRef{actor:liderAnalisis}{Líder de análisis}, \cdtRef{actor:analista}{Analista}}
	\UCitem{Propósito}{
		Registrar una trayectoria principal o alternativa.
	}
	\UCitem{Entradas}{
		\begin{UClist}
			\UCli \cdtRef{Trayectoria:Alternativa}{Principal o alternativa}: \ioSeleccionar.
			\UCli \cdtRef{Trayectoria:Identificador}{Nombre de la trayectoria}: \ioEscribir.
		\end{UClist}
	}
	\UCitem{Salidas}{
		\begin{UClist}
			\UCli \cdtRef{Paso}{Pasos} de la trayectoria.
		\end{UClist}
	}
	
	\UCitem{Mensajes}{
		\begin{UClist}
			\UCli \cdtIdRef{MSG1}{Operación exitosa}: Se muestra en la pantalla \cdtIdRef{IU 5.1.1}{Gestionar trayectorias} para indicar que el registro fue exitoso.
		\end{UClist}
	}

	\UCitem{Precondiciones}{
		Ninguna
	}
	
	\UCitem{Postcondiciones}{
		\begin{UClist}
			\UCli Los elementos utilizados en los pasos y que no estén descritos en la sección ``Descripción del Caso de uso'' de la pantalla 
			      \cdtIdRef{CU 5.1}{Registrar caso de uso} se agregarán a la sección automáticamente.
		\end{UClist}
	}

	\UCitem{Errores}{
			\UCli \cdtIdRef{MSG4}{Dato obligatorio}: Se muestra en la pantalla \cdtIdRef{CU 5.1.1.1}{Registrar trayectoria} cuando no se ha ingresado un dato marcado como obligatorio.
			\UCli \cdtIdRef{MSG5}{Dato incorrecto}: Se muestra en la pantalla \cdtIdRef{CU 5.1.1.1}{Registrar trayectoria} cuando el tipo de dato ingresado no cumple con el tipo de dato solicitado en el campo.
			\UCli \cdtIdRef{MSG6}{Longitud inválida}: Se muestra en la pantalla \cdtIdRef{CU 5.1.1.1}{Registrar trayectoria} cuando se ha excedido la longitud de alguno de los campos.
			\UCli \cdtIdRef{MSG7}{Registro repetido}: Se muestra en la pantalla \cdtIdRef{CU 5.1.1.1}{Registrar trayectoria} cuando se haya ingresado un nombre que ya esté registrado.
	}

	\UCitem{Tipo}{
		Secundario, extiende del caso de uso \cdtIdRef{CU 5.1.1}{Gestionar trayectorias}.
	}
\end{UseCase}
%-------------------------------------------------------%trayectoria Principal-----------------------------------------------
 \begin{UCtrayectoria}
	\UCpaso[\UCactor] Solicita registrar una trayectoria oprimiento el botón \cdtButton{Registrar} de la la pantalla \cdtIdRef{IU 5.1.1}{Gestionar trayectorias}. 
	\UCpaso[\UCsist] Muestra la pantalla \cdtIdRef{IU 5.1.1.1}{Registrar trayectoria} con el mensaje \cdtIdRef{MSG2}{No existe información} debido a que no hay pasos registrados para la trayectoria. 
	
	\UCpaso[\UCactor] Selecciona que es la trayectoria principal. \label{cu5.1.1.1:ingresaDatos}
	\UCpaso[\UCactor] Ingresa el nombre de la trayectoria.
	
	\UCpaso[\UCactor] Gestiona los pasos de la trayectoria. \label{cu5.1.1.1:gestionaPasos}
	\UCpaso[\UCactor] Solicita registrar la trayectoria oprimiendo el botón \cdtButton{Aceptar} de la pantalla \cdtIdRef{IU 5.1.1.1}{Registrar trayectoria}. \refTray{A} 
	
	\UCpaso[\UCsist] Verifica que el actor haya ingresado todos los campos obligatorios con base en la regla de negocio \cdtIdRef{RN-S8}{Datos obligatorios}. \refTray{B}
	\UCpaso[\UCsist] Verifica que la trayectoria no se encuentre registrada en el sistema con base en la regla de negocio \cdtIdRef{RN-N6}{Unicidad de nombres}. \refTray{C}
	\UCpaso[\UCsist] Verifica que los datos requeridos sean proporcionados correctamente como se especifica en la regla de negocio \cdtIdRef{RN-S7}{Información correcta}. \refTray{D} \refTray{E}
	
	\UCpaso[\UCsist] Registra la información de la trayectoria en el sistema.
	
	\UCpaso[\UCsist] Muestra el mensaje \cdtIdRef{MSG1}{Operación exitosa} en la pantalla \cdtIdRef{IU 5.1.1}{Gestionar trayectorias} 
	para indicar al actor que el registro se ha realizado exitosamente.
    
\end{UCtrayectoria}
%     \UCpaso[\UCactor] Selecciona si el actor o el sistema realiza el paso.
%     \UCpaso[\UCactor] Ingresa la redacción del paso.
%     \UCpaso[\UCactor] Solicita registrar la  oprimiendo el botón \cdtButton{Terminar} de la pantalla \cdtIdRef{IU 5.1}{Registrar caso de uso}. \refTray{C}  \refTray{D} 
%     \UCpaso[\UCsist] Verifica que el actor ingrese todos los campos obligatorios con base en la regla de negocio  \cdtIdRef{RN-S8}{Datos obligatorios}. \refTray{E}
%     \UCpaso[\UCsist] Verifica que el nombre del caso de uso no se encuentre registrado en el sistema con base en la regla de negocio  \cdtIdRef{RN-N6}{Unicidad de nombres}. \refTray{F}
%     \UCpaso[\UCsist] Verifica que los datos requeridos sean proporcionados correctamente como se especifica en la regla de negocio \cdtIdRef{RN-S7}{Información correcta}. \refTray{G} \refTray{H}
%     \UCpaso[\UCsist] Verifica que todos los elementos del caso de uso esten referenciados en las trayectorias. \refTray{I}
%     \UCpaso[\UCsist] Muestra el mensaje \cdtIdRef{MSG8}{Caso de uso terminado} para preguntar al actor si desea continuar con la operación.
    
 %----------------------------------------------------------%trayectoria A---------------------------------------------------- 
 \begin{UCtrayectoriaA}[Fin del caso de uso]{A}{El actor desea cancelar la operación.}
    \UCpaso[\UCactor] Solicita cancelar la operación oprimiendo el botón \cdtButton{Cancelar} de la pantalla \cdtIdRef{IU 5.1.1.1}{Registrar trayectoria}.
    \UCpaso[\UCsist] Muestra la pantalla \cdtIdRef{IU 5.1.1}{Gestionar trayectorias}.
 \end{UCtrayectoriaA}
 
 %----------------------------------------------------------%trayectoria B---------------------------------------------------- 
 \begin{UCtrayectoriaA}{B}{El actor no ingresó algún dato marcado como obligatorio.}
    \UCpaso[\UCsist] Muestra el mensaje \cdtIdRef{MSG4}{Dato obligatorio} y señala el campo que presenta el error en la pantalla 
	    \cdtIdRef{CU 5.1.1.1}{Registrar trayectoria}, indicando al actor que el dato es obligatorio.
    \UCpaso[] Continúa con el paso \ref{cu5.1.1.1:ingresaDatos} de la trayectoria principal.
 \end{UCtrayectoriaA}
 %----------------------------------------------------------%trayectoria C---------------------------------------------------- 
 \begin{UCtrayectoriaA}{C}{El actor ingresó un nombre de trayectoria repetido.}
    \UCpaso[\UCsist] Muestra el mensaje \cdtIdRef{MSG7}{Registro repetido} y señala el campo que presenta la duplicidad en la pantalla 
	    \cdtIdRef{CU 5.1.1.1}{Registrar trayectoria}, indicando al actor que existe una trayectoria con el mismo nombre.
    \UCpaso[] Continúa con el paso \ref{cu5.1.1.1:ingresaDatos} de la trayectoria principal.
 \end{UCtrayectoriaA}
 %----------------------------------------------------------%trayectoria D---------------------------------------------------- 
 \begin{UCtrayectoriaA}{D}{El actor ingresó un tipo de dato incorrecto.}
    \UCpaso[\UCsist] Muestra el mensaje \cdtIdRef{MSG4}{Formato incorrecto} y señala el campo que presenta el dato inválido en la 
    pantalla \cdtIdRef{CU 5.1.1.1}{Registrar trayectoria} para indicar que se ha ingresado un tipo de dato inválido.
    \UCpaso[] Continúa con el paso \ref{cu5.1.1.1:ingresaDatos} de la trayectoria principal.
 \end{UCtrayectoriaA}
 %----------------------------------------------------------%trayectoria E--------------------------------------------------  
 \begin{UCtrayectoriaA}{E}{El actor proporciona un dato que excede la longitud máxima.}
    \UCpaso[\UCsist] Muestra el mensaje \cdtIdRef{MSG5}{Se ha excedido la longitud máxima del campo} y señala el campo que excede la 
    longitud en la pantalla \cdtIdRef{CU 5.1.1.1}{Registrar trayectoria}, para indicar que el dato excede el tamaño máximo permitido.
    \UCpaso[] Continúa con el paso \ref{cu5.1.1.1:ingresaDatos} de la trayectoria principal.
 \end{UCtrayectoriaA}

\subsection{Puntos de extensión}

\UCExtensionPoint{El actor requiere registrar un paso de la trayectoria}
	{Paso \ref{cu5.1.1.1:gestionaPasos}}
	{\cdtIdRef{CU 5.1.1.1.1}{Registrar paso}}
\UCExtensionPoint{El actor requiere modificar un paso de la trayectoria}
	{Paso \ref{cu5.1.1.1:gestionaPasos}}
	{\cdtIdRef{CU 5.1.1.1.2}{Modificar paso}}	
\UCExtensionPoint{El actor requiere eliminar un paso de la trayectoria}
	{Paso \ref{cu5.1.1.1:gestionaPasos}}
	{\cdtIdRef{CU 5.1.1.1.3}{Eliminar paso}}
  