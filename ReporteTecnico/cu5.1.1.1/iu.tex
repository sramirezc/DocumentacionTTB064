\subsection{IU 5.1.1.1 Registrar trayectoria}

\subsubsection{Objetivo}
	
	Esta pantalla permite al actor registrar la información de una trayectoria, así como gestionar los pasos de la misma.

\subsubsection{Diseño}

    En la figura ~\ref{IU 5.1.1.1} se muestra la pantalla ``Registrar trayectoria'' que permite registrar una trayectoria. El actor deberá ingresar la información solicitada, 
    esto incluye los pasos de la trayectoria.\\
    
    
    Una vez ingresada la información solicitada en la pantalla, deberá oprimir el botón 
    \cdtButton{Aceptar}, el sistema validará y registrará la información sólo si se han cumplido todas las reglas de negocio establecidas.  \\
    
    Finalmente se mostrará el mensaje \cdtIdRef{MSG1}{Operación exitosa} en la pantalla \cdtIdRef{IU 5.1.1}{Gestionar trayectorias},
    para indicar que la información de la trayectoria
    se ha registrado correctamente.        


    \IUfig[.9]{cu5.1.1.1/images/iu.png}{IU 5.1.1.1}{Registrar trayectoria}


\subsubsection{Comandos}
\begin{itemize}
	\item \cdtButton{Registrar}: Permite al actor solicitar el registro de un paso, dirige a la pantalla \cdtIdRef{IU 5.1.1.1.1}{Registrar paso}.
	\item \btnEditar[Modificar]: Permite al actor solicitar la modificación de un paso, dirige a la pantalla \cdtIdRef{IU 5.1.1.1.2}{Modificar paso}.
	\item \btnEliminar[Eliminar]: Permite al actor solicitar la eliminación de un paso, dirige a la pantalla \cdtIdRef{IU 5.1.1.1.3}{Eliminar paso}.
	\item \cdtButton{Aceptar}: Permite al actor guardar el registro de la trayectoria, dirige a la pantalla \cdtIdRef{IU 5.1.1}{Gestionar trayectorias}.
	\item \cdtButton{Cancelar}: Permite al actor cancelar el registro de la trayectoria, dirige a la pantalla \cdtIdRef{IU 5.1.1}{Gestionar trayectorias}.
\end{itemize}

\subsubsection{Mensajes}

	
\begin{description}
	\item[ \cdtIdRef{MSG1}{Operación exitosa}:] Se muestra en la pantalla \cdtIdRef{IU 5.1.1}{Gestionar trayectorias} para indicar que el registro fue exitoso.
	\item[\cdtIdRef{MSG4}{Dato obligatorio}:] Se muestra en la pantalla \cdtIdRef{CU 5.1.1.1}{Registrar trayectoria} cuando no se ha ingresado un dato marcado como obligatorio.
	\item[\cdtIdRef{MSG5}{Dato incorrecto}:] Se muestra en la pantalla \cdtIdRef{CU 5.1.1.1}{Registrar trayectoria} cuando el tipo de dato ingresado no cumple con el tipo de dato solicitado en el campo.
	\item[\cdtIdRef{MSG6}{Longitud inválida}:] Se muestra en la pantalla \cdtIdRef{CU 5.1.1.1}{Registrar trayectoria} cuando se ha excedido la longitud de alguno de los campos.
	\item[\cdtIdRef{MSG7}{Registro repetido}:] Se muestra en la pantalla \cdtIdRef{CU 5.1.1.1}{Registrar trayectoria} cuando se haya ingresado una clave que ya esté registrada.
	\item[\cdtIdRef{MSG15}{Dato no registrado}:] Se muestra en la pantalla \cdtIdRef{CU 5.1.1.1}{Registrar trayectoria} cuando un elemento referenciado no existe en el sistema.
	\item[\cdtIdRef{MSG23}{Caracteres inválidos}:] Se muestra en la pantalla \cdtIdRef{CU 5.1.1.1}{Registrar trayectoria} cuando el nombre del caso de uso contiene un caracter no válido.
	\item[\cdtIdRef{MSG27}{Token incorrecto}:] Se muestra en la pantalla \cdtIdRef{CU 5.1.1.1}{Registrar trayectoria} cuando el token ingresado está mal formado.
\end{description}
