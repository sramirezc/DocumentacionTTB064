\begin{UseCase}{CU 11.1.2}{Modificar atributo}
	{
		Este caso de uso permite al analista modificar la información de un atributo. 
	}
	
	\UCitem{Actor}{\cdtRef{actor:liderAnalisis}{Líder de análisis}, \cdtRef{actor:analista}{Analista}}
	\UCitem{Propósito}{
		Registrar la información de un atributo.
	}
	\UCitem{Entradas}{
		\begin{UClist}
			\UCli \cdtRef{Elemento:Nombre}{Nombre}: \ioEscribir.
 			\UCli \cdtRef{Atributo:Descripcion}{Descripción}: \ioEscribir.
 			\UCli \cdtRef{Atributo:Obligatorio}{Obligatorio}: \ioCheckBox.
 			\UCli \cdtRef{Atributo:Longitud}{Longitud}: \ioEscribir.
  			\UCli \cdtRef{gls:TipoDeDato}{Tipo de Dato}: \ioSeleccionar.
  			\UCli \cdtRef{}{Formato de archivo}: \ioEscribir.
  			\UCli \cdtRef{}{Tamaño de archivo}: \ioEscribir.
  			\UCli \cdtRef{}{Unidad}: \ioSeleccionar.
  			\UCli \cdtRef{}{Otro tipo de dato}: \ioEscribir.
		\end{UClist}
	}
	\UCitem{Salidas}{
		\begin{UClist}
			\UCli \cdtRef{Elemento:Nombre}{Nombre}: \ioObtener.
 			\UCli \cdtRef{Atributo:Descripcion}{Descripción}: \ioObtener.
 			\UCli \cdtRef{Atributo:Obligatorio}{Obligatorio}: \ioObtener.
 			\UCli \cdtRef{Atributo:Longitud}{Longitud}: \ioObtener.
  			\UCli \cdtRef{gls:TipoDeDato}{Tipo de Dato}: \ioObtener.
  			\UCli \cdtRef{}{Formato de archivo}: \ioObtener.
  			\UCli \cdtRef{}{Tamaño de archivo}: \ioObtener.
  			\UCli \cdtRef{}{Unidad}: \ioObtener.
  			\UCli \cdtRef{}{Otro tipo de dato}: \ioObtener.
		\end{UClist}
	}
	
	\UCitem{Mensajes}{
		Ninguno
	}

	\UCitem{Precondiciones}{
		Ninguna
	}
	
	\UCitem{Postcondiciones}{
		Ninguna
	}

	\UCitem{Errores}{
		\begin{UClist}
			\UCli \cdtIdRef{MSG4}{Dato obligatorio}: Se muestra en la pantalla \cdtIdRef{IU 11.1.2}{Modificar atributo} cuando no se ha ingresado un dato marcado como obligatorio.
			\UCli \cdtIdRef{MSG5}{Dato incorrecto}: Se muestra en la pantalla \cdtIdRef{IU 11.1.2}{Modificar atributo} cuando el tipo de dato ingresado no cumple con el tipo de dato solicitado en el campo.
			\UCli \cdtIdRef{MSG6}{Longitud inválida}: Se muestra en la pantalla \cdtIdRef{IU 11.1.2}{Modificar atributo} cuando se ha excedido la longitud de alguno de los campos.
			\UCli \cdtIdRef{MSG7}{Registro repetido}: Se muestra en la pantalla \cdtIdRef{IU 11.1.2}{Modificar atributo} cuando se registre un atributo con un nombre que ya este registrado.
			\UCli \cdtIdRef{MSG23}{Caracteres inválidos}: Se muestra en la pantalla \cdtIdRef{IU 11.1.2}{Modificar atributo} cuando el nombre del atributo contiene un caracter no válido.
		\end{UClist}
	}

	\UCitem{Tipo}{
		Secundario, extiende de los casos de uso \cdtIdRef{CU 11.1}{Registrar entidad} y \cdtIdRef{CU 11.2}{Modificar entidad}.
	}
\end{UseCase}
%-------------------------------------------------------%trayectoria Principal-----------------------------------------------
 \begin{UCtrayectoria}
    \UCpaso[\UCactor] Oprime el botón \btnEditar del atributo que desea modificar en la pantalla \cdtIdRef{IU 11.1}{Registrar entidad} o \cdtIdRef{IU 11.2}{Modificar entidad}.
	\UCpaso[\UCsist] Busca la información del atributo.
    \UCpaso[\UCsist] Muestra la información encontrada en la pantalla \cdtIdRef{IU 11.1.2}{Modificar atributo}. 
    \UCpaso[\UCactor] Modifica la información del atributo en la pantalla \cdtIdRef{IU 11.1.2}{Modificar atributo}. \label{cu11.1.2:ingresaDatos}
    \UCpaso[\UCactor] Solicita guardar el atributo oprimiendo el botón \cdtButton{Aceptar} de la pantalla \cdtIdRef{IU 11.1.2}{Modificar atributo}. \refTray{A}
    \UCpaso[\UCsist] Verifica que el actor ingrese todos los campos obligatorios con base en la regla de negocio  \cdtIdRef{RN8}{Datos obligatorios}. \refTray{B}
    \UCpaso[\UCsist] Verifica que el nombre del atributo no se encuentre registrado en el sistema con base en la regla de negocio  \cdtIdRef{RN6}{Unicidad de nombres}. \refTray{C}
    \UCpaso[\UCsist] Verifica que los datos requeridos sean proporcionados correctamente como se especifica en la regla de negocio \cdtIdRef{RN7}{Información correcta}. \refTray{D} \refTray{E}
    \UCpaso[\UCsist] Verifica que el nombre no contenga caracteres inválidos con base en la regla de negocio \cdtIdRef{RN2}{Nombres de los elementos}. \refTray{F}
    \UCpaso[\UCsist] Modifica la información del atributo.
 \end{UCtrayectoria}
 
 %----------------------------------------------------------%trayectoria A---------------------------------------------------- 
 \begin{UCtrayectoriaA}[Fin del caso de uso]{A}{El actor desea cancelar la operación.}
    \UCpaso[\UCactor] Solicita cancelar la operación oprimiendo el botón \cdtButton{Cancelar} de la pantalla \cdtIdRef{IU 11.1.2}{Modificar atributo}.
    \UCpaso[\UCsist] Muestra la pantalla donde se solicitó la operación.
 \end{UCtrayectoriaA}
  %----------------------------------------------------------%trayectoria B---------------------------------------------------- 
 \begin{UCtrayectoriaA}{B}{El actor no ingresó algún dato marcado como obligatorio.}
    \UCpaso[\UCsist] Muestra el mensaje \cdtIdRef{MSG4}{Dato obligatorio} y señala el campo que presenta el error en la pantalla 
	    \cdtIdRef{IU 11.1.2}{Modificar atributo}, indicando al actor que el dato es obligatorio.
    \UCpaso[] Continúa con el paso \ref{cu11.1.2:ingresaDatos} de la trayectoria principal.
 \end{UCtrayectoriaA}
 %----------------------------------------------------------%trayectoria C---------------------------------------------------- 
 \begin{UCtrayectoriaA}{C}{El actor ingresó un nombre de entidad repetido.}
    \UCpaso[\UCsist] Muestra el mensaje \cdtIdRef{MSG7}{Registro repetido} y señala el campo que presenta la duplicidad en la pantalla 
	    \cdtIdRef{IU 11.1.2}{Modificar atributo}, indicando al actor que existe un atributo con el mismo nombre.
    \UCpaso[] Continúa con el paso \ref{cu11.1.2:ingresaDatos} de la trayectoria principal.
 \end{UCtrayectoriaA}
 
 %----------------------------------------------------------%trayectoria D----------------------------------------------------  
 \begin{UCtrayectoriaA}{D}{El actor proporciona un dato que excede la longitud máxima.}
    \UCpaso[\UCsist] Muestra el mensaje \cdtIdRef{MSG5}{Se ha excedido la longitud máxima del campo} y señala el campo que excede la 
    longitud en la pantalla \cdtIdRef{IU 11.1.2}{Modificar atributo}, para indicar que el dato excede el tamaño máximo permitido.
    \UCpaso[] Continúa con el paso \ref{cu11.1.2:ingresaDatos} de la trayectoria principal.
 \end{UCtrayectoriaA}
 %----------------------------------------------------------%trayectoria E---------------------------------------------------- 
 \begin{UCtrayectoriaA}{E}{El actor ingresó un tipo de dato incorrecto.}
    \UCpaso[\UCsist] Muestra el mensaje \cdtIdRef{MSG4}{Formato incorrecto} y señala el campo que presenta el dato inválido en la 
    pantalla \cdtIdRef{IU 11.1.2}{Modificar atributo} para indicar que se ha ingresado un tipo de dato inválido.
    \UCpaso[] Continúa con el paso \ref{cu11.1.2:ingresaDatos} de la trayectoria principal.
 \end{UCtrayectoriaA}
 %----------------------------------------------------------%trayectoria F---------------------------------------------------- 
 \begin{UCtrayectoriaA}{F}{El actor ingresó un nombre con caracteres inválidos.}
    \UCpaso[\UCsist] Muestra el mensaje \cdtIdRef{MSG23}{Caracteres inválidos} y señala el campo que contiene los caracteres inválidos.
    \UCpaso[] Continúa con el paso \ref{cu11.1.2:ingresaDatos} de la trayectoria principal.
 \end{UCtrayectoriaA}
