\subsection{IU 11.1.2 Modificar atributo}

\subsubsection{Objetivo}
	
	Esta pantalla permite al actor modificar la información de un atributo.

\subsubsection{Diseño}

    En la figura ~\ref{IU 11.1.2} se muestra la pantalla ``Modificar atributo'' que permite modificar un atributo. El actor deberá ingresar la información solicitada,
    y oprimir el botón \cdtButton{Aceptar}. El sistema validará y registrará la información sólo si se han cumplido todas las reglas de negocio establecidas.  \\
    
    Para aplicar los cambios realizados, el actor deberá confirmar la operación en la pantalla \cdtIdRef{IU 11.1}{Registrar entidad} o en la pantalla \cdtIdRef{IU 11.2}{Modificar entidad}. \\
	
	La información mostrada dependerá del tipo de dato:
	\begin{itemize}
		\item Tipo de dato ``Archivo'', el sistema mostrará los formatos válidos así como el tamaño máximo, como se muestra en la figura ~\ref{IU 11.1.2A}.
		\item Tipo de dato ``Booleano'' o ``Fecha'', el sistema no mostrará ningún dato extra, como se muestra en la figura ~\ref{IU 11.1.2B}.
		\item Tipo de dato ``Cadena'' o ``Entero'' o ``Flotante'', el sistema mostrará la longitud máxima, como se muestra en la figura ~\ref{IU 11.1.2C}.
		\item Tipo de dato ``Otro'', el sistema mostrará el tipo especificado por el actor, como se muestra en la figura ~\ref{IU 11.1.2D}.
	\end{itemize}
	
	\IUfig[.8]{cu11.1.2/images/iu.png}{IU 11.1.2}{Modificar atributo}
    \IUfig[.8]{cu11.1.2/images/iu.png}{IU 11.1.2A}{Modificar atributo: Archivo}
    \IUfig[.8]{cu11.1.2/images/iu.png}{IU 11.1.2B}{Modificar atributo: Booleano, Fecha}
    \IUfig[.8]{cu11.1.2/images/iu.png}{IU 11.1.2C}{Modificar atributo: Cadena, Entero, Flotante}
    \IUfig[.8]{cu11.1.2/images/iu.png}{IU 11.1.2D}{Modificar atributo: Otro}
	

\subsubsection{Comandos}
\begin{itemize}
	\item \cdtButton{Aceptar}: Permite al actor guardar el registro del atributo, dirige a la pantalla \cdtIdRef{IU 11.1}{Registrar entidad} o \cdtIdRef{IU 11.2}{Modificar entidad}, según corresponda.
	\item \cdtButton{Cancelar}: Permite al actor cancelar el registro del atributo, dirige a la pantalla \cdtIdRef{IU 11.1}{Registrar entidad} o \cdtIdRef{IU 11.2}{Modificar entidad}, según corresponda.
\end{itemize}

\subsubsection{Mensajes}

	
\begin{description}
	\item[\cdtIdRef{MSG4}{Dato obligatorio}:] Se muestra en la pantalla \cdtIdRef{IU 11.1.2}{Modificar atributo} cuando no se ha ingresado un dato marcado como obligatorio.
	\item[\cdtIdRef{MSG5}{Dato incorrecto}:] Se muestra en la pantalla \cdtIdRef{IU 11.1.2}{Modificar atributo} cuando el tipo de dato ingresado no cumple con el tipo de dato solicitado en el campo.
	\item[\cdtIdRef{MSG6}{Longitud inválida}:] Se muestra en la pantalla \cdtIdRef{IU 11.1.2}{Modificar atributo} cuando se ha excedido la longitud de alguno de los campos.
	\item[\cdtIdRef{MSG7}{Registro repetido}:] Se muestra en la pantalla \cdtIdRef{IU 11.1.2}{Modificar atributo} cuando se registre un atributo con un nombre que ya esté registrado.
\end{description}
