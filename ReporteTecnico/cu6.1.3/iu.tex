\subsection{IU 6.1.3 Eliminar acción}

\subsubsection{Objetivo}
	
	Esta pantalla permite al actor eliminar la información de una pantalla.

\subsubsection{Diseño}

    En la figura ~\ref{IU 6.1.3} se muestra la pantalla ``Eliminar acción'' que permite confirmar la eliminación de una acción. \\
    
    Si al solicitar la operación la pantalla está siendo referenciada por algún caso de uso el sistema mostrará el mensaje \cdtIdRef{MSG14}{Eliminación no permitida} en una pantalla emergente, en caso contrario el usuario 
    deberá confirmar la operación presionando el botón \cdtButton{Aceptar}.\\
    
    Una vez que se confirma la operación la acción será eliminada de la tabla de la pantalla \cdtIdRef{CU 6.1}{Registrar pantalla} o de la pantalla \cdtIdRef{CU 6.2}{Modificar pantalla}, según corresponda.  \\


    \IUfig[.9]{cu6.1.3/images/iu.png}{IU 6.1.3}{Eliminar acción}

\subsubsection{Comandos}
\begin{itemize}
	\item \cdtButton{Aceptar}: Permite al actor confirmar la eliminación de la acción o cerrar el mensaje \cdtIdRef{MSG14}{Eliminación no permitida} de la pantalla emergente, dirige a la pantalla donde se solicitó la operación.
	\item \cdtButton{Cancelar}: Permite al actor cancelar la eliminación de la pantalla, dirige a la pantalla donde se solicitó la operación.
\end{itemize}

\subsubsection{Mensajes}
	
\begin{description}
	\item[\cdtIdRef{MSG14}{Eliminación no permitida}:] Se muestra en la pantalla \cdtIdRef{IU 6.1.3}{Eliminar acción} cuando no se pueda eliminar la acción debido a que está siendo referenciada en algún caso de uso.	
	
\end{description}
