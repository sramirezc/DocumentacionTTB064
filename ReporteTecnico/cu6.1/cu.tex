\begin{UseCase}{CU 6.1}{Registrar pantalla}
	{
		Las pantallas son la interfaz del sistema con el usuario, permiten solicitar o mostrarle información. Este caso de uso permite al analista registrar la información de una pantalla. 
	}
	
	\UCitem{Actor}{\cdtRef{actor:liderAnalisis}{Líder de análisis}, \cdtRef{actor:analista}{Analista}}
	\UCitem{Propósito}{
		Registrar la información de una pantalla.
	}
	\UCitem{Entradas}{
		\begin{UClist}
			\UCli \cdtRef{Elemento:Numero}{Número}: \ioEscribir.
			\UCli \cdtRef{Elemento:Nombre}{Nombre}: \ioEscribir.
			\UCli \cdtRef{Elemento:Descripcion}{Descripción}: \ioEscribir.
			\UCli \cdtRef{Pantalla:Imagen}{Imagen}: \ioArchivo.
		\end{UClist}
	}
	\UCitem{Salidas}{
		\begin{UClist}
			\UCli \cdtRef{Elemento:Clave}{Clave}: \ioCalcular{\cdtIdRef{RN12}{Identificador de elemento}}.
		\end{UClist}
	}
	
	\UCitem{Mensajes}{
		\begin{UClist}
			\UCli \cdtIdRef{MSG1}{Operación exitosa}: Se muestra en la pantalla \cdtIdRef{IU 6}{Gestionar pantallas} para indicar que el registro fue exitoso.
		\end{UClist}
	}

	\UCitem{Precondiciones}{
		Ninguna
	}
	
	\UCitem{Postcondiciones}{
		Ninguna
	}

	\UCitem{Errores}{
		\begin{UClist}
			\UCli \cdtIdRef{MSG4}{Dato obligatorio}: Se muestra en la pantalla \cdtIdRef{IU 6.1}{Registrar pantalla} cuando no se ha ingresado un dato marcado como obligatorio.
			\UCli \cdtIdRef{MSG5}{Dato incorrecto}: Se muestra en la pantalla \cdtIdRef{IU 6.1}{Registrar pantalla} cuando el tipo de dato ingresado no cumple con el tipo de dato solicitado en el campo.
			\UCli \cdtIdRef{MSG6}{Longitud inválida}: Se muestra en la pantalla \cdtIdRef{IU 6.1}{Registrar pantalla} cuando se ha excedido la longitud de alguno de los campos.
			\UCli \cdtIdRef{MSG7}{Registro repetido}: Se muestra en la pantalla \cdtIdRef{IU 6.1}{Registrar pantalla} cuando se registre una pantalla con un nombre o número que ya está registrado.
			\UCli \cdtIdRef{MSG23}{Caracteres inválidos}: Se muestra en la pantalla \cdtIdRef{IU 6.1}{Registrar pantalla} cuando el nombre de la pantalla contenga un carácter no válido.
			\UCli \cdtIdRef{MSG24}{Formato de archivo incorrecto}: Se muestra en la pantalla \cdtIdRef{IU 6.1}{Registrar pantalla} cuando la imagen de la pantalla no cumpla con el formato especificado.
			\UCli \cdtIdRef{MSG25}{Se ha excedido el tamaño del archivo}: Se muestra en la pantalla \cdtIdRef{IU 6.1}{Registrar pantalla} cuando la imagen de la pantalla exceda el tamaño especificado.
		\end{UClist}
	}

	\UCitem{Tipo}{
		Secundario, extiende del casos de uso \cdtIdRef{CU 6}{Gestionar pantallas}.
	}
\end{UseCase}
%-------------------------------------------------------%trayectoria Principal-----------------------------------------------
 \begin{UCtrayectoria}
    \UCpaso[\UCactor] Solicita registrar una pantalla oprimiendo el botón \cdtButton{Registrar} de la pantalla \cdtIdRef{IU 6}{Gestionar pantallas}.
    \UCpaso[\UCsist] Muestra la pantalla \cdtIdRef{IU 6.1}{Registrar pantalla} en la cual se realizará el registro de la pantalla. 
    \UCpaso[\UCactor] Ingresa el número, el nombre y la descripción de la pantalla. \label{cu6.1:ingresaDatos}
    \UCpaso[\UCactor] Selecciona la imagen de sus archivos locales.\label{cu6.1:seleccionaImagen}
    \UCpaso[\UCactor] Gestiona las acciones de la pantalla.\label{cu6.1:gestionaAcciones}
    \UCpaso[\UCactor] Solicita guardar la pantalla oprimiendo el botón \cdtButton{Aceptar} de la pantalla \cdtIdRef{IU 6.1}{Registrar pantalla}. \refTray{A} 
    
    \UCpaso[\UCsist] Verifica que el actor ingrese todos los campos obligatorios con base en la regla de negocio  \cdtIdRef{RN8}{Datos obligatorios}. \refTray{B}
    \UCpaso[\UCsist] Verifica que el nombre de la pantalla no se encuentre registrado en el sistema con base en la regla de negocio  \cdtIdRef{RN6}{Unicidad de nombres}. \refTray{C}
    \UCpaso[\UCsist] Verifica que los datos requeridos sean proporcionados correctamente como se especifica en la regla de negocio \cdtIdRef{RN7}{Información correcta}. \refTray{D} \refTray{E} \refTray{F} \refTray{G}
    \UCpaso[\UCsist] Verifica que el nombre no contenga caracteres inválidos con base en la regla de negocio \cdtIdRef{RN2}{Nombres de los elementos}. \refTray{H}
    \UCpaso[\UCsist] Verifica que el número del caso de uso no se encuentre registrado en el sistema con base en la regla de negocio  \cdtIdRef{RN1}{Unicidad de números}. \refTray{I}    
    
    \UCpaso[\UCsist] Registra la información de la pantalla en el sistema.
    \UCpaso[\UCsist] Muestra el mensaje \cdtIdRef{MSG1}{Operación exitosa} en la pantalla \cdtIdRef{IU 6}{Gestionar pantallas}
    para indicar al actor que el registro se ha realizado exitosamente.
 \end{UCtrayectoria}
 
 %----------------------------------------------------------%trayectoria A---------------------------------------------------- 
 \begin{UCtrayectoriaA}[Fin del caso de uso]{A}{El actor desea cancelar la operación.}
    \UCpaso[\UCactor] Solicita cancelar la operación oprimiendo el botón \cdtButton{Cancelar} de la pantalla \cdtIdRef{IU 6.1}{Registrar pantalla}.
    \UCpaso[\UCsist] Muestra la pantalla donde se solicitó la operación.
 \end{UCtrayectoriaA} 
  %----------------------------------------------------------%trayectoria B---------------------------------------------------- 
 \begin{UCtrayectoriaA}{B}{El actor no ingresó algún dato marcado como obligatorio.}
    \UCpaso[\UCsist] Muestra el mensaje \cdtIdRef{MSG4}{Dato obligatorio} y señala el campo que presenta el error en la pantalla 
	    \cdtIdRef{IU 6.1}{Registrar pantalla}, indicando al actor que el dato es obligatorio.
    \UCpaso[] Continúa con el paso \ref{cu6.1:ingresaDatos} de la trayectoria principal.
 \end{UCtrayectoriaA}
 %----------------------------------------------------------%trayectoria C---------------------------------------------------- 
 \begin{UCtrayectoriaA}{C}{El actor ingresó un nombre de entidad repetido.}
    \UCpaso[\UCsist] Muestra el mensaje \cdtIdRef{MSG7}{Registro repetido} y señala el campo que presenta la duplicidad en la pantalla 
	    \cdtIdRef{IU 6.1}{Registrar pantalla}, indicando al actor que existe una regla de negocio con el mismo nombre.
    \UCpaso[] Continúa con el paso \ref{cu6.1:ingresaDatos} de la trayectoria principal.
 \end{UCtrayectoriaA}
 
 %----------------------------------------------------------%trayectoria D----------------------------------------------------  
 \begin{UCtrayectoriaA}{D}{El actor proporciona un dato que excede la longitud máxima.}
    \UCpaso[\UCsist] Muestra el mensaje \cdtIdRef{MSG5}{Se ha excedido la longitud máxima del campo} y señala el campo que excede la 
    longitud en la pantalla \cdtIdRef{IU 6.1}{Registrar pantalla}, para indicar que el dato excede el tamaño máximo permitido.
    \UCpaso[] Continúa con el paso \ref{cu6.1:ingresaDatos} de la trayectoria principal.
 \end{UCtrayectoriaA}
 
 %----------------------------------------------------------%trayectoria E---------------------------------------------------- 
 \begin{UCtrayectoriaA}{E}{El actor ingresó un tipo de dato incorrecto.}
    \UCpaso[\UCsist] Muestra el mensaje \cdtIdRef{MSG4}{Formato incorrecto} y señala el campo que presenta el dato inválido en la 
    pantalla \cdtIdRef{IU 6.1}{Registrar pantalla} para indicar que se ha ingresado un tipo de dato inválido.
    \UCpaso[] Continúa con el paso \ref{cu6.1:ingresaDatos} de la trayectoria principal.
 \end{UCtrayectoriaA}

 %----------------------------------------------------------%trayectoria F----------------------------------------------------  
 \begin{UCtrayectoriaA}{F}{El actor proporciona una imagen de formato incorrecto.}
    \UCpaso[\UCsist] Muestra el mensaje \cdtIdRef{MSG24}{Formato de archivo incorrecto} y señala el campo donde se solicita la imagen
    en la pantalla \cdtIdRef{IU 6.1}{Registrar pantalla}, para indicar que el formato es incorrecto.
    \UCpaso[] Continúa con el paso \ref{cu6.1:seleccionaImagen} de la trayectoria principal.
 \end{UCtrayectoriaA}
 
 %----------------------------------------------------------%trayectoria G----------------------------------------------------  
 \begin{UCtrayectoriaA}{G}{El actor proporciona una imagen que excede el tamaño máximo}
    \UCpaso[\UCsist] Muestra el mensaje \cdtIdRef{MSG25}{Se ha excedido el tamaño del archivo} y señala el campo donde se solicita la imagen
    en la pantalla \cdtIdRef{IU 6.1}{Registrar pantalla}, para indicar que el tamaño máximo ha sido rebasado.
    \UCpaso[] Continúa con el paso \ref{cu6.1:seleccionaImagen} de la trayectoria principal.
 \end{UCtrayectoriaA}
 
 %----------------------------------------------------------%trayectoria H---------------------------------------------------- 
 \begin{UCtrayectoriaA}{H}{El actor ingresó un nombre con caracteres inválidos.}
    \UCpaso[\UCsist] Muestra el mensaje \cdtIdRef{MSG23}{Caracteres inválidos} y señala el campo que contiene los caracteres inválidos.
    \UCpaso[] Continúa con el paso \ref{cu6.1:ingresaDatos} de la trayectoria principal.
 \end{UCtrayectoriaA}
 %----------------------------------------------------------%trayectoria I---------------------------------------------------- 
 \begin{UCtrayectoriaA}{I}{El actor ingresó un número de regla de negocio repetido.}
    \UCpaso[\UCsist] Muestra el mensaje \cdtIdRef{MSG7}{Registro repetido} y señala el campo que presenta la duplicidad en la pantalla 
	    \cdtIdRef{IU 6.1}{Registrar pantalla}, indicando al actor que existe una regla de negocio con el mismo número.
    \UCpaso[] Continúa con el paso \ref{cu6.1:ingresaDatos} de la trayectoria principal.
 \end{UCtrayectoriaA}

\subsection{Puntos de extensión}

\UCExtensionPoint{El actor requiere registrar una acción}
	{Paso \ref{cu6.1:gestionaAcciones} de la trayectoria principal}
	{\cdtIdRef{CU 6.1.1}{Registrar acción}}
\UCExtensionPoint{El actor requiere modificar una acción}
	{Paso \ref{cu6.1:gestionaAcciones} de la trayectoria principal}
	{\cdtIdRef{CU 6.1.2}{Modificar acción}}	
\UCExtensionPoint{El actor requiere eliminar una acción}
	{Paso \ref{cu6.1:gestionaAcciones} de la trayectoria principal}
	{\cdtIdRef{CU 6.1.3}{Eliminar acción}}