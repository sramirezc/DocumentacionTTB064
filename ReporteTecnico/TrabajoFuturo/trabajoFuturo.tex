A lo largo del desarrollo del prototipo se identificaron diferentes áreas de oportunidad para el prototipo, a continuación se explica cada una de ellas:

\subsubsection{Soporte para reglas de negocio con base en algún estándar} 
	Actualmente el sistema únicamente brinda soporte para un conjunto limitado de reglas de negocio. Otorgar soporte para reglas de negocio con base en un estándar, permitiría al prototipo identificar con mayor facilidad el funcionamiento descrito en la regla de negocio, y con ello el espacio de escenarios válidos para la generación de casos de prueba aumentaría.
	
\subsubsection{Crear un estándar para la redacción de los casos de uso}
	Dado que el idioma español es ambiguo, resulta muy complicado comprender de manera explícita lo que indican los pasos de un caso de uso, frecuentemente para comprenderlo se recurre al contexto, sin embargo para un sistema informático esta tarea no es trivial. Crear un estándar que establezca la manera en que se debe redactar un caso de uso, aportaría una mejor comprensión de las redacciones y con ello el espacio de escenarios a probar también podría crecer.	
	
\subsubsection{Automatizar el proceso de configuración de una prueba}
	Para que el prototipo genere casos de prueba funcionales, es necesario que el usuario indique los valores de una gran cantidad de parámetros. Una parte de estos valores suelen localizarse en diferentes artefactos de software desarrollados durante la etapa de {\it Implementación}. Diseñar un mecanimo para analizar estos artefactos de software, permitiría automatizar el proceso de configuración de la prueba, por lo que se podría reducir considerablemente el tiempo dedicado a esta tarea.