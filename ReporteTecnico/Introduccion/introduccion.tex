Actualmente existen herramientas que permiten automatizar diferentes actividades del proceso de software, esto permite algunas mejoras en la calidad y productividad del software. Estas actividades pueden ser ubicadas en alguna de las etapas del proceso software, que generalmente son: {\it Análisis}, {\it Diseño}, {\it Implementación} y {\it Pruebas}. El presente reporte expone las actividades realizadas y los resultados obtenidos de desarrollar una herramienta que permite asistir a la etapa de {\it Pruebas}, a través de la generación semiautomática de casos de prueba funcionales basados en la documentación de análisis.


\section{Problemática}
\cfinput{Introduccion/problematica}

\section{Propuesta}
\cfinput{Introduccion/propuesta}

\section{Objetivos}
\cfinput{Introduccion/objetivos}

\section{Estructura del documento}
\cfinput{Introduccion/estructuraDocumento}

