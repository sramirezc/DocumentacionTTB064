
El \emph{proceso de software} tiene como etapas principales el análisis, diseño, implementación y pruebas, 
en proyectos grandes regularmente cada una de estas etapas se lleva a cabo por diferentes equipos
de trabajo. En la etapa de \emph{Análisis}, los analistas son los encargados definir qué hará el sistema
mediante la descripción del ámbito del sistema, los requerimientos del sistema, el análisis del comportamiento que debe tener el sistema para satisfacer 
las necesidades del usuario, así como documentar todo lo anterior, entre otras cosas. En la etapa de \emph{Diseño} el equipo de trabajo encargado
(diseñadores) describe cómo debe implementarse el código, cómo se guardará la información, la distribución de las pantallas, entre otras cosas, e igual
que los analistas deben documentar todas las actividades. El equipo de desarrollo es el encargado de codificar el sistema con base en el análisis 
y diseño, y por último el equipo de pruebas valida y verifica el funcionamiento del sistema.\\

Dentro del documento generado en la etapa de análisis se incluye la descripción de los casos de uso, las entidades, 
las reglas de negocio, los mensajes, las interfaces, los actores, la trayectoria principal
y las trayectorias alternativas del sistema. Toda la información recabada es utilizada a lo largo del proceso de software 
por los equipos de diseño, desarrollo y pruebas, los integrantes de estos
equipos deben interpretar la información del documento de análisis para diseñar, codificar el sistema y ejecutar las pruebas. 
Cabe mencionar que en muchas ocasiones en el documento de análisis se incluyen secciones que van más relacionadas con la etapa de diseño.\\

La documentación de análisis debe contener los elementos necesarios para describir el funcionamiento completo del sistema, considerando lo anterior
la información de este documento puede ser utilizada para asistir a todos los colaboradores del proyecto en distintas etapas
del proceso de software. Si suponemos una estructura del documento de análisis
bien definida con cada elemento organizado y las relaciones entre estos completamente descritas, toda la información es materia prima para construir algunos
artefactos de software tales como scripts para la base de datos, codificación automática de algunos componentes, generación automática de guiones de prueba, entre otros,
este mapeo del documento de análisis a otro artefacto podría ser viable cuando la estructura del documento es bien conocida, 
sin embargo hay que considerar que exiten diversas maneras y estilos para 
documentar sistemas.\\

Para lograr un mapeo automático del documento de análisis a otro artefacto de software es necesario conocer la redacción utilizada, ya que la manera de escribir 
el documento de análisis depende del analista y de las reglas definidas por la empresa u organización. No existe un lenguaje definido para realizar esta tarea
(regularmente se redacta en lenguaje natural) y esto nos impide acotar las posibles redacciones que se puedan utilizar.\\

El lenguaje natural no puede ser completamente modelado, así que debe buscarse una alternativa para mitigar este hecho, la forma en que trabaja un lenguaje de programación 
puede ser una primera aproximación; los lenguajes de programación son modelados en gramáticas y esto permite traducir cualquier código sin importar el estilo de cada pogramador. \\

Obligar a los analistas a aprender un lenguaje para escribir el documento de análisis resulta complicado y reduce la flexibilidad de escritura. \\

Para evitar que el analista aprenda un lenguaje de programación, se decidió construir una herramienta que permita la
captura de la información de manera organizada mediante formularios con una
estructura definida.\\

Lo anterior permite obtener la información necesaria para construir el documento de análisis y 
organizar todos los elementos de tal forma que puedan ser utilizados posteriormente como base de otros 
artefactos de software.\\

La etapa de pruebas representa aproximadamente un 40 \% del costo total del sistema debido a que se llevan a cabo, entre otras actividades, la ejecución de 
caso de prueba para validar la funcionalidad del sistema y estas pruebas son en la mayoría de los casos repetitivas y no es posible contemplar todos los escenarios
debido al número de validaciones y combinaciones entre ellas, por lo que se ubican como área de oportunidad
para reducir los costos del proceso completo del software.\\