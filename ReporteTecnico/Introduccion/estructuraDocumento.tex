	El presente documento está dirigido a los sinodales del Trabajo Terminal TTB064, retoma los objetivos descritos en el protocolo, considerando las observaciones realizadas en la primera evaluación del trabajo.\\
	
	Se destaca que existe un documento adicional, denominado {\it Documento de análisis} donde se describen los casos de uso con los que se desarrolló cada uno de los módulos.
	
	Los capítulos de este reporte son:

\begin{Citemize}
	\item \cdtRef{chp:antecedentes}{Capítulo 2. Antecedentes.} En este capítulo se exponen algunas herramientas para automatizar el proceso de software.
	\item \cdtRef{chp:metodologia}{Capítulo 3. Metodología.} En este capítulo se describe el proceso que se llevó a cabo para la construcción del prototipo.
	\item \cdtRef{chp:modulosDesarrollados}{Capítulo 4. Módulos desarrollados.} En este capítulo se describen los módulos y submódulos desarrollados.
	\item \cdtRef{chp:algoritmosDesarrollados}{Capítulo 5. Algoritmos desarrollados.} En este capítulo se muestran los principales algoritmos utilizados en el prototipo.
	\item \cdtRef{chp:pruebas}{Capítulo 6. Pruebas.} En este capítulo se muestran las pruebas realizadas al prototipo.
	\item \cdtRef{chp:resultados}{Capítulo 7. Resultados.} En este capítulo se exponen los resultados obtenidos del desarrollo del prototipo.
	\item \cdtRef{chp:conclusiones}{Capítulo 8. Conclusiones.} En este capítulo se expone un breve análisis sobre los resultados obtenidos y las experiencias adquiridas.
	\item \cdtRef{chp:trabajoFuturo}{Capítulo 9. Trabajo a futuro.} En este capítulo se describe la posible continuación del presente trabajo.

\end{Citemize}


