	%En la etapa de {\it Análisis} se definen elementos muy importantes para el desarrollo de un sistema \cite{kendall2005analisis}. Toda la información debe estar organizada para permitir a los colaboradores entender su contenido sin dejar espacio a las ambigüedades, para ello se construye un documento de análisis donde se define la información con la que se construirá el sistema, la estructura donde esta será organizada, las relaciones entre todos los elementos definidos, así como el comportamiento que deberá tener el sistema.\\

%Las herramientas que existen actualmente para generar un documento de análisis están limitadas debido a que no permiten realizar las siguientes tareas en conjunto:
%\begin{itemize}
 %\item Organizar y relacionar las entidades de un sistema con los casos de uso.
 %\item Proveer un mecanismo para organizar y relacionar las reglas de negocio utilizadas en la documentación de casos de uso.
 %\item Proveer un mecanismo para organizar y relacionar los mensajes utilizados en la documentación de casos de uso.
 %\item Entrelazar los casos de uso con las reglas de negocio, mensajes y entidades. 
 %\item Documentar el comportamiento que tendrá el sistema a través de las trayectorias principales y alternativas.
%\end{itemize}

	En la etapa de {\it Pruebas} resulta muy costoso en tiempo y recursos desarrollar las pruebas funcionales de un sistema debido a la cantidad de escenarios y validaciones que se deben de probar \cite{pressman2005software}. Además, las personas responsables de realizar las pruebas funcionales de un sistema requieren conocer el negocio, las validaciones y el comportamiento del sistema a través del estudio del documento de análisis o del sistema a probar.
	
	Algunas de las desventajas de las herramientas que existen actualmente para la automatización de pruebas son:
	
\begin{itemize}
 \item No reutilizan la información de los datos de entrada que se definieron en el documento de análisis.
 \item No reutilizan la información de las salidas esperadas que se describieron en el documento de análisis.
 \item No permiten reutilizar todas las validaciones descritas en las trayectorias de los casos de uso.
 \item No proveen un mecanismo que transforme las trayectorias de los casos de uso en casos de prueba funcionales.
\end{itemize}

Si existiera un base de datos con la información organizada del documento de análisis, sería posible automatizar el proceso de creación de casos de prueba funcionales y con ello posiblemente se podría disminuir el tiempo y los recursos necesarios para la etapa de {\it Pruebas}.