El contenido del documento se encuentra estructurado de la siguiente forma:

\begin{Citemize}
	\item En el capítulo 2 \cdtRef{chp:marcoTeorico}{Marco teórico} se describen algunos aspectos generales relacionados al prototipo.
	\item En el capítulo 3 \cdtRef{chp:requerimientos}{Requerimientos del sistema} se enlistan los requerimientos funcionales del prototipo descritos de manera general.
	\item En el capítulo 4 \cdtRef{chp:riesgos}{Analisis de riesgos} se muestran los posibles problemas que se pueden presentar al desarrollar o utilizar la herramienta, así como la estrategia de reducción estos riesgos.
	\item En el capítulo 5 \cdtRef{chp:modeloNegocios}{Modelo de negocio} se presenta el Modelo de negocio que consiste en la descripción de las reglas de negocio del prototipo y el ciclo de vida de los elementos del sistema a documentar.
	\item En el capítulo 6 \cdtRef{chp:modeloComportamiento}{Modelo de comportamiento} se presenta el Modelo de comportamiento donde se describen los casos de uso más representativos del prototipo.
	\item En el capítulo 7 \cdtRef{chp:modeloInteraccionUsuario}{Modelo de interacción con el usuario} se detalla el Modelo de interacción donde se describe la documentación de las pantallas y de los mensajes del prototipo.
\end{Citemize}


