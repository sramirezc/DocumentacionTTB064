\subsection{CU 5.1.1.1.2 Modificar paso}

\subsubsection{Objetivo}
	
	Esta pantalla permite al actor modificar la información de un paso.

\subsubsection{Diseño}

    En la figura ~\ref{IU 5.1.1.1.2} se muestra la pantalla ``Modificar paso'' que permite registrar un paso. El actor deberá ingresar la información solicitada, 
    esto incluye indicar quién realiza la acción y la redacción del paso.\\
    
    
    Una vez ingresada la información solicitada en la pantalla, deberá oprimir el botón 
    \cdtButton{Aceptar}, el sistema validará y modificará la información sólo si se han cumplido todas las reglas de negocio establecidas.  \\
    
    Finalmente se mostrará el mensaje \cdtIdRef{MSG1}{Operación exitosa} en la pantalla donde se solicitó la operación,
    para indicar que la información del paso
    se ha modificado correctamente.        


    \IUfig[.9]{cu5.1.1.1.2/images/iu.png}{IU 5.1.1.1.2}{Modificar paso}


\subsubsection{Comandos}
\begin{itemize}
	\item \cdtButton{Aceptar}: Permite al actor guardar la modificación del paso, dirige a la pantalla \cdtIdRef{IU 5.1.1.1}{Registrar trayectoria}.
	\item \cdtButton{Cancelar}: Permite al actor cancelar la modificación del paso, dirige a la pantalla \cdtIdRef{IU 5.1.1.1}{Registrar trayectoria}.
\end{itemize}

\subsubsection{Mensajes}

	
\begin{description}

	\item[ \cdtIdRef{MSG1}{Operación exitosa}:] Se muestra en la pantalla \cdtIdRef{IU 5.1.1.1}{Registrar trayectoria} para indicar que la modificación fue exitosa.
	\item[ \cdtIdRef{MSG4}{Dato obligatorio}:] Se muestra en la pantalla \cdtIdRef{IU 5.1.1.1.2}{Registrar paso} cuando no se ha ingresado un dato marcado como obligatorio.
	\item[ \cdtIdRef{MSG6}{Longitud inválida}:] Se muestra en la pantalla \cdtIdRef{IU 5.1.1.1.2}{Registrar paso} cuando se ha excedido la longitud de alguno de los campos.	
\end{description}
