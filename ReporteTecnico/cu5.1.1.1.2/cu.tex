\begin{UseCase}{CU 5.1.1.1.2}{Modificar paso}
	{
		Este caso de uso permite al actor modificar la información previamente registrada de un paso 
		de la trayectoria principal o de alguna trayectoria alternativa.
		
	}
	%\UCitem{\DONEUC}{Edición}
	\UCitem{Versión}{1.0}
	\UCccsection{Administración de Requerimientos}	
	\UCitem{Autor}{Natalia Giselle Hernández Sánchez}	
	\UCccitem{Evaluador}{}
	\UCitem{Operación}{Administración}
	\UCccitem{Prioridad}{Alta}
	\UCccitem{Complejidad}{Media}
	\UCccitem{Volatilidad}{Alta}
	\UCccitem{Madurez}{Baja}
	\UCitem{Estatus}{Terminado}
	\UCitem{Fecha del último estatus}{}
	\UCsection{Atributos}
	\UCitem{Actor}{\cdtRef{actor:liderAnalisis}{Líder de análisis}, \cdtRef{actor:analista}{Analista}}
	\UCitem{Propósito}{
		Modificar los pasos de la trayectoria principal o de alguna trayectoria alternativa.
	}
	\UCitem{Entradas}{
		\begin{UClist}
 			\UCli \cdtRef{Paso:RealizaActor}{Quien realiza el paso}: \ioSeleccionar.
 			\UCli \cdtRef{Paso:Redaccion}{Redacción del paso}: \ioEscribir.
		\end{UClist}
	}
	\UCitem{Salidas}{
		\begin{UClist}
			\UCli Número del paso.
		\end{UClist}
	}
	
	\UCitem{Mensajes}{
		\begin{UClist}
			\UCli \cdtIdRef{MSG1}{Operación exitosa}: Se muestra en la pantalla \cdtIdRef{CU 5.1.1.1}{Registrar trayectoria} o \cdtIdRef{CU 5.1.1.2}{Modificar trayectoria}
			para indicar que la modificación ha sido exitosa.
		\end{UClist}
	}

	\UCitem{Precondiciones}{
		\begin{UClist}
			\UCli Ninguna
		\end{UClist}
	}
	
	\UCitem{Postcondiciones}{
		\begin{UClist}
			\UCli Ninguna
		\end{UClist}
	}

	\UCitem{Errores}{
		\begin{UClist}
			\UCli \cdtIdRef{MSG4}{Dato obligatorio}: Se muestra en la pantalla \cdtIdRef{CU 5.1.1.1.2}{Modificar paso} cuando no se ha ingresado un dato marcado como obligatorio.
			\UCli \cdtIdRef{MSG6}{Longitud inválida}: Se muestra en la pantalla \cdtIdRef{CU 5.1.1.1.2}{Modificar paso} cuando se ha excedido la longitud de alguno de los campos.
		\end{UClist}
	}

	\UCitem{Tipo}{
		Secundario, extiende de los casos de uso \cdtIdRef{CU 5.1.1.1}{Registrar trayectoria} y \cdtIdRef{CU 5.1.1.2}{Modificar trayectoria}.
	}
\end{UseCase}
%-------------------------------------------------------%trayectoria Principal-----------------------------------------------
 \begin{UCtrayectoria}
	\UCpaso[\UCactor] Solicita modificar un paso oprimiento el botón \btnEditar de la pantalla \cdtIdRef{IU 5.1.1.1}{Registrar trayectoria} o la pantalla \cdtIdRef{CU 5.1.1.2}{Modificar trayectoria}. 
	\UCpaso[\UCsist] Muestra la pantalla \cdtIdRef{IU 5.1.1.1.2}{Modificar paso}. 
	
	\UCpaso[\UCactor] Selecciona si el actor o el sistema realiza el paso.\label{cu5.1.1.1.2:ingresaDatos}
	\UCpaso[\UCactor] Ingresa la redacción del paso.
	
	\UCpaso[\UCactor] Solicita modificar el paso oprimiendo el botón \cdtButton{Aceptar} de la pantalla \cdtIdRef{IU 5.1.1.1.2}{Modificar paso}. \refTray{A} 
	
	\UCpaso[\UCsist] Verifica que el actor haya ingresado todos los campos obligatorios con base en la regla de negocio \cdtIdRef{RN-S8}{Datos obligatorios}. \refTray{B}
	\UCpaso[\UCsist] Verifica que los datos requeridos sean proporcionados correctamente como se especifica en la regla de negocio \cdtIdRef{RN-S7}{Información correcta}. \refTray{C} 
	
	\UCpaso[\UCsist] Registra la información del paso en el sistema.
	
	\UCpaso[\UCsist] Muestra el mensaje \cdtIdRef{MSG1}{Operación exitosa} en la pantalla donde se solicitó la operación
	para indicar al actor que la modificación se ha realizado exitosamente.
    
\end{UCtrayectoria}

    
 %----------------------------------------------------------%trayectoria A---------------------------------------------------- 
 \begin{UCtrayectoriaA}[Fin del caso de uso]{A}{El actor desea cancelar la operación.}
    \UCpaso[\UCactor] Solicita cancelar la operación oprimiendo el botón \cdtButton{Cancelar} de la pantalla \cdtIdRef{CU 5.1.1.1.2}{Modificar paso}.
    \UCpaso[\UCsist] Muestra la pantalla \cdtIdRef{IU 5.1.1.1}{Registrar trayectoria} o \cdtIdRef{CU 5.1.1.2}{Modificar trayectoria} según corresponda.
 \end{UCtrayectoriaA}
 
 %----------------------------------------------------------%trayectoria B---------------------------------------------------- 
 \begin{UCtrayectoriaA}{B}{El actor no ingresó algún dato marcado como obligatorio.}
    \UCpaso[\UCsist] Muestra el mensaje \cdtIdRef{MSG4}{Dato obligatorio} y señala el campo que presenta el error en la pantalla 
	    \cdtIdRef{CU 5.1.1.1.2}{Modificar paso}, indicando al actor que el dato es obligatorio.
    \UCpaso[] Continúa con el paso \ref{cu5.1.1.1.2:ingresaDatos} de la trayectoria principal.
 \end{UCtrayectoriaA}
 %----------------------------------------------------------%trayectoria C --------------------------------------------------  
 \begin{UCtrayectoriaA}{C}{El actor proporciona un dato que excede la longitud máxima.}
    \UCpaso[\UCsist] Muestra el mensaje \cdtIdRef{MSG5}{Se ha excedido la longitud máxima del campo} y señala el campo que excede la 
    longitud en la pantalla \cdtIdRef{CU 5.1.1.1.2}{Modificar paso}, para indicar que el dato excede el tamaño máximo permitido.
    \UCpaso[] Continúa con el paso \ref{cu5.1.1.1.2:ingresaDatos} de la trayectoria principal.
 \end{UCtrayectoriaA}

  