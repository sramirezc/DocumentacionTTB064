\begin{UseCase}{CU 9.1}{Registrar mensaje}
	{
		Este caso de uso permite al analista registrar la información de un mensaje. 
	}
	
	\UCitem{Actor}{\cdtRef{actor:liderAnalisis}{Líder de análisis}, \cdtRef{actor:analista}{Analista}}
	\UCitem{Propósito}{
		Registrar la información de un mensaje.
	}
	\UCitem{Entradas}{
		\begin{UClist}
			\UCli \cdtRef{Elemento:Numero}{Número}: \ioEscribir.
			\UCli \cdtRef{Elemento:Nombre}{Nombre}: \ioEscribir.
			\UCli \cdtRef{Elemento:Descripcion}{Descripción}: \ioEscribir.
 			\UCli \cdtRef{Mensaje:Redaccion}{Redacción}: \ioEscribir.
 			\UCli \cdtRef{Mensaje:Parametrizado}{Parametrizado}: \ioCheckBox.
 			\UCli \cdtRef{ValorDelParametroEnMensaje}{Parámetros}: \ioEscribir.
		\end{UClist}
	}
	\UCitem{Salidas}{
		\begin{UClist}
			\UCli \cdtRef{Elemento:Clave}{Clave}: \ioCalcular{\cdtIdRef{RN12}{Identificador de elemento}}.
		\end{UClist}
	}
	
	\UCitem{Mensajes}{
		\begin{UClist}
			\UCli \cdtIdRef{MSG1}{Operación exitosa}: Se muestra en la pantalla \cdtIdRef{IU 9}{Gestionar mensajes} para indicar que el registro fue exitoso.
		\end{UClist}
	}

	\UCitem{Precondiciones}{
		Ninguna
	}
	
	\UCitem{Postcondiciones}{
		Ninguna
	}

	\UCitem{Errores}{
		\begin{UClist}
			\UCli \cdtIdRef{MSG4}{Dato obligatorio}: Se muestra en la pantalla \cdtIdRef{IU 9.1}{Registrar mensaje} cuando no se ha ingresado un dato marcado como obligatorio.
			\UCli \cdtIdRef{MSG5}{Dato incorrecto}: Se muestra en la pantalla \cdtIdRef{IU 9.1}{Registrar mensaje} cuando el tipo de dato ingresado no cumple con el tipo de dato solicitado en el campo.
			\UCli \cdtIdRef{MSG6}{Longitud inválida}: Se muestra en la pantalla \cdtIdRef{IU 9.1}{Registrar mensaje} cuando se ha excedido la longitud de alguno de los campos.
			\UCli \cdtIdRef{MSG7}{Registro repetido}: Se muestra en la pantalla \cdtIdRef{IU 9.1}{Registrar mensaje} cuando se registre un mensaje con un nombre o número que ya este registrado.
			\UCli \cdtIdRef{MSG23}{Caracteres inválidos}: Se muestra en la pantalla \cdtIdRef{IU 9.1}{Registrar mensaje} cuando el nombre del mensaje contiene un caracter no válido.
		\end{UClist}
	}

	\UCitem{Tipo}{
		Secundario, extiende del caso de uso \cdtIdRef{CU 9}{Gestionar mensajes}.
	}
\end{UseCase}
%-------------------------------------------------------%trayectoria Principal-----------------------------------------------
 \begin{UCtrayectoria}
    \UCpaso[\UCactor] Solicita registrar un mensaje oprimiendo el botón \cdtButton{Registrar} de la pantalla \cdtIdRef{IU 9}{Gestionar mensajes}.
    \UCpaso[\UCsist] Muestra la pantalla \cdtIdRef{IU 9.1}{Registrar mensaje} en la cual se realizará el registro del mensaje.
    \UCpaso[\UCactor] Ingresa el número, nombre y descripción del mensaje. \label{cu9.1:ingresaDatos}
    \UCpaso[\UCactor] Ingresa la redacción del mensaje en la pantalla \cdtIdRef{IU 9.1a}{Registrar mensaje: Parametrizado}. \refTray{A} \label{cu9.1:ingresaParam}
    \UCpaso[\UCactor] Presiona el botón \cdtButton{Aceptar}.
    \UCpaso[\UCsist] Verifica que el mensaje sea parametrizado. \refTray{B}
    \UCpaso[\UCsist] Muestra la pantalla \cdtIdRef{IU 9.1a}{Registrar mensaje: Parametrizado}.
    \UCpaso[\UCactor] Ingresa la descripción de cada parámetro. \label{cu9.1:ingresaPaso}
    \UCpaso[\UCactor] Solicita guardar el mensaje oprimiendo el botón \cdtButton{Aceptar} de la pantalla \cdtIdRef{IU 9.1}{Registrar mensaje}. \refTray{C} \label{cu9.1:guardarInfo}
    \UCpaso[\UCsist] Verifica que el actor ingrese todos los campos obligatorios con base en la regla de negocio  \cdtIdRef{RN8}{Datos obligatorios}. \refTray{D}
    \UCpaso[\UCsist] Verifica que el nombre del mensaje no se encuentre registrado en el sistema con base en la regla de negocio  \cdtIdRef{RN6}{Unicidad de nombres}. \refTray{E}
    \UCpaso[\UCsist] Verifica que los datos requeridos sean proporcionados correctamente como se especifica en la regla de negocio \cdtIdRef{RN7}{Información correcta}. \refTray{F}
    
    \UCpaso[\UCsist] Verifica que el nombre no contenga caracteres inválidos con base en la regla de negocio \cdtIdRef{RN2}{Nombres de los elementos}. \refTray{G}
    \UCpaso[\UCsist] Verifica que el número del caso de uso no se encuentre registrado en el sistema con base en la regla de negocio  \cdtIdRef{RN1}{Unicidad de números}. \refTray{H}
    
    \UCpaso[\UCsist] Registra la información del mensaje en el sistema.
    \UCpaso[\UCsist] Muestra el mensaje \cdtIdRef{MSG1}{Operación exitosa} en la pantalla \cdtIdRef{IU 9}{Gestionar mensajes}
    para indicar al actor que el registro se ha realizado exitosamente.
 \end{UCtrayectoria}
 
 
  %----------------------------------------------------------%trayectoria A---------------------------------------------------- 
 \begin{UCtrayectoriaA}{A}{El actor desea ingresar un parámetro.}
 	 \UCpaso[\UCactor] Ingresa el token ``PARAM.''.
 	\UCpaso[\UCsist] Busca las sugerencias de parametro en el sistema. 
 	\UCpaso[\UCsist] Muestra la lista de sugerencias encontradas.
 	\UCpaso[\UCactor] Selecciona un parámetro de la lista.
  	\UCpaso[\UCsist] Oculta el texto ``PARAM.'' y muestra únicamente el nombre del parámetro.
  	\UCpaso[\UCsist] Agrega a la pantalla un campo que tenga como etiqueta el nombre del parámetro. 
    \UCpaso[] Continúa con el paso \ref{cu9.1:ingresaParam} de la trayectoria principal.
 \end{UCtrayectoriaA}
 %----------------------------------------------------------%trayectoria B---------------------------------------------------- 
 \begin{UCtrayectoriaA}{B}{El mensaje no es parametrizado.}
	\UCpaso[\UCsist] Muestra la pantalla \cdtIdRef{IU 9.1b}{Registrar mensaje: No parametrizado}.
	\UCpaso[] Continúa con el paso \ref{cu9.1:guardarInfo} de la trayectoria principal.
 \end{UCtrayectoriaA} 
 %----------------------------------------------------------%trayectoria C---------------------------------------------------- 
 \begin{UCtrayectoriaA}[Fin del caso de uso]{C}{El actor desea cancelar la operación.}
    \UCpaso[\UCactor] Solicita cancelar la operación oprimiendo el botón \cdtButton{Cancelar} de la pantalla \cdtIdRef{IU 9.1}{Registrar mensaje}.
    \UCpaso[\UCsist] Muestra la pantalla donde se solicitó la operación.
 \end{UCtrayectoriaA}
  %----------------------------------------------------------%trayectoria D---------------------------------------------------- 
 \begin{UCtrayectoriaA}{D}{El actor no ingresó algún dato marcado como obligatorio.}
    \UCpaso[\UCsist] Muestra el mensaje \cdtIdRef{MSG4}{Dato obligatorio} y señala el campo que presenta el error en la pantalla 
	    \cdtIdRef{IU 9.1}{Registrar mensaje}, indicando al actor que el dato es obligatorio.
    \UCpaso[] Continúa con el paso \ref{cu9.1:ingresaDatos} de la trayectoria principal.
 \end{UCtrayectoriaA}
 %----------------------------------------------------------%trayectoria E---------------------------------------------------- 
 \begin{UCtrayectoriaA}{E}{El actor ingresó un nombre de mensaje repetido.}
    \UCpaso[\UCsist] Muestra el mensaje \cdtIdRef{MSG7}{Registro repetido} y señala el campo que presenta la duplicidad en la pantalla 
	    \cdtIdRef{IU 9.1}{Registrar mensaje}, indicando al actor que existe un mensaje con el mismo nombre.
    \UCpaso[] Continúa con el paso \ref{cu9.1:ingresaDatos} de la trayectoria principal.
 \end{UCtrayectoriaA}
 
 %----------------------------------------------------------%trayectoria F----------------------------------------------------  
 \begin{UCtrayectoriaA}{F}{El actor proporciona un dato que excede la longitud máxima.}
    \UCpaso[\UCsist] Muestra el mensaje \cdtIdRef{MSG5}{Se ha excedido la longitud máxima del campo} y señala el campo que excede la 
    longitud en la pantalla \cdtIdRef{IU 9.1}{Registrar mensaje}, para indicar que el dato excede el tamaño máximo permitido.
    \UCpaso[] Continúa con el paso \ref{cu9.1:ingresaDatos} de la trayectoria principal.
 \end{UCtrayectoriaA}
 
%----------------------------------------------------------%trayectoria G---------------------------------------------------- 
 \begin{UCtrayectoriaA}{G}{El actor ingresó un nombre con caracteres inválidos.}
    \UCpaso[\UCsist] Muestra el mensaje \cdtIdRef{MSG23}{Caracteres inválidos} y señala el campo que contiene los caracteres inválidos.
    \UCpaso[] Continúa con el paso \ref{cu9.1:ingresaDatos} de la trayectoria principal.
 \end{UCtrayectoriaA}
 %----------------------------------------------------------%trayectoria H---------------------------------------------------- 
 \begin{UCtrayectoriaA}{H}{El actor ingresó un número de mensaje repetido.}
    \UCpaso[\UCsist] Muestra el mensaje \cdtIdRef{MSG7}{Registro repetido} y señala el campo que presenta la duplicidad en la pantalla 
	    \cdtIdRef{IU 9.1}{Registrar mensaje}, indicando al actor que existe un mensaje con el mismo número.
    \UCpaso[] Continúa con el paso \ref{cu9.1:ingresaDatos} de la trayectoria principal.
 \end{UCtrayectoriaA}