\section{Ingeniería de software}
En la actualidad el software juega un papel muy importante a lo largo de todo el mundo. El software está presente en todos los ámbitos,
la fabricación industrial, el comercio, la salud, la educación son algunos ejemplos significativos de las áreas donde están presentes los 
sistemas informáticos. Por lo tanto, el desarrollo de software es un factor dominante en la economía de los países industrializados.\\

La ingeniería del software es una disciplina de la ingeniería cuya meta es el desarrollo costeable de sistemas de software. El software es
abstracto e intangible y no está restringido por materiales, o gobernado por leyes físicas o por procesos de manufactura. De alguna forma, esto
simplifica la ingeniería del software ya que no existen limitaciones físicas del potencial del software. 
Sin embargo, esta falta de restricciones naturales significa que el software puede llegar a ser extremadamente complejo y, por lo tanto, muy difícil de entender (SMV, 4).\\

La noción \emph{ingeniería de software} fue propuesta en una conferencia en 1968 donde se discutía acerca de la \emph{crisis del software}. Esta crisis del software
fue el resultado de la introducción de las nuevas computadoras hardware basadas en circuitos integrados. Su poder hizo que las aplicaciones
hasta ese entonces irrealizables fueran una propuesta factible. El software resultante fue de órdenes de magnitud más grande y más
complejo que los sistemas de software previos (SMV, 4).\\

Las notables mejoras en el funcionamiento del hardware, los profundos cambios en las arquitecturas de computadora, 
el gran incremento en la memoria y capacidad de almacenamiento, y
una amplia variedad de opciones de entradas y salidas exóticas han propiciado la existencia de
sistemas basados en computadora más sofisticados y complejos (PRESS,3).\\

Todos estos cambios mostraron la importancia de un enfoque formal al construir software, desde ese entonces 
se comenzó a definir el proceso que debía seguirse para construir software.\\

La ingeniería de software ha ayudado a mejorar considerablemente el software que se produce y a desarrollar métodos efectivos para la especificación, diseño 
e implementación de este. Las técnicas y herramientas utilizadas para producir software reducen el esfuerzo requerido para producir sistemas grandes y complejos.\\

Los ingenieros de software pueden estar orgullosos de sus logros. Sin software complejo
no habríamos explorado el espacio, no tendríamos Internet y telecomunicaciones modernas,
y todas las formas de viajar serían más peligrosas y caras. Dicha ingeniería ha hecho enormes
contribuciones, y no cabe dudad de que, en cuanto la disciplina madure, su contribución en el
siglo XXI será aún más grande (SMV, 4).\\

La ingeniería de software es una tecnología con varias capas (PRESS, 12). Como se aprecia en la figura
\refIM{fig:capasIS}{Capas de la ingeniería de software}, cualquier enfoque de ingeniería debe basarse en un \emph{compromiso
organizacional con la calidad}. El fundamento en el que se apoya la ingeniería de software es el compromiso con la calidad.\\

\IMfig[.9]{MarcoTeorico/images/capasIS.png}{fig:capasIS}{Capas de la ingeniería de software}

El fundamento para la ingeniería de software es la capa \emph{proceso}. El proceso de software permite desarrollar software en tiempo y de manera organizada.
El proceso de software define una estructura para elaborar el producto esperado de una forma eficaz, en este proceso
se utilizan métodos técnicos y se generan productos de trabajo (modelos, documentos, datos,
reportes, formatos, etc.).\\

Los \emph{métodos} de la ingeniería de software proporcionan la experiencia técnica para elaborar
software. Incluyen un conjunto amplio de tareas, como comunicación, análisis de los requerimientos, 
modelación del diseño, construcción del programa, pruebas y apoyo. Los métodos de
la ingeniería de software se basan en un conjunto de principios fundamentales que gobiernan
cada área de la tecnología e incluyen actividades de modelación y otras técnicas descriptivas. \\

Las \emph{herramientas} de la ingeniería de software proporcionan un apoyo automatizado o semiautomatizado 
para el proceso y los métodos. Cuando se integran las herramientas de modo
que la información creada por una pueda ser utilizada por otra, queda establecido un sistema
llamado \emph{ingeniería de software asistida por computadora} que apoya el desarrollo de software.\\

\section{Proceso de software}
Un \emph{proceso de software} es un conjunto coherente de actividades que se llevan a cabo para la producción de un software. 
El modelo básico para un proceso de software contiene cuatro etapas \emph{Análisis}, \emph{Diseño}, \emph{Implementación} y \emph{Pruebas}, 
cada una de estas etapas contiene una serie de tareas que se describen brevemente a continuación:

\begin{itemize}
	\item Análisis. En esta etapa se especifica qué hará el software, para ello se recaban los requerimientos del sistema, se construyen diagramas y se elabora el documento de análisis.
	\item Diseño. Esta etapa consiste en especificar cómo debe implementarse el software, con base en el análisis realizado. Para ello se analiza cómo estará organizada la información, se definen las tecnologías que deben utilizarse, 
		se crea la base de datos y se diseñan las pantallas del sistema.
	\item Implementación. En esta etapa se codifica el software.
	\item Pruebas. Consiste en verificar y validar que el software se ajusta a su especificación, para ello se ejecutan pruebas de funcionalidad, desempeño, usabilidad, entre otras.
\end{itemize}

Durante todas las etapas del proceso se busca asegurar la calidad del sistema a desarrollar, para ello es necesario tomar en cuenta los requerimientos y evaluar en qué grado se están cumpliendo. Dentro el proceso de
software también es importante reducir en lo mayor posible los costos, esto se logra minimizando el tiempo y el esfuerzo de las tareas de cada etapa del proceso.\\


\section{Etapa de análisis}
Como se mencionó antes, una de las tareas de la etapa de análisis es recabar los requerimientos del sistema, este es el punto de partida para la construcción de un sistema. Para comprender la naturaleza de un sistema, 
el analista debe conocer el dominio de información del sistema, así como la función requerida, el comportamiento, el rendimiento y la interconexión [Pressman, 20].\\

El trabajo de un analista es refinar el detalle de los requerimientos de un sistema y la funcionalidad que este va a tener. Debe construir modelos de los requerimientos, del flujo de información y del comportamiento operativo. Un análisis completo
y bien hecho asegura un producto de calidad que cumple con las necesidades que el cliente desea satisfacer. \\

\subsection{Casos de uso}
Una vez recopilados los requerimientos, el analista puede crear un conjunto de escenarios para facilitar la descripción de cómo el usuario utilizará al sistema. Estos escenarios se describen mediante casos de uso. Un modelo de caso de uso 
describe lo que hacer el sistema sin describir cómo lo hace. El modelo de caso de uso refleja la vista del sistema desde la perspectiva de los usuarios y divide la funcionalidad del sistema en comportamientos, servicios y respuestas que son 
significativos para los mismos usuarios (KK, 666). Desde la perspectiva del usuario, un caso de uso debe producir algo que es de valor.\\

Un caso de uso puede verse como una secuencia de transacciones en un sistema. Un caso de uso siempre describe tres cosas: un actor que inicia un evento; el evento que activa el caso de uso y el caso de uso que desempeña las acciones activadas
por el evento.\\

La trayectoria principal del caso de uso consiste en un flujo ideal del sistema, la trayectoria principal representa la realización normal, esperada y exitosa del caso de uso. Las variaciones o excepciones (trayectorias alternativas) también 
se pueden describir. Cada caso de uso tiene una descripción, no existe un formato estándar para describir los escenarios del caso de uso, por lo que cada organización se enfrenta con especificar qué lineamientos se deben seguir. Las áreas principales 
que debe tener la descripción de un caso de uso son:
\begin{itemize}
 \item {\bf Identificadores e iniciadores de caso de uso.} En esta área se agrega el identificador y nombre del caso de uso, los actores involucrados, una breve descripción de lo que logra el caso de uso y la iniciación (activación), es decir, lo que ocasionó que empezara el caso de uso.
 \item {\bf Pasos desempeñados (trayectorias).} Aquí se incluye el flujo del caso de uso representado en pasos y la información requerida para cada uno de estos.
 \item {\bf Condiciones, suposiciones y preguntas.} Aquí se incluyen la condición del sistema antes de que se pudiera desempeñar el caso de uso, así como el estado del sistema después de que el caso de uso se haya terminado.
\end{itemize}

\subsubsection{Principios del análisis}

Los sistemas se construyen para procesar datos, para obtener una salida específica a partir de una entrada mediante alguna función definida. A partir de esto uno de los principios del análisis requiere el examen del dominio de la información
y la creación de un modelo de datos que describa la organización de los elementos que participan como entradas o salidas del sistema.\\

Los modelos se crean para entender mejor la entidad que se va a construir. Cuando la entidad es algo físico (un edificio, un avión, una máquina), se puede construir un modelo idéntico en forma, pero más pequeño. Sin embargo, cuando la entidad en un sistema,
el modelo debe tomar una forma diferente. Debe ser capaz de modelar la información que transforma el sistema y las funciones que permiten que ocurran estas transformaciones, así como el comportamiento que tendrá el sistema a partir de ello.\\

Como consecuencia existen otros dos principios del análisis que se refieren a la construcción de modelos de función y comportamiento.

\subsection{Modelos funcionales y de comportamiento}
Los sistemas deben realizar al menos tres funciones genéricas para transformar la información: entrada, procesamiento y salida. Después es necesario descomponer cada función en funciones más específicas buscando que con cada iteración 
las funciones se detallen, hasta que se consigue representar una minuciosa definición de toda la funcionalidad del sistema.\\

La mayoría de los sistemas responden a acontecimientos del mundo exterior, esta característica de estímulo-respuesta forma la base del modelo de comportamiento. Un programa de computadora siempre está en un estado, un modo de comportamiento observable
exteriormente que cambia solamente cuando ocurre un proceso. Un modelo de comportamiento crea una representación de los estados del software y de los sucesos que causan que cambie de estado.\\

Como resultado de la etapa de análsis se desarrolla la \emph{especificación de requerimientos del software}, esta especificación puede verse como un proceso de representación donde todo el comportamiento del sistema está descrito.\\
