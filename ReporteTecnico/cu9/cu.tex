\begin{UseCase}{CU 9}{Gestionar mensajes}
	{
		Este caso de uso permite al analista visualizar los registros de los mensajes del sistema. También permite 
		al actor acceder a las operaciones de registro, modificación y eliminación de un mensaje.
	}
	
	\UCitem{Actor}{\cdtRef{actor:liderAnalisis}{Líder de análisis}, \cdtRef{actor:analista}{Analista}}
	\UCitem{Propósito}{
		Gestionar los mensajes registrados en la herramienta.
	}
	\UCitem{Entradas}{
		Ninguna
	}
	\UCitem{Salidas}{
		\begin{UClist}
			\UCli \cdtRef{Modulo:Clave}{Clave del módulo}: \ioObtener.
			\UCli \cdtRef{Modulo:Nombre}{Nombre del módulo}: \ioObtener.
			\UCli \cdtRef{Modulo:Descripcion}{Descripción del módulo}: \ioObtener.
			\UCli \cdtRef{Mensaje}{Mensaje}: \ioTabla{el \cdtRef{Elemento:Nombre}{Nombre}}{de los mensajes}.
		\end{UClist}
	}
	
	\UCitem{Mensajes}{
		\begin{UClist}
			\UCli \cdtIdRef{MSG2}{No existe información}: Se muestra en la pantalla \cdtIdRef{CU 9}{Gestionar mensajes} cuando no existen mensajes registrados.
		\end{UClist}
	}

	\UCitem{Precondiciones}{
		Ninguna
	}
	
	\UCitem{Postcondiciones}{
		\begin{UClist}
			\UCli Se podrá solicitar el registro de un mensaje por medio del caso de uso \cdtIdRef{CU 9.1}{Registrar mensaje}.
			\UCli Se podrá solicitar la modificación de un mensaje por medio del caso de uso \cdtIdRef{CU 9.2}{Modificar mensaje}.
			\UCli Se podrá solicitar la eliminación de un mensaje por medio del caso de uso \cdtIdRef{CU 9.3}{Eliminar mensaje}.
		\end{UClist}
	}

	\UCitem{Errores}{
		Ninguno
	}

	\UCitem{Tipo}{
		Primario
	}
\end{UseCase}
%-------------------------------------------------------%trayectoria Principal-----------------------------------------------
 \begin{UCtrayectoria}
    \UCpaso[\UCactor] Solicita gestionar los mensajes seleccionando la opción ``Mensajes'' del \cdtRef{menu:principal}{Menú principal}.
    \UCpaso[\UCsist] Busca la información de los mensajes registrados en el sistema. \refTray{A}
    \UCpaso[\UCsist] Muestra la información de los mensajes en la pantalla \cdtIdRef{IU 9}{Gestionar mensajes} y las operaciones 
    disponibles de acuerdo a la regla de negocio \cdtIdRef{RN-S15}{Elementos en uso}. 
    \UCpaso[\UCactor] Gestiona los mensajes a través de los botones: \cdtButton{Registrar}, \btnEditar y \btnEliminar. \label{cu8:gestiona}
 \end{UCtrayectoria}
 
 \begin{UCtrayectoriaA}[Fin del caso de uso]{A}{No existen registros de mensajes.}
    \UCpaso[\UCsist] Muestra el mensaje \cdtIdRef{MSG2}{No existe información} en pantalla \cdtIdRef{IU 9}{Gestionar mensajes} 
    para indicar que no hay registros de mensajes para mostrar.
 \end{UCtrayectoriaA}
 

\subsection{Puntos de extensión}

\UCExtensionPoint{El actor requiere registrar un mensaje}
	{Paso \ref{cu8:gestiona}}
	{\cdtIdRef{CU 9.1}{Registrar mensaje}}
\UCExtensionPoint{El actor requiere modificar un mensaje}
	{Paso \ref{cu8:gestiona}}
	{\cdtIdRef{CU 9.2}{Modificar mensaje}}	
\UCExtensionPoint{El actor requiere eliminar un mensaje}
	{Paso \ref{cu8:gestiona}}
	{\cdtIdRef{CU 9.3}{Eliminar mensaje}}
  