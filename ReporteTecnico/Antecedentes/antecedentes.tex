Un proceso de software es un conjunto coherente de actividades que se llevan a cabo para la producción de un software. Los procesos de software son complejos y los intentos por automatizarlos han tenido un éxito limitado debido a su inmensa diversidad. Aunque existen muchos procesos diferentes, algunas etapas son comunes entre ellos \cite{sommerville1992software}, \cite{pressman2005software}:
\begin{itemize}
	\item Análisis
	\item Diseño
	\item Implementación
	\item Pruebas
\end{itemize}


Si se considera que el costo total del desarrollo de un sistema de software es de 100 unidades de costo, la figura~ref{fig:costos}  muestra cómo se gastan estas en las diferentes actividades del proceso \cite{sommerville1992software}. 

	 \IUfigNoId[.7]{images/costosProceso.png}{fig:costos}{Distribución de costos}

\section{Etapa de Pruebas}
\cfinput{Antecedentes/etapaPruebas}

\section{Etapa de Análisis}
\cfinput{Antecedentes/etapaAnalisis}



