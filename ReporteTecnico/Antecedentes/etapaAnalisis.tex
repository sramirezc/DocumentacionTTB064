La etapa de {\it Análisis} consiste en identificar las necesidades del cliente y la funcionalidad que tendrá el sistema con base en los procesos del negocio para describir el comportamiento que tendrá el mismo, además de definir la información que será almacenada en una estructura de datos. \\

Las necesidades del cliente se transforman en requerimientos funcionales y no funcionales, estos describen el comportamiento del sistema, parte de este comportamiento es modelado mediante casos de uso (considerando el paradigma orientado a objetos). El producto de la etapa de Análisis es un documento donde se incluye la definición de los casos de uso, la cual requiere la descripción de entidades, especificación de los atributos,  reglas de negocio, interfaces, mensajes y perfiles de usuario que intervienen en el sistema.\\

La etapa de {\it Análisis} es muy importante ya que en ella se define el comportamiento del sistema y los errores u omisiones generados durante esta etapa se propagarán a todas las demás etapas del proceso de construcción del software \cite{sommerville1992software}, \cite{kendall2005analisis}.\\


Algunas de las herramientas que se utilizan para realizar la documentación de análisis, son:

\begin{longtable}{| p{.30\textwidth} | p{.10\textwidth} | p{.10\textwidth} | p{.10\textwidth} |}%
	\arrayrulecolor{black}%
	\rowcolor{black}%
	{\color{white}Herramienta} & {\color{white}UML} & {\color{white}Bases de datos} & {\color{white}Código abierto}\\ \hline
	\endhead%
	\arrayrulecolor{black}%
	Visual Paradigm & Sí & No & No\\ \hline
	Rational Rose & Sí & No & No\\ \hline
	Argo UML &  Sí & No & Sí\\ \hline
	Microsfot Word &  No & No & No\\ \hline
	\caption{Herramientas para documentar software}\label{fig:tablaAnalisis}
\end{longtable}%


Debido a las carencias de cada herramienta, las organizaciones recurren a una suite de herramientas para documentar sus sistemas. 




