\begin{UseCase}{CU 1}{Gestionar proyectos de Administrador}
	{
		Este caso de uso permite al Administrador visualizar todos los proyectos, así como solicitar el registro, consulta, modificación y eliminación de un proyecto.
	}
	
	\UCitem{Actor}{\cdtRef{actor:administrador}{Administrador}}
	\UCitem{Propósito}{
		Visualizar y gestionar los proyectos.
	}
	\UCitem{Entradas}{
		Ninguna
	}
	\UCitem{Salidas}{
		\begin{UClist}
			\UCli \cdtRef{Proyecto}{Proyecto}: \ioTabla{\cdtRef{Proyecto:Clave}{Clave}, \cdtRef{Proyecto:Nombre}{Nombre} y el \cdtRef{Colaborador}{Lídel del Proyecto}}{de los proyectos}.
		\end{UClist}
	}
	
	\UCitem{Mensajes}{
		\begin{UClist}
			\UCli \cdtIdRef{MSG2}{No existe información}: Se muestra en la pantalla \cdtIdRef{IU 1}{Gestionar proyectos de Administrador} cuando no existen proyectos registrados.
		\end{UClist}
	}

	\UCitem{Precondiciones}{
		Ninguna
	}
	
	\UCitem{Postcondiciones}{
		Ninguna
	}

	\UCitem{Errores}{
		Ninguno
	}

	\UCitem{Tipo}{
		Primario
	}
\end{UseCase}
%-------------------------------------------------------%trayectoria Principal-----------------------------------------------
 \begin{UCtrayectoria}
    \UCpaso[\UCactor] Solicita gestionar los proyectos presionando la opción ``Proyectos'' del menú \cdtRef{menu:principal}{Menú principal}.
    \UCpaso[\UCsist] Busca la información de los proyectos registrados en cualquier estado. \refTray{A}
    \UCpaso[\UCsist] Muestra la información de los proyectos en la pantalla \cdtIdRef{IU 1}{Gestionar proyectos de Administrador} y las operaciones 
    disponibles de acuerdo al estado en que se encuentra.
    \UCpaso[\UCactor] Gestiona los proyectos a través de los botones: \cdtButton{Registrar}, \btnConsulta, \btnEditar y \btnEliminar. \label{cu1:gestiona}
 \end{UCtrayectoria}
 
 \begin{UCtrayectoriaA}[Fin del caso de uso]{A}{No existen registros de proyectos.}
    \UCpaso[\UCsist] Muestra el mensaje \cdtIdRef{MSG2}{No existe información} en pantalla \cdtIdRef{IU 1}{Gestionar proyectos de Administrador} 
    para indicar que no hay registros de proyectos para mostrar.
 \end{UCtrayectoriaA}
 

\subsection{Puntos de extensión}

\UCExtensionPoint{El actor requiere registrar un proyecto}
	{Paso \ref{cu1:gestiona} de la trayectoria principal}
	{\cdtIdRef{CU 1.1}{Registrar proyecto}}
\UCExtensionPoint{El actor requiere modificar un proyecto}
	{Paso \ref{cu1:gestiona} de la trayectoria principal}
	{\cdtIdRef{CU 1.2}{Modificar proyecto}}	
\UCExtensionPoint{El actor requiere eliminar un proyecto}
	{Paso \ref{cu1:gestiona} de la trayectoria principal}
	{\cdtIdRef{CU 1.3}{Eliminar proyecto}}
\UCExtensionPoint{El actor requiere consultar un proyecto}
	{Paso \ref{cu1:gestiona} de la trayectoria principal}
	{\cdtIdRef{CU 1.4}{Consultar proyecto}}