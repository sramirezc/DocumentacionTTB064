En este capítulo se describen los casos de uso referentes al registro y modificación la información de las escuelas. \bigskip
     
    \begin{objetivos}[Elementos de un caso de uso]
	\item {\bf Resumen:} Descripción textual del caso de uso.
	\item {\bf Actores:} Lista de los actores que intervienen en el caso de uso.
	\item {\bf Propósito:} Una breve descripción del objetivo que busca el actor al ejecutar el caso de uso.
	\item {\bf Entradas:} Lista de los datos de entrada requeridos durante la ejecución del caso de uso.
	\item {\bf Salidas:} Lista de los datos de salida que presenta el sistema durante la ejecución del caso de uso.
	\item {\bf Precondiciones:} Descripción de las operaciones o condiciones que se deben cumplir previamente para que el caso de uso pueda ejecutarse correctamente.
	\item {\bf Postcondiciones:} Lista de los cambios que ocurrirán en el sistema después de la ejecución del caso de uso y de las consecuencias en el sistema.
	\item {\bf Reglas de negocio:} Lista de las reglas que describen, limitan o controlan algún aspecto del negocio del caso de uso.
	\item {\bf Errores:} Lista de los posibles errores que pueden surgir durante la ejecución del caso de uso.
	\item {\bf Trayectorias:} Secuencia de los pasos que ejecutará el caso de uso.
    \end{objetivos}

%=========================================================
\section{Actores del sistema}\label{sec:Comportamiento:ActoresSistema}
Los actores del sistema son los perfiles que tiene un usuario que interactúa con la herramienta. Los perfiles que puede tener un usuario
son: administrador, líder de análisis y analista.\\

%--------------------------------------------------------------------------------------------------
\begin{actor}{Administrador}{administrador}{Es la persona encargada de registrar los proyectos y al personal de la organización.}
    \item[Responsabilidades:] \hspace{1pt}
    \begin{itemize}
	\item Registrar y modificar la información de los proyectos.
	\item Asignar un líder de análisis a los proyectos.
	\item Registrar y modificar la información del personal.
    \end{itemize}
    \item[Perfil:] \hspace{1pt}
    \begin{itemize}
	\item Debe conocer los proyectos que la organización va a  comenzar.
	\item Debe conocer toda la información del personal que colabora en la organización.
    \end{itemize}
    \item[Cantidad:] Uno
\end{actor}

%--------------------------------------------------------------------------------------------------
\begin{actor}{Líder de análisis}{liderAnalisis}{Es el analista encargado de un proyecto.}
    \item[Responsabilidades:] \hspace{1pt}
    \begin{itemize}
	\item Asignar los colaboradores del proyecto que lidera.
	\item Registrar y modificar módulos para los casos de uso y pantallas.
	\item Registrar y modificar los \cdtRef{gls:elemento}{elementos} del sistema.
	\RCitem{PC1}{\TODO{Falta indicar si el indica el inicio y fin del proyecto.}}{}
	\item Revisar y realizar observaciones de los elementos para enviarlos a corrección.
	\item Liberar los elementos que considere completos y correctos.
    \end{itemize}
    \item[Perfil:] \hspace{1pt}
    \begin{itemize}
	\item Debe tener conocimientos referentes a la documentación de sistemas.
	\item Debe saber organizar las actividades que debe hacer.
	\item Debe tener habilidad para comunicarse efectivamente con los demás colaboradores del proyecto.
	\item Debe saber asignar trabajo.
    \end{itemize}
    \item[Cantidad:] Uno por proyecto.
\end{actor}

%--------------------------------------------------------------------------------------------------
\begin{actor}{Analista}{analista}{Es el encargado de documentar los elementos que componen el proyecto.}
    \item[Responsabilidades:] \hspace{1pt}
    \begin{itemize}
	\item Registrar y modificar módulos para los casos de uso y pantallas.
	\item Registrar y modificar los \cdtRef{gls:elemento}{elementos} del sistema.
	\item Corregir la información de los elementos con base en las observaciones realizadas por el líder de análisis.
    \end{itemize}
    \item[Perfil:] \hspace{1pt}
    \begin{itemize}
	\item Debe tener conocimientos referentes a la documentación de sistemas.
	\item Debe tener habilidad para comunicarse efectivamente con los demás colaboradores del proyecto.
    \end{itemize}
    \item[Cantidad:] De acuerdo al personal de la organización.
\end{actor}


%====================================================================================
\clearpage
\section{Casos de uso}
Los casos de uso que describen el funcionamiento del prototipo se muestran en la figura \refIM{fig:cuAdmin}{Casos de uso del módulo: Administrador}, 
los casos de uso se dividieron por módulos para organizar de una mejor manera la visualización y el desarrollo. Los módulos
que conforman el editor son:
\begin{itemize}
	\item Administrador
	\item Líder de Análisis
	\item Módulos
	\item Casos de Uso
	\item Pantallas
	\item Entidades
	\item Reglas de negocio
	\item Mensajes
	\item Actores
	\item Términos del Glosario
\end{itemize}

\IMfig[.9]{ModeloComportamiento/images/CasosUso_General.png}{fig:cuAdmin}{Casos de uso del módulo: Administrador}

\clearpage
La figura \refIM{fig:cuAdmin}{Casos de uso del módulo: Administrador} muestra los casos de uso del administrador que incluyen la gestión de proyectos y del personal.\\

\IMfig[.9]{ModeloComportamiento/images/CasosUso_Adminstrador.png}{fig:cuAdmin}{Casos de uso del módulo: Administrador}

\clearpage
La figura \refIM{fig:cuLider}{Casos de uso del módulo: Líder de Análisis} muestra los casos de uso del líder de análisis que incluyen elegir colaboradores y gestionar proyectos.\\

\IMfig[.9]{ModeloComportamiento/images/CasosUso_Lider.png}{fig:cuLider}{Casos de uso del módulo: Líder de Análisis}

\clearpage
La figura \refIM{fig:cuModulos}{Casos de uso del módulo: Módulos} muestra los casos de uso referentes a la gestión de los módulos.\\

\IMfig[.9]{ModeloComportamiento/images/CasosUso_Modulos.png}{fig:cuModulos}{Casos de uso del módulo: Módulos}

\clearpage
La figura \refIM{fig:cuCU}{Casos de uso del módulo: Casos de Uso} muestra los casos de uso referentes a la gestión de los casos de uso.\\

\IMfig[.9]{ModeloComportamiento/images/CasosUso_CU.png}{fig:cuCU}{Casos de uso del módulo: Casos de Uso}

\clearpage
La figura \refIM{fig:cuIU}{Casos de uso del módulo: Pantallas} muestra los casos de uso referentes a la gestión de las pantallas y los comandos de estas.\\

\IMfig[.9]{ModeloComportamiento/images/CasosUso_Pantallas.png}{fig:cuIU}{Casos de uso del módulo: Pantallas}

\clearpage
La figura \refIM{fig:cuEntidades}{Casos de uso del módulo: Entidades} muestra los casos de uso referentes a la gestión de las entidades y atributos que las describen.\\

\IMfig[.9]{ModeloComportamiento/images/CasosUso_Entidades.png}{fig:cuEntidades}{Casos de uso del módulo: Entidades}

\clearpage
La figura \refIM{fig:cuRN}{Casos de uso del módulo: Reglas de Negocio} muestra los casos de uso referentes a la gestión de las reglas de negocio.\\

\IMfig[.9]{ModeloComportamiento/images/CasosUso_ReglasNegocio.png}{fig:cuRN}{Casos de uso del módulo: Reglas de Negocio}

\clearpage
La figura \refIM{fig:cuMsj}{Casos de uso del módulo: Mensajes} muestra los casos de uso referentes a la gestión de los mensajes.\\

\IMfig[.9]{ModeloComportamiento/images/CasosUso_Mensajes.png}{fig:cuMsj}{Casos de uso del módulo: Mensajes}

\clearpage
La figura \refIM{fig:cuActores}{Casos de uso del módulo: Actores} muestra los casos de uso referentes a la gestión de los actores.\\

\IMfig[.9]{ModeloComportamiento/images/CasosUso_Actores.png}{fig:cuActores}{Casos de uso del módulo: Actores}

\clearpage
La figura \refIM{fig:cuGlosario}{Casos de uso del módulo: Términos del Glosario} muestra los casos de uso referentes a la gestión de los términos del glosario.\\

\IMfig[.9]{ModeloComportamiento/images/CasosUso_Glosario.png}{fig:cuGlosario}{Casos de uso del módulo: Términos del Glosario}



