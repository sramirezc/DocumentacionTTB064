%=========================================================
\section{Módulos del sistema}

El sistema se encuentra organizado por módulos con la finalidad de agrupar y administrar de mejor manera los requerimientos funcionales del sistema. Dividir el sistema en módulos permite visualizar e identificar rápidamente aquellos aspectos funcionales que pueden tratarse conjuntamente. \\

La figura \ref{fig:ModulosPAPEAR} muestra los módulos que conforman el subsistema de \gie. Cada uno de estos módulos agrupan los casos de uso que poseen funcionalidad similar o que trabajan en conjunto para alcanzar un aspecto funcional del sistema. Cada uno de los módulos que se muestran en la figura se describe a continuación:

% \begin{figure}[h!]
%   \begin{center}
% 	  \fbox{\includegraphics[width=\textwidth]{images/modulosGral.png}}
%   \caption{Módulos del \seipae.}
%   \label{fig:ModulosPAPEAR}
%   \end{center}
% \end{figure}

% \begin{itemize}
% 	\item {\bf Administrar eventos:} Agrupa los casos de uso que tienen que ver con el registro de la información cartográfica y georreferencia de los eventos. En este se integran los casos de uso de cada uno de los eventos manejados por PROBOSQUE para el SIG-SEIPA de forma independiente: \cdtRef{gls:incendio}{``Incendios''}, \cdtRef{gls:pagoServicios}{``Pagos por servicios ambientales hidrológicos''} y \cdtRef{gls:reforestacion}{``Reforestaciónes'}. Este módulo está diseñado para cada una de las áreas responsables de los eventos manejados por PROBOSQUE.
% 	\item {\bf Administrar predios:} Agrupa los casos de uso que tienen que ver con el registro y modificación de la información cartográfica y georreferencia de predios. Este módulo está diseñado para el área responsable de los predios que maneja PROBOSQUE.
% 	\item {\bf Administrar cartografía:} Agrupa los casos de uso que tienen que ver con la consulta de la información cartográfica y georreferencia de los eventos y predios.
% 	\item {\bf Capas cartográficas:} Contiene las capas cartográficas correspondientes a datos geográficos de división política, climas, regiones, localidades, entre otros.
% 	\item {\bf Administrar cuentas:} Integra los casos de uso referentes a credenciales, usuarios y control de acceso al sistema. 
% \end{itemize}

%=========================================================
\section{Actores del sistema}\label{sec:Comportamiento:ActoresSistema}
Los actores son los perfiles asociados a las diversas áreas y/u organizaciones que intervienen en el proceso. Se han identificado los actores de acuerdo a las actividades y responsabilidades dentro del \papear - \gie, los cuales se describen a continuación.

% \begin{figure}[h!]
%   \begin{center}
%       \includegraphics[width=0.6\textwidth]{images/actores.png}
%   \caption{Perfiles identificados.}
%   \label{fig:ModulosSIGSEIPA}
%   \end{center}
% \end{figure}

%--------------------------------------------------------------------------------------------------
\begin{actor}{Coordinador del programa}{usuarioEscuela}{Se refiere a cualquier persona perteneciente a la escuela participante.}
    \item[Área:] Escuela participante.
    \item[Responsabilidades:] \hspace{1pt}
    \begin{itemize}
	\item Registrar a su escuela en el programa.
	\item Registrar a los miembros del comité de su escuela en el programa.
	\item Visualizar la información de registro de su escuela.
	\item Visualizar la información de registro de los miembros del comité de su escuela.
	\item Modificar la información de registro correspondiente a su escuela.
	\item Modificar la información de registro correspondiente a los miembros del comité de su escuela.
	\item Dar de baja a su escuela del programa.
	\item Dar de baja a miembros del comité de su escuela.
    \end{itemize}
    \item[Perfil:] \hspace{1pt}
    \begin{itemize}
	\item Conocer el objetivo del programa.
	\item Conocimientos básicos en materia ambiental.
	\item Conocimientos en el uso de computadora.
	\item Contar con una cuenta de correo electrónico.
    \end{itemize}
    \item[Cantidad:] Uno por cada turno escolar de la escuela participante.
\end{actor}

%--------------------------------------------------------------------------------------------------
\begin{actor}{Director de Programa}{directorPrograma}{Director del Programa de Acreditación de Escuelas Ambientalmente Responsables en SMAGEM}
    \item[Área:] SMAGEM.
    \item[Responsabilidades:] \hspace{1pt}
    \begin{itemize}
	\item Visualizar las escuelas inscritas en el programa.
	\item Visualizar la información registrada de las escuelas inscritas en el programa.
	\item Visualizar la información registrada del comité de las escuelas inscritas en el programa.
	\item Dar de baja una escuela inscrita en el programa.
    \end{itemize}
    \item[Perfil:] \hspace{1pt}
    \begin{itemize}
	\item Conocimientos sobre la operación del subsistema GIE.
	\item Contar con un amplio conocimiento sobre el PAEAR.
	\item Conocimientos en el uso de computadora.
	\item Contar con una cuenta de correo electrónico.
    \end{itemize}
    \item[Cantidad:] Uno.
\end{actor}

%--------------------------------------------------------------------------------------------------
\begin{actor}{Administrador}{administrador}{Persona encargada de administrar y dar soporte técnico al sistema}
    \item[Área:] SMAGEM.
    \item[Responsabilidades:] \hspace{1pt}
    \begin{itemize}
	\item Dar soporte técnico al sistema, en lo referente a conectividad y acceso por parte de los usuarios.
	\item Puede realizar todas las operaciones de los demás usuarios por instrucciones del Director del Programa.
    \end{itemize}
    \item[Perfil:] \hspace{1pt}
    \begin{itemize}
	\item Conocimientos técnicos y de operación del subsistema GIE.
	\item Conocimientos del sistema operativo sobre el cual se instale el sistema.
	\item Conocimientos acerca de la administración de sistemas.
	\item Contar con una cuenta de correo electrónico.
    \end{itemize}
    \item[Cantidad:] Uno.
\end{actor}


%====================================================================================
\section{Casos de Uso del módulo Administración de Escuelas}
La figura \ref{fig:casosUso:escuelas} muestra los casos de uso que integran la funcionalidad de la Administración de Escuelas, que se refieren al registro, modificación y visualización de su información.

% \begin{figure}[h!]
%     \begin{center}
%         \fbox{\includegraphics[width=\textwidth]{images/CasosUso/escuelas.png}}
%     \caption{Diagrama de casos de uso para la administración de escuelas. \label{fig:casosUso:escuelas}}
%     \end{center}
% \end{figure}

\section{Casos de Uso del módulo Administración de Comité}
La figura \ref{fig:casosUso:comite} muestra los casos de uso que integran la funcionalidad de la Administración de Comité, que se refieren al registro, modificación y visualización de su información.

% \begin{figure}[h!]
%     \begin{center}
%         \fbox{\includegraphics[angle=90, height=0.9\textheight]{images/CasosUso/comite.png}}
%     \caption{Diagrama de casos de uso para la administración de comité.}
%     \label{fig:casosUso:comite}
%     \end{center}
% \end{figure}

\section{Casos de Uso del módulo Administración de Cuentas de usuario}
La figura \ref{fig:casosUso:cuentasUsuario} muestra los casos de uso que integran la funcionalidad de la Administración de Cuentas de usuario y el acceso al sistema. 

\begin{figure}[h!]
    \begin{center}
        \fbox{\includegraphics[width=\textwidth]{images/CasosUso/cuentasUsuario.png}}
    \caption{Diagrama de casos de uso para la administración de cuentas de usuario.}
    \label{fig:casosUso:cuentasUsuario}
    \end{center}
\end{figure}
