\begin{UseCase}{CU 5.1}{Registrar caso de uso}
	{
		Este caso de uso permite al actor registrar un caso de uso. El actor puede dejar inconcluso el registro de un 
		caso de uso o puede marcarlo como terminado, si esto sucede la herramienta validará y almacenará la información.
	}
	
	\UCsection{Atributos}
	\UCitem{Actor}{\cdtRef{actor:liderAnalisis}{Líder de análisis}, \cdtRef{actor:analista}{Analista}}
	\UCitem{Propósito}{
		Registrar la información de un caso de uso.
	}
	\UCitem{Entradas}{
		\begin{UClist} 
			\UCli De la sección {\bf Información general del caso de uso}: 
				\begin{Citemize}
				        \item \cdtRef{Elemento:Nombre}{Nombre}: \ioEscribir.
					\item \cdtRef{Elemento:Descripcion}{Descripcion}: \ioEscribir.
				\end{Citemize}
			\UCli De la sección {\bf Descripción del caso de uso}:
				\begin{Citemize}
				        \item \cdtRef{Actor}{Actores}: \ioEscribirSeleccionar.
				        \item \cdtRef{Entrada}{Entradas}: \ioEscribirSeleccionar.
				        \item \cdtRef{Salida}{Salidas}: \ioEscribirSeleccionar.
					\item \cdtRef{RegladeNegocio}{Reglas de negocio}: \ioEscribirSeleccionar.
					\item \cdtRef{Error}{Errores}: \ioEscribirSeleccionar.
				\end{Citemize}
			\UCli \cdtRef{Precondiciones}{Precondiciones}: \ioEscribir.
			\UCli \cdtRef{Postcondiciones}{Postcondiciones}: \ioEscribir.
			\UCli \cdtRef{Extension}{Puntos de extensión}: \ioEscribir.
			
			\RCitem{PC1}{\TODO{Falta describir cómo registrar las precondiciones, postcondiciones y los puntos de extensión.}}{}
		\end{UClist}
	}	
	
	\UCitem{Salidas}{	
		\begin{UClist}
			\UCli \cdtRef{Elemento:Clave}{Clave}: \ioObtener.
			\UCli \cdtRef{Elemento:Numero}{Número}: \ioCalcular{\cdtIdRef{RN05}{Numeración de elementos}}.
		\end{UClist}
	}
	
	\UCitem{Mensajes}{	
		\begin{UClist}
			\UCli \cdtIdRef{MSG1}{Operación exitosa}: Se muestra en la pantalla \cdtIdRef{IU 5}{Gestionar casos de uso} para indicar que el registro fue exitoso.
			\UCli \cdtIdRef{MSG3}{Caso de uso terminado}: Se muestra en la pantalla \cdtIdRef{IU 5}{Gestionar casos de uso} para indicar que el 
			caso de uso se ha terminado de registrar y la información se guardó exitosamente.
		\end{UClist}
	}
	
	\UCitem{Precondiciones}{ 
		Ninguna
	}			
	\UCitem{Postcondiciones}{ 
		\begin{UClist}
			\UCli El estado del caso de uso cambiará a ``Terminado'' o ``En edición''.
			\UCli Se almacenará en el sistema la fecha de actualización y el mensaje referente a esta actualización.
	 	\end{UClist}
	}
	\UCitem{Errores}{
			\UCli \cdtIdRef{MSG4}{Dato obligatorio}: Se muestra en la pantalla \cdtIdRef{IU 5.1}{Registrar caso de uso} cuando no se ha ingresado un dato marcado como obligatorio.
			\UCli \cdtIdRef{MSG5}{Dato incorrecto}: Se muestra en la pantalla \cdtIdRef{IU 5.1}{Registrar caso de uso} cuando el tipo de dato ingresado no cumple con el tipo de dato solicitado en el campo.
			\UCli \cdtIdRef{MSG6}{Longitud inválida}: Se muestra en la pantalla \cdtIdRef{IU 5.1}{Registrar caso de uso} cuando se ha excedido la longitud de alguno de los campos.
			\UCli \cdtIdRef{MSG7}{Registro repetido}: Se muestra en la pantalla \cdtIdRef{IU 5.1}{Registrar caso de uso} cuando se hayan registrado dos casos de uso con el mismo nombre.
			\UCli \cdtIdRef{MSG9}{Elemento no referenciado}: Se muestra en la pantalla \cdtIdRef{IU 5.1}{Registrar caso de uso} cuando un actor, entrada, salida, regla de negocio o mensaje no está siendo utilizado en las trayectorias.
	}
	\UCitem{Tipo}{Secundario, extiende del caso de uso \cdtIdRef{CU 5}{Gestionar casos de uso}.}

	
\end{UseCase}
 %-------------------------------------------------------%trayectoria Principal-----------------------------------------------
 \begin{UCtrayectoria}
    \UCpaso[\UCactor] Solicita registrar un caso de uso oprimiendo el botón \cdtButton{Registrar} de la pantalla \cdtIdRef{IU 5}{Gestionar casos de uso}.
    \UCpaso[\UCsist] Guarda el caso de uso con estado ``En registro''.
    \UCpaso[\UCsist] Muestra la pantalla \cdtIdRef{IU 5.1}{Registrar caso de uso} en la cual se realizará el registro del caso de uso. 
    \UCpaso[\UCactor] Ingresa la información general del caso de uso en la pantalla \cdtIdRef{IU 5.1}{Registrar caso de uso}. 
    
    \UCpaso[\UCactor] Ingresa los actores, las entradas, las salidas, las reglas de negocio y los errores del caso de uso en la pantalla \cdtIdRef{IU 5.1}{Registrar caso de uso}.\label{cu5.1:ingresaDatos}
    
    \UCpaso[\UCactor] Enlista los casos de uso que debieron ejecutarse como precondiciones.
    \UCpaso[\UCactor] Enlista los casos de uso que podrán ejecutarse como postcondiciones.
    
    \UCpaso[\UCactor] Registra los pasos de la trayectoria principal del caso de uso.  \label{cu5.1:ingresaTrayP}
%     \UCpaso[\UCsist] Verifica que existan trayectorias alternativas. \refTray{A}
%     \UCpaso[\UCsist] Agrega las secciones de las trayectorias alternativas que se referencian en la trayectoria principal. \label{cu5.1:agregaSecc}
%     \UCpaso[\UCactor] Registra los pasos de las las trayectorias alternativas del caso de uso en la pantalla \cdtIdRef{IU 5.1}{Registrar caso de uso}. \label{cu5.1:ingresaTrayA}
%     \UCpaso[\UCsist] Verifica que no se hayan referenciado más trayectorias alternativas. \refTray{B}

    \UCpaso[\UCactor] Agrega los punto de extensión. \label{cu5.1:ingresaPE}
    \UCpaso[\UCactor] Solicita terminar el caso de uso oprimiendo el botón \cdtButton{Terminar} de la pantalla \cdtIdRef{IU 5.1}{Registrar caso de uso}. \refTray{C}  \refTray{D} 
    \UCpaso[\UCsist] Verifica que el actor ingrese todos los campos obligatorios con base en la regla de negocio  \cdtIdRef{RN-S8}{Datos obligatorios}. \refTray{E}
    \UCpaso[\UCsist] Verifica que el nombre del caso de uso no se encuentre registrado en el sistema con base en la regla de negocio  \cdtIdRef{RN-N6}{Unicidad de nombres}. \refTray{F}
    \UCpaso[\UCsist] Verifica que los datos requeridos sean proporcionados correctamente como se especifica en la regla de negocio \cdtIdRef{RN-S7}{Información correcta}. \refTray{G} \refTray{H}
    \UCpaso[\UCsist] Verifica que todos los elementos del caso de uso esten referenciados en las trayectorias. \refTray{I}
    \UCpaso[\UCsist] Muestra el mensaje \cdtIdRef{MSG8}{Caso de uso terminado} para preguntar al actor si desea continuar con la operación.
    \UCpaso[\UCactor] Oprime el botón \cdtButton{Aceptar} del mensaje de confirmación. \refTray{J}
    \UCpaso[\UCsist] Registra la información del caso de uso en el sistema.
    \UCpaso[\UCsist] Cambia el estado del caso de uso a ``Terminado''.
    \UCpaso[\UCsist] Actualiza la fecha de la última actualización y solicita las observaciones de la actualización.
    \UCpaso[\UCsist] Muestra el mensaje \cdtIdRef{MSG3}{Caso de uso terminado} en la pantalla \cdtIdRef{IU 5}{Gestionar casos de uso} 
    para indicar al actor que el registro se ha realizado exitosamente.
 \end{UCtrayectoria}
 
%  %----------------------------------------------------------%trayectoria A---------------------------------------------------- 
%  \begin{UCtrayectoriaA}{A}{El actor no utilizó trayectorias alternativas.}
%     \UCpaso[] Continúa con el paso \ref{cu5.1:registrarPtoExt} de la trayectoria principal.
%  \end{UCtrayectoriaA}
%  %----------------------------------------------------------%trayectoria B---------------------------------------------------- 
%  \begin{UCtrayectoriaA}{B}{El actor referenció más trayectorias alternativas.}
%     \UCpaso[] Continúa con el paso \ref{cu5.1:agregaSecc} de la trayectoria principal.
%  \end{UCtrayectoriaA}

 %----------------------------------------------------------%trayectoria C---------------------------------------------------- 
 \begin{UCtrayectoriaA}[Fin del caso de uso]{C}{El actor no desea terminar la edición.}
    \UCpaso[\UCactor] Solicita guardar el caso de uso oprimiendo el botón \cdtButton{Aceptar} de la pantalla \cdtIdRef{IU 5.1}{Registrar caso de uso}.
    \UCpaso[\UCsist] Cambia el estado del caso de uso a ``En edición''.
    \UCpaso[\UCsist] Actualiza la fecha de la última actualización y solicita las observaciones de la actualización.
    \UCpaso[\UCsist] Muestra el mensaje \cdtIdRef{MSG1}{Operación exitosa} en la pantalla \cdtIdRef{IU 5}{Gestionar casos de uso} 
    para indicar al actor que el registro se ha realizado exitosamente.
 \end{UCtrayectoriaA}
 %----------------------------------------------------------%trayectoria D---------------------------------------------------- 
 \begin{UCtrayectoriaA}[Fin del caso de uso]{D}{El actor desea cancelar la operación.}
    \UCpaso[\UCactor] Solicita cancelar la operación oprimiendo el botón \cdtButton{Cancelar} de la pantalla \cdtIdRef{IU 5.1}{Registrar caso de uso}.
    \UCpaso[\UCsist] Muestra la pantalla \cdtIdRef{IU 5}{Gestionar casos de uso}.
 \end{UCtrayectoriaA}
  %----------------------------------------------------------%trayectoria E---------------------------------------------------- 
 \begin{UCtrayectoriaA}{E}{El actor no ingresó algún dato marcado como obligatorio.}
    \UCpaso[\UCsist] Muestra el mensaje \cdtIdRef{MSG4}{Dato obligatorio} y señala el campo que presenta el error en la pantalla 
	    \cdtIdRef{CU 5.1}{Registrar caso de uso}, indicando al actor que el dato es obligatorio.
    \UCpaso[] Continúa con el paso \ref{cu5.1:ingresaDatos} de la trayectoria principal.
 \end{UCtrayectoriaA}
 %----------------------------------------------------------%trayectoria F---------------------------------------------------- 
 \begin{UCtrayectoriaA}{F}{El actor ingresó un nombre de caso de uso repetido.}
    \UCpaso[\UCsist] Muestra el mensaje \cdtIdRef{MSG7}{Registro repetido} y señala el campo que presenta la duplicidad en la pantalla 
	    \cdtIdRef{CU 5.1}{Registrar caso de uso}, indicando al actor que existe un caso de uso con el mismo nombre.
    \UCpaso[] Continúa con el paso \ref{cu5.1:ingresaDatos} de la trayectoria principal.
 \end{UCtrayectoriaA}
 %----------------------------------------------------------%trayectoria G---------------------------------------------------- 
 \begin{UCtrayectoriaA}{G}{El actor ingresó un tipo de dato incorrecto.}
    \UCpaso[\UCsist] Muestra el mensaje \cdtIdRef{MSG4}{Formato incorrecto} y señala el campo que presenta el dato inválido en la 
    pantalla \cdtIdRef{IU 5.1}{Registrar caso de uso} para indicar que se ha ingresado un tipo de dato inválido.
    \UCpaso[] Continúa con el paso \ref{cu5.1:ingresaDatos} de la trayectoria principal.
 \end{UCtrayectoriaA}
 %----------------------------------------------------------%trayectoria H----------------------------------------------------  
 \begin{UCtrayectoriaA}{H}{El actor proporciona un dato que excede la longitud máxima.}
    \UCpaso[\UCsist] Muestra el mensaje \cdtIdRef{MSG5}{Se ha excedido la longitud máxima del campo} y señala el campo que excede la 
    longitud en la pantalla \cdtIdRef{IU 5.1}{Registrar caso de uso}, para indicar que el dato excede el tamaño máximo permitido.
    \UCpaso[] Continúa con el paso \ref{cu5.1:ingresaDatos} de la trayectoria principal.
 \end{UCtrayectoriaA}
 %----------------------------------------------------------%trayectoria I---------------------------------------------------- 
 \begin{UCtrayectoriaA}{I}{Alguno de los elementos no está siendo utilizado en las trayectorias.}
    \UCpaso[\UCsist] Muestra el mensaje \cdtIdRef{MSG9}{Elemento no referenciado} en el campo del elemento
    que no está siendo utilizado.
    \UCpaso[] Continúa con el paso \ref{cu5.1:ingresaDatos} de la trayectoria principal.
 \end{UCtrayectoriaA}
 
 %----------------------------------------------------------%trayectoria J---------------------------------------------------- 
 \begin{UCtrayectoriaA}{J}{El actor desea cancelar el envío de la información.}
    \UCpaso[\UCactor] Solicita cancelar el envío de la información oprimiendo el botón \cdtButton{Cancelar} del mensaje \cdtIdRef{MSG8}{Caso de uso terminado}.
    \UCpaso[] Continúa con el paso \ref{cu5.1:ingresaDatos} de la trayectoria principal.
 \end{UCtrayectoriaA}
 
 

\subsection{Puntos de extensión}
\UCExtensionPoint{El actor requiere registrar la trayectoria principal del caso de uso}
	{Pasos \ref{cu5.1:ingresaTrayP} y \ref{cu5.1:ingresaTrayA} de la trayectoria principal}
	{\cdtIdRef{CU 5.1.1}{Gestionar trayectorias}}
\UCExtensionPoint{El actor requiere registrar un punto de extensión}
	{Pasos \ref{cu5.1:ingresaPE} de la trayectoria principal}
	{\cdtIdRef{CU 5.1.2}{Registrar punto de extensión}}
\UCExtensionPoint{El actor requiere eliminar un punto de extensión}
	{Pasos \ref{cu5.1:ingresaPE} de la trayectoria principal}
	{\cdtIdRef{CU 5.1.3}{Eliminar punto de extensión}}
	

