\section{Reglas de negocio}
%---------------------------------------------------------
% \begin{BusinessRule}{RN01}{Eliminación de elementos}
%     {Restricción de operación}
%     {Controla la operación}
%     \BRitem{Versión}{0.1}
%     \BRitem{Autor}{Natalia Giselle Hernández Sánchez}
%     \BRitem{Estatus}{Edición}
%     \BRitem{Descripción}{Un elemento podrá ser eliminado del sistema cuando no tenga asociado ningún otro elemento.}	
% \end{BusinessRule}
% 
% \begin{BusinessRule}{RN02}{Eliminación de módulos}
%     {Restricción de operación}
%     {Controla la operación}
%     \BRitem{Versión}{0.1}
%     \BRitem{Autor}{Natalia Giselle Hernández Sánchez}
%     \BRitem{Estatus}{Edición}
%     \BRitem{Descripción}{Los módulos del sistema podrán ser eliminados cuando no contengan casos de uso o pantallas asociados.}	
% \end{BusinessRule}

\begin{BusinessRule}{RN03}{Líder de análisis}
    {Restricción de operación}
    {Controla la operación}
    \BRitem{Versión}{0.1}
    \BRitem{Autor}{Natalia Giselle Hernández Sánchez}
    \BRitem{Estatus}{Edición}
    \BRitem{Descripción}{Los proyectos deben tener asignado solamente un líder de análisis.}	
\end{BusinessRule}

\begin{BusinessRule}{RN04}{Integrantes de un proyecto}
    {Restricción de operación}
    {Controla la operación}
    \BRitem{Versión}{0.1}
    \BRitem{Autor}{Natalia Giselle Hernández Sánchez}
    \BRitem{Estatus}{Edición}
    \BRitem{Descripción}{Los proyectos deben tener al menos un integrante.}	
\end{BusinessRule}

\begin{BusinessRule}{RN05}{Numeración de elementos}
    {Restricción de operación}
    {Controla la operación}
    \BRitem{Versión}{0.1}
    \BRitem{Autor}{Natalia Giselle Hernández Sánchez}
    \BRitem{Estatus}{Edición}
    \BRitem{Descripción}{Para asignar el número de un elemento en el registro, el sistema buscará el último número asociado a la clave y le sumará 1. }
\end{BusinessRule}

\begin{BusinessRule}{RN-N6}{Unicidad de nombres}
    {Restricción de operación}
    {Controla la operación}
    \BRitem{Versión}{0.1}
    \BRitem{Autor}{Natalia Giselle Hernández Sánchez}
    \BRitem{Estatus}{Edición}
    \BRitem{Descripción}{El nombre de los elementos no puede repetirse.}	
\end{BusinessRule}

\begin{BusinessRule}{RN-S7}{Información correcta}
    {Restricción de operación}
    {Controla la operación}
    \BRitem{Versión}{0.1}
    \BRitem{Autor}{Natalia Giselle Hernández Sánchez}
    \BRitem{Estatus}{Edición}
    \BRitem{Descripción}{La información que el usuario proporcione, debe ser del tipo y longitud definida en el modelo conceptual.}	
\end{BusinessRule}

\begin{BusinessRule}{RN-S8}{Datos obligatorios}
    {Restricción de operación}
    {Controla la operación}
    \BRitem{Versión}{0.1}
    \BRitem{Autor}{Natalia Giselle Hernández Sánchez}
    \BRitem{Estatus}{Edición}
    \BRitem{Descripción}{El usuario debe ingresar toda la información marcada como obligatoria en el modelo conceptual.}	
\end{BusinessRule}

\begin{BusinessRule}{RN-S9}{Operaciones disponibles de casos de uso}
    {Restricción de operación}
    {Controla la operación}
    \BRitem{Versión}{0.1}
    \BRitem{Autor}{Natalia Giselle Hernández Sánchez}
    \BRitem{Estatus}{Edición}
    \BRitem{Descripción}{
      Los estados de los casos de uso y el rol del actor determinan las operaciones que pueden solicitarse desde la gestión:
      
      \begin{longtable}{| p{.2\textwidth} | p{.3\textwidth} | p{.3\textwidth} |}%
		\arrayrulecolor{black}%
		\rowcolor{black}%
		\color{white} Estado &  \color{white}Operaciones Analista & \color{white}Operaciones Líder de análisis\\ \hline
		\endhead%
		\arrayrulecolor{black}
		En registro & N/A & N/A\\ \hline 
		Pendiente de corrección & Consulta, edición & N/A\\ \hline
		En edición & Consulta, edición (solamente el analista que esté editando el caso de uso) & \\ \hline
		Terminado & Consulta, revisión, eliminación, edición (solamente el analista que terminó el caso de uso) & \\ \hline
		En revisión & Consulta, revisión (solamente el analista que esté realizando la revisión) & Liberar\\ \hline
		Liberado & Consulta & Habilitar correcciones\\ \hline
		
      \end{longtable}
    }	
\end{BusinessRule}

\begin{BusinessRule}{RN-S10}{Referencias a elementos}
    {Restricción de operación}
    {Controla la operación}
    \BRitem{Versión}{0.1}
    \BRitem{Autor}{Natalia Giselle Hernández Sánchez}
    \BRitem{Estatus}{Edición}
    \BRitem{Descripción}{Los elementos que podrán ser referenciados desde los casos de uso son aquellos que ya han sido registrados.}	
    % Se sacrifica la usabilidad por el tiempo reducido de desarrollo
\end{BusinessRule}

\begin{BusinessRule}{RN-S11}{Registro de trayectorias}
    {Restricción de operación}
    {Controla la operación}
    \BRitem{Versión}{0.1}
    \BRitem{Autor}{Natalia Giselle Hernández Sánchez}
    \BRitem{Estatus}{Edición}
    \BRitem{Descripción}{Al menos una de las trayectorias registradas debe ser marcada como principal.}	
    % Se sacrifica la usabilidad por el tiempo reducido de desarrollo
\end{BusinessRule}

\begin{BusinessRule}{RN-S12}{Identificador de elemento}
    {Restricción de operación}
    {Controla la operación}
    \BRitem{Versión}{0.1}
    \BRitem{Autor}{Natalia Giselle Hernández Sánchez}
    \BRitem{Estatus}{Edición}
    \BRitem{Descripción}{El identificador de cada elemento se compone de un nombre, número y una clave. Donde el nombre es el que le asigna el usuario, el número 
    es secuencial y la clave define el tipo de elemento: ``ENT'' para las entidades, ``ACT'' para los actores, ``CU'' para los casos de uso, ``IU'' para
    las pantallas, ``MSJ'' para los mensajes, ``RN'' para las reglas de negocio y ``GLS'' para los términos del glosario.}	
    % Se sacrifica la usabilidad por el tiempo reducido de desarrollo
\end{BusinessRule}

\begin{BusinessRule}{RN-S13}{Modificación del identificador}
    {Restricción de operación}
    {Controla la operación}
    \BRitem{Versión}{0.1}
    \BRitem{Autor}{Natalia Giselle Hernández Sánchez}
    \BRitem{Estatus}{Edición}
    \BRitem{Descripción}{Una vez registrado un elemento no se podrá modificar el nombre, el número o la clave del identificador.}
\end{BusinessRule}

\begin{BusinessRule}{RN-S14}{Salidas del caso de uso}
    {Restricción de operación}
    {Controla la operación}
    \BRitem{Versión}{0.1}
    \BRitem{Autor}{Natalia Giselle Hernández Sánchez}
    \BRitem{Estatus}{Edición}
    \BRitem{Descripción}{En las salidas del caso de uso podrán enlistarse mensajes y atributos de las entidades.}
\end{BusinessRule}

\begin{BusinessRule}{RN-S15}{Elementos en uso}
    {Restricción de operación}
    {Controla la operación}
    \BRitem{Versión}{0.1}
    \BRitem{Autor}{Natalia Giselle Hernández Sánchez}
    \BRitem{Estatus}{Edición}
    \BRitem{Descripción}{Cuando un analista esté modificando un elemento que no sea un caso de uso, ningún otro analista podrá solicitar 
    su edición, ni eliminación. La única operación disponible será la consulta.}
\end{BusinessRule}
