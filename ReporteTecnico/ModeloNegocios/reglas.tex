\section{Reglas de negocio}
%---------------------------------------------------------

% \begin{BusinessRule}{RN01}{Eliminación de elementos}
%     {Restricción de operación}
%     {Controla la operación}
%     \BRitem{Versión}{0.1}
%     \BRitem{Autor}{Natalia Giselle Hernández Sánchez}
%     \BRitem{Estatus}{Edición}
%     \BRitem{Descripción}{Un elemento podrá ser eliminado del sistema cuando no tenga asociado ningún otro elemento.}	
% \end{BusinessRule}
% 
% \begin{BusinessRule}{RN02}{Eliminación de módulos}
%     {Restricción de operación}
%     {Controla la operación}
%     \BRitem{Versión}{0.1}
%     \BRitem{Autor}{Natalia Giselle Hernández Sánchez}
%     \BRitem{Estatus}{Edición}
%     \BRitem{Descripción}{Los módulos del sistema podrán ser eliminados cuando no contengan casos de uso o pantallas asociados.}	
% \end{BusinessRule}
\begin{BusinessRule}{RN01}{Unicidad de números}
    {Restricción de operación}
    {Controla la operación}
    \BRitem{Versión}{1.0}
    \BRitem{Autor}{Natalia Giselle Hernández Sánchez}
    \BRitem{Estatus}{Edición}
    \BRitem{Descripción}{El número de los elementos del mismo tipo no pueden repetirse.}	
\end{BusinessRule}
\begin{BusinessRule}{RN02}{Nombres de los elementos}
    {Restricción de operación}
    {Controla la operación}
    \BRitem{Versión}{1.0}
    \BRitem{Autor}{Natalia Giselle Hernández Sánchez}
    \BRitem{Estatus}{Edición}
    \BRitem{Descripción}{Los nombres de los elementos no pueden contener coma, punto, punto medio, dos puntos o guión bajo.}
\end{BusinessRule}
\begin{BusinessRule}{RN03}{Líder de análisis}
    {Restricción de operación}
    {Controla la operación}
    \BRitem{Versión}{1.0}
    \BRitem{Autor}{Natalia Giselle Hernández Sánchez}
    \BRitem{Estatus}{Edición}
    \BRitem{Descripción}{Los proyectos deben tener asignado solamente un líder de análisis.}	
\end{BusinessRule}

\begin{BusinessRule}{RN04}{Integrantes de un proyecto}
    {Restricción de operación}
    {Controla la operación}
    \BRitem{Versión}{1.0}
    \BRitem{Autor}{Natalia Giselle Hernández Sánchez}
    \BRitem{Estatus}{Edición}
    \BRitem{Descripción}{Los proyectos deben tener al menos un integrante.}	
\end{BusinessRule}

\begin{BusinessRule}{RN05}{Liberación de elementos}
    {Restricción de operación}
    {Controla la operación}
    \BRitem{Versión}{1.0}
    \BRitem{Autor}{Sergio Ramírez Camacho}
    \BRitem{Estatus}{Edición}
    \BRitem{Descripción}{No es posible modificar entidades, reglas de negocio, actores, términos del glosario, pantallas y/o mensajes que se encuentren asociados a 						casos de uso con estado ``Liberado''.}	
\end{BusinessRule}

%\begin{BusinessRule}{RN05}{Numeración de elementos}
%     {Restricción de operación}
%     {Controla la operación}
%     \BRitem{Versión}{0.1}
%     \BRitem{Autor}{Natalia Giselle Hernández Sánchez}
%     \BRitem{Estatus}{Edición}
%     \BRitem{Descripción}{Para asignar el número de un elemento en el registro, el sistema buscará el último número asociado a la clave y le sumará 1. }
% \end{BusinessRule}

\begin{BusinessRule}{RN6}{Unicidad de nombres}
    {Restricción de operación}
    {Controla la operación}
    \BRitem{Versión}{1.0}
    \BRitem{Autor}{Natalia Giselle Hernández Sánchez}
    \BRitem{Estatus}{Edición}
    \BRitem{Descripción}{El nombre de los elementos del mismo tipo no puede repetirse.}	
\end{BusinessRule}

\begin{BusinessRule}{RN7}{Información correcta}
    {Restricción de operación}
    {Controla la operación}
    \BRitem{Versión}{1.0}
    \BRitem{Autor}{Natalia Giselle Hernández Sánchez}
    \BRitem{Estatus}{Edición}
    \BRitem{Descripción}{La información que el usuario proporcione, debe ser del tipo y longitud definida en el modelo conceptual.}	
\end{BusinessRule}

\begin{BusinessRule}{RN8}{Datos obligatorios}
    {Restricción de operación}
    {Controla la operación}
    \BRitem{Versión}{1.0}
    \BRitem{Autor}{Natalia Giselle Hernández Sánchez}
    \BRitem{Estatus}{Edición}
    \BRitem{Descripción}{El usuario debe ingresar toda la información marcada como obligatoria en el modelo conceptual.}	
\end{BusinessRule}

\begin{BusinessRule}{RN9}{Operaciones disponibles de casos de uso}
    {Restricción de operación}
    {Controla la operación}
    \BRitem{Versión}{1.0}
    \BRitem{Autor}{Natalia Giselle Hernández Sánchez}
    \BRitem{Estatus}{Edición}
    \BRitem{Descripción}{
      Los estados de los casos de uso y el rol del actor determinan las operaciones que pueden solicitarse desde la gestión:
      
      \begin{longtable}{| p{.2\textwidth} | p{.3\textwidth} | p{.3\textwidth} |}%
		\arrayrulecolor{black}%
		\rowcolor{black}%
		\color{white} Estado &  \color{white}Operaciones Analista & \color{white}Operaciones Líder de análisis\\ \hline
		\endhead%
		\arrayrulecolor{black}
		Edición & Consultar y solamente el analista que esté editando el caso de uso puede editar, gestionar trayectorias y gestionar puntos de extensión 
		& Consultar y solamente el analista que esté editando el caso de uso puede editar, gestionar trayectorias y gestionar puntos de extensión \\ \hline
		Terminado & Consultar, revisar, eliminar y solamente el analista que esté editando el caso de uso puede editar, gestionar trayectorias y gestionar puntos de extensión 
		& Consultar, revisar, eliminar y solamente el analista que esté editando el caso de uso puede editar, gestionar trayectorias y gestionar puntos de extensión\\ \hline
		Revisión & Consultar y solamente el analista que esté realizando la revisión puede revisar & Liberar, consultar, revisar (solamente el líder que esté realizando la revisión)\\ \hline
		Pendiente de corrección & Consultar, editar & N/A\\ \hline
		Por Liberar & Consulta & Habilitar correcciones\\ \hline
		Liberado & Consulta & Habilitar correcciones\\ \hline
      \end{longtable}
    }	
\end{BusinessRule}

\begin{BusinessRule}{RN10}{Referencias a elementos}
    {Restricción de operación}
    {Controla la operación}
    \BRitem{Versión}{1.0}
    \BRitem{Autor}{Natalia Giselle Hernández Sánchez}
    \BRitem{Estatus}{Edición}
    \BRitem{Descripción}{Los elementos que podrán ser referenciados desde los casos de uso son aquellos que ya han sido registrados.}	
    % Se sacrifica la usabilidad por el tiempo reducido de desarrollo
\end{BusinessRule}

\begin{BusinessRule}{RN11}{Registro de trayectorias}
    {Restricción de operación}
    {Controla la operación}
    \BRitem{Versión}{1.0}
    \BRitem{Autor}{Natalia Giselle Hernández Sánchez}
    \BRitem{Estatus}{Edición}
    \BRitem{Descripción}{Al menos una de las trayectorias registradas debe ser marcada como principal.}	
    % Se sacrifica la usabilidad por el tiempo reducido de desarrollo
\end{BusinessRule}

\begin{BusinessRule}{RN12}{Identificador de elemento}
    {Restricción de operación}
    {Controla la operación}
    \BRitem{Versión}{1.0}
    \BRitem{Autor}{Natalia Giselle Hernández Sánchez}
    \BRitem{Estatus}{Edición}
    \BRitem{Descripción}{El identificador de cada elemento se compone de un nombre, número y una clave. Donde el nombre es el que le asigna el usuario, el número 
    es secuencial y la clave define el tipo de elemento: ``ENT'' para las entidades, ``ACT'' para los actores, ``CU'' para los casos de uso, ``IU'' para
    las pantallas, ``MSJ'' para los mensajes, ``RN'' para las reglas de negocio y ``GLS'' para los términos del glosario.}	
    % Se sacrifica la usabilidad por el tiempo reducido de desarrollo
\end{BusinessRule}

\begin{BusinessRule}{RN13}{Modificación del identificador}
    {Restricción de operación}
    {Controla la operación}
    \BRitem{Versión}{1.0}
    \BRitem{Autor}{Natalia Giselle Hernández Sánchez}
    \BRitem{Estatus}{Edición}
    \BRitem{Descripción}{Una vez registrado un elemento no se podrá modificar el nombre, el número o la clave del identificador.}
\end{BusinessRule}

\begin{BusinessRule}{RN14}{Salidas del caso de uso}
    {Restricción de operación}
    {Controla la operación}
    \BRitem{Versión}{1.0}
    \BRitem{Autor}{Natalia Giselle Hernández Sánchez}
    \BRitem{Estatus}{Edición}
    \BRitem{Descripción}{En las salidas del caso de uso podrán enlistarse mensajes y atributos de las entidades.}
\end{BusinessRule}

% \begin{BusinessRule}{RN15}{Elementos en uso}
%     {Restricción de operación}
%     {Controla la operación}
%     \BRitem{Versión}{1.0}
%     \BRitem{Autor}{Natalia Giselle Hernández Sánchez}
%     \BRitem{Estatus}{Edición}
%     \BRitem{Descripción}{Cuando un analista esté modificando una entidad, regla de negocio, actor, término del glosario, pantalla y/o mensaje, 
%     ningún otro analista podrá solicitar su edición, ni eliminación. La única operación disponible será la consulta.}
% \end{BusinessRule}

\begin{BusinessRule}{RN15}{Operaciones disponibles}
    {Restricción de operación}
    {Controla la operación}
    \BRitem{Versión}{1.0}
    \BRitem{Autor}{Natalia Giselle Hernández Sánchez}
    \BRitem{Estatus}{Edición}
    \BRitem{Descripción}{Cuando una entidad, regla de negocio, actor, término del glosario, pantalla y/o mensaje, 
    están en estado ``Edición'' es posible solicitar su consulta, modificación y eliminación. Cuando alguno de estos elementos se encuentre asociado 
    a un caso de uso con estado ``Liberado'' solamente estará disponible la operación de consulta.}
\end{BusinessRule}

\begin{BusinessRule}{RN16}{Nombres de las trayecorias}
    {Restricción de operación}
    {Controla la operación}
    \BRitem{Versión}{1.0}
    \BRitem{Autor}{Natalia Giselle Hernández Sánchez}
    \BRitem{Estatus}{Edición}
    \BRitem{Descripción}{Los nombres de las trayectorias no pueden contener espacio, coma, punto, punto medio, dos puntos o guión bajo.}
\end{BusinessRule}

\begin{BusinessRule}{RN17}{Unicidad de puntos de extensión}
    {Restricción de operación}
    {Controla la operación}
    \BRitem{Versión}{1.0}
    \BRitem{Autor}{Natalia Giselle Hernández Sánchez}
    \BRitem{Estatus}{Edición}
    \BRitem{Descripción}{No puede existir más de un punto de extensión con el mismo caso de uso origen y el mismo caso de uso destino.}
\end{BusinessRule}

\begin{BusinessRule}{RN18}{Eliminación de elementos}
    {Restricción de operación}
    {Controla la operación}
    \BRitem{Versión}{1.0}
    \BRitem{Autor}{Natalia Giselle Hernández Sánchez}
    \BRitem{Estatus}{Edición}
    \BRitem{Descripción}{No es posible eliminar entidades, reglas de negocio, actores, términos del glosario, pantallas y/o mensajes que se encuentren asociados a casos de uso con estado ``Liberado''.}
\end{BusinessRule}