\section{Glosario}
\label{sec:glosario}
    En esta sección se describen algunos términos utilizados a lo largo del documento que se considera importante detallar. 
  
\begin{description}
%	\BRterm{gls:nombre}{nombre.} Descripción.
	\BRterm{gls:atributo}{Atributo.} Característica que define o identifica una \cdtRef{gls:entidad}{entidad} en un conjunto de entidades.

	\BRterm{gls:archivoDigital}{Archivo digital.} Equivalente digital de los archivos escritos en libros, tarjetas, libretas, papel o microfichas del entorno de oficina tradicional.

	\BRterm{gls:booleano}{Booleano.} Es un \cdtRef{gls:tipoDato}{tipo de dato} que puede tomar los valores verdadero o falso (1 ó 0).
	
	\BRterm{gls:cadena}{Cadena.} Es el \cdtRef{gls:tipoDato}{tipo de dato} definido por cualquier valor que se compone de una secuencia de caracteres, con o sin acentos, espacios, dígitos y 
	signos de puntuación. Existen tres tipos de cadenas: Palabra, frase y párrafo.
	
	\BRterm{gls:Cardinalidad}{Cardinalidad.} Es el número . Es un \cdtRef{gls:tipoDato}{tipo de dato} para el sistema y puede tomar alguno de los siguientes valores:
	Uno, Muchos u Otro.

	\BRterm{gls:elemento}{Elemento.} Se utiliza para referirse a los casos de uso, pantallas, reglas de negocio, entidades, término del glosario, mensajes y actores.
	
	\BRterm{gls:entero}{Entero} Es el \cdtRef{gls:tipoDato}{tipo de dato} \cdtRef{gls:numerico}{numérico} definido por todos los valores numéricos enteros, tanto positivos como negativos.
	
	\BRterm{gls:entidad}{Entidad.} Término genérico que se utiliza para determinar un ente el cual puede ser concreto, abstracto o conceptual por ejemplo: Caso de uso, 
		proyecto, módulo, etc. La entidades se caracterizan con atributos que la definen.		
		
	\BRterm{gls:EstadoDelElemento}{Estado del Elemento.} Es un identificador que indica la situación de un elemento. Es un \cdtRef{gls:tipoDato}{tipo de dato} para el sistema y puede tomar alguno de los siguientes valores:
	Pre-registro, Edición, Terminado, Pendiente de Corrección, Por Liberar o Liberado.
	
	\BRterm{gls:EstadoDelProyecto}{Estado del Proyecto.} Es un identificador que indica la situación de un proyecto. Es un \cdtRef{gls:tipoDato}{tipo de dato} para el sistema y puede tomar alguno de los siguientes valores:
	En Negociación, Iniciado o Terminado.

	\BRterm{gls:fecha}{Fecha.} Es un \cdtRef{gls:tipoDato}{tipo de dato} que indica un día único en referencia al calendario gregoriano. Los tipos de fecha utilizados son: \cdtRef{gls:fechaCorta}
	{fecha corta} y \cdtRef{gls:fechaLarga}{fecha larga}. %con formato DD/MM/YYYY, por ejemplo: 24/02/2013.

	\BRterm{gls:fechaCorta}{Fecha corta.} Es la representación del \cdtRef{gls:tipoDato}{tipo de dato} \cdtRef{gls:fecha}{fecha} en la forma DD/MM/YYYY, por ejemplo: 24/02/2013.
	
	\BRterm{gls:fechaInicioProy}{Fecha de inicio del proyecto.} Es la fecha de inicio el proceso de software del sistema.

	\BRterm{gls:fechaLarga}{Fecha larga.} Es la representación del \cdtRef{gls:tipoDato}{tipo de dato} \cdtRef{gls:fecha}{fecha} en la forma DD de MM del YYYY, por ejemplo: 24 de febrero del 2013.

	\BRterm{gls:frase}{Frase.} Es un \cdtRef{gls:tipoDato}{tipo de dato} conformado por \cdtRef{gls:palabra}{palabras} y espacios.
	
	\BRterm{gls:imagen}{Imagen.} Es una imagen de tamaño pequeño (no más de un Megabyte) en formato jpeg.
	
	\BRterm{gls:numerico}{Numérico.} Es un \cdtRef{gls:tipoDato}{tipo de dato} que se compone de la combinación de los símbolos \textit{0, 1, 2, 3, 4, 5, 6, 7, 8, 9, . y -},  que expresan una cantidad 
	en relación a su unidad.
	
	\BRterm{gls:opcional}{Opcional.} Es un elemento que el actor puede o no proporcionar en el formulario o la pantalla, su decisión no afectará la ejecución de la operación solicitada.

	\BRterm{gls:palabra}{Palabra.} Es un \cdtRef{gls:tipoDato}{tipo de dato} \cdtRef{gls:cadena}{cadena} conformado por el alfabeto y símbolos especiales como son 
	\textit{\#, -, \$, \%, \&, (,), etc.} y se caracteriza por no tener espacios.
	
	\BRterm{gls:ParametroDelMensaje}{Parámetro del Mensaje.} Es una palabra que se solicita en un mensaje parametrizado. Es un \cdtRef{gls:tipoDato}{tipo de dato} para el sistema y puede tomar alguno de los siguientes valores:
	Determinado, Indeterminado, Operación, Atributo, Entidad, Regla de negocio, entre otros.
	
	\BRterm{gls:ParametroDelPaso}{Parámetro del Paso.} Es un elemento que se solicita en un paso. Es un \cdtRef{gls:tipoDato}{tipo de dato} para el sistema y puede tomar alguno de los siguientes valores:
	Atributo, Casos de Uso, Pantalla, Regla de Negocio, Entidad, Término del Glosario, Mensaje, Actor, Paso o Acción.
	
	\BRterm{gls:parrafo}{Párrafo.} Es un \cdtRef{gls:tipoDato}{tipo de dato}  conformado por \cdtRef{gls:frase}{frases}.

	\BRterm{gls:requerido}{Requerido.} Es un tipo de dato que debe proporcionarse siempre sin excepción en el formulario.
	
	\BRterm{gls:Rol}{Rol.} Es la función de un colaborador dentro de un proyecto. Es un \cdtRef{gls:tipoDato}{tipo de dato} para el sistema y puede tomar alguno de los siguientes valores:
	Analista o Líder de Análisis.
	
	\BRterm{gls:seccion}{Sección.} Es el área del caso de uso que se revisa y sobre la que se hacen observaciones. Es un \cdtRef{gls:tipoDato}{tipo de dato} para el sistema y puede tomar alguno de los siguientes valores:
	Información General, Descripción, Precondiciones, Postcondiciones, Trayectorias o Puntos de extensión.
	
	\BRterm{gls:Telefono}{Teléfono.} Secuencia de dígitos utilizada para identificar una línea telefónica. Es un \cdtRef{gls:tipoDato}{tipo de dato} para el sistema.
	
	\BRterm{gls:TipoDeAccion}{Tipo de Acción.} Es un elemento que permite solicitar una operación desde la pantalla. Es un \cdtRef{gls:tipoDato}{tipo de dato} para el sistema y puede tomar alguno de los siguientes valores:
	Botón, Liga, Opción del Menú, entre otros.
	
	\BRterm{gls:tipoDato}{Tipo de dato.} Es el dominio o conjunto de valores que puede tomar un atributo de una \cdtRef{gls:entidad}{entidad} en el modelo de información. Los tipos de datos 
	utilizados son: \cdtRef{gls:palabra}{palabra}, \cdtRef{gls:frase}{frase}, \cdtRef{gls:parrafo}{párrafo}, \cdtRef{gls:numerico}{numérico}, \cdtRef{gls:fecha}{fecha} y 
	\cdtRef{gls:booleano}{booleano}.
	
	\BRterm{gls:TipoDeDato}{Tipo de Dato (herramienta).} Es el tipo de dato que puede tener un atributo. Es un \cdtRef{gls:tipoDato}{tipo de dato} para el sistema y puede tomar alguno de los siguientes valores:
	Entero, Flotante, Booleano, Cadena o Fecha.
	
	\BRterm{gls:TipoDeReglaDeNegocio}{Tipo de Regla de Negocio.} Es la categoría a la que pertenece una regla de negocio de acuerdo a las descritas en los requerimientos del sistema.
	Es un \cdtRef{gls:tipoDato}{tipo de dato} para el sistema y puede tomar alguno de los siguientes valores:
	Verificación de catálogos, Operaciones aritméticas, Unicidad de parámetros, Datos obligatorios, Longitud correcta, Tipo de dato correcto, Formato de archivos, 
	Tamaño de archivos, Intervalo de fechas correctas, Formato correcto u Otro.
	
\end{description}
