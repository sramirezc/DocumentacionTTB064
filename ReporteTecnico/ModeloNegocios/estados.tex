\section{Diagramas de estados}
Es esta sección se muestran los diagramas de estados y la descripción de cada uno de ellos.\\

\subsection{Diagrama de estados de un caso de uso}

En la figura \refIM{fig:estadosElemento}{Diagrama de estados de un caso de uso} se muestran los estados que tiene un caso de uso a lo largo 
del sistema. Estos estados cambian conforme la operación que se esté realizando sobre ellos.\\

\IMfig[1]{ModeloNegocios/images/estadosElementos.png}{fig:estadosElemento}{Diagrama de estados de un caso de uso}

\clearpage
Cada estado se describe a continuación:

\begin{itemize}
 \item {\bf Pre-registro.} Cuando un analista solicite el registro de un caso de uso, este tendrá el estado ``Pre-registro''.
 Los casos de uso que se encuentran en este estado están temporalmente en la base de datos hasta que el analista decide guardar la información.
 
 \item {\bf Edición.} Un caso de uso pasará a este estado cuando: 
 \begin{itemize}
  \item El analista guarda la información del caso de uso que solicitó registrar.
  \item Cuando un analista solicita modificar la información del caso de uso que no ha terminado.
  \item Cuando un analista solicita corregir un caso de uso.
  \item Cuando un caso de uso está liberado y un analista solicita modicar su información.
 \end{itemize}
 
 \item {\bf Terminado.} Cuando el analista haya concluido el registro del caso de uso este pasará a estado ``Terminado''. Cuando el caso de uso está en este estado
 el analista que lo haya terminado podrá modificarlo y cualquier otro analista podrá revisarlo.
 \item {\bf Revisión.} Cuando algún analista solicite revisar el caso de uso este pasará a estado ``Revisión''. Cuando el caso de uso está en este estado, el revisor podrá solicitar correcciones. En caso de que no existan correcciones, el caso de uso estará a la espera de su liberación, no obstante si quién determina que el caso de uso es correcto es el Líder de Análisis, será liberado inmediatamente.


 \item {\bf Pendiente de corrección.} Un caso de uso pasará a este estado cuando:
 \begin{itemize}
  \item Cuando un analista revisa el caso de uso y solicita correcciones.
  \item Cuando el líder de análisis decide no liberar el caso de uso y solicita correcciones.
  \item Cuando un analista solicita corregir un caso de uso.
  \item Cuando un caso de uso está liberado y un analista solicita modicar su información.
 \end{itemize}

 \item {\bf Por liberar.} Cuando un analista revisa un caso de uso y no realiza observaciones, este pasará a estado ``Por liberar''. El líder de análisis podrá
 decidir liberar el caso de uso o solicitar correcciones.

 \item {\bf Liberado.} Cuando el líder de análisis decide liberar el caso de uso, este pasará a estado ``Liberado''. Cuando el caso de uso está en este estado, cualquier analista podrá solicitar modificarlo. En caso de que se modifique algún elemento asociado al caso de uso, este pasará a estar pendiente de corrección.

\end{itemize}

\subsection{Diagrama de estados de un proyecto}

En la figura \refIM{fig:estadosProyecto}{Diagrama de estados de un proyecto} se muestra el diagrama de estados que corresponde a un proyecto. Cada estado se
detalla a continuación:

\IMfig[.9]{ModeloNegocios/images/estadosProyectos.png}{fig:estadosProyecto}{Diagrama de estados de un proyecto}

\begin{itemize}
 \item {\bf En negociación.} Cuando un proyecto esté siendo negociado con el cliente se puede indicar 
 que el estado de este es ``En negociación''. 
 \RCitem{PC1}{\TODO{Falta indicar si se puede realizar operaciones sobre los elementos en este estado.}}{}
 \item {\bf Iniciado.} Cuando un proyecto haya iniciado antes de ser registrado en la herramienta, se puede indicar que el estado es ``Iniciado''.
 Se puede realizar cualquier operación sobre los elementos de los proyectos en este estado.
 \item {\bf Terminado.} Cuando un proyecto haya terminado, se puede indicar que el estado de este es ``Terminado'' y por lo tanto no se podrán realizar
 modificaciones o registros de elementos, solo consulta.
\end{itemize}