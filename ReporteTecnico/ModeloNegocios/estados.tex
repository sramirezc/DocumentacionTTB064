\section{Diagramas de estados}
Es esta sección se muestran los diagramas de estados y la descripción de cada uno de ellos.\\

\subsection{Diagrama de estados de un caso de uso}

En la figura \refIM{fig:estadosElemento}{Diagrama de estados de un caso de uso} se muestran los estados que tiene un caso de uso a lo largo 
del sistema. Estos estados cambian conforme la operación que se esté realizando sobre ellos.\\

\IMfig[.9]{ModeloNegocios/images/estadosElementos.png}{fig:estadosElemento}{Diagrama de estados de un elemento}

Cada estado se describe a continuación:

\begin{itemize}
 \item {\bf Pre-registro.} Cuando un analista solicite el registro de un caso de uso, este tendrá el estado ``Pre-registro''.
 Los elementos que se encuentran en este estado están temporalmente en la base de datos hasta que el analista decide guardar la información.
 
 \item {\bf Edición.} Un elemento pasará a este estado cuando: 
 \begin{itemize}
  \item El analista guarda la información del caso de uso que solicitó registrar.
  \item Cuando un analista solicita modificar la información del caso de uso que no ha terminado.
  \item Cuando un analista solicita corregir un caso de uso.
  \item Cuando un caso de uso está liberado y un analista solicita modicar su información.
 \end{itemize}
 El analista que haya guardado el caso de uso puede solicitar la edición de este mientras no lo termine.
 
 \item {\bf Terminado.} Cuando el analista haya concluido el registro del caso de uso este pasará a estado ``Terminado''. Cuando el caso de uso está en este estado
 el analista que lo haya terminado podrá modificarlo y cualquier otro analista podrá revisarlo.
 \item {\bf Pendiente de corrección.} Cuando un analista revisa un caso de uso y solicita correcciones, este pasará a estado ``Pendiente de corrección''. Cuando el 
 líder de análisis decide que no puede liberar el caso de uso debido a que tiene errores este pasará a estado ``Pendiente de corrección''. Cualquier 
 analista puede corregir un caso de uso en este estado.
 \item {\bf Por liberar.} Cuando un analista revisa un caso de uso y no realiza observaciones, este pasará a estado ``Por liberar''. El líder de análisis podrá
 decidir liberar el caso de uso o mandarlo a corregir.
 \item {\bf Liberado.} Cuando un caso de uso está en estado ``Por liberar'' y el líder de análisis decide liberar el caso de uso, este pasará a estado ``Liberado''. 
 Cuando el líder de análisis revisa el caso de uso y decide liberarlos, este pasará a estado ``Liberado''.
 Cualquier analista puede modificar la información de un caso de uso en este estado.
%  \item {\bf En registro.} Cuando un analista solicita el registro de un elemento, este tendrá el estado ``En registro'', el elemento se mantiene en 
%  este estado mientras el analista no guarde la información del elemento. Este estado permite asociar información al elemento que se registra aunque aún no 
%  haya sido almacenado en el sistema.
%  \item {\bf En edición.} Cuando un analista guarde la información de un elemento, este tendrá el estado ``En edición'', el elemento se mantiene en 
%  este estado mientras el analista no termine el registro. Si el elemento tiene observaciones para corrección o el líder de análisis habilita la edición,
%  también pasará a este estado.
%  \item {\bf Terminado.} El elemento pasará a este estado cuando el analista decida que ha terminado de registrar toda la información solicitada.
%  \item {\bf En revisión.} Cuando un analista solicite revisar la información de un elemento, el estado de este pasará a ``En revisión''.
%  \item {\bf Liberado.} Si el líder de análisis registra un elemento y está seguro de que toda la información es correcta, entonces podrá liberarlo y el estado cambiará
%  a ``Liberado'', esto significa que ya no necesita cambios. También cuando un líder está revisando la información que ha registrado otro analista
%  podrá decidir si el elemento es correcto y cambiar su estado a ``Liberado''.
\end{itemize}

\subsection{Diagrama de estados de un proyecto}

En la figura \refIM{fig:estadosProyecto}{Diagrama de estados de un proyecto} se muestra el diagrama de estados que corresponde a un proyecto. Cada estado se
detalla a continuación:

\IMfig[.9]{ModeloNegocios/images/estadosProyectos.png}{fig:estadosProyecto}{Diagrama de estados de un proyecto}

\begin{itemize}
 \item {\bf En negociación.} Cuando un proyecto esté siendo negociado con el cliente se puede indicar 
 que el estado de este es ``En negociación''. 
 \RCitem{PC1}{\TODO{Falta indicar si se puede realizar operaciones sobre los elementos en este estado.}}{}
 \item {\bf Iniciado.} Cuando un proyecto haya iniciado antes de ser registrado en la herramienta, se puede indicar que el estado es ``Iniciado''.
 Se puede realizar cualquier operación sobre los elementos de los proyectos en este estado.
 \item {\bf Terminado.} Cuando un proyecto haya terminado, se puede indicar que el estado de este es ``Terminado'' y por lo tanto no se podrán realizar
 modificaciones o registros de elementos, solo consulta.
\end{itemize}