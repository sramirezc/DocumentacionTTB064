En este capítulo se describen las pruebas realizadas al sistema, se describe el total de pruebas realizadas por incremento:
\begin{itemize}
 \item Editor de casos de uso
 \item Generador de casos de prueba 
\end{itemize}

La pruebas que se llevaron a cabo permitieron comprobar el funcionamiento de la herramienta, así como detectar los diferentes errores de todo el proceso de desarrollo. 
Los errores se clasificaron en errores de funcionalidad, excepciones, errores visuales y mejoras al sistema.\\

\section{Editor de casos de uso}

Para probar el {\it Editor de casos de uso} se realizaron registros de diferentes elementos: actores, mensajes, reglas de negocio, entidades, atributos, términos del glosario, pantallas y casos de uso de
dos sistemas distintos:

\begin{itemize}
 \item E-Commerce
 \item PRISMA
\end{itemize}

En la tabla \ref{tabla:vectorEditor} se muestra la cantidad de elementos registrados con la herramienta en relación al total de elementos pertenecientes al sistema PRISMA.

\begin{longtable}{| p{.30\textwidth} | p{.30\textwidth} |}%
	\arrayrulecolor{black}%
	\rowcolor{black}%
	{\color{white}Elemento} & {\color{white}Total registrado}\\ \hline
	\endhead%
	\arrayrulecolor{black}%
	Casos de uso & 9 / 83 \\ \hline
	Pantallas & 10 / 70 \\ \hline
	Mensajes & 10 / 36 \\ \hline
	Entidades & 6 / 27 \\ \hline
	Reglas de negocio & 10 / 26 \\ \hline
	Actores & 3 / 3 \\ \hline
	Términos del glosario  & 1 / 30 \\ \hline
	\caption{Elementos de PRISMA}\label{tabla:vectorEditor}
\end{longtable}%

En la tabla \ref{tabla:vectorEditor2} se muestra el total de elementos registrados en la herramienta pertenecientes a E-Commerce.

\begin{longtable}{| p{.30\textwidth} | p{.30\textwidth} |}%
	\arrayrulecolor{black}%
	\rowcolor{black}%
	{\color{white}Elemento} & {\color{white}Total registrado}\\ \hline
	\endhead%
	\arrayrulecolor{black}%
	Casos de uso & 4 \\ \hline
	Pantallas & 5 \\ \hline
	Mensajes & 9 \\ \hline
	Entidades & 2 \\ \hline
	Reglas de negocio & 6 \\ \hline
	Actores & 2 \\ \hline
	Términos del glosario  & 0 \\ \hline
	\caption{Elementos de E-Commerce}\label{tabla:vectorEditor2}
\end{longtable}%

\subsection*{Resultados}

Con base a los registros realizados fue posible generar la base de datos y el documento de análisis de manera satisfactoria para los dos sistemas. \\

Al realizar las pruebas mencionadas, se detectaron diferentes tipos de errores. Los resultados de las pruebas
realizadas al {\it Editor de casos de uso} se muestran en la tabla \ref{tabla:erroresEditor}, donde también se muestra el total de errores corregidos.

\begin{longtable}{| p{.30\textwidth} | p{.15\textwidth} | p{.15\textwidth} |}%
	\arrayrulecolor{black}%
	\rowcolor{black}%
	{\color{white}Tipo de error} & {\color{white}Total errores detectados} & {\color{white}Total errores corregidos}\\ \hline
	\endhead%
	\arrayrulecolor{black}%
	{\bf Excepción} & 3 & 3 \\ \hline
	{\bf Funcionalidad} & 59 & 50 \\ \hline
	{\bf Visual} & 12 & 11 \\ \hline
	{\bf Mejora al sistema} & 2 & 0 \\ \hline
	\caption{Resultados de las pruebas del Editor de casos de uso}\label{tabla:erroresEditor}
\end{longtable}%

\section{Generador de casos de prueba}

Para probar el {\it Generador de casos de prueba} se utilizaron algunos casos de uso de los registrados en las pruebas del editor. Los casos de uso que se probaron pertenecen a los sistemas:

\begin{itemize}
 \item E-commerce
 \item PRISMA
\end{itemize}

\subsection*{Desarrollo de la prueba realizada al caso de uso CUE2.1 Registrar persona}

A continuación se describirá a detalle el desarrollo de una prueba realizada al {\it Generador de casos de prueba}, con la finalidad de mostrar el número
casos de prueba que se pueden generar, mostrar el tiempo de generación de los casos de prueba, así como mostrar la cantidad de valores que el usuario debe cambiar 
directamente en JMeter para completar al 100 \% una prueba, de un caso de uso con determinadas características.\\

\subsection*{Características del caso de uso}
Esta prueba consiste en la configuración y 
generación de los casos de prueba para el caso de uso {\bf CUE2.1 Registrar persona}, el cual posee las siguientes características:

\begin{itemize}
 \item Tiene un total de 6 entradas de tipo cadena, de las cuales 5 son obligatorias y una es opcional.
 \item Las reglas de negocio que son validadas son: Datos obligatorios, unicidad de parámetros, tipo de dato correcto y longitud correcta.
 \item Los casos de uso previos que deben ser configurados son: CUE13 Iniciar sesión, CUE1 Gestionar proyectos de Administrador y CUE2 Gestionar personal.
\end{itemize}

\subsection*{Total de casos de prueba generados}
Como resultado de esta configuración se generaron \totalCasosPruebaCURegistrarPersona casos de prueba, para completar la prueba
fue necesario ingresar 6 valores directamente en JMeter, estos valores no podían ser ingresados a la configuración. 

\subsection*{Tiempo de respuesta}
Se solicitó 5 veces la generación de la prueba, el tiempo promedio de respuesta del sistema fue de 4.26 segundos.


\subsection*{Resultados}

De las pruebas realizadas al generador, los resultados de las se muestran en la tabla \ref{tabla:erroresGenerador}, donde también se muestra el total de errores corregidos.

\begin{longtable}{| p{.30\textwidth} | p{.15\textwidth} | p{.15\textwidth} |}%
	\arrayrulecolor{black}%
	\rowcolor{black}%
	{\color{white}Tipo de error} & {\color{white}Total errores detectados} & {\color{white}Total errores corregidos}\\ \hline
	\endhead%
	\arrayrulecolor{black}%
	{\bf Excepción} & 3 & 3 \\ \hline
	{\bf Funcionalidad} & 25 & 17 \\ \hline
	{\bf Visual} & 13 & 12 \\ \hline
	{\bf Mejora al sistema} & 2 & 0 \\ \hline
	\caption{Resultados de las pruebas del Generador de casos de prueba}\label{tabla:erroresGenerador}
\end{longtable}

