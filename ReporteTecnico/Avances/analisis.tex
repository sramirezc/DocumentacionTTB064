\section{Análisis}



La etapa de análisis del presente proyecto se desarrolló con el objetivo particular de asistir el proceso de pruebas de un sistema. Debido a que este proceso es muy extenso y a que existe una gran diversidad en las pruebas, se requería acotar la participación de nuestro prototipo en dicho proceso, para realizar esta delimitación se observó un proceso real de pruebas de un sistema y además fue posible participar en él.\\

%Imagen de pruebas

Notamos que en gran medida las pruebas realizadas se encontraban basadas en la documentación de casos de uso, específicamente en las trayectorias. Parte de estas pruebas eran altamente repetitivas, por ejemplo, para cada formulario, la trayectoria describía que el sistema debía validar que los datos marcados como obligatorios fueran ingresados. En el caso anterior, el encargado de realizar la prueba manual tenía que comprobar que dicho funcionamiento se cumpliera y repetirlo para cada uno de los escenarios presentados, para este mismo caso, el encargado de realizar la prueba automatizada, tenía que configurar la prueba indicando todos los comportamientos que debían validarse, en ambos casos fue necesario estudiar el comportamiento descrito en los casos de uso, así como definir todos los datos de entrada necesarios.\\

%Imagen de casos de uso  y pruebas

Con lo anterior el objetivo se limitó a asistir al proceso de pruebas de un sistema utilizando la documentación de casos de uso. Posteriormente se revisaron los casos de uso para conocer cómo se encontraban estructurados y qué elementos los componían. Debido a que los elementos que intervenían en los casos de uso eran bastantes, se realizó un ensayo de la automatización de una prueba con el objetivo de identificar qué elementos resultaban indispensables para una prueba, por lo que con base en este ensayo y al estudio de la documentación de análisis se determinó que para construir un caso de uso y que además resultara útil para una posible automatización de las pruebas, también era necesario contar con los siguientes elementos:

	\begin{itemize}
		\item Pantallas
		\item Entidades
		\item Mensajes
		\item Reglas de negocio
		\item Términos del glosario
		\item Actores
	\end{itemize}
	
%Imagen elementos, casos de uso y pruebas


La información recabada anteriormente permitió plantear la problemática, los objetivos y una propuesta de solución. Las tareas siguientes para esta etapa consistieron en definir de forma explícita qué haría el sistema, para ello se identificaron los principales requerimientos, las reglas de negocio y los actores participantes en el proceso de software. Con base en esta información se planteó un modelo conceptual que describía el funcionamiento general del prototipo, una vez definido este modelo se procedió a definir los casos de uso con sus respectivas pantallas y mensajes. El resultado de las tareas mencionadas anteriormente se encuentran detalladas a lo largo de este documento.
