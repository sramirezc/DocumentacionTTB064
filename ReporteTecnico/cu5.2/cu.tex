\begin{UseCase}{CU 5.2}{Modificar caso de uso}
	{
		Este caso de uso permite al actor modificar la información general, la descripción, las precondiciones y las postcondiciones de un caso de uso. En la sección de descripción del caso de uso se puede utilizar el token para visualizar los elementos registrados y hacerles referencia.
		
	}
	\UCitem{Actor}{\cdtRef{actor:liderAnalisis}{Líder de análisis}, \cdtRef{actor:analista}{Analista}}
	\UCitem{Propósito}{
		Modificar la información general y la descripción de un caso de uso, así como gestionar las precondiciones y postcondiciones.
	}
	\UCitem{Entradas}{
		\begin{UClist} 
			\UCli De la sección {\bf Información general del caso de uso}: 
				\begin{Citemize}
					\item \cdtRef{Elemento:Numero}{Número}: \ioEscribir.
				    \item \cdtRef{Elemento:Nombre}{Nombre}: \ioEscribir.
					\item \cdtRef{Elemento:Resumen}{Resumen}: \ioEscribir.
				\end{Citemize}
			\UCli De la sección {\bf Descripción del caso de uso}:
				\begin{Citemize}
				        \item \cdtRef{Actor}{Actores}: \ioEscribirSeleccionar.
				        \item \cdtRef{Entrada}{Entradas}: \ioEscribirSeleccionar.
				        \item \cdtRef{Salida}{Salidas}: \ioEscribirSeleccionar.
					\item \cdtRef{RegladeNegocio}{Reglas de negocio}: \ioEscribirSeleccionar.
				\end{Citemize}
			
		\end{UClist}
	}	
	
	\UCitem{Salidas}{	
		\begin{UClist}			
			\UCli \cdtRef{Elemento:Clave}{Clave}: \ioCalcular{\cdtIdRef{RN12}{Identificador de elemento}}.
			\UCli De la sección {\bf Información general del caso de uso}: 
				\begin{Citemize}
					\item \cdtRef{Elemento:Numero}{Número}: \ioEscribir.
				    \item \cdtRef{Elemento:Nombre}{Nombre}: \ioEscribir.
					\item \cdtRef{Elemento:Resumen}{Resumen}: \ioEscribir.
				\end{Citemize}
			\UCli De la sección {\bf Descripción del caso de uso}:
				\begin{Citemize}
				        \item \cdtRef{Actor}{Actores}: \ioEscribirSeleccionar.
				        \item \cdtRef{Entrada}{Entradas}: \ioEscribirSeleccionar.
				        \item \cdtRef{Salida}{Salidas}: \ioEscribirSeleccionar.
					\item \cdtRef{RegladeNegocio}{Reglas de negocio}: \ioEscribirSeleccionar.
				\end{Citemize}
		\end{UClist}
	}
	
	\UCitem{Mensajes}{	
		\begin{UClist}
			\UCli \cdtIdRef{MSG1}{Operación exitosa}: Se muestra en la pantalla \cdtIdRef{IU 5}{Gestionar casos de uso} para indicar que el registro fue exitoso.
		\end{UClist}
	}
	
	\UCitem{Precondiciones}{ 
		Ninguna
	}			
	\UCitem{Postcondiciones}{ 
		\begin{UClist}
			\UCli El estado del caso de uso cambiará a ``Edición''.
	 	\end{UClist}
	}
	\UCitem{Errores}{
		\begin{UClist}
			\UCli \cdtIdRef{MSG4}{Dato obligatorio}: Se muestra en la pantalla \cdtIdRef{IU 5.2}{Modificar caso de uso} cuando no se ha ingresado un dato marcado como obligatorio.
			\UCli \cdtIdRef{MSG5}{Dato incorrecto}: Se muestra en la pantalla \cdtIdRef{IU 5.2}{Modificar caso de uso} cuando el tipo de dato ingresado no cumple con el tipo de dato solicitado en el campo.
			\UCli \cdtIdRef{MSG6}{Longitud inválida}: Se muestra en la pantalla \cdtIdRef{IU 5.2}{Modificar caso de uso} cuando se ha excedido la longitud de alguno de los campos.
			\UCli \cdtIdRef{MSG7}{Registro repetido}: Se muestra en la pantalla \cdtIdRef{IU 5.2}{Modificar caso de uso} cuando se registre un caso de uso con un nombre o número que ya este registrado.
			\UCli \cdtIdRef{MSG15}{Dato no registrado}: Se muestra en la pantalla \cdtIdRef{IU 5.2}{Modificar caso de uso} cuando un elemento referenciado no existe en el sistema.
			\UCli \cdtIdRef{MSG23}{Caracteres inválidos}: Se muestra en la pantalla \cdtIdRef{IU 5.2}{Modificar caso de uso} cuando el nombre del caso de uso contiene un caracter no válido.
			\UCli \cdtIdRef{MSG27}{Token incorrecto}: Se muestra en la pantalla \cdtIdRef{IU 5.2}{Modificar caso de uso} cuando el token ingresado está mal formado.
		\end{UClist}
	}
	
	\UCitem{Tipo}{
		Secundario, extiende del caso de uso \cdtIdRef{CU 5}{Gestionar casos de uso}.
	}

	
\end{UseCase}
 %-------------------------------------------------------%trayectoria Principal-----------------------------------------------
 \begin{UCtrayectoria}
    \UCpaso[\UCactor] Solicita registrar un caso de uso oprimiendo el botón \cdtButton{Registrar} de la pantalla \cdtIdRef{IU 5}{Gestionar casos de uso}.
    \UCpaso[\UCsist] Calcula la clave del caso de uso.
    \UCpaso[\UCsist] Muestra la pantalla \cdtIdRef{IU 5.2}{Modificar caso de uso} en la cual se realizará el registro del caso de uso. 
    \UCpaso[\UCactor] Ingresa la información general del caso de uso en la pantalla \cdtIdRef{IU 5.2}{Modificar caso de uso}. \label{cu5.2:ingresaDatos}
    \UCpaso[\UCactor] Ingresa los actores en la pantalla \cdtIdRef{IU 5.2}{Modificar caso de uso}. \refTray{A} \label{cu5.2:ingresaActores}
    \UCpaso[\UCactor] Ingresa las entradas en la pantalla \cdtIdRef{IU 5.2}{Modificar caso de uso}. \refTray{B} \refTray{C} \label{cu5.2:ingresaEntradas}
    \UCpaso[\UCactor] Ingresa las salidas en la pantalla \cdtIdRef{IU 5.2}{Modificar caso de uso}. \refTray{B} \refTray{C} \refTray{D} \label{cu5.2:ingresaSalidas}
    \UCpaso[\UCactor] Ingresa las reglas de negocio en la pantalla \cdtIdRef{IU 5.2}{Modificar caso de uso}. \refTray{E} \label{cu5.2:ingresaReglasNegocio}
    \UCpaso[\UCactor] Gestiona las precondiciones.\label{cu5.2:ingresaPrecond}
    \UCpaso[\UCactor] Gestiona las postcondiciones.\label{cu5.2:ingresaPostcond}
    \UCpaso[\UCactor] Solicita registrar el caso de uso oprimiendo el botón \cdtButton{Aceptar} de la pantalla \cdtIdRef{IU 5.2}{Modificar caso de uso}. \refTray{F}
    \UCpaso[\UCsist] Verifica que el actor ingrese todos los campos obligatorios con base en la regla de negocio  \cdtIdRef{RN8}{Datos obligatorios}. \refTray{G}
    \UCpaso[\UCsist] Verifica que el nombre del caso de uso no se encuentre registrado en el sistema con base en la regla de negocio  \cdtIdRef{RN6}{Unicidad de nombres}. \refTray{H}
    
    \UCpaso[\UCsist] Verifica que el nombre no contenga caracteres inválidos con base en la regla de negocio \cdtIdRef{RN2}{Nombres de los elementos}. \refTray{I}
    \UCpaso[\UCsist] Verifica que el número del caso de uso no se encuentre registrado en el sistema con base en la regla de negocio  \cdtIdRef{RN1}{Unicidad de números}. \refTray{J}
    
    \UCpaso[\UCsist] Verifica que los datos requeridos sean proporcionados correctamente como se especifica en la regla de negocio \cdtIdRef{RN7}{Información correcta}. \refTray{K} \refTray{L}
    
    \UCpaso[\UCsist] Verifica que los tokens utilizados esten bien formados. \refTray{M}
    \UCpaso[\UCsist] Verifica que los elementos referenciados existan. \refTray{N}
    
    \UCpaso[\UCsist] Registra la información del caso de uso en el sistema.
    \UCpaso[\UCsist] Cambia el estado del caso de uso a ``Edición''.
    \UCpaso[\UCsist] Muestra el mensaje \cdtIdRef{MSG1}{Operación exitosa} en la pantalla \cdtIdRef{IU 5}{Gestionar casos de uso} 
    para indicar al actor que el registro se ha realizado exitosamente.
 \end{UCtrayectoria}
 
  %----------------------------------------------------------%trayectoria A---------------------------------------------------- 
 \begin{UCtrayectoriaA}{A}{El actor desea ingresar un actor.}
 	\UCpaso[\UCactor] Ingresa el token ``ACT·''.
 	\UCpaso[\UCsist] Busca los actores registrados en el sistema. 
 	\UCpaso[\UCsist] Muestra la lista de actores encontrados.
 	\UCpaso[\UCactor] Selecciona un actor de la lista.
  	\UCpaso[\UCsist] Oculta la lista y agrega el nombre del actor al texto.
    \UCpaso[] Continúa con el paso \ref{cu5.2:ingresaActores} de la trayectoria principal.
 \end{UCtrayectoriaA}
  %----------------------------------------------------------%trayectoria B---------------------------------------------------- 
 \begin{UCtrayectoriaA}{B}{El actor desea ingresar un término del glosario.}
 	\UCpaso[\UCactor] Ingresa el token ``GLS·''.	
 	\UCpaso[\UCsist] Busca los términos del glosario registrados en el sistema. 
 	\UCpaso[\UCsist] Muestra la lista de términos del glosario.
 	\UCpaso[\UCactor] Selecciona un término del glosario de la lista.
  	\UCpaso[\UCsist] Oculta la lista y agrega el nombre del término del glosario al texto.
    \UCpaso[] Continúa con el paso \ref{cu5.2:ingresaEntradas} o el paso \ref{cu5.2:ingresaSalidas} de la trayectoria principal.
 \end{UCtrayectoriaA}

  %----------------------------------------------------------%trayectoria C---------------------------------------------------- 
 \begin{UCtrayectoriaA}{C}{El actor desea ingresar un atributo de una entidad.}
 	\UCpaso[\UCactor] Ingresa el token ``ATR·''.
  	\UCpaso[\UCsist] Busca los atributos de las entidades registradas.
  	\UCpaso[\UCsist] Muestra la lista de atributos encontrados.
 	\UCpaso[\UCactor] Selecciona un atributo de la lista.
  	\UCpaso[\UCsist] Oculta la lista y agrega el nombre del atributo al texto.
    \UCpaso[] Continúa con el paso \ref{cu5.2:ingresaEntradas} o el paso \ref{cu5.2:ingresaSalidas} de la trayectoria principal.
 \end{UCtrayectoriaA}
%----------------------------------------------------------%trayectoria D---------------------------------------------------- 
 \begin{UCtrayectoriaA}{D}{El actor desea ingresar un mensaje.}
 	 \UCpaso[\UCactor] Ingresa el token ``MSG·''.	
 	\UCpaso[\UCsist] Busca los mensajes registrados en el sistema. 
 	\UCpaso[\UCsist] Muestra la lista de mensajes.
 	\UCpaso[\UCactor] Selecciona un mensaje de la lista.
  	\UCpaso[\UCsist] Oculta la lista y agrega el nombre del mensaje al texto.
    \UCpaso[] Continúa con el paso \ref{cu5.2:ingresaSalidas} de la trayectoria principal.
 \end{UCtrayectoriaA}
 %----------------------------------------------------------%trayectoria E---------------------------------------------------- 
 \begin{UCtrayectoriaA}{E}{El actor desea ingresar una regla de negocio.}
 	\UCpaso[\UCactor] Ingresa el token ``RN·''.	
 	\UCpaso[\UCsist] Busca las reglas de negocio registradas en el sistema. 
 	\UCpaso[\UCsist] Muestra la lista de reglas de negocio.
 	\UCpaso[\UCactor] Selecciona una regla de negocio de la lista.
  	\UCpaso[\UCsist] Oculta la lista y agrega el nombre de la regla de negocio al texto.
    \UCpaso[] Continúa con el paso \ref{cu5.2:ingresaReglasNegocio} de la trayectoria principal.
 \end{UCtrayectoriaA}
 %----------------------------------------------------------%trayectoria F---------------------------------------------------- 
 \begin{UCtrayectoriaA}[Fin del caso de uso]{F}{El actor desea cancelar la operación.}
    \UCpaso[\UCactor] Solicita cancelar la operación oprimiendo el botón \cdtButton{Cancelar} de la pantalla \cdtIdRef{IU 5.2}{Modificar caso de uso}.
    \UCpaso[\UCsist] Elimina la información registrada en el sistema.
    \UCpaso[\UCsist] Muestra la pantalla \cdtIdRef{IU 5}{Gestionar casos de uso}.
 \end{UCtrayectoriaA}
  %----------------------------------------------------------%trayectoria G---------------------------------------------------- 
 \begin{UCtrayectoriaA}{G}{El actor no ingresó algún dato marcado como obligatorio.}
    \UCpaso[\UCsist] Muestra el mensaje \cdtIdRef{MSG4}{Dato obligatorio} y señala el campo que presenta el error en la pantalla 
	    \cdtIdRef{CU 5.2}{Modificar caso de uso}, indicando al actor que el dato es obligatorio.
    \UCpaso[] Continúa con el paso \ref{cu5.2:ingresaDatos} de la trayectoria principal.
 \end{UCtrayectoriaA}
 %----------------------------------------------------------%trayectoria H---------------------------------------------------- 
 \begin{UCtrayectoriaA}{H}{El actor ingresó un nombre de caso de uso repetido.}
    \UCpaso[\UCsist] Muestra el mensaje \cdtIdRef{MSG7}{Registro repetido} y señala el campo que presenta la duplicidad en la pantalla 
	    \cdtIdRef{CU 5.2}{Modificar caso de uso}, indicando al actor que existe un caso de uso con el mismo nombre.
    \UCpaso[] Continúa con el paso \ref{cu5.2:ingresaDatos} de la trayectoria principal.
 \end{UCtrayectoriaA}
%----------------------------------------------------------%trayectoria I---------------------------------------------------- 
 \begin{UCtrayectoriaA}{I}{El actor ingresó un nombre con caracteres inválidos.}
    \UCpaso[\UCsist] Muestra el mensaje \cdtIdRef{MSG23}{Caracteres inválidos} y señala el campo que contiene los caracteres inválidos.
    \UCpaso[] Continúa con el paso \ref{cu5.2:ingresaDatos} de la trayectoria principal.
 \end{UCtrayectoriaA}
 %----------------------------------------------------------%trayectoria J---------------------------------------------------- 
 \begin{UCtrayectoriaA}{J}{El actor ingresó un número de caso de uso repetido.}
    \UCpaso[\UCsist] Muestra el mensaje \cdtIdRef{MSG7}{Registro repetido} y señala el campo que presenta la duplicidad en la pantalla 
	    \cdtIdRef{CU 5.2}{Modificar caso de uso}, indicando al actor que existe un caso de uso con el mismo número.
    \UCpaso[] Continúa con el paso \ref{cu5.2:ingresaDatos} de la trayectoria principal.
 \end{UCtrayectoriaA}
 %----------------------------------------------------------%trayectoria K---------------------------------------------------- 
 \begin{UCtrayectoriaA}{K}{El actor ingresó un tipo de dato incorrecto.}
    \UCpaso[\UCsist] Muestra el mensaje \cdtIdRef{MSG4}{Formato incorrecto} y señala el campo que presenta el dato inválido en la 
    pantalla \cdtIdRef{IU 5.2}{Modificar caso de uso} para indicar que se ha ingresado un tipo de dato inválido.
    \UCpaso[] Continúa con el paso \ref{cu5.2:ingresaDatos} de la trayectoria principal.
 \end{UCtrayectoriaA}
 %----------------------------------------------------------%trayectoria L----------------------------------------------------  
 \begin{UCtrayectoriaA}{L}{El actor proporciona un dato que excede la longitud máxima.}
    \UCpaso[\UCsist] Muestra el mensaje \cdtIdRef{MSG5}{Se ha excedido la longitud máxima del campo} y señala el campo que excede la 
    longitud en la pantalla \cdtIdRef{IU 5.2}{Modificar caso de uso}, para indicar que el dato excede el tamaño máximo permitido.
    \UCpaso[] Continúa con el paso \ref{cu5.2:ingresaDatos} de la trayectoria principal.
 \end{UCtrayectoriaA}
 %----------------------------------------------------------%trayectoria M---------------------------------------------------- 
 \begin{UCtrayectoriaA}{M}{El actor ingresó un token mal formado.}
    \UCpaso[\UCsist] Muestra el mensaje \cdtIdRef{MSG27}{Token incorrecto} mencionando que el token utilizado no es correcto.
    \UCpaso[] Continúa con el paso \ref{cu5.2:ingresaDatos} de la trayectoria principal.
 \end{UCtrayectoriaA}
 %----------------------------------------------------------%trayectoria N---------------------------------------------------- 
 \begin{UCtrayectoriaA}{N}{Alguno de los elementos referenciados no existe en el sistema.}
    \UCpaso[\UCsist] Muestra el mensaje \cdtIdRef{MSG15}{Dato no registrado} mencionando el elemento que no está registrado en el sistema.
    \UCpaso[] Continúa con el paso \ref{cu5.2:ingresaDatos} de la trayectoria principal.
 \end{UCtrayectoriaA}
 

\subsection{Puntos de extensión}	
\UCExtensionPoint{El actor requiere registrar una precondición}
	{Paso \ref{cu5.2:ingresaPrecond} de la trayectoria principal}
	{\cdtIdRef{CU 5.2.2}{Registrar precondición}}
	
\UCExtensionPoint{El actor requiere registrar una postcondición}
	{Paso \ref{cu5.2:ingresaPostcond} de la trayectoria principal}
	{\cdtIdRef{CU 5.2.3}{Registrar postcondición}}