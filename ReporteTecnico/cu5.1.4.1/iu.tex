\subsection{IU 5.1.4.1 Registrar punto de extensión}

\subsubsection{Objetivo}
	
	Esta pantalla permite al actor registrar la información de un punto de extensión.

\subsubsection{Diseño}

    En la figura ~\ref{IU 5.1.4.1} se muestra la pantalla ``Registrar punto de extensión'' que permite registrar un punto de extensión. El actor deberá ingresar la información solicitada, esto incluye indicar la casusa de la extensión, la región de la trayectoria en la que ocurre y el caso de uso al que extiende.
    
    Una vez ingresada la información solicitada en la pantalla, el actor deberá oprimir el botón 
    \cdtButton{Aceptar}, el sistema validará y registrará la información sólo si se han cumplido todas las reglas de negocio establecidas.  \\
    
    Finalmente se mostrará el mensaje \cdtIdRef{MSG1}{Operación exitosa} en la pantalla donde se solicitó la operación, para indicar que la información del punto de extensión se ha registrado correctamente.        


    \IUfig[.9]{cu5.1.4.1/images/iu.png}{IU 5.1.4.1}{Registrar punto de extensión}


\subsubsection{Comandos}
\begin{itemize}
	\item \cdtButton{Aceptar}: Permite al actor guardar el registro del punto de extensión, dirige a la pantalla \cdtIdRef{IU 5.1}{Registrar caso de uso} o \cdtIdRef{IU 5.2}{Modificar caso de uso}.
	\item \cdtButton{Cancelar}: Permite al actor cancelar el registro del punto de extensión, dirige a la pantalla \cdtIdRef{IU 5.1}{Registrar caso de uso} o \cdtIdRef{IU 5.2}{Modificar caso de uso}.
\end{itemize}

\subsubsection{Mensajes}

	
\begin{description}

	\item[ \cdtIdRef{MSG1}{Operación exitosa}:] Se muestra en la pantalla \cdtIdRef{IU 5.1}{Registrar caso de uso} o \cdtIdRef{IU 5.2}{Modificar caso de uso} para indicar que el registro fue exitoso.
	\item[ \cdtIdRef{MSG4}{Dato obligatorio}:] Se muestra en la pantalla \cdtIdRef{IU 5.1.4.1}{Registrar punto de extensión} cuando no se ha ingresado un dato marcado como obligatorio.
	\item[ \cdtIdRef{MSG6}{Longitud inválida}:] Se muestra en la pantalla \cdtIdRef{IU 5.1.4.1}{Registrar punto de extensión} cuando se ha excedido la longitud de alguno de los campos.
\end{description}
