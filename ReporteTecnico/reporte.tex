 \documentclass[10pt]{book}
\usepackage{cdt/cdtBusiness}
\usepackage{prisma}
\usepackage{enumerate}
\usepackage{subfigure}
\usepackage{cite}
\usepackage{ragged2e}


%%%%%%%%%%%%%%%%%%%%%%%%%%%%%%%%%%%%%%%%%%%%%%%%%%%%%%%%%%%%%%%%
% Datos del proyecto

% \tituloTrabajo
% \noRegistro
% \resumen
% \palabrasClave
% \autores
% \directores

% \organizacion[ESCOM--IPN]{ESCOM-IPN}
% \autor[Natalia Giselle Hernández Sánchez, Sergio Ramírez Camacho]{\autores}
% 
% \sistema[PRISMA]{\prisma}
% \proyecto[Prototipo de editor de casos de uso para la construcción asistida de casos de prueba]{\editor}
% \documento{RT}{Reporte técnico}{\DRAFT{\today}} %{\RELEASE{2.0}}
% 
% \entregable{}{\Large{Editor de casos de uso}}

% Descomentar y establecer la fecha cuando se desee congelar la fecha del documento.
%\fecha{12 de Abril de 2013}

%%%%%%%%%%%%%%%%%%%%%%%%%%%%%%%%%%%%%%%%%%%%%%%%%%%%%%%%%%%%%%%%
% Datos para revisión
% \elaboro[Líder de proyecto IPN-ESCOM]{M. en C. Idalia Maldonado Castillo\vspace{0.4cm}} % Responsable del contenido (IPN) Lic. Ulises Vélez Saldaña
% \superviso[Jefe del Área SMAGEM]{Lic. Mario Alberto Pichardo Mejía} % Quien recibe el documento (Contraparte)
% \aprobo[Jefe del Departamento de Educación y Difusión de la Cultura Ambiental SMAGEM]{Emmanuel Macedonio Flores Campuzano} % Responsable Técnico (Contraparte)

%\title{\varProyecto}
%\subtitle{\varCveDocumento--\varDocumento}

%%%%%%%%%%%%%%%%%%%%%%%%%%%%%%%%%%%%%%%%%%%%%%%%%%%%%%%%%%%%%%%%
% Documentos relacionados con el documento actual

% TODO: Escriba los documentos en los que está basado este documento.
% \docRefs{
%      \docItem{Catálogo de Escuelas}{}{Catálogo de Escuelas de la DGAIR (Dirección General de Acreditación, Incorporación y Revalidación) de la SEP (Secretaría de Educación Pública)}
%      \docItem{M-5RT}{1.0}{ \cdtLabel{M-5RT}{Minuta de la quinta reunión de trabajo}, para presentar el proyecto SIG para la SMAGEM.}
% }

%%%%%%%%%%%%%%%%%%%%%%%%%%%%%%%%%%%%%%%%%%%%%%%%%%%%%%%%%%%%%%%%
% Elementos contenidos en el documento

% TODO: Al finalizar el análisis resuma aquí todos los elementos del componente: RN, CU, IU, MSG.
% \elemRefs{
%   \elemItem{Glosario de términos}{1.0}{Descripción de los terminos técnicos y de negocio usados.}                  

%Reglas de negocio
%    \elemItem{RN-S1}{1.0}{Información correcta.}          
  
%Mensajes
%    \elemItem{MSG1}{1.0}{Operación realizada exitosamente}
  
%Casos de uso
%   \elemItem{CUEI 1}{1.0}{Administrar incendios (Responsable)}  
% 
%   \elemItem{IUEI 1}{1.0}{Administrar incendios (Responsable)}  
% }

%%%%%%%%%%%%%%%%%%%%%%%%%%%%%%%%%%%%%%%%%%%%%%%%%%%%%%%%%%%%%%%%
\begin{document}
\bibliographystyle{IEEEtran}
%=========================================================
% Portada
\ThisLRCornerWallPaper{1}{cdt/theme/agua.png}
\thispagestyle{empty}

\cfinput{portada}
% \makeFirmas

%=========================================================
% Indices del documento
\frontmatter
% \LRCornerWallPaper{1}{cdt/theme/pleca.png}
\tableofcontents
\listoffigures
%\listoftables
\mainmatter

% Para esconder la información del documentador se descomenta el \hideControlVersion
 \hideControlVersion

%=========================================================
\chapter{Introducción}\label{chp:introduccion}
\cfinput{Introduccion/introduccion}
%=========================================================

\chapter{Marco teórico}\label{chp:marcoTeorico} 
\hypertarget{chp:marcoTeorico}{}
\cfinput{MarcoTeorico/marcoTeorico}
%=========================================================
\chapter{Requerimientos del sistema}\label{chp:requerimientos}
\hypertarget{chp:requerimientos}{}
\cfinput{ModeloNegocios/requerimientos}
%=========================================================
\chapter{Análisis de riesgos}\label{chp:riesgos}
\cfinput{AnalisisRiesgos/analisisRiesgos}

%=========================================================
\chapter{Modelo de negocios}\label{chp:modeloNegocios}
En este capítulo se mostrará aquella información que define el negocio de la herramienta: el glosario, los diagramas de estado, el modelo conceptual 
y las reglas de negocio.\\
	\cfinput{ModeloNegocios/glosario}
	\cfinput{ModeloNegocios/estados}
	\cfinput{ModeloNegocios/conceptual}
	\cfinput{ModeloNegocios/conceptualProyecto}
	\cfinput{ModeloNegocios/reglas}
%===========================================================
 \chapter{Modelo de comportamiento}\label{chp:modeloComportamiento}
	\cfinput{ModeloComportamiento/comportamiento.tex}
	\cfinput{cu5/cu} %Gestionar casos de uso
	\cfinput{cu5.1/cu}
	\cfinput{cu5.1.1/cu}
	\cfinput{cu5.1.1.1/cu}
	\cfinput{cu5.1.1.1.1/cu}
% 	\cfinput{cu5.1.1.1.2/cu}
	\cfinput{cu5.1.2/cu}
	\cfinput{cu5.1.2.1/cu}
	\cfinput{cu5.1.3/cu}
	\cfinput{cu5.1.3.1/cu}
	\cfinput{cu5.1.4/cu}
	\cfinput{cu5.1.4.1/cu}
	\cfinput{cu8/cu}
	\cfinput{cu8.1/cu}
	\cfinput{cu9/cu}
	\cfinput{cu9.1/cu}
	\cfinput{cu11/cu}
	\cfinput{cu11.1/cu}
	\cfinput{cu11.5/cu}
	
% 	\cfinput{cu/cu}

\chapter{Modelo de interacción con el usuario}\label{chp:modeloInteraccionUsuario}
% \cfinput{ModeloInteraccion/interaccion}

%---------------------------------------------------------------------
\section{Interfaces del prototipo}
	\cfinput{cu5/iu}
	\cfinput{cu5.1/iu}
	\cfinput{cu5.1.1/iu}
	\cfinput{cu5.1.1.1/iu}
	\cfinput{cu5.1.1.1.1/iu}
% 	\cfinput{cu5.1.1.1.2/iu}
	\cfinput{cu5.1.2.1/iu}
	\cfinput{cu5.1.3.1/iu}
	\cfinput{cu5.1.4.1/iu}
	\cfinput{cu8/iu}
	\cfinput{cu8.1/iu}
	\cfinput{cu9/iu}
	\cfinput{cu9.1/iu}
	\cfinput{cu11/iu}
	\cfinput{cu11.1/iu}
	\cfinput{cu11.5/iu}
	
%---------------------------------------------------------------------
\section{Diseño de mensajes}
	\cfinput{ModeloInteraccion/mensajes}

% 	\cfinput{Avances/avances}

% \clossing

%---------------------------------------------------------------------

\bibliography{Referencias}

\end{document}

