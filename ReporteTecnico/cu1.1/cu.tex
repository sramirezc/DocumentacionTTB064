\begin{UseCase}{CU 1.1}{Registrar proyecto}
	{
		Este caso de uso permite registrar la información de un proyecto.
	}
	
	\UCitem{Actor}{\cdtRef{actor:administrador}{Administrador}}
	\UCitem{Propósito}{
		Registrar la información de un proyecto.
	}
	\UCitem{Entradas}{
		\begin{UClist}
			\UCli \cdtRef{Proyecto:Clave}{Clave}: \ioEscribir.
			\UCli \cdtRef{Proyecto:Nombre}{Nombre}: \ioEscribir.
			\UCli \cdtRef{Proyecto:FechaDeInicio}{Fecha de inicio}: \ioCalendario.
			\UCli \cdtRef{Proyecto:FechaDeTermino}{Fecha de término}: \ioCalendario.
			\UCli \cdtRef{Proyecto:FechaDeInicioProgramada}{Fecha de Inicio Programada}: \ioCalendario.
			\UCli \cdtRef{Proyecto:FechaDeTerminoProgramada}{Fecha de Término Programada}: \ioCalendario.
			\UCli \cdtRef{Proyecto:Descripcion}{Descripcion}: \ioEscribir.
			\UCli \cdtRef{Proyecto:Contraparte}{Contraparte}: \ioEscribir.
			\UCli \cdtRef{Proyecto:Presupuesto}{Presupuesto}: \ioEscribir.
		\end{UClist}
	}
	\UCitem{Salidas}{
		Ninguna
	}
	
	\UCitem{Mensajes}{
		\begin{UClist}
			\UCli \cdtIdRef{MSG1}{Operación exitosa}: Se muestra en la pantalla \cdtIdRef{IU 1}{Gestionar proyectos de Administrador} para indicar que el registro fue exitoso.
		\end{UClist}
	}

	\UCitem{Precondiciones}{
		\begin{UClist}
			\UCli Que exista al menos un colaborador registrado.
		\end{UClist}
	}
	
	\UCitem{Postcondiciones}{
		\begin{UClist}
			\UCli Se registrará un proyecto en el sistema.
		\end{UClist}
	}

	\UCitem{Errores}{
		\begin{UClist}
			\UCli \cdtIdRef{MSG4}{Dato obligatorio}: Se muestra en la pantalla \cdtIdRef{IU 1.1}{Registrar proyecto} cuando no se ha ingresado un dato marcado como obligatorio.
			\UCli \cdtIdRef{MSG5}{Dato incorrecto}: Se muestra en la pantalla \cdtIdRef{IU 1.1}{Registrar proyecto} cuando el tipo de dato ingresado no cumple con el tipo de dato solicitado en el campo.
			\UCli \cdtIdRef{MSG6}{Longitud inválida}: Se muestra en la pantalla \cdtIdRef{IU 1.1}{Registrar proyecto} cuando se ha excedido la longitud de alguno de los campos.
			\UCli \cdtIdRef{MSG7}{Registro repetido}: Se muestra en la pantalla \cdtIdRef{IU 1.1}{Registrar proyecto} cuando se registre un proyecto con un nombre que ya exista.
		\end{UClist}
	}

	\UCitem{Tipo}{
		Secundario, extiende del casos de uso \cdtIdRef{CU 1}{Gestionar proyectos de Administrador}.
	}
\end{UseCase}
%-------------------------------------------------------%trayectoria Principal-----------------------------------------------
 \begin{UCtrayectoria}
    \UCpaso[\UCactor] Solicita registrar un proyecto oprimiendo el botón \cdtButton{Registrar} de la pantalla \cdtIdRef{IU 1}{Gestionar proyectos de Administrador}.
    \UCpaso[\UCsist] Busca los colaboradores registrados en el sistema. 
    \UCpaso[\UCsist] Verifica que exista al menos un colaborador. \refTray{A}
    \UCpaso[\UCsist] Muestra la pantalla \cdtIdRef{IU 1.1}{Registrar proyecto} en la cual se realizará el registro del proyecto. 
    \UCpaso[\UCactor] Ingresa la información solicitada en la pantalla. \label{cu1.1:ingresaDatos}
    \UCpaso[\UCactor] Solicita guardar el proyecto oprimiendo el botón \cdtButton{Aceptar} de la pantalla \cdtIdRef{IU 1.1}{Registrar proyecto}. \refTray{B} 
    
    \UCpaso[\UCsist] Verifica que el actor ingrese todos los campos obligatorios con base en la regla de negocio  \cdtIdRef{RN8}{Datos obligatorios}. \refTray{C}
    \UCpaso[\UCsist] Verifica que el nombre del proyecto no se encuentre registrado en el sistema con base en la regla de negocio  \cdtIdRef{RN6}{Unicidad de nombres}. \refTray{D}
    \UCpaso[\UCsist] Verifica que los datos requeridos sean proporcionados correctamente como se especifica en la regla de negocio \cdtIdRef{RN7}{Información correcta}. \refTray{E} \refTray{F}
    \UCpaso[\UCsist] Registra la información del proyecto en el sistema.
    \UCpaso[\UCsist] Muestra el mensaje \cdtIdRef{MSG1}{Operación exitosa} en la pantalla \cdtIdRef{IU 1}{Gestionar proyectos de Administrador}
    para indicar al actor que el registro se ha realizado exitosamente.
 \end{UCtrayectoria}

 %----------------------------------------------------------%trayectoria A---------------------------------------------------- 
 \begin{UCtrayectoriaA}[Fin del caso de uso]{A}{No hay ningún colaborador registrado.}
    \UCpaso[\UCsist] Muestra el mensaje \cdtIdRef{MSG25}{Falta de información} en la la pantalla \cdtIdRef{IU 1}{Gestionar proyectos de Administrador}.
 \end{UCtrayectoriaA} 
 %----------------------------------------------------------%trayectoria B---------------------------------------------------- 
 \begin{UCtrayectoriaA}[Fin del caso de uso]{B}{El actor desea cancelar la operación.}
    \UCpaso[\UCactor] Solicita cancelar la operación oprimiendo el botón \cdtButton{Cancelar} de la pantalla \cdtIdRef{IU 1.1}{Registrar proyecto}.
    \UCpaso[\UCsist] Muestra la pantalla donde se solicitó la operación.
 \end{UCtrayectoriaA} 
  %----------------------------------------------------------%trayectoria C---------------------------------------------------- 
 \begin{UCtrayectoriaA}{C}{El actor no ingresó algún dato marcado como obligatorio.}
    \UCpaso[\UCsist] Muestra el mensaje \cdtIdRef{MSG4}{Dato obligatorio} y señala el campo que presenta el error en la pantalla 
	    \cdtIdRef{IU 1.1}{Registrar proyecto}, indicando al actor que el dato es obligatorio.
    \UCpaso[] Continúa con el paso \ref{cu1.1:ingresaDatos} de la trayectoria principal.
 \end{UCtrayectoriaA}
 %----------------------------------------------------------%trayectoria D---------------------------------------------------- 
 \begin{UCtrayectoriaA}{D}{El actor ingresó un nombre de proyecto repetido.}
    \UCpaso[\UCsist] Muestra el mensaje \cdtIdRef{MSG7}{Registro repetido} y señala el campo que presenta la duplicidad en la pantalla 
	    \cdtIdRef{IU 1.1}{Registrar proyecto}, indicando al actor que existe un proyecto con el mismo nombre.
    \UCpaso[] Continúa con el paso \ref{cu1.1:ingresaDatos} de la trayectoria principal.
 \end{UCtrayectoriaA}
 
 %----------------------------------------------------------%trayectoria E----------------------------------------------------  
 \begin{UCtrayectoriaA}{E}{El actor proporciona un dato que excede la longitud máxima.}
    \UCpaso[\UCsist] Muestra el mensaje \cdtIdRef{MSG5}{Se ha excedido la longitud máxima del campo} y señala el campo que excede la 
    longitud en la pantalla \cdtIdRef{IU 1.1}{Registrar proyecto}, para indicar que el dato excede el tamaño máximo permitido.
    \UCpaso[] Continúa con el paso \ref{cu1.1:ingresaDatos} de la trayectoria principal.
 \end{UCtrayectoriaA}
 
 %----------------------------------------------------------%trayectoria F---------------------------------------------------- 
 \begin{UCtrayectoriaA}{F}{El actor ingresó un tipo de dato incorrecto.}
    \UCpaso[\UCsist] Muestra el mensaje \cdtIdRef{MSG4}{Formato incorrecto} y señala el campo que presenta el dato inválido en la 
    pantalla \cdtIdRef{IU 1.1}{Registrar proyecto} para indicar que se ha ingresado un tipo de dato inválido.
    \UCpaso[] Continúa con el paso \ref{cu1.1:ingresaDatos} de la trayectoria principal.
 \end{UCtrayectoriaA}
 