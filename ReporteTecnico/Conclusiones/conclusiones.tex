	Las pruebas de software con frecuencia representan un costo muy elevado en relación al invertido en las demás actividades del proceso de software. Debido a ello, se han realizado diversos intentos por automatizar su proceso, como lo son herramientas de software o metodologías de pruebas.\\

	El desarrollo del presente trabajo demostró que es posible automatizar las pruebas de software a través de mecanismos semiautomáticos que reutilizan la información de la documentación de análisis.\\
	
	Debido a que la especificación de casos de uso es realizada por diferentes analistas, con frecuencia se encuentran diferentes redacciones para un mismo objetivo, lo cual complica bastante el entendimiento de la información para una computadora, por lo que analizar los pasos de un caso de uso y traducirlos a componentes de pruebas resulta altamente complejo.\\
		
	Por otro lado, se identificó que tener la información de la documentación de análisis en una base de datos permite mantener la información organizada y centralizada. En general el contar con una base de datos de esta información abre las puertas a otros sistemas para automatizar algún proceso dentro del desarrollo de un software, en nuestro caso, por ejemplo, permitió automatizar la generación de casos de prueba funcionales.