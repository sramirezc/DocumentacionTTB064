\begin{UseCase}{CU 5}{Gestionar casos de uso}
	{
		Este caso de uso permite al analista visualizar los registros de los casos de uso del sistema. También permite 
		al actor acceder a las operaciones de registro, consulta, modificación y eliminación de un caso de uso.
	}
	%\UCitem{\DONEUC}{Edición}
	\UCitem{Versión}{1.0}
	\UCccsection{Administración de Requerimientos}	
	\UCitem{Autor}{Natalia Giselle Hernández Sánchez}	
	\UCccitem{Evaluador}{}
	\UCitem{Operación}{Administración}
	\UCccitem{Prioridad}{Alta}
	\UCccitem{Complejidad}{Media}
	\UCccitem{Volatilidad}{Alta}
	\UCccitem{Madurez}{Baja}
	\UCitem{Estatus}{Terminado}
	\UCitem{Fecha del último estatus}{}
	\UCsection{Atributos}
	\UCitem{Actor}{\cdtRef{actor:liderAnalisis}{Líder de análisis}, \cdtRef{actor:analista}{Analista}}
	\UCitem{Propósito}{
		Permitir al actor gestionar los casos de uso registrados en la herramienta.
	}
	\UCitem{Entradas}{
		Ninguna
	}
	\UCitem{Salidas}{
		\begin{UClist}
			\UCli \cdtRef{Modulo:Clave}{Clave del módulo}: \ioObtener.
			\UCli \cdtRef{Modulo:Nombre}{Nombre del módulo}: \ioObtener.
			\UCli \cdtRef{Modulo:Descripcion}{Descripción del módulo}: \ioObtener.
			\UCli \cdtRef{CasodeUso}{Caso de uso}: \ioTabla{\cdtRef{Elemento:Clave}{Clave}, \cdtRef{Elemento:Numero}{Número}, \cdtRef{Elemento:Nombre}{Nombre} y el \cdtRef{Estado_Elemento}{Estado} }{de los casos de uso}.
		\end{UClist}
	}
	
	\UCitem{Mensajes}{
		\begin{UClist}
			\UCli \cdtIdRef{MSG2}{No existe información}: Se muestra en la pantalla \cdtIdRef{CU 5}{Gestionar casos de uso} cuando no existen casos de uso registrados.
		\end{UClist}
	}

	\UCitem{Precondiciones}{
		Ninguna
	}
	
	\UCitem{Postcondiciones}{
		\begin{UClist}
			\UCli Se podrá solicitar el registro de un caso de uso por medio del caso de uso \cdtIdRef{CU 5.1}{Registrar caso de uso}.
			\UCli Se podrá solicitar la modificación de un caso de uso por medio del caso de uso \cdtIdRef{CU 5.2}{Modificar caso de uso}.
			\UCli Se podrá solicitar la eliminación de un caso de uso por medio del caso de uso \cdtIdRef{CU 5.3}{Eliminar caso de uso}.
		\end{UClist}
	}

	\UCitem{Errores}{
		Ninguno
	}

	\UCitem{Tipo}{
		Primario
	}
\end{UseCase}
%-------------------------------------------------------%trayectoria Principal-----------------------------------------------
 \begin{UCtrayectoria}
    \UCpaso[\UCactor] Solicita gestionar los casos de uso presionando el botón \btnCU de la pantalla \cdtIdRef{Gestionar módulos}.
    \UCpaso[\UCsist] Busca la información de los casos de uso en cualquier estado. \refTray{A}
    \UCpaso[\UCsist] Muestra la información de los casos de uso en la pantalla \cdtIdRef{IU 5}{Gestionar casos de uso}. 
    \UCpaso[\UCactor] Gestiona los casos de uso a través de los botones: \btnRevisar, \btnNuevo, \btnConsulta, \btnEditar y \btnEliminar. \label{cu5:gestionaCU}
 \end{UCtrayectoria}
 
 \begin{UCtrayectoriaA}[Fin del caso de uso]{A}{No existen registros de casos de uso.}
    \UCpaso[\UCsist] Muestra el mensaje \cdtIdRef{MSG2}{No existe información} en pantalla \cdtIdRef{IU 5}{Gestionar casos de uso} 
    para indicar que no hay registros de casos de uso para mostrar.
 \end{UCtrayectoriaA}
 

\subsection{Puntos de extensión}

\UCExtensionPoint{El actor requiere registrar un caso de uso}
	{Paso \ref{cu5:gestionaCU}}
	{\cdtIdRef{CU 5.1}{Registrar caso de uso}}
\UCExtensionPoint{El actor requiere modificar un caso de uso}
	{Paso \ref{cu5:gestionaCU}}
	{\cdtIdRef{CU 5.2}{Registrar caso de uso}}	
\UCExtensionPoint{El actor requiere eliminar un caso de uso}
	{Paso \ref{cu5:gestionaCU}}
	{\cdtIdRef{CU 5.3}{Eliminar caso de uso}}
\UCExtensionPoint{El actor requiere revisar un caso de uso}
	{Paso \ref{cu5:gestionaCU}}
	{\cdtIdRef{CU 5.5}{Revisar caso de uso}}
  