\subsection{IU 5 Gestionar casos de uso}
\subsubsection{Objetivo}
	
	En esta pantalla el actor puede visualizar algunos atributos de los casos de uso y las operaciones disponibles de acuerdo su estado.

\subsubsection{Diseño}

    En la figura ~\ref{IU 5} se muestra la pantalla ``Gestionar casos de uso'', por medio de la cual 
    se podrán gestionar los casos de uso a través de una tabla.
    El actor podrá podrá solicitar la revisión, el registro, la modificación y la eliminación de un caso de uso mediante los botones
    \btnRevisar, \cdtButton{Registrar}, \btnConsulta, \btnEditar y \btnEliminar respectivamente. \\
    
    Los botones correspondientes a las opciones mencionadas para cada uno de los registros aparecerán en la tabla dependiendo del estado
    del caso de uso y el rol del actor, con base en la regla de negocio \cdtIdRef{RN9}{Operaciones disponibles de casos de uso}.
    	
    \begin{itemize}
	    \item {\bf En registro:} Los elementos en este estado no se muestran en la tabla de gestión.
	    \item {\bf Edición:} Para este estado se muestra el botón \btnConsulta, si el actor es el que lo está editando o es el líder de análisis además se mostrarán los botones \btnEditar, \btnTray, \btnExt.
            \item {\bf Terminado:} Para este estado se muestran los botones \btnConsulta \btnRevisar \btnEliminar, si el actor es el líder de análisis además se mostrarán los botones \btnEditar, \btnTray, \btnExt.
            \item {\bf Revisión:} Para este estado se muestra el botón \btnConsulta, si el actor es el que lo está revisando además se mostrará el botón \btnRevisar.
            \item {\bf Liberado:} Para este estado se muestra el botón \btnConsulta, si el actor es el líder de análisis además se mostrarán los botones \btnEditar, \btnTray, \btnExt.
    \end{itemize}


    \IUfig[.9]{cu5/images/iu.png}{IU 5}{Gestionar casos de uso}


\subsubsection{Comandos}
\begin{itemize}
	\item \cdtButton{Registrar}: Permite al actor solicitar el registro de un caso de uso, dirige a la pantalla \cdtIdRef{IU 5.1}{Registrar caso de uso}.
	\item \btnEditar[Modificar]: Permite al actor solicitar la modificación de un caso de uso, dirige a la pantalla \cdtIdRef{IU 5.2}{Modificar caso de uso}.
	\item \btnTray[Gestionar Trayectorias]: Permite al actor gestionar las trayectorias del caso de uso, dirige a la pantalla \cdtIdRef{IU 5.1.1}{Gestionar trayectorias}.
	\item \btnExt[Gestionar Puntos de extensión]: Permite al actor gestionar los puntos de extensión del caso de uso, dirige a la pantalla \cdtIdRef{IU 5.1.4}{Gestionar puntos de extensión}.
	\item \btnEliminar[Eliminar]: Permite al actor solicitar la eliminación de un caso de uso, dirige a la pantalla \cdtIdRef{IU 5.3}{Eliminar caso de uso}.
	\item \btnConsulta[Consultar]: Permite al actor solicitar la consulta de un caso de uso, dirige a la pantalla \cdtIdRef{IU 5.4}{Consultar caso de uso}.
	\item \btnRevisar[Revisar]: Permite al actor solicitar la revisión de un caso de uso, dirige a la pantalla \cdtIdRef{IU 5.5}{Revisar caso de uso}.
\end{itemize}

\subsubsection{Mensajes}

	
\begin{description}
	\item[\cdtIdRef{MSG2}{No existe información}:] Se muestra en la pantalla \cdtIdRef{IU 5}{Gestionar casos de uso} cuando no existen casos de uso registrados.
\end{description}
