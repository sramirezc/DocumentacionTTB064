En este capítulo se especificarán las amenazas al pĺan de proyecto. Para clasificar los riesgos, se utilizará la categorización propuesta por
Charette la cual propone tres categorías de riesgos:
\begin{itemize}
 \item {\bf Conocidos.} Son aquellos que pueden descubrirse después de una evaluación cuidadosa del plan del proyecto, de los requerimientos del sistma y del 
	entorno de desarrollo del sistema.
 \item {\bf Predecibles.} Pueden definirse de acuerdo a la experiencia de proyectos anteriores.
 \item {\bf Impredecibles.} Son aquellos riesgos que difícilmente se pueden identificar por adelantado.
\end{itemize}

\section{Tabla de riesgos}
A continuación se detallan los riesgos con base en las categorías explicadas anteriormente, es importante mencionar que los riesgos impredecibles no
se mencionan debido a la naturaleza de los mismos.\\

En la primer columna se describe el riesgo, en la segunda columna se indica a qué categoría pertenece, la tercer columna indica la 
probabilidad que se percibe (valoración colectiva del equipo de trabajo), la cuarta columna indica el nivel de impacto de acuerdo a los siguientes valores:

\begin{enumerate}
 \item Catastrófico
 \item Crítico
 \item Marginal 
 \item Despreciable
\end{enumerate}

Los riesgos de alta probabilidad y alto impacto se ubican en
la parte superior de la tabla y los riesgos de baja probabilidad se ubican en el fondo. Esto logra
una priorización de riesgos.\\

Posibles riesgos:
\begin{itemize}
 \item No terminar el proyecto debido al tamaño del mismo.
 \item 
\end{itemize}




\section{Riesgos conocidos}
\subsection{Genéricos}
\begin{enumerate}
 \item El tamaño del producto esperado crece debido al cambio constante de requerimientos
\end{enumerate}

\subsection{Específicos}


\section{Riesgos predecibles}
\subsection{Genéricos}
\subsection{Específicos}

