En este capítulo se especificarán los riesgos del desarrollo del prototipo, así como aquellos
que pueden surgir al utilizarlo, también la estrategia de reducción que se lleva a cabo. 


% Para clasificar los riesgos, se utilizará la categorización propuesta por
% Charette la cual propone tres categorías de riesgos:
% \begin{itemize}
%  \item {\bf Conocidos.} Son aquellos que se relacionan directamente con el negocio del sistema.
%  \item {\bf Predecibles.} Pueden definirse de acuerdo a la experiencia de proyectos anteriores.
%  \item {\bf Impredecibles.} Son aquellos riesgos que difícilmente se pueden identificar por adelantado.
% \end{itemize}

\section{Tabla de riesgos}
La siguiente tabla muestra los riesgos detectados en el análisis del prototipo.
En la primer columna se describe el riesgo, en la segunda columna se indica la 
probabilidad que se percibe (valoración colectiva del equipo de trabajo), la tercer columna indica el nivel de impacto de acuerdo a los siguientes valores:

\begin{enumerate}
 \item Catastrófico
 \item Crítico
 \item Marginal 
 \item Despreciable
\end{enumerate}

La última columna indica la estrategia de reducción del riesgo. Los riesgos de alta probabilidad y alto impacto deben ser atendidos con mayor prioridad.\\

\clearpage
\begin{riesgos}
%%Referentes al desarrollo del proyecto
	\riesgo{La curva de aprendizaje de las tecnologías impacta en el tiempo de desarrollo.}{90 \%}{1} 
		{Se implementarán los módulos más representativos del prototipo con la finalidad de demostrar la funcionalidad.}
	\riesgo{La duración del semestre no es suficiente para concluir el desarrollo del prototipo.}{90 \%}{1} 
		{Se implementarán los módulos más representativos del prototipo con la finalidad de demostrar la funcionalidad.}
	\riesgo{El constante cambio de los requerimientos del sistema retardan las demás etapas del proceso de software del prototipo.}{90 \%}{1} 
		{Se limitarán los casos de uso a implementar y se evaluará la posibilidad de un incremento posterior.}
%%Referentes al uso del proyecto
	\riesgo{El usuario ha olvidado su nombre de usuario o su contraseña.}{90 \%}{4} 
		{El nombre de usuario es el correo electrónico del usuario y en la sección ``Recuperar contraseña'' se puede solicitar el envío de una contraseña nueva.}
	\riesgo{El usuario desea recuperar un elemento eliminado.}{80 \%}{4}
		{Antes de que se elimine definitivamente un elemento, el sistema alerta al usuario para que esté consiente de que no podrá recuperarlo posteriormente.}
	\riesgo{El usuario desea revertir las modificaciones realizadas a un elemento.}{80 \%}{4}
		{Antes de guardar los cambios definitivamente, el sistema alerta al usuario para que esté consiente de que no se podrán revertir los cambios.}
	\riesgo{El documento de análisis no se puede generar debido a que faltan elementos por registrar.}{80 \%}{4}
		{Cuando el analista desee generar el documento de análisis y hay elementos pendientes de registrar, se notificará al usuario que es posible que el documento 
		no se genere completo.}
	\riesgo{Alguno de los analistas desea consultar la información de otro analista.}{70 \%}{3}
		{Debido al enfoque del proyecto, no es prioridad en este momento la comunicación directa entre colaboradores.}
	\riesgo{Los analistas registran dos elementos idénticos semánticamente pero con distinta clave.}{60 \%}{2}
 		{Todos los elementos pueden ser eliminados cuando no tienen asociado ningún otro elemento, la comunicación entre analistas es un punto clave para evitar esta situación de duplicidad.}
 	\riesgo{El analista necesita un tipo de regla de negocio que no está en la lista.}{80 \%}{2}
 		{Es posible registrar un tipo de regla de negocio que no esté en la lista mediante la opción ``Otro'' y describiendola en lenguaje natural.}
 	 \riesgo{El usuario desea agregar más datos de los que se solicitan para generar el documento de análisis.}{100 \%}{2}
 		{El usuario podrá descargar el documento de análisis en formato tex para modificar lo que requiera. El usuario debe tener experiencia utilizando Latex.}
 	\riesgo{Un analista no experimentado realiza cambios en elementos que no son parte de su trabajo y sobreescribe información importante.}{80 \%}{2}
		{Se registra el responsable y la fecha de actualización.}
	\riesgo{El analista desea realizar algún cambio a un elemento que ya está liberado.}{40 \%}{1}
		{Se contempló que cada elemento pasa por una serie de revisiones por parte del líder de análisis que debería evitar esta situación.}
	
% 	\riesgo{Riesgo}{probabilidad}{impacto}
% 		{estrategia}
\end{riesgos}



