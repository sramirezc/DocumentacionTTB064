\begin{UseCase}{CU 6.1.1}{Registrar acción}
	{
		Una acción es una operación que se puede solicitar desde una pantalla, regularmente son botones u opciones de un menú. Este caso de uso permite al analista registrar la información de una acción. 
	}
	
	\UCitem{Actor}{\cdtRef{actor:liderAnalisis}{Líder de análisis}, \cdtRef{actor:analista}{Analista}}
	\UCitem{Propósito}{
		Registrar la información de una acción.
	}
	\UCitem{Entradas}{
		\begin{UClist}
			\UCli \cdtRef{Accion:Nombre}{Nombre}: \ioEscribir.
 			\UCli \cdtRef{Accion:Descripcion}{Descripción}: \ioEscribir.
 			\UCli \cdtRef{Accion:Imagen}{Imagen}: \ioArchivo.
		\end{UClist}
	}
	\UCitem{Salidas}{
		Ninguna
	}
	
	\UCitem{Mensajes}{
		Ninguno
	}

	\UCitem{Precondiciones}{
		Ninguna
	}
	
	\UCitem{Postcondiciones}{
		\begin{UClist}
			\UCli La acción se agregará a la tabla de la pantalla \cdtIdRef{IU 6.1}{Registrar pantalla} o \cdtIdRef{IU 6.2}{Modificar pantalla}.
		\end{UClist}
	}

	\UCitem{Errores}{
		\begin{UClist}
			\UCli \cdtIdRef{MSG4}{Dato obligatorio}: Se muestra en la pantalla \cdtIdRef{IU 6.1.1}{Registrar acción} cuando no se ha ingresado un dato marcado como obligatorio.
			\UCli \cdtIdRef{MSG5}{Dato incorrecto}: Se muestra en la pantalla \cdtIdRef{IU 6.1.1}{Registrar acción} cuando el tipo de dato ingresado no cumple con el tipo de dato solicitado en el campo.
			\UCli \cdtIdRef{MSG6}{Longitud inválida}: Se muestra en la pantalla \cdtIdRef{IU 6.1.1}{Registrar acción} cuando se ha excedido la longitud de alguno de los campos.
			\UCli \cdtIdRef{MSG7}{Registro repetido}: Se muestra en la pantalla \cdtIdRef{IU 6.1.1}{Registrar acción} cuando se registre una acción con un nombre que ya esté asociado a la pantalla.
			\UCli \cdtIdRef{MSG23}{Caracteres inválidos}: Se muestra en la pantalla \cdtIdRef{IU 6.1.1}{Registrar acción} cuando el nombre de la acción contiene un caracter no válido.
			\UCli \cdtIdRef{MSG24}{Formato de archivo incorrecto}: Se muestra en la pantalla \cdtIdRef{IU 6.1.1}{Registrar acción} cuando la imagen de la acción no cumpla con el formato especificado.
			\UCli \cdtIdRef{MSG25}{Se ha excedido el tamaño del archivo}: Se muestra en la pantalla \cdtIdRef{IU 6.1.1}{Registrar acción} cuando la imagen de la acción exceda el tamaño especificado.
		\end{UClist}
	}

	\UCitem{Tipo}{
		Secundario, extiende de los casos de uso \cdtIdRef{CU 6.1}{Registrar pantalla} y \cdtIdRef{CU 6.2}{Modificar pantalla}.
	}
\end{UseCase}
%-------------------------------------------------------%trayectoria Principal-----------------------------------------------
 \begin{UCtrayectoria}
    \UCpaso[\UCactor] Solicita registrar una acción oprimiendo el botón \cdtButton{Registrar} de la pantalla \cdtIdRef{IU 6.1}{Registrar pantalla} o \cdtIdRef{IU 6.2}{Modificar pantalla}.
    \UCpaso[\UCsist] Muestra la pantalla \cdtIdRef{IU 6.1.1}{Registrar acción} en la cual se realizará el registro de la acción. 
    \UCpaso[\UCactor] Ingresa la información de la acción en la pantalla \cdtIdRef{IU 6.1.1}{Registrar acción}. \label{cu6.1.1:ingresaDatos}
    \UCpaso[\UCactor] Solicita guardar la acción oprimiendo el botón \cdtButton{Aceptar} de la pantalla \cdtIdRef{IU 6.1.1}{Registrar acción}. \refTray{A}
    \UCpaso[\UCsist] Verifica que el actor ingrese todos los campos obligatorios con base en la regla de negocio  \cdtIdRef{RN8}{Datos obligatorios}. \refTray{B}
    \UCpaso[\UCsist] Verifica que el nombre de la acción no se encuentre asociado a la pantalla con base en la regla de negocio \cdtIdRef{RN6}{Unicidad de nombres}. \refTray{C}
    \UCpaso[\UCsist] Verifica que los datos requeridos sean proporcionados correctamente como se especifica en la regla de negocio \cdtIdRef{RN7}{Información correcta}. \refTray{D} \refTray{E} \refTray{F} \refTray{G}
    \UCpaso[\UCsist] Verifica que el nombre no contenga caracteres inválidos con base en la regla de negocio \cdtIdRef{RN2}{Nombres de los elementos}. \refTray{H}
    
    \UCpaso[\UCsist] Agrega la acción a la tabla de la pantalla \cdtIdRef{IU 6.1}{Registrar pantalla} o \cdtIdRef{IU 6.2}{Modificar pantalla}.
 \end{UCtrayectoria}
 
 %----------------------------------------------------------%trayectoria A---------------------------------------------------- 
 \begin{UCtrayectoriaA}[Fin del caso de uso]{A}{El actor desea cancelar la operación.}
    \UCpaso[\UCactor] Solicita cancelar la operación oprimiendo el botón \cdtButton{Cancelar} de la pantalla \cdtIdRef{IU 6.1.1}{Registrar acción}.
    \UCpaso[\UCsist] Muestra la pantalla donde se solicitó la operación.
 \end{UCtrayectoriaA}
  %----------------------------------------------------------%trayectoria B---------------------------------------------------- 
 \begin{UCtrayectoriaA}{B}{El actor no ingresó algún dato marcado como obligatorio.}
    \UCpaso[\UCsist] Muestra el mensaje \cdtIdRef{MSG4}{Dato obligatorio} y señala el campo que presenta el error en la pantalla 
	    \cdtIdRef{IU 6.1.1}{Registrar acción}, indicando al actor que el dato es obligatorio.
    \UCpaso[] Continúa con el paso \ref{cu6.1.1:ingresaDatos} de la trayectoria principal.
 \end{UCtrayectoriaA}
 %----------------------------------------------------------%trayectoria C---------------------------------------------------- 
 \begin{UCtrayectoriaA}{C}{El actor ingresó un nombre de acción que ya está asociado a la pantalla.}
    \UCpaso[\UCsist] Muestra el mensaje \cdtIdRef{MSG7}{Registro repetido} y señala el campo que presenta la duplicidad en la pantalla 
	    \cdtIdRef{IU 6.1.1}{Registrar acción}, indicando al actor que existe una acción con el mismo nombre.
    \UCpaso[] Continúa con el paso \ref{cu6.1.1:ingresaDatos} de la trayectoria principal.
 \end{UCtrayectoriaA}
 
 %----------------------------------------------------------%trayectoria D----------------------------------------------------  
 \begin{UCtrayectoriaA}{D}{El actor proporciona un dato que excede la longitud máxima.}
    \UCpaso[\UCsist] Muestra el mensaje \cdtIdRef{MSG5}{Se ha excedido la longitud máxima del campo} y señala el campo que excede la 
    longitud en la pantalla \cdtIdRef{IU 6.1.1}{Registrar acción}, para indicar que el dato excede el tamaño máximo permitido.
    \UCpaso[] Continúa con el paso \ref{cu6.1.1:ingresaDatos} de la trayectoria principal.
 \end{UCtrayectoriaA}
 %----------------------------------------------------------%trayectoria E---------------------------------------------------- 
 \begin{UCtrayectoriaA}{E}{El actor ingresó un tipo de dato incorrecto.}
    \UCpaso[\UCsist] Muestra el mensaje \cdtIdRef{MSG4}{Formato incorrecto} y señala el campo que presenta el dato inválido en la 
    pantalla \cdtIdRef{IU 6.1.1}{Registrar acción} para indicar que se ha ingresado un tipo de dato inválido.
    \UCpaso[] Continúa con el paso \ref{cu6.1.1:ingresaDatos} de la trayectoria principal.
 \end{UCtrayectoriaA}
  %----------------------------------------------------------%trayectoria F----------------------------------------------------  
 \begin{UCtrayectoriaA}{F}{El actor proporciona una imagen de formato incorrecto.}
    \UCpaso[\UCsist] Muestra el mensaje \cdtIdRef{MSG24}{Formato de archivo incorrecto} y señala el campo donde se solicita la imagen
    en la pantalla \cdtIdRef{IU 6.1.1}{Registrar acción}, para indicar que el formato es incorrecto.
    \UCpaso[] Continúa con el paso \ref{cu6.1.1:ingresaDatos} de la trayectoria principal.
 \end{UCtrayectoriaA}
 
 %----------------------------------------------------------%trayectoria G----------------------------------------------------  
 \begin{UCtrayectoriaA}{G}{El actor proporciona una imagen que excede el tamaño máximo.}
    \UCpaso[\UCsist] Muestra el mensaje \cdtIdRef{MSG25}{Se ha excedido el tamaño del archivo} y señala el campo donde se solicita la imagen
    en la pantalla \cdtIdRef{IU 6.1.1}{Registrar acción}, para indicar que el tamaño máximo ha sido rebasado.
    \UCpaso[] Continúa con el paso \ref{cu6.1.1:ingresaDatos} de la trayectoria principal.
 \end{UCtrayectoriaA}
 %----------------------------------------------------------%trayectoria H---------------------------------------------------- 
 \begin{UCtrayectoriaA}{H}{El actor ingresó un nombre con caracteres inválidos.}
    \UCpaso[\UCsist] Muestra el mensaje \cdtIdRef{MSG23}{Caracteres inválidos} y señala el campo que contiene los caracteres inválidos.
    \UCpaso[] Continúa con el paso \ref{cu6.1.1:ingresaDatos} de la trayectoria principal.
 \end{UCtrayectoriaA}
