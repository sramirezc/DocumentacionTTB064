%SUPERFICIES FORESTALES

%----------------------------------------------------------
\section{Entorno de trabajo}

    El entorno de trabajo es el medio por el cual el usuario interactúa con el sistema para poder gestionar la información 
    referente a las Áreas Naturales Protegidas, los instrumentos de evaluación, las categorías, los criterios, los predios y los eventos. En este capítulo se describe el comportamiento y los elementos que conforman el entorno de 
    trabajo del seipa, como son: la disposición de los elementos principales y comunes de las pantallas, los colores, la iconografía, componentes, etc. \bigskip

    \begin{objetivos}
      \item Describir las áreas principales del entorno de trabajo.
      \item Describir la iconografía utilizada en las pantallas.
      \item Describir el mapa de navegación del sistema.
      \item Describir los componentes principales de las pantallas, tales como: controles de entrada, datos obligatorios, separadores, tablas de resultados, entre otros.
    \end{objetivos}
\\\\\\\\\\\\\\\\
%----------------------------------------------------------

\subsection{Diseño}

  El diseño de las pantallas del sistema sigue un enfoque minimalista que permite a los usuarios trabajar sin gran dificultad y sin distracción. 
  Las pantallas son consistentes, ya que tienen un diseño homogéneo y cuentan con componentes comunes; la consistencia facilita al usuario la interacción
  con el sistema a medida que hace uso del mismo. En la figura~\ref{fig:entornoDeTrabajo} se muestran los elementos principales que conforman las pantallas del sistema, 
  dichos elementos se describen a continuación:

  \begin{figure}[ht!]
      \begin{center}
	  \fbox{\includegraphics[width=.8\textwidth]{images/pantallas/general/entornoTrabajo}}
	  \caption{Entorno de trabajo del sistema.}
	  \label{fig:entornoDeTrabajo}
      \end{center}
  \end{figure}
  
  	\begin{enumerate}
		\item {\bf Datos de Sesión:} Esta sección será visible solo cuando un actor ingrese al sistema. En ella se mostrarán los datos del actor que haya iniciado sesión, así como las opciones para el cierre de sesión y el 
		cambio de contraseña.
		
		\item {\bf Baner de la página:} Tiene la finalidad de mostrar la imagen institucional de la dependencia a la cual pertenece el sistema. Aquí se muestran los logotipos de la Secretaría del Medio Ambiente y del 
		Gobierno del Estado de México.
		
		\item {\bf Menú superior:} Muestra las opciones generales de navegación para los diferentes tipos de actores.
		
		\item {\bf Área de trabajo:} En esta sección los actores visualizarán los elementos que el sistema proporciona para la realización de las tareas contempladas en el mismo. Aquí se desplegarán formularios para 
		captura, tablas, imágenes, gráficas y demás elementos contenidos en el sistema. La documentación de las pantallas se explica a partir de esta sección.
		
		\item {\bf Pie de página:} En esta sección aparecen los datos de contacto de la  Unidad de Información, Planeación, Programación y Evaluación de la Secretaría del Medio Ambiente del Gobierno del Estado de México y 
		los del soporte técnico del sistema.
		
		\item {\bf Información legal:} Muestra una leyenda con información legal referente a la propiedad y uso del sistema.
			
	\end{enumerate}


%----------------------------------------------------------
\subsection{Pantalla de bienvenida}
\label{ch:Interaccion:PantallaBienvenida}

En la figura \ref{fig:inicio} se muestra la pantalla de bienvenida, en la cual se mostrará el nombre completo del usuario, así como una leyenda de bienvenida.
\begin{figure}[htbp!]
    \begin{center}
	\fbox{\includegraphics[width=\textwidth]{images/inicio.png}}
	\caption{Pantalla de bienvenida.}
	\label{fig:inicio}
	\cdtLabel{fig:inicio}{}
    \end{center}
\end{figure}

%----------------------------------------------------------
\subsection{Datos de sesión}
\label{ch:Interaccion:DatosSesion}

En la figura \ref{fig:sesion} se muestra la sección ``Datos de Sesión'', en la cual se mostrará el nombre completo del usuario, así como las opciones para cerrar sesión y cambiar contraseña.
\begin{figure}[htbp!]
    \begin{center}
	\fbox{\includegraphics[width=\textwidth]{images/datosSesion.png}}
	\caption{Datos de sesión}
	\label{fig:sesion}
	\cdtLabel{fig:sesion}{}
    \end{center}
\end{figure}

\subsection{Iconografía}

  En las pantallas se utilizan diversos íconos para denotar las operaciones que el actor puede realizar sobre el sistema. Los íconos se diseñaron con base en los perfiles de actor y en la operación que podrán realizar 
  después del evento {\it clic} sobre ellos.  A continuación se describe la la funcionalidad de cada uno de ellos:\\\\

  \begin{UClist}
    \UCli \botOk Se utiliza para aprobar un predio o evento, para que sea publicado.
    \UCli \botEdit Se utiliza para modificar la información de un predio o de una cuenta de usuario.
    \UCli \botReg Se utiliza para solicitar la modificación de los datos del predio por parte del cartógrafo.
    \UCli \botGeoref Se utiliza para georreferenciar una entidad en el sistema.
    \UCli \botErrorInfo Se utiliza para notificar al responsable que la información proporcionada contiene errores.
    \UCli \botCalendar Se utiliza para ingresar una fecha por medio de un calendario.
    \UCli \botLock Se utiliza para desbloquear una cuenta de usuario.
    \UCli \botReestCont Se utiliza para restablecer la contraseña de una cuenta de usuario. 
  \end{UClist}

%----------------------------------------------------------
\subsection{Componentes utilizados}

  \subsubsection{Pantalla emergente}
    Algunos mensajes se muestran en pantallas emergentes, las cuales cuentan con dos botones: \cdtButton{Aceptar} y \cdtButton{Cancelar}, que permiten confimar la acción que se muestra en el mensaje.
    En la figura \ref{fig:pantallaEmergente} se muestra una pantalla emergente de ejemplo.
%       \begin{figure}[htbp!]
% 	\begin{center}
% 	  \fbox{\includegraphics[width=.7\textwidth]{images/pantallas/general/IUP9.png}}
% 	  \caption{Pantalla emergente.}
% 	  \label{fig:pantallaEmergente}
% 	  \cdtLabel{fig:pantallaEmergente}{}
% 	\end{center}
%       \end{figure}
  
    \subsubsection{Tabla de selección de múltiple}
      Este componente permite realizar una selección múltiple de valores, los cuales corresponden a las opciones disponibles en un catálogo. Es posible realizar búsquedas y seleccionar la cantidad de registros a mostrar por página, así como navegar el paginado con las opciones ``Anterior'' y ``Siguiente''.
      Muestra los botones \cdtButton{Aceptar} y \cdtButton{Cancelar}, que permiten confirmar la selección o cancelarla respectivamente.
      En la figura \ref{fig:tablaSeleccion} se muestra un ejemplo de este componente. 
%       \begin{figure}[htbp!]
% 	  \begin{center}
% 	      \fbox{\includegraphics[width=.7\textwidth]{images/pantallas/general/tablaSeleccion}}
% 	      \caption{Tabla de selección múltiple.}
% 	      \label{fig:tablaSeleccion}
% 	      \cdtLabel{fig:tablaSeleccion}{}
% 	  \end{center}
%       \end{figure}
  
  
\subsection{Organización}
En la parte superior de la pantalla en la interfaz de usuario, se mostrará un menú correspondiente a cada perfil que ingrese al sistema, el cual tendrá las opciones ``Inicio'' y ``Mapa'', además de las opciones a las que cada perfil tiene acceso. La opción ``Inicio'' dirige a la pantalla de bienvenida de sistema, mostrada en la figura~\ref{fig:inicio}. La opción ``Mapa'' dirige a la pantalla de administración de cartografía mostrada en la figura~\ref{IUM 1}. A continuación se describen los menús correspondientes a cada perfil en el sistema.

\subsubsection{Menú del Administrador}
En la figura~\ref{MN1} se muestra el menú de \cdtRef{actor:Administrador}{Administrador}, con las opciones:

    \begin{Citemize}
	    \item {\bf Inicio:} dirige a la pantalla de inicio.
	    \item {\bf Mapa:} dirige a la pantalla de administración de cartografía.
	    \item {\bf Administrar cuentas de usuario:} Permite acceder a la pantalla de Administración de cuentas de usuario, para visualizar el estado de las cuentas registradas en el sistema.
    \end{Citemize}

%     \IUfig[.7]{menus/administrador}{MN1}{Menú del Administrador}

\subsubsection{Menú del Usuario SMAGEM}

    En la figura~\ref{MN2} se muestran las opciones del menú superior que serán visibles para los actores \cdtRef{actor:usuarioSMAGEM}{Usuario SMAGEM} y \cdtRef{actor:Administrador}{Administrador}. Las opciones del menú se enlistan a continuación:

    \begin{Citemize}
	    \item Inicio
	    \item Mapa
    \end{Citemize}

%     \IUfig[.5]{menus/usuarioSMAGEM}{MN2}{Menú del Usuario SMAGEM}
    
    
\subsubsection{Menú del Cartógrafo del evento y Responsable del evento}

    En la figura~\ref{MN3} se muestran las opciones del menú superior que serán visibles para los actores \cdtRef{actor:cartografo-evento}{Cartógrafo del evento} y \cdtRef{actor:responsable-evento}{Responsable del evento}. Las opciones del menú se enlistan a continuación:

    \begin{Citemize}
      \item Inicio
      \item Mapa
      \item Evento
    \end{Citemize}

%     \IUfig[.5]{menus/cart_respEvento}{MN3}{Menú del Cartógrafo del evento y Responsable del evento}
    
\subsubsection{Menú del Cartógrafo del predio y Responsable del predio}

    En la figura~\ref{MN4} se muestran las opciones del menú superior que serán visibles para los actores \cdtRef{actor:cartografo-predios}{Cartógrafo del predio} y \cdtRef{actor:responsable-predios}{Responsable del predio}. Las opciones del menú se enlistan a continuación:

    \begin{Citemize}
      \item Inicio
      \item Mapa
      \item Predio
    \end{Citemize}

%     \IUfig[.5]{menus/cart_respPredio}{MN4}{Menú del Cartógrafo del predio y Responsable del predio}

\subsubsection{Menú del Cartógrafo del ANP}

    En la figura~\ref{MN5} se muestran las opciones del menú superior que serán visibles para el actor \cdtRef{actor:cartografoANP}{Cartógrafo del ANP}. Las opciones del menú se enlistan a continuación:

    \begin{Citemize}
	    \item Inicio
	    \item Mapa
	    \item ANP
    \end{Citemize}

%     \IUfig[.5]{menus/cartografoANP}{MN5}{Menú del Cartógrafo del ANP}

\subsubsection{Menú del Responsable del ANP}

    En la figura~\ref{MN6} se muestran las opciones del menú superior que serán visibles para el actor \cdtRef{actor:responsableANP}{Responsable del ANP}. Las opciones del menú se enlistan a continuación:

    \begin{Citemize}
      \item Inicio
      \item Mapa
      \item Instrumento de evaluación: permite acceder a las opciones:
	\begin{itemize}
	  \item Administrar instrumento de evaluación
	  \item Administrar indicadores
	  \item Aplicar instrumento de evaluación activo
	\end{itemize}
      \item ANP\\\\\\
    \end{Citemize}

%     \IUfig[.5]{menus/responsableANP}{MN6}{Menú del Responsable del ANP}

