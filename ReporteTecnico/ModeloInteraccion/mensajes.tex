%\DONE{} 
    En esta sección se describen los mensajes utilizados en el prototipo actual del sistema. Los mensajes se refieren a todos
    aquellos avisos que el sistema muestra al actor a través de la pantalla debido a diversas
    razones, por ejemplo: informar acerca de algún fallo en el sistema, notificar acerca de alguna operación importante sobre
    la información, etc.\\ %Los mensajes se clasifican en dos grupos principales: mensajes y registros.\\

%===========================================================removiendo los puntos generados
\subsection{Parámetros comunes}
    Cuando un mensaje es recurrente se parametrizan sus elementos, por ejemplo los mensajes: ``El {\em incendio} se registró correctamente'', ``El {\em pago de servicios ambientales hidrológicos} se registró 
    correctamente'', ``La {\em reforestación} se registró correctamente'', etcétera tienen una estructura similar, por lo tanto se utilizan {\em parámetros} para definir un mensaje 
    genérico que pueda utilizarse en todos los casos que se considere necesario.\\
    
    Los parámetros también se utilizan cuando la redacción del mensaje tiene datos que son introducidos por el actor o que dependen del resultado de la operación, por ejemplo: 
    ``El incendio  {\em 2014-003} ya se encuentra registrado''. En este caso la redacción se presenta parametrizada de la forma: ``La ENTIDAD1 VALOR1 ya se encuentra registrada'' y los 
    parámetros se describen de la siguiente forma:
    
    \begin{itemize}
	\item ENTIDAD1: Elemento del modelo estructural al que pertenece el ``VALOR1''.
	\item VALOR1: Identificador del incendio obtenido automáticamente por el sistema.
    \end{itemize}

    En el ejemplo anterior se hace referencia a VALOR1, es decir: {\em 2014-003} es el {\bf valor}  de la entidad {\bf incendio}. Cada mensaje lista los parámetros 
    que utiliza, sin embargo aquí se definen los más comunes a fin de simplificar la descripción de los mensajes:

    \begin{description}
	\item [ARTÍCULO:] Se refiere a un {\em artículo} el cual puede ser DETERMINADO (El $\mid$ La $\mid$ Lo $\mid$ Los $\mid$ Las) o INDETERMINADO (Un $\mid$ Una $\mid$ 
	Uno $\mid$ Unos $\mid$Unas) se aplica generalmente sobre una ENTIDAD, ATRIBUTO o VALOR.
	\item [CAMPO:] Se refiere a un campo del formulario. Por lo regular es el nombre de un atributo en una entidad.
	\item [CONDICIÓN:] Define una expresión booleana cuyo resultado deriva en {\em falso} o {\em verdadero} y suele ser la causa del mensaje.
	\item [DATO:] Es un sustantivo y generalmente se refiere a un atributo de una entidad descrito en el modelo estructural del negocio, por ejemplo: número de incendio,
	brigada de apoyo del incendio, uso de suelo autorizado del predio, etc. %ATRIBUTO
	\item [ENTIDAD:] Es un sustantivo y generalmente se refiere a una entidad del modelo estructural del negocio, por ejemplo: incendio, pago por servicios ambientales hidrológicos, reforestación, etc.
	\item [OPERACIÓN:] Se refiere a una acción que se debe realizar sobre los datos de una o varias entidades. Por ejemplo: registrar, eliminar, actualizar, etc. Comúnmente 
	la OPERACIÓN va concatenada con el sustantivo, por ejemplo: Registro de un nuevo beneficio, registro de una actividad, eliminar una tarea, etc.
	\item [VALOR:] Es un sustantivo concreto y generalmente se refiere a un valor en específico. Por ejemplo: ``2014-003'', que es un valor concreto del DATO de la 
	ENTIDAD ``incendio''.
	\item [TAMAÑO:] Es el tamaño del atributo de una entidad, el cual se encuentra definido en el diccionario de datos.
	\item [MOTIVO:] Es una explicación acerca de la operación que se pretende realizar.
    \end{description}


\subsection{Mensajes a través de la pantalla}

%===========================  MSG1 ==================================
\begin{mensaje}{MSG1}{Operación realizada exitosamente}{Confirmación}
    \item[Estatus:] Pendiente
    \item[Objetivo:] Notificar al actor que la acción solicitada fue realizada exitosamente.
    \item[Redacción:] DETERMINADO ENTIDAD VALOR ha sido OPERACIÓN exitosamente.
    \item[Parámetros:] El mensaje se muestra con base en los siguientes parámetros:
    \begin{Citemize}
	\item DETERMINADO ENTIDAD: Es un artículo determinado más el nombre de la entidad sobre la cual se realizó la acción.
	\item VALOR: Es el valor asignado al atributo de la entidad, generalmente es el nombre o la clave.
	\item OPERACIÓN: Es la acción que el actor solicitó realizar.
    \end{Citemize}
    \item[Ejemplo:] {\em La escuela 15DPR2497-K} ha sido {\em registrada} exitosamente.
\end{mensaje}

%============================== MSG2 =================================
\begin{mensaje}{MSG2}{No existe información registrada por el momento}{Notificación}
    \item[Estatus:] Pendiente
    \item[Objetivo:] Notificar al actor que aún no existe información registrada en el sistema.
    \item[Redacción:] Aún no existen registros de ENTIDAD en el sistema.
    \item[Parámetros:] El mensaje se muestra con base en los siguientes parámetros:
    \begin{Citemize}
	\item ENTIDAD: Especifica la entidad sobre la que se está realizando la consulta.
    \end{Citemize}
    \item[Ejemplo:] Aún no existen registros de {\em escuelas} en el sistema.
\end{mensaje}

% %============================== MSG3 ========================================
 \begin{mensaje}{MSG3}{Confirmación de envío de información}{Notificación}
 	\item[Estatus:] Pendiente
 	\item[Objetivo:] Indicar al usuario que es necesario que verifique la información antes de ser enviada debido a que no se podrán realizar modificaciones posteriormente.
 	\item[Redacción:] ¿Está seguro que los datos proporcionados son correctos? Una vez que envíe la solicitud no podrá modificarlos.
 \end{mensaje}

%===========================  MSG4 ==================================
\begin{mensaje}{MSG4}{No se encontró información sustantiva}{Error}
	\item[Estatus:] Pendiente
	\item[Objetivo:] Informar al actor que no se puede ejecutar la operación debido a que el sistema no tiene información base.
	\item[Redacción:] Error, no se encontró información registrada en DETERMINADO ENTIDAD[, DETERMINADO-2 ENTIDAD-2, ... , DETERMINADO-N ENTIDAD-N]. Favor de contactar al administrador del sistema.
	\item[Parámetros:] El mensaje se muestra con base en los siguientes parámetros:
	\begin{Citemize}
		\item DETERMINADO ENTIDAD: Es un artículo determinado más el(los) nombre(s) de la(s) entidad(es) que no tienen información.
	\end{Citemize}
	\item[Ejemplo:] Error, no se encontró información registrada en {\em el catalogo de escuelas}. Favor de contactar al administrador del sistema.
\end{mensaje}

%===========================  MSG5 ==================================
\begin{mensaje}{MSG5}{Falta un dato requerido para efectuar la operación solicitada}{Error}
	\item[Estatus:] Pendiente
	\item[Objetivo:] Indicar al actor la omisión de algún \cdtRef{gls:requerido}{dato requerido} para ejecutar la operación solicitada.
	\item[Redacción:] El VARIABLE CAMPO es requerido para realizar la operación.
	\item[Parámetros:] El mensaje se muestra con base en los siguientes parámetros:
	\begin{Citemize}
		\item VARIABLE: Indica la etiqueta ``campo'' o ``elemento'' según corresponda. Se definirá la etiqueta campo cuando se refiera a una omisión de un campo en el formulario. Se definirá la etiqueta elemento cuando la omisión que se presenta sea un elemento en el formulario. 		
		\item CAMPO: Indica el campo del formulario que presenta el error de omisión.
	\end{Citemize}
	\item[Ejemplo:] El campo {\em clave de centro de trabajo} es requerido para realizar la operación.
\end{mensaje}

%===========================  MSG6 ==================================
\begin{mensaje}{MSG6}{Formato incorrecto}{Error}
	\item[Estatus:] Pendiente
	\item[Objetivo:] Indicar al actor que uno de los campos ingresados en el formulario no cumple con el tipo de dato definido en el diccionario de datos.
	\item[Redacción:] El valor del campo CAMPO es incorrecto, favor de introducir un dato válido.
	\item[Parámetros:] El mensaje se muestra con base en los siguientes parámetros:
	\begin{Citemize}
		\item CAMPO: Indica el campo del formulario que presenta el error de formato.
	\end{Citemize}
	\item[Ejemplo:] El valor del campo {\em número de docentes femeninos} es incorrecto, favor de introducir un dato válido.
\end{mensaje}

%===========================  MSG7 ==================================
\begin{mensaje}{MSG7}{Se ha excedido la longitud máxima del campo}{Error}
	\item[Estatus:] Pendiente
	\item[Objetivo:] Indicar al actor que el valor ingresado en uno de los campos del formulario no cumple o rebasa la longitud especificada en el diccionario de datos.
	\item[Redacción:] La longitud del campo CAMPO es incorrecta, favor de introducir un dato válido. La longitud debe ser menor a TAMAÑO.
	\item[Parámetros:] El mensaje se muestra con base en los siguientes parámetros:
	\begin{Citemize}
		\item CAMPO: Indica el campo del formulario que presenta el error de longitud.
		\item TAMAÑO: Especifica la longitud del atributo definido en el diccionario de datos.
	\end{Citemize}
	\item[Ejemplo:] La longitud del \textit{clave de centro de trabajo} es incorrecta, favor de introducir un dato válido. La longitud debe ser igual a \textit{10}.
\end{mensaje}

%============================== MSG8 =================================
\begin{mensaje}{MSG8}{Registro repetido}{Error}
\item[Estatus:] Pendiente
\item[Objetivo:] Informar al actor que ya existe un registro con los mismos datos.
\item[Redacción:] Error, ya se OPERACIÓN INDETERMINADO ENTIDAD con el mismo valor en el atributo ATRIBUTO, favor de verificarlo.
\item[Parámetros:] El mensaje se muestra con base en los siguientes parámetros:
	\begin{Citemize}
		\item INDETERMINADO ENTIDAD: Es un artículo indeterminado más el nombre de la entidad sobre la cual se desea realizar la operación.
		\item OPERACIÓN: Es la acción que el actor solicita realizar.
		\item ATRIBUTO: Es un atributo de la entidad.
	\end{Citemize}
	\item[Ejemplo:] Error, ya se {\em registró una escuela} con el mismo valor en el atributo {\em clave de centro de trabajo}, favor de verificarlo.
\end{mensaje}

%============================== MSG9 =================================
\begin{mensaje}{MSG9}{Confirmar la modificación de un registro}{Notificación}
    \item[Estatus:] Pendiente
    \item[Objetivo:] Indicar al actor que está a punto de modificar un registro y se necesita su aprobación para ello.
    \item[Redacción:] Se modificará INDETERMINADO ENTIDAD. ¿Está seguro que desea continuar? Al modificar esta información se perderá la información previa.
    \item[Parámetros:] El mensaje se muestra con base en los siguientes parámetros:
    \begin{Citemize}
	\item INDETERMINADO ENTIDAD: Es un artículo indeterminado más el nombre de la entidad sobre la cual se realiza la acción.
    \end{Citemize}
    \item[Ejemplo:] Se modificará {\em la escuela}. ¿Está seguro que desea continuar? Al modificar esta información se perderá la información previa.
\end{mensaje}

%============================== MSG10 =================================
\begin{mensaje}{MSG10}{Confirmar la eliminación de un registro}{Confirmación}
\item[Estatus:] Terminado
\item[Objetivo:] Indicar al actor que está a punto de eliminar un registro y se necesita su aprobación para ello.
\item[Redacción:] Se eliminará INDETERMINADO ENTIDAD del sistema. ¿Está seguro que desea continuar? Al eliminar esta información ya no podrá recuperarla posteriormente.
\item[Parámetros:] El mensaje se muestra con base en los siguientes parámetros:
	\begin{Citemize}
		\item INDETERMINADO ENTIDAD: Es un artículo indeterminado más el nombre de la entidad sobre la cual se realiza la acción.
	\end{Citemize}
	\item[Ejemplo:] Se eliminará un {\em ANP} del sistema. ¿Está seguro que desea continuar? Al eliminar esta información ya no podrá recuperarla posteriormente.
\end{mensaje}

%============================== MSG11 =================================
\begin{mensaje}{MSG11}{Operación no permitida por existencia de comité}{Error}
\item[Estatus:] Terminado
\item[Objetivo:] Indicar al actor que no se permite realizar la operación sobre la entidad.
 	\item[Redacción:] La operación OPERACIÓN no se puede realizar sobre DETERMINADO ENTIDAD ya que aún cuenta con un comité asociado.
 	\item[Parámetros:] El mensaje se muestra con base en los siguientes parámetros:
 	\begin{Citemize}
  		\item OPERACIÓN: Nombre de la acción que se desea realizar.
		\item DETERMINADO ENTIDAD: Es un artículo determinado más el nombre de la entidad sobre la cual se realiza la acción.
 	\end{Citemize}
 	\item[Ejemplo:] La operación {\em eliminar} no se puede realizar sobre {\em la escuela} ya que aún cuenta con un comité asociado.
\end{mensaje}

