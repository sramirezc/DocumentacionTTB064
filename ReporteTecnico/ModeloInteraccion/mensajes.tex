%\DONE{} 
    En esta sección se describen los mensajes utilizados en el prototipo actual del sistema. Los mensajes se refieren a todos
    aquellos avisos que el sistema muestra al actor a través de la pantalla debido a diversas
    razones, por ejemplo: informar acerca de algún fallo en el sistema, notificar acerca de alguna operación importante sobre
    la información, etc.\\

%===========================================================removiendo los puntos generados
\subsection{Parámetros comunes}
    Los mensajes que pueden reutilizarse al cambiar solo algunas palabras pueden ser parametrizados, esto es que se indica qué parte del mensaje 
    debe cambiar para poderse reutilizar, los parámetros más comunes son:

    \begin{description}
	\item [ARTÍCULO:] Se refiere a un {\em artículo} el cual puede ser DETERMINADO (El $\mid$ La $\mid$ Lo $\mid$ Los $\mid$ Las) o INDETERMINADO (Un $\mid$ Una $\mid$ 
	Uno $\mid$ Unos $\mid$Unas) se aplica generalmente sobre una ENTIDAD, ATRIBUTO o VALOR.
	\item [CAMPO:] Se refiere a un campo del formulario. Por lo regular es el nombre de un atributo en una entidad.
	\item [ENTIDAD:] Es un sustantivo y generalmente se refiere a una entidad del modelo estructural del negocio.
	\item [OPERACIÓN:] Se refiere a una acción que se debe realizar sobre los datos de una o varias entidades. Por ejemplo: registrar, eliminar, modificar, etc.
	\item [VALOR:] Es un sustantivo concreto y generalmente se refiere a un valor en específico. 
	\item [TAMAÑO:] Es el tamaño del atributo de una entidad, el cual se encuentra definido en el modelo conceptual. \
    \end{description}
    
    Los tipos de mensajes que se manejan en la herramienta son: notificación, alerta, error y confirmación.\\


\subsection{Mensajes}

%===========================  MSG1 ==================================
\begin{mensaje}{MSG1}{Operación exitosa}{Notificación}
    \item[Objetivo:] Notificar al actor que la operación se ha realizado de forma exitosa.
    \item[Redacción:] DETERMINADO ENTIDAD ha sido OPERACIÓN exitosamente.
    \item[Parámetros:] El mensaje se muestra con base en los siguientes parámetros:
    \begin{Citemize}
	\item DETERMINADO ENTIDAD: Artículo determinado más el nombre de la entidad sobre la que se realiza la operación.
	\item OPERACIÓN: Es la acción que el actor solicitó realizar. Puede ser registro, eliminación, modificación o revisión.
    \end{Citemize}
    \item[Ejemplo:] {\em El actor} ha sido {\em registrado} exitosamente.
\end{mensaje}

%============================== MSG2 =================================
\begin{mensaje}{MSG2}{No existe información}{Notificación}
    \item[Objetivo:] Notificar al actor que aún no existe información registrada en el editor.
    \item[Redacción:] No se han encontrado registros.
\end{mensaje}

%============================== MSG3 =================================
\begin{mensaje}{MSG3}{Caso de uso terminado}{Notificación}
    \item[Objetivo:] Notificar al actor que ha terminado de registrar el caso de uso y este está listo para ser revisado.
    \item[Redacción:] El registro del caso de uso ha sido terminado y registrado exitosamente. Ahora el caso de uso puede ser revisado.
\end{mensaje}

%============================== MSG4 =================================
\begin{mensaje}{MSG4}{Dato obligatorio}{Error}
    \item[Objetivo:] Notificar al actor que el dato es obligatorio.
    \item[Redacción:] Dato obligatorio.
\end{mensaje}

%============================== MSG5 =================================
\begin{mensaje}{MSG5}{Dato incorrecto}{Error}
    \item[Objetivo:] Notificar al actor que el dato no tiene el formato solicitado.
    \item[Redacción:] Dato incorrecto, ingrese un TIPODATO.
    \item[Parámetros:] El mensaje se muestra con base en los siguientes parámetros:
    \begin{Citemize}
	\item TIPODATO: Indica el tipo de dato, por ejemplo cadena o número.
    \end{Citemize}
    \item[Ejemplo:] Dato incorrecto, ingrese una cadena.
\end{mensaje}

%============================== MSG6 =================================
\begin{mensaje}{MSG6}{Longitud inválida}{Error}
    \item[Objetivo:] Notificar al actor que el dato no cumple con la longitud solicitada.
    \item[Redacción:] Escriba menos de TAMAÑO TIPODATO.
    \item[Parámetros:] El mensaje se muestra con base en los siguientes parámetros:
    \begin{Citemize}
	\item TAMAÑO: Indica el tamaño requerido del campo.
	\item TIPODATO: Indica la unidad en que se mide la longitud del campo.
    \end{Citemize}
    \item[Ejemplo:] Escriba menos de {\em 30 letras}.
\end{mensaje}

%============================== MSG7 =================================
\begin{mensaje}{MSG7}{Registro repetido}{Error}
    \item[Objetivo:] Notificar al actor que el nombre del elemento ya existe.
    \item[Redacción:] El nombre que desea registrar ya existe.
\end{mensaje}

%============================== MSG8 =================================
\begin{mensaje}{MSG8}{Caso de uso terminado}{Confirmación}
    \item[Objetivo:] Preguntar al usuario si desea continuar con la operación debido a que no podrá realizar ningún cambio.
    \item[Redacción:] "¿Está seguro de marcar como terminado este elemento? La información será enviada a revisión y no podrá realizar ningún cambio."
\end{mensaje}

%============================== MSG9 =================================
\begin{mensaje}{MSG9}{Elemento no referenciado}{Error}
    \item[Objetivo:] Notificar al actor que un actor, entrada, salida, regla de negocio o mensaje no está siendo utilizado en las trayectorias.
    \item[Redacción:] Hay elementos que no están siendo utilizados en las trayectorias.
\end{mensaje}

%============================== MSG10 =================================
\begin{mensaje}{MSG10}{Error en la operación}{Error}
    \item[Objetivo:] Notificar al actor que ocurrión un problema al intentar ejecutar la operación solicitada.
    \item[Redacción:] Hubo un error al realizar la operación, vuelva a intentarlo.
\end{mensaje}

%============================== MSG11 =================================
\begin{mensaje}{MSG11}{Confirmar eliminación}{Confirmación}
    \item[Objetivo:] Preguntar al actor si desea confirmar la eliminación.
    \item[Redacción:] ¿Está seguro de que quiere eliminar este registro?
\end{mensaje}

%============================== MSG12 =================================
\begin{mensaje}{MSG12}{Elemento no agregado}{Error}
    \item[Objetivo:] Informar al actor que algunos elementos mencionados en las trayectorias no están en la descripción del caso de uso.
    \item[Redacción:] Hay elementos que están siendo utilizados en las trayectorias pero no están en la descripción del caso de uso.
\end{mensaje}
%============================== MSG15 =================================
\begin{mensaje}{MSG15}{Dato no registrado}{Error}
    \item[Objetivo:] Informar al actor que el dato que ingresó o referenció no existe en el sistema.
    \item[Redacción:] Dato incorrecto, DETERMINADO ELEMENTO VALOR no se encuentra registrado en el sistema.
    \item[Parámetros:] El mensaje se muestra con base en los siguientes parámetros:
    \begin{Citemize}
	\item DETERMINADO ENTIDAD: Artículo determinado más el nombre de un elemento.
	\item VALOR: El nombre o identificador de la entidad que no está registrada en el sistema.
    \end{Citemize}
    \item[Ejemplo:] Dato incorrectos, { \em el actor Alumno} no se encuentra registrado en el sistema.
\end{mensaje}
%============================== MSG17 =================================
\begin{mensaje}{MSG17}{Registro incorrecto}{Error}
    \item[Objetivo:] Informar al actor que alguno de los registros de alguna gestión no es correcto.
    \item[Redacción:] Alguno de DETERMINADO ENTIDAD no es correcto.
    \item[Parámetros:] El mensaje se muestra con base en los siguientes parámetros:
    \begin{Citemize}
	\item DETERMINADO ENTIDAD: Artículo determinado más el nombre de un elemento.
    \end{Citemize}
    \item[Ejemplo:] Alguna de {\em las acciones} no es correcta.
\end{mensaje}
%============================== MSG18 =================================
\begin{mensaje}{MSG18}{Registro necesario}{Error}
    \item[Objetivo:] Informar al actor que debe realizar el registro de al menos un elemento para continuar con la operación.
    \item[Redacción:] Ingrese al menos INDETERMINADO ENTIDAD.
    \item[Parámetros:] El mensaje se muestra con base en los siguientes parámetros:
    \begin{Citemize}
	\item INDETERMINADO ENTIDAD: Artículo indeterminado más el nombre de una entidad.
    \end{Citemize}
    \item[Ejemplo:] Ingrese al menos {\em un paso}.
\end{mensaje}
%============================== MSG23 =================================
\begin{mensaje}{MSG23}{Caracteres inválidos}{Error}
    \item[Objetivo:] Informar al actor que el dato que ingresó no puede contener coma, punto, punto medio, dos puntos o guión bajo.
    \item[Redacción:] DETERMINADO ATRIBUTO no puede contener coma, punto, punto medio, dos puntos o guión bajo.
    \item[Parámetros:] El mensaje se muestra con base en los siguientes parámetros:
    \begin{Citemize}
	\item DETERMINADO ATRIBUTO: Artículo determinado más el nombre del atributo que no puede contener coma, punto, punto medio, dos puntos o guión bajo.
    \end{Citemize}
    \item[Ejemplo:] { \em El nombre} no puede contener coma, punto, punto medio, dos puntos o guión bajo.
\end{mensaje}
%============================== MSG28 =================================
\begin{mensaje}{MSG28}{Formato de archivo incorrecto}{Error}
    \item[Objetivo:] Informar al actor que el archivo seleccionado no cumple con el formato especificado en el model conceptual.
    \item[Redacción:] Formato incorrecto, seleccione un archivo con formato FORMATO.
    \item[Parámetros:] El mensaje se muestra con base en los siguientes parámetros:
    \begin{Citemize}
	\item FORMATO: Es el formato o formatos que se permiten para el archivo de acuerdo al modelo conceptual.
    \end{Citemize}
    \item[Ejemplo:] Formato incorrecto, seleccione un archivo con formato {\em jpeg, jpg, png}.
\end{mensaje}
%============================== MSG29 =================================
\begin{mensaje}{MSG29}{Se ha excedido el tamaño del archivo}{Error}
    \item[Objetivo:] Informar al actor que el archivo seleccionado excede el tamaño máximo especificado en el modelo conceptual.
    \item[Redacción:] El archivo no puede exceder TAMAÑO UNIDAD.
    \item[Parámetros:] El mensaje se muestra con base en los siguientes parámetros:
    \begin{Citemize}
	\item TAMAÑO: Es el tamaño máximo permitido especificado en el modelo conceptual.
	\item UNIDAD: Es la unidad en que se especificó el tamaño máximo del archivo.
    \end{Citemize}
    \item[Ejemplo:] El archivo no puede exceder {\em 2 MB}.
\end{mensaje}