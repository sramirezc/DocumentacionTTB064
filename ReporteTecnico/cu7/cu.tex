\begin{UseCase}{CU 7}{Gestionar actores}
	{
		Este caso de uso permite al analista visualizar los registros de los actores registrados en el sistema. También permite 
		al actor acceder a las operaciones de registro, consulta, modificación, 
		y eliminación de un actor.
	}
	
	\UCitem{Actor}{\cdtRef{actor:liderAnalisis}{Líder de análisis}, \cdtRef{actor:analista}{Analista}}
	\UCitem{Propósito}{
		Revisar y gestionar los actores registrados en la herramienta.
	}
	\UCitem{Entradas}{
		Ninguna
	}
	\UCitem{Salidas}{
		\begin{UClist}
			\UCli \cdtRef{Modulo:Clave}{Clave del módulo}: \ioObtener.
			\UCli \cdtRef{Modulo:Nombre}{Nombre del módulo}: \ioObtener.
			\UCli \cdtRef{Modulo:Descripcion}{Descripción del módulo}: \ioObtener.
			\UCli \cdtRef{Actor}{Actor}: \ioTabla{\cdtRef{Elemento:Clave}{Clave} y el \cdtRef{Elemento:Nombre}{Nombre}}{de los actores}.
		\end{UClist}
	}
	
	\UCitem{Mensajes}{
		\begin{UClist}
			\UCli \cdtIdRef{MSG2}{No existe información}: Se muestra en la pantalla \cdtIdRef{IU 7}{Gestionar actores} cuando no existen actores registrados.
		\end{UClist}
	}

	\UCitem{Precondiciones}{
		Ninguna
	}
	
	\UCitem{Postcondiciones}{
		Ninguna
	}

	\UCitem{Errores}{
		Ninguno
	}

	\UCitem{Tipo}{
		Primario
	}
\end{UseCase}
%-------------------------------------------------------%trayectoria Principal-----------------------------------------------
 \begin{UCtrayectoria}
	\UCpaso[\UCactor] Oprime el botón \btnEntrar del proyecto con el que desea trabajar, en la pantalla \cdtIdRef{IU 1}{Gestionar proyectos}.
    \UCpaso[\UCactor] Solicita gestionar los actores seleccionando la opción ``Actores'' del \cdtRef{menu:principal}{Menú principal}.
    \UCpaso[\UCsist] Busca la información de los actores registrados en el sistema. \refTray{A}
    \UCpaso[\UCsist] Muestra la información de los actores en la pantalla \cdtIdRef{IU 7}{Gestionar actores} y las operaciones 
    disponibles de acuerdo a la regla de negocio \cdtIdRef{RN15}{Operaciones disponibles}. 
    \UCpaso[\UCactor] Gestiona los actores a través de los botones: \cdtButton{Registrar}, \btnConsulta, \btnEditar y \btnEliminar. \label{cu7:gestiona}
 \end{UCtrayectoria}
 
 \begin{UCtrayectoriaA}[Fin del caso de uso]{A}{No existen registros de actores.}
    \UCpaso[\UCsist] Muestra el mensaje \cdtIdRef{MSG2}{No existe información} en pantalla \cdtIdRef{IU 7}{Gestionar actores} 
    para indicar que no hay registros de actores para mostrar.
 \end{UCtrayectoriaA}
 

\subsection{Puntos de extensión}

\UCExtensionPoint{El actor requiere registrar un actor}
	{Paso \ref{cu7:gestiona}}
	{\cdtIdRef{CU 7.1}{Registrar actor}}
\UCExtensionPoint{El actor requiere modificar un actor}
	{Paso \ref{cu7:gestiona}}
	{\cdtIdRef{CU 7.2}{Modificar actor}}	
\UCExtensionPoint{El actor requiere eliminar un actor}
	{Paso \ref{cu7:gestiona}}
	{\cdtIdRef{CU 7.3}{Eliminar actor}}
\UCExtensionPoint{El actor requiere consultar un actor}
	{Paso \ref{cu7:gestiona}}
	{\cdtIdRef{CU 7.4}{Consultar actor}}  