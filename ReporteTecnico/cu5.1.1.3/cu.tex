\begin{UseCase}{CU 5.1.1.3}{Eliminar trayectoria}
	{
		Este caso de uso permite al actor eliminar del sistema una trayectoria.
	}
	
	\UCitem{Actor}{\cdtRef{actor:liderAnalisis}{Líder de análisis}, \cdtRef{actor:analista}{Analista}}
	\UCitem{Propósito}{
		Eliminar la información de una trayectoria.
	}
	\UCitem{Entradas}{
		Ninguna
	}
	\UCitem{Salidas}{
		\begin{UClist}
			\UCli \cdtIdRef{MSG1}{Operación exitosa}: Se muestra en la pantalla \cdtIdRef{IU 5.1.1}{Gestionar trayectorias} para indicar que la eliminación fue exitosa.
		\end{UClist}
	}
	\UCitem{Mensajes}{
		\begin{UClist}
			\UCli \cdtIdRef{MSG11}{Confirmar eliminación}: Se muestra para que el actor confirme la eliminación.
		\end{UClist}
	}

	\UCitem{Precondiciones}{
		Ninguna
	}
	
	\UCitem{Postcondiciones}{
		\begin{UClist}
			\UCli Se eliminará la trayectoria seleccionada del sistema.
		\end{UClist}
	}

	\UCitem{Errores}{
		\begin{UClist}
			\UCli \cdtIdRef{MSG14}{Eliminación no permitida}: Se muestra en una pantalla emergente cuando no se pueda eliminar la trayectoria debido a que está siendo referenciada en algún caso de uso.
		\end{UClist}
	}

	\UCitem{Tipo}{
		Secundario, extiende del casos de uso \cdtIdRef{CU 5.1.1}{Gestionar trayectorias}.
	}
\end{UseCase}
%-------------------------------------------------------%trayectoria Principal-----------------------------------------------
 \begin{UCtrayectoria}
    \UCpaso[\UCactor] Solicita eliminar una trayectoria oprimiendo el botón \btnEliminar del registro que desea eliminar de la pantalla \cdtIdRef{IU 5.1.1}{Gestionar trayectorias}.
    \UCpaso[\UCsist] Busca los casos de uso que estén referenciando a la trayectoria.
    \UCpaso[\UCsist] Busca los casos de uso que estén referenciando algún paso de la trayectoria.
    \UCpaso[\UCsist] Verifica que ningún caso de uso esté referenciando a la trayectoria. \refTray{A}
    \UCpaso[\UCsist] Verifica que ningún caso de uso esté referenciando algún paso de la trayectoria. \refTray{B}
    \UCpaso[\UCsist] Muestra el mensaje \cdtIdRef{MSG11}{Confirmar eliminación} en una pantalla emergente con los botones \cdtButton{Aceptar} y \cdtButton{Cancelar}.
    \UCpaso[\UCactor] Confirma la eliminación de la trayectoria oprimiendo el botón \cdtButton{Aceptar} de la pantalla emergente. \refTray{C}
    \UCpaso[\UCsist] Elimina la información referente a la trayectoria.
    \UCpaso[\UCsist] Muestra el mensaje \cdtIdRef{MSG1}{Operación exitosa} en la pantalla \cdtIdRef{IU 5.1.1}{Gestionar trayectorias}
    para indicar al actor que se ha eliminado el registro exitosamente.
 \end{UCtrayectoria}
 
 %----------------------------------------------------------%trayectoria A---------------------------------------------------- 
 \begin{UCtrayectoriaA}[Fin del caso de uso]{A}{La trayectoria está siendo referenciada en algún caso de uso.}
    \UCpaso[\UCsist] Muestra el mensaje \cdtIdRef{MSG14}{Eliminación no permitida} en una pantalla emergente
    con la lista de casos de uso que están referenciando a la trayectoria.
    \UCpaso[\UCactor] Oprime el botón \cdtButton{Aceptar} de la pantalla emergente.
    \UCpaso[\UCsist] Muestra la pantalla \cdtIdRef{IU 5.1.1}{Gestionar trayectorias}.
 \end{UCtrayectoriaA}
 %----------------------------------------------------------%trayectoria B---------------------------------------------------- 
 \begin{UCtrayectoriaA}[Fin del caso de uso]{B}{Algún paso de la trayectoria está siendo referenciado en algún caso de uso.}
    \UCpaso[\UCsist] Muestra el mensaje \cdtIdRef{MSG14}{Eliminación no permitida} en una pantalla emergente
    con la lista de casos de uso que están referenciando al paso.
    \UCpaso[\UCactor] Oprime el botón \cdtButton{Aceptar} de la pantalla emergente.
    \UCpaso[\UCsist] Muestra la pantalla \cdtIdRef{IU 5}{Gestionar casos de uso}.
 \end{UCtrayectoriaA}
 %----------------------------------------------------------%trayectoria C---------------------------------------------------- 
 \begin{UCtrayectoriaA}[Fin del caso de uso]{C}{El actor desea cancelar la operación.}
    \UCpaso[\UCactor] Solicita cancelar la operación oprimiendo el botón \cdtButton{Cancelar} de la pantalla emergente.
    \UCpaso[\UCsist] Muestra la pantalla donde se solicitó la operación.
 \end{UCtrayectoriaA} 