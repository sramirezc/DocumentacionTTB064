\subsection{IU 5.1.1.2 Modificar trayectoria}

\subsubsection{Objetivo}
	
	Esta pantalla permite al actor modificar la información de una trayectoria, así como gestionar los pasos de la misma.

\subsubsection{Diseño}

    En la figura ~\ref{IU 5.1.1.2} se muestra la pantalla ``Modificar trayectoria'' en la cual el actor deberá ingresar la nueva información y gestionar los pasos.\\
      
    Una vez ingresada la información solicitada en la pantalla, deberá oprimir el botón \cdtButton{Aceptar}, el sistema mostrará la pantalla emergente ~\ref{IU 5.1.1.2A} para que el actor ingrese los comentarios referentes a la modificación, posteriormente deberá oprimir el botón \cdtButton{Aceptar} de la pantalla emergente, el sistema validará la información y en caso de cumplor con todas las reglas de negocio establecidas, se realizará la modificación de la información.
	    
    Finalmente se mostrará el mensaje \cdtIdRef{MSG1}{Operación exitosa} en la pantalla \cdtIdRef{IU 5.1.1}{Gestionar trayectorias}, para indicar que la información de la trayectoria se ha modificado correctamente.        


    \IUfig[.9]{cu5.1.1.2/images/iu.png}{IU 5.1.1.2}{Modificar trayectoria}
    \IUfig[.9]{cu5.1.1.2/images/iuA.png}{IU 5.1.1.2A}{Modificar trayectoria: Comentarios}



\subsubsection{Comandos}
\begin{itemize}
	\item \cdtButton{Registrar}: Permite al actor solicitar el registro de un paso, dirige a la pantalla \cdtIdRef{IU 5.1.1.2.1}{Registrar paso}.
	\item \btnEditar[Modificar]: Permite al actor solicitar la modificación de un paso, dirige a la pantalla \cdtIdRef{IU 5.1.1.2.2}{Modificar paso}.
	\item \btnEliminar[Eliminar]: Permite al actor solicitar la eliminación de un paso, dirige a la pantalla \cdtIdRef{IU 5.1.1.2.3}{Eliminar paso}.
	\item \cdtButton{Aceptar}: Permite al actor confirmar la modificación de la trayectoria, dirige a la pantalla \cdtIdRef{IU 5.1.1}{Gestionar trayectorias}.
	\item \cdtButton{Cancelar}: Permite al actor cancelar la modificación de la trayectoria, dirige a la pantalla \cdtIdRef{IU 5.1.1}{Gestionar trayectorias}.
\end{itemize}

\subsubsection{Mensajes}

	
\begin{description}
	\item[ \cdtIdRef{MSG1}{Operación exitosa}:] Se muestra en la pantalla \cdtIdRef{IU 5.1.1}{Gestionar trayectorias} para indicar que la modificación fue exitosa.
	\item[\cdtIdRef{MSG4}{Dato obligatorio}:] Se muestra en la pantalla \cdtIdRef{CU 5.1.1.2}{Modificar trayectoria} o en la pantalla emergente \cdtIdRef{CU 5.1.1.2A}{Modificar trayectoria: Comentarios} cuando no se ha ingresado un dato marcado como obligatorio.
	\item[\cdtIdRef{MSG5}{Dato incorrecto}:] Se muestra en la pantalla \cdtIdRef{CU 5.1.1.2}{Modificar trayectoria} cuando el tipo de dato ingresado no cumple con el tipo de dato solicitado en el campo.
	\item[\cdtIdRef{MSG6}{Longitud inválida}:] Se muestra en la pantalla \cdtIdRef{CU 5.1.1.2}{Modificar trayectoria} o en la pantalla emergente \cdtIdRef{CU 5.1.1.2A}{Modificar trayectoria: Comentarios} cuando se ha excedido la longitud de alguno de los campos.
	\item[\cdtIdRef{MSG7}{Registro repetido}:] Se muestra en la pantalla \cdtIdRef{CU 5.1.1.2}{Modificar trayectoria} cuando se haya ingresado una clave que ya esté registrada.
	\item[\cdtIdRef{MSG15}{Dato no registrado}:] Se muestra en la pantalla \cdtIdRef{CU 5.1.1.2}{Modificar trayectoria} cuando un elemento referenciado no existe en el sistema.
	\item[\cdtIdRef{MSG13}{Ha ocurrido un error}:] Se muestra en la pantalla \cdtIdRef{IU 5}{Gestionar casos de uso} cuando el estado del caso de uso sea inválido.
	\item[\cdtIdRef{MSG23}{Caracteres inválidos}:] Se muestra en la pantalla \cdtIdRef{CU 5.1.1.2}{Modificar trayectoria} cuando la clave de la trayectoria contiene un carácter no válido.
	\item[\cdtIdRef{MSG27}{Token incorrecto}:] Se muestra en la pantalla \cdtIdRef{CU 5.1.1.2}{Modificar trayectoria} cuando el token ingresado está mal formado.
\end{description}
