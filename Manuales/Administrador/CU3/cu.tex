
En este capítulo se describe cómo gestionar los Módulos, lo cual corresponde a registrar Módulo, editar Módulo y eliminar Módulo.\\


\begin{objetivos}
	\item Describir cómo registrar un Módulo.
	\item Describir cómo modificar un Módulo.
	\item Describir cómo eliminar un Módulo.
\end{objetivos}

\section{Gestionar Módulos}

Para gestionar los Módulos es necesario que entre a un Proyecto como se indica en la sección \ref{cap:gestionarProyectos}, el sistema mostrará la pantalla \cdtFigureRef{fig:gestionarModulos}{Gestionar Módulos}.

\IUfig[.9]{CU3/images/gestionarModulos}{fig:gestionarModulos}{Gestionar Módulos}

\subsubsection{Acciones}

\begin{enumerate}
	\item \btnCU[Gestionar Casos de uso]: Mediante este botón puede ingresar a la gestión de Casos de uso.
	\item \btnIU[Gestionar Pantallas]: Mediante este botón puede ingresar a la gestión de Pantallas.
	\item \cdtButton{Registrar}: Mediante este botón puede solicitar el registro de un Módulo.
	\item \btnEditar[Modificar Módulo]: Mediante este botón puede solicitar la modificación de la información del Módulo.
	\item \btnEliminar[Eliminar Módulo]: Mediante este botón puede solicitar la eliminación de un Módulo.
\end{enumerate}

\begin{UseCase}{CU 5.1.3}{Gestionar postcondiciones}
	{
		Este caso de uso permite al analista visualizar las postcondiciones del caso de uso. También permite 
		al actor acceder a las operaciones de registro, modificación y eliminación de las postcondiciones.
	}

	\UCitem{Actor}{\cdtRef{actor:liderAnalisis}{Líder de análisis}, \cdtRef{actor:analista}{Analista}}
	\UCitem{Propósito}{
		Revisar y gestionar las postcondiciones de un caso de uso.
	}
	\UCitem{Entradas}{
		Ninguna
	}
	\UCitem{Salidas}{
		\begin{UClist}
			\UCli \cdtRef{Postcondicion}{Postcondición}: \ioTabla{el \cdtRef{Postcondicion:Numero}{Número} y la \cdtRef{Postcondicion:Redacción}{Redacción}}.
		\end{UClist}
	}
	
	\UCitem{Mensajes}{
		\begin{UClist}
			\UCli \cdtIdRef{MSG2}{No existe información}: Se muestra en la pantalla \cdtIdRef{IU 5.1}{Registrar caso de uso} o \cdtIdRef{IU 5.2}{Modificar caso de uso} cuando no existen postcondiciones registradas.
		\end{UClist}
	}

	\UCitem{Postcondiciones}{
		Ninguna
	}
	
	\UCitem{Postcondiciones}{
		\begin{UClist}
			\UCli Se podrá solicitar el registro de una postcondición por medio del caso de uso \cdtIdRef{CU 5.1.3.1}{Registrar postcondición}.
			\UCli Se podrá solicitar la modificación de una postcondición por medio del caso de uso \cdtIdRef{CU 5.1.3.2}{Modificar postcondición}.
			\UCli Se podrá solicitar la eliminación de una postcondición por medio del caso de uso \cdtIdRef{CU 5.1.3.3}{Eliminar postcondición}.
		\end{UClist}
	}

	\UCitem{Errores}{
		Ninguno
	}

	\UCitem{Tipo}{
		Secundario, extiende del caso de uso \cdtIdRef{CU 5.1}{Registrar caso de uso}.
	}
\end{UseCase}
%-------------------------------------------------------%trayectoria Principal-----------------------------------------------
 \begin{UCtrayectoria}
    \UCpaso[\UCactor] Solicita gestionar las postcondiciones del caso de uso seleccionando la opción ``Postcondiciones'' de la pantalla \cdtIdRef{IU 5.1}{Registrar caso de uso}.
    \UCpaso[\UCsist] Busca información de las postcondiciones del caso de uso. \refTray{A}
    \UCpaso[\UCsist] Muestra la información de las postcondiciones en la pantalla \cdtIdRef{IU 5.1}{Registrar caso de uso} o \cdtIdRef{IU 5.2}{Modificar caso de uso}. 
    \UCpaso[\UCactor] Gestiona las postcondiciones a través de los botones: \btnAgregar, \btnEditar y \btnEliminar. \label{cu5.1.3:gestionaCU}
 \end{UCtrayectoria}
 
 \begin{UCtrayectoriaA}[Fin del caso de uso]{A}{No existen registros de postcondiciones.}
    \UCpaso[\UCsist] Muestra el mensaje \cdtIdRef{MSG2}{No existe información} en pantalla \cdtIdRef{IU 5.1.3}{Gestionar postcondiciones} 
    para indicar que no hay registros de postcondiciones para mostrar.
 \end{UCtrayectoriaA}
 

\subsection{Puntos de extensión}

\UCExtensionPoint{El actor requiere registrar una postcondición}
	{Paso \ref{cu5.1.3:gestionaCU}}
	{\cdtIdRef{CU 5.1.3.1}{Registrar postcondición}}
\UCExtensionPoint{El actor requiere modificar una precondición}
	{Paso \ref{cu5.1.3:gestionaCU}}
	{\cdtIdRef{CU 5.1.3.2}{Modificar postcondición}}
\UCExtensionPoint{El actor requiere eliminar una precondición}
	{Paso \ref{cu5.1.3:gestionaCU}}
	{\cdtIdRef{CU 5.1.3.3}{Eliminar postcondición}}
  
\begin{UseCase}{CU 5.1.3}{Gestionar postcondiciones}
	{
		Este caso de uso permite al analista visualizar las postcondiciones del caso de uso. También permite 
		al actor acceder a las operaciones de registro, modificación y eliminación de las postcondiciones.
	}

	\UCitem{Actor}{\cdtRef{actor:liderAnalisis}{Líder de análisis}, \cdtRef{actor:analista}{Analista}}
	\UCitem{Propósito}{
		Revisar y gestionar las postcondiciones de un caso de uso.
	}
	\UCitem{Entradas}{
		Ninguna
	}
	\UCitem{Salidas}{
		\begin{UClist}
			\UCli \cdtRef{Postcondicion}{Postcondición}: \ioTabla{el \cdtRef{Postcondicion:Numero}{Número} y la \cdtRef{Postcondicion:Redacción}{Redacción}}.
		\end{UClist}
	}
	
	\UCitem{Mensajes}{
		\begin{UClist}
			\UCli \cdtIdRef{MSG2}{No existe información}: Se muestra en la pantalla \cdtIdRef{IU 5.1}{Registrar caso de uso} o \cdtIdRef{IU 5.2}{Modificar caso de uso} cuando no existen postcondiciones registradas.
		\end{UClist}
	}

	\UCitem{Postcondiciones}{
		Ninguna
	}
	
	\UCitem{Postcondiciones}{
		\begin{UClist}
			\UCli Se podrá solicitar el registro de una postcondición por medio del caso de uso \cdtIdRef{CU 5.1.3.1}{Registrar postcondición}.
			\UCli Se podrá solicitar la modificación de una postcondición por medio del caso de uso \cdtIdRef{CU 5.1.3.2}{Modificar postcondición}.
			\UCli Se podrá solicitar la eliminación de una postcondición por medio del caso de uso \cdtIdRef{CU 5.1.3.3}{Eliminar postcondición}.
		\end{UClist}
	}

	\UCitem{Errores}{
		Ninguno
	}

	\UCitem{Tipo}{
		Secundario, extiende del caso de uso \cdtIdRef{CU 5.1}{Registrar caso de uso}.
	}
\end{UseCase}
%-------------------------------------------------------%trayectoria Principal-----------------------------------------------
 \begin{UCtrayectoria}
    \UCpaso[\UCactor] Solicita gestionar las postcondiciones del caso de uso seleccionando la opción ``Postcondiciones'' de la pantalla \cdtIdRef{IU 5.1}{Registrar caso de uso}.
    \UCpaso[\UCsist] Busca información de las postcondiciones del caso de uso. \refTray{A}
    \UCpaso[\UCsist] Muestra la información de las postcondiciones en la pantalla \cdtIdRef{IU 5.1}{Registrar caso de uso} o \cdtIdRef{IU 5.2}{Modificar caso de uso}. 
    \UCpaso[\UCactor] Gestiona las postcondiciones a través de los botones: \btnAgregar, \btnEditar y \btnEliminar. \label{cu5.1.3:gestionaCU}
 \end{UCtrayectoria}
 
 \begin{UCtrayectoriaA}[Fin del caso de uso]{A}{No existen registros de postcondiciones.}
    \UCpaso[\UCsist] Muestra el mensaje \cdtIdRef{MSG2}{No existe información} en pantalla \cdtIdRef{IU 5.1.3}{Gestionar postcondiciones} 
    para indicar que no hay registros de postcondiciones para mostrar.
 \end{UCtrayectoriaA}
 

\subsection{Puntos de extensión}

\UCExtensionPoint{El actor requiere registrar una postcondición}
	{Paso \ref{cu5.1.3:gestionaCU}}
	{\cdtIdRef{CU 5.1.3.1}{Registrar postcondición}}
\UCExtensionPoint{El actor requiere modificar una precondición}
	{Paso \ref{cu5.1.3:gestionaCU}}
	{\cdtIdRef{CU 5.1.3.2}{Modificar postcondición}}
\UCExtensionPoint{El actor requiere eliminar una precondición}
	{Paso \ref{cu5.1.3:gestionaCU}}
	{\cdtIdRef{CU 5.1.3.3}{Eliminar postcondición}}
  
\begin{UseCase}{CU 5.1.3}{Gestionar postcondiciones}
	{
		Este caso de uso permite al analista visualizar las postcondiciones del caso de uso. También permite 
		al actor acceder a las operaciones de registro, modificación y eliminación de las postcondiciones.
	}

	\UCitem{Actor}{\cdtRef{actor:liderAnalisis}{Líder de análisis}, \cdtRef{actor:analista}{Analista}}
	\UCitem{Propósito}{
		Revisar y gestionar las postcondiciones de un caso de uso.
	}
	\UCitem{Entradas}{
		Ninguna
	}
	\UCitem{Salidas}{
		\begin{UClist}
			\UCli \cdtRef{Postcondicion}{Postcondición}: \ioTabla{el \cdtRef{Postcondicion:Numero}{Número} y la \cdtRef{Postcondicion:Redacción}{Redacción}}.
		\end{UClist}
	}
	
	\UCitem{Mensajes}{
		\begin{UClist}
			\UCli \cdtIdRef{MSG2}{No existe información}: Se muestra en la pantalla \cdtIdRef{IU 5.1}{Registrar caso de uso} o \cdtIdRef{IU 5.2}{Modificar caso de uso} cuando no existen postcondiciones registradas.
		\end{UClist}
	}

	\UCitem{Postcondiciones}{
		Ninguna
	}
	
	\UCitem{Postcondiciones}{
		\begin{UClist}
			\UCli Se podrá solicitar el registro de una postcondición por medio del caso de uso \cdtIdRef{CU 5.1.3.1}{Registrar postcondición}.
			\UCli Se podrá solicitar la modificación de una postcondición por medio del caso de uso \cdtIdRef{CU 5.1.3.2}{Modificar postcondición}.
			\UCli Se podrá solicitar la eliminación de una postcondición por medio del caso de uso \cdtIdRef{CU 5.1.3.3}{Eliminar postcondición}.
		\end{UClist}
	}

	\UCitem{Errores}{
		Ninguno
	}

	\UCitem{Tipo}{
		Secundario, extiende del caso de uso \cdtIdRef{CU 5.1}{Registrar caso de uso}.
	}
\end{UseCase}
%-------------------------------------------------------%trayectoria Principal-----------------------------------------------
 \begin{UCtrayectoria}
    \UCpaso[\UCactor] Solicita gestionar las postcondiciones del caso de uso seleccionando la opción ``Postcondiciones'' de la pantalla \cdtIdRef{IU 5.1}{Registrar caso de uso}.
    \UCpaso[\UCsist] Busca información de las postcondiciones del caso de uso. \refTray{A}
    \UCpaso[\UCsist] Muestra la información de las postcondiciones en la pantalla \cdtIdRef{IU 5.1}{Registrar caso de uso} o \cdtIdRef{IU 5.2}{Modificar caso de uso}. 
    \UCpaso[\UCactor] Gestiona las postcondiciones a través de los botones: \btnAgregar, \btnEditar y \btnEliminar. \label{cu5.1.3:gestionaCU}
 \end{UCtrayectoria}
 
 \begin{UCtrayectoriaA}[Fin del caso de uso]{A}{No existen registros de postcondiciones.}
    \UCpaso[\UCsist] Muestra el mensaje \cdtIdRef{MSG2}{No existe información} en pantalla \cdtIdRef{IU 5.1.3}{Gestionar postcondiciones} 
    para indicar que no hay registros de postcondiciones para mostrar.
 \end{UCtrayectoriaA}
 

\subsection{Puntos de extensión}

\UCExtensionPoint{El actor requiere registrar una postcondición}
	{Paso \ref{cu5.1.3:gestionaCU}}
	{\cdtIdRef{CU 5.1.3.1}{Registrar postcondición}}
\UCExtensionPoint{El actor requiere modificar una precondición}
	{Paso \ref{cu5.1.3:gestionaCU}}
	{\cdtIdRef{CU 5.1.3.2}{Modificar postcondición}}
\UCExtensionPoint{El actor requiere eliminar una precondición}
	{Paso \ref{cu5.1.3:gestionaCU}}
	{\cdtIdRef{CU 5.1.3.3}{Eliminar postcondición}}
  

