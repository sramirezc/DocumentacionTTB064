\subsection{Configurar Caso de uso}
En esta subsección se describe el procedimiento a seguir para configurar el Caso de uso a probar.

\subsubsection{Datos de entrada}
\begin{description}
	\item Para configurar el Caso de uso debe contar con los siguientes datos: \hspace{10pt}
	\begin{description}
	    \item \textbf{Configuración de las Entradas}
	    \begin{itemize}
		  \item Name: Es el valor del atributo \emph{Name} del input de la entrada indicada.
		  \item Value: Es el valor del atributo \emph{Value} del input de la entrada indicada.
	    \end{itemize}
	    \item \textbf{Configuración de las URL}
	    \begin{itemize}
		  \item URL: Es la URL a la que dirige la acción indicada.
		  \item Método: Permite indicar si la URL es solicitada con el método GET o POST.
	    \end{itemize}
	    \item \textbf{Configuración de las Reglas de negocio}
	    \begin{itemize}
		  \item Query: Es la query para realizar una consulta a la base de datos. Las Reglas de negocio de unicidad requieren una query que consulte las Entidades cuyo Atributo identificador sea igual al proporcionado en las Entradas. 
			En el caso de las Reglas de negocio de verificación de catálogos, deberá ingresar una query que consulte todos los registros del catálogo.
	    \end{itemize}
	    \item \textbf{Configuración de los Parámetros}
	    \begin{itemize}
		  \item Valor del Parámetro: Es el valor que toma el Parámetro del Mensaje en el momento de ejecución del Caso de uso. Si deja este valor vacío, el sistema guardará como valor el nombre del Parámetro.
	    \end{itemize}
	    \item \textbf{Configuración de las Pantallas}
	    \begin{itemize}
		  \item Patrón: Es una cadena que contiene la Pantalla indicada.
	    \end{itemize}
	 \end{description}
\end{description}

\subsubsection{Acciones}
\begin{itemize}
    \item \cdtButton{Anterior}: Mediante este botón puede regresar a la configuración de los Casos de uso previos.
    \item \cdtButton{Guardar}: Mediante este botón puede guardar la información ingresada en la pantalla.
    \item \cdtButton{Finalizar}: Mediante este botón puede configurar la información del Caso de uso.
    \item \cdtButton{Cancelar}: Mediante este botón puede cancelar la configuración del Caso de uso.
\end{itemize}
	
	
\subsubsection{Procedimiento}
\begin{enumerate}
	\item Oprima el botón \cdtButton{Siguiente} de la pantalla \cdtFigureRef{fig:configurarCasosUsoPrevios}{Configurar Casos de uso previos}, el sistema mostrará la pantalla \cdtFigureRef{fig:configurarCasosUso}{Configurar Caso de uso}. 

	%--------------------------------------------------------------------------------------------------------
	      \IUfig[.45]{CU5.7.3/images/configurarCasoUso}{fig:configurarCasosUso}{Configurar Caso de uso}
	%--------------------------------------------------------------------------------------------------------
	
	\item Ingrese la información solicitada en la pantalla.
	
	\item Oprima el botón \cdtButton{Finalizar} de la pantalla \cdtFigureRef{fig:configurarCasosUso}{Configurar Caso de uso}, el sistema mostrará la pantalla 
	      \cdtFigureRef{fig:gestionarCasosUso}{Gestionar Casos de uso} con el mensaje de éxito.
\end{enumerate}



