\subsection{Modificar Pantalla}
En esta subsección se describe el procedimiento a seguir para modificar una Pantalla.

\subsubsection{Datos de entrada}
\begin{description}
	\item Para modificar una Pantalla debe contar con los siguientes datos: \hspace{10pt}
	\begin{description}
	    \item \textbf{Información general de la Pantalla}
	    \begin{itemize}
		  \item Número: Es un número que sirve para identificar a la Pantalla.
		  \item Nombre: Es el nombre de la Pantalla.
		  \item Descripción: Es un texto que detalla la información general de la Pantalla.
		  \item Imagen: Es un archivo en formato PNG de la Pantalla.
	    \end{itemize}
	    \item \textbf{Información general de la Acción}
	    \begin{itemize}
		  \item Nombre: Es el nombre de la Acción.
		  \item Descripción: Es un texto que detalla la información general de la Acción.
		  \item Imagen: Es un archivo en formato PNG de la Acción.
		  \item Tipo de Acción: Sive para definir si la Acción es un botón, liga, opción del menú u otro tipo.
		  \item Pantalla destino: Es la Pantalla a la que dirige la Acción.
	    \end{itemize}
	 \end{description}
\end{description}

\subsubsection{Acciones}
\begin{itemize}
 \item Sección \textbf{Información general de la Pantalla}
  \begin{enumerate}
	\item \btnEliminar[Eliminar]: Mediante este botón puede eliminar la imagen seleccionada de la Pantalla.
  \end{enumerate}
  \item Sección \textbf{Acciones de la Pantalla}
  \begin{enumerate}
	\item \cdtButton{Registrar}: Mediante este botón puede solicitar el registro de una Acción.
	\item \btnEditar[Modificar Acción]: Mediante este botón puede solicitar la modificación de una Acción.
	\item \btnEliminar[Eliminar Acción]: Mediante este botón puede solicitar la eliminación de una Acción.
  \end{enumerate}
  \item Pantalla emergente \textbf{Acción}
  \begin{enumerate}
	\item \btnEliminar[Eliminar]: Mediante este botón puede eliminar la imagen seleccionada de la Acción.
	\item \cdtButton{Aceptar}: Mediante este botón puede registrar la Acción.
	\item \cdtButton{Cancelar}: Mediante este botón puede cancelar el registro de la Acción.
  \end{enumerate}
 \item \cdtButton{Aceptar}: Mediante este botón puede modificar la Pantalla.
 \item \cdtButton{Cancelar}: Mediante este botón puede cancelar la modificación de la Pantalla.
\end{itemize}
	
	
\subsubsection{Procedimiento}
\begin{enumerate}
	\item Oprima el botón \btnEditar de la Pantalla que desea modificar en la pantalla \cdtFigureRef{fig:gestionarPantallas}{Gestionar Pantallas}, el sistema mostrará la pantalla \cdtFigureRef{fig:modificarPantalla}{Modificar Pantalla}. 

	%--------------------------------------------------------------------------------------------------------
	      \IUfig[.9]{CU6.2/images/modificarPantalla}{fig:modificarPantalla}{Modificar Pantalla}
	%--------------------------------------------------------------------------------------------------------
	
	\item Modifique la \textbf{Información general de la Pantalla}.
	
	\item Oprima el botón \cdtButton{Registrar} de la sección \textbf{Acciones de la Pantalla}, el sistema mostrará la pantalla emergente \cdtFigureRef{fig:registrarAccion}{Registrar Acción}. \label{cu6.2:paso:registrarAccion}
	
	\item Ingrese la información de la pantalla emergente \textbf{Acción}.
	
	\item Oprima el botón \cdtButton{Aceptar}, el sistema mostrará la pantalla \cdtFigureRef{fig:modificarPantalla}{Modificar Pantalla} con la Acción en la sección \textbf{Acciones de la Pantalla}.
	
	\item Registre todas las Acciones repitiendo el procedimiento que se indica a partir del paso \ref{cu6.2:paso:registrarAccion}.
	
	\item Oprima el botón \cdtButton{Aceptar}, el sistema mostrará la pantalla \cdtFigureRef{fig:gestionarPantallas}{Gestionar Pantallas} con el mensaje de éxito.
\end{enumerate}


