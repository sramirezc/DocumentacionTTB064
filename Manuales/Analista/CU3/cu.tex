
En este capítulo se describe cómo gestionar los Módulos, lo cual corresponde a registrar Módulo, editar Módulo y eliminar Módulo.\\


\begin{objetivos}
	\item Describir cómo registrar un Módulo.
	\item Describir cómo modificar un Módulo.
	\item Describir cómo eliminar un Módulo.
\end{objetivos}

\section{Gestionar Módulos}

Para gestionar los Módulos es necesario que entre a un Proyecto como se indica en la sección \ref{cap:gestionarProyectos}, el sistema mostrará la pantalla \cdtFigureRef{fig:gestionarModulos}{Gestionar Módulos}.

\IUfig[.9]{CU3/images/gestionarModulos}{fig:gestionarModulos}{Gestionar Módulos}

\subsubsection{Acciones}

\begin{enumerate}
	\item \btnCU[Gestionar Casos de uso]: Mediante este botón puede ingresar a la gestión de Casos de uso.
	\item \btnIU[Gestionar Pantallas]: Mediante este botón puede ingresar a la gestión de Pantallas.
	\item \cdtButton{Registrar}: Mediante este botón puede solicitar el registro de un Módulo.
	\item \btnEditar[Modificar Módulo]: Mediante este botón puede solicitar la modificación de la información del Módulo.
	\item \btnEliminar[Eliminar Módulo]: Mediante este botón puede solicitar la eliminación de un Módulo.
\end{enumerate}

\subsection{Eliminar Actor}

En esta subsección se describe el procedimiento a seguir para eliminar un Actor, se podrá eliminar un Actor cuando este no tenga elementos asociados.

\subsubsection{Acciones}
\begin{itemize}
  \item Pantalla emergente \textbf{Confirmación}
  \begin{enumerate}
	\item \cdtButton{Aceptar}: Mediante este botón puede confirmar la eliminación del Actor.
	\item \cdtButton{Cancelar}: Mediante este botón puede cancelar la eliminación del Actor.
  \end{enumerate}
\end{itemize}

\subsubsection{Procedimiento}
\begin{enumerate}
	\item Oprima el botón \btnEliminar del registro que desea eliminar en la pantalla \cdtFigureRef{fig:gestionarActores}{Gestionar Actores}, el sistema mostrará una pantalla emergente de confirmación.
	
	\item Oprima el botón \cdtButton{Aceptar} de la pantalla de confirmación, el sistema mostrará la pantalla \cdtFigureRef{fig:gestionarActores}{Gestionar Actores} con el mensaje de éxito.
\end{enumerate}


\subsection{Eliminar Actor}

En esta subsección se describe el procedimiento a seguir para eliminar un Actor, se podrá eliminar un Actor cuando este no tenga elementos asociados.

\subsubsection{Acciones}
\begin{itemize}
  \item Pantalla emergente \textbf{Confirmación}
  \begin{enumerate}
	\item \cdtButton{Aceptar}: Mediante este botón puede confirmar la eliminación del Actor.
	\item \cdtButton{Cancelar}: Mediante este botón puede cancelar la eliminación del Actor.
  \end{enumerate}
\end{itemize}

\subsubsection{Procedimiento}
\begin{enumerate}
	\item Oprima el botón \btnEliminar del registro que desea eliminar en la pantalla \cdtFigureRef{fig:gestionarActores}{Gestionar Actores}, el sistema mostrará una pantalla emergente de confirmación.
	
	\item Oprima el botón \cdtButton{Aceptar} de la pantalla de confirmación, el sistema mostrará la pantalla \cdtFigureRef{fig:gestionarActores}{Gestionar Actores} con el mensaje de éxito.
\end{enumerate}


\subsection{Eliminar Actor}

En esta subsección se describe el procedimiento a seguir para eliminar un Actor, se podrá eliminar un Actor cuando este no tenga elementos asociados.

\subsubsection{Acciones}
\begin{itemize}
  \item Pantalla emergente \textbf{Confirmación}
  \begin{enumerate}
	\item \cdtButton{Aceptar}: Mediante este botón puede confirmar la eliminación del Actor.
	\item \cdtButton{Cancelar}: Mediante este botón puede cancelar la eliminación del Actor.
  \end{enumerate}
\end{itemize}

\subsubsection{Procedimiento}
\begin{enumerate}
	\item Oprima el botón \btnEliminar del registro que desea eliminar en la pantalla \cdtFigureRef{fig:gestionarActores}{Gestionar Actores}, el sistema mostrará una pantalla emergente de confirmación.
	
	\item Oprima el botón \cdtButton{Aceptar} de la pantalla de confirmación, el sistema mostrará la pantalla \cdtFigureRef{fig:gestionarActores}{Gestionar Actores} con el mensaje de éxito.
\end{enumerate}



