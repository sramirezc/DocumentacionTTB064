\subsection{Registrar Actor}
En esta subsección se describe el procedimiento a seguir para registrar un Actor.

\subsubsection{Datos de entrada}
\begin{description}
	\item Para registrar un Actor debe contar con los siguientes datos: \hspace{10pt}
	\begin{description}
	    \item \textbf{Información general del Actor}
	    \begin{itemize}
		  \item Nombre: Es el nombre del Actor.
		  \item Descripción: Es un texto que detalla la información general del Actor.
		  \item Cardinalidad: Sirve para indicar cuántos actores son necesarios para el sistema, uno, muchos u otra cardinalidad.
		  \item Otra cardinalidad: Es la especificación de la otra cardinalidad, se solicita cuando se selecciona la opción ``Otra'' como cardinalidad.
	    \end{itemize}
	 \end{description}
\end{description}

\subsubsection{Acciones}
\begin{itemize}
 \item \cdtButton{Aceptar}: Mediante este botón puede registrar el Actor.
 \item \cdtButton{Cancelar}: Mediante este botón puede cancelar el registro del Actor.
\end{itemize}
	
	
\subsubsection{Procedimiento}
\begin{enumerate}
	\item Oprima el botón \cdtButton{Registrar} de la pantalla \cdtFigureRef{fig:gestionarActores}{Gestionar Actores}, el sistema mostrará la pantalla \cdtFigureRef{fig:registrarActor}{Registrar Actor}. 

	%--------------------------------------------------------------------------------------------------------
	      \IUfig[.9]{CU7.1/images/registrarActor}{fig:registrarActor}{Registrar Actor}
	%--------------------------------------------------------------------------------------------------------
	
	\item Ingrese la \textbf{Información general del Actor}.
	
	\item Oprima el botón \cdtButton{Aceptar}, el sistema mostrará la pantalla \cdtFigureRef{fig:gestionarActores}{Gestionar Actores} con el mensaje de éxito.
\end{enumerate}


