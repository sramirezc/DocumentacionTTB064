\subsection{Modificar Módulo}

En esta subsección se describe el procedimiento a seguir para modificar un Módulo.


\subsubsection{Datos de entrada}
\begin{description}
	\item Para modificar un Módulo debe contar con los siguientes datos: \hspace{10pt}
	
	\begin{description}
	    \item \textbf{Información general del Módulo}
	    \begin{itemize}
		  \item Nombre: Es el nombre del Módulo.
		  \item Descripción: Es un texto que indica la información general del Módulo.
	    \end{itemize}
	 \end{description}
\end{description}

\subsubsection{Acciones}
\begin{itemize}
  \item \cdtButton{Aceptar}: Mediante este botón puede modificar el Módulo.
  \item \cdtButton{Cancelar}: Mediante este botón puede cancelar la modificación del Módulo.
\end{itemize}

\subsubsection{Procedimiento}
\begin{enumerate}
	\item Oprima el botón \btnEditar de la pantalla \cdtFigureRef{fig:gestionarModulos}{Gestionar Módulos}, el sistema mostrará la pantalla \cdtFigureRef{fig:modificarModulo}{Modificar Módulo}. 

	%--------------------------------------------------------------------------------------------------------
	      \IUfig[.9]{CU3.2/images/modificarModulo}{fig:modificarModulo}{Modificar Módulo}
	%--------------------------------------------------------------------------------------------------------
	
	\item Ingrese la \textbf{Información general del Módulo} en la pantalla.
	
	\item Oprima el botón \cdtButton{Aceptar}, el sistema mostrará la pantalla \cdtFigureRef{fig:gestionarModulos}{Gestionar Módulos} con el mensaje de éxito.
\end{enumerate}

