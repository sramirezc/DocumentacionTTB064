\subsection{Solicitar correcciones del Caso de uso}

En esta subsección se describe el procedimiento a seguir para solicitar correcciones de un Caso de uso.


\subsubsection{Datos de entrada}
\begin{description}
	\item Para solicitar las correcciones de un Caso de uso debe contar con los siguientes datos: \hspace{10pt}
	\begin{description}
	    \item Observaciones del resumen del Caso de uso: Es una descripción de los errores encontrados en la sección de información general del Caso de uso.
	    \item Observaciones de las Trayectorias del Caso de uso: Es una descripción de los errores encontrados en la sección de Trayectorias del Caso de uso.
	    \item Observaciones de los Puntos de extensión del Caso de uso: Es una descripción de los errores encontrados en la sección de Puntos de extensión del Caso de uso.
	 \end{description}
\end{description}

\subsubsection{Acciones}
\begin{itemize}
  \item \cdtButton{Aceptar}: Mediante este botón puede solicitar las correcciones del Caso de uso.
  \item \cdtButton{Cancelar}: Mediante este botón puede cancelar la solicitud de correcciones del Caso de uso.
\end{itemize}


\subsubsection{Procedimiento}
\begin{enumerate}
	\item Oprima el botón \btnSolicitarCorrecciones de la pantalla \cdtFigureRef{fig:gestionarCasosUso}{Gestionar Casos de uso}, el sistema mostrará la pantalla \cdtFigureRef{fig:solicitarCorrecciones}{Solicitar correcciones del Caso de uso}. 

	%--------------------------------------------------------------------------------------------------------
	      \IUfig[.7]{CU4.4/images/solicitarCorrecciones}{fig:solicitarCorrecciones}{Solicitar correcciones del Caso de uso}
	%--------------------------------------------------------------------------------------------------------	
	 
	\item Seleccione \textbf{Sí} para indicar que la sección de información general del Caso de uso es correcta, seleccione \textbf{No} si considera que la información no es correcta y anote las observaciones correspondientes.
	
	\item Seleccione \textbf{Sí} para indicar que la sección de Trayectorias del Caso de uso es correcta, seleccione \textbf{No} si considera que la información no es correcta y anote las observaciones correspondientes.
	
	\item Seleccione \textbf{Sí} para indicar que la sección de Puntos de extensión del Caso de uso es correcta, seleccione \textbf{No} si considera que la información no es correcta y anote las observaciones correspondientes.
	
	\item Oprima el botón \cdtButton{Aceptar}, el sistema mostrará la pantalla \cdtFigureRef{fig:gestionarCasosUso}{Gestionar Casos de uso} con el mensaje de éxito. Se solicitarán correcciones si alguna de las secciones tiene observaciones.
\end{enumerate}

