
En este capítulo se describe cómo gestionar los Términos del glosario, lo cual corresponde a registrar, modificar, eliminar y
consultar Término del glosario.\\


\begin{objetivos}
	\item Describir cómo registrar un Término del glosario.
	\item Describir cómo modificar un Término del glosario.
	\item Describir cómo eliminar un Término del glosario.
	\item Describir cómo consultar un Término del glosario.
\end{objetivos}

\section{Gestionar Términos del glosario}

Para gestionar los Términos del glosario es necesario que solicite esta operación desde el menú superior una vez que haya entrado a un Proyecto como se indica en la sección \ref{cap:gestionarProyectos}, el sistema mostrará la pantalla \cdtFigureRef{fig:gestionarTerminosGlosario}{Gestionar Términos del glosario}.

\IUfig[.9]{CU10/images/gestionarTerminosGlosario}{fig:gestionarTerminosGlosario}{Gestionar Términos del glosario}

\subsubsection{Acciones}

\begin{enumerate}
	\item \cdtButton{Registrar}: Mediante este botón puede solicitar el registro de un Término del glosario.
	\item \btnEditar[Modificar Término del glosario]: Mediante este botón puede solicitar la modificación de un Término del glosario.
	\item \btnEliminar[Eliminar Término del glosario]: Mediante este botón puede solicitar la eliminación de un Término del glosario.
	\item \btnConsulta[Consultar Término del glosario]: Mediante este botón puede solicitar la consulta de un Término del glosario.
\end{enumerate}

\subsection{Eliminar Actor}

En esta subsección se describe el procedimiento a seguir para eliminar un Actor, se podrá eliminar un Actor cuando este no tenga elementos asociados.

\subsubsection{Acciones}
\begin{itemize}
  \item Pantalla emergente \textbf{Confirmación}
  \begin{enumerate}
	\item \cdtButton{Aceptar}: Mediante este botón puede confirmar la eliminación del Actor.
	\item \cdtButton{Cancelar}: Mediante este botón puede cancelar la eliminación del Actor.
  \end{enumerate}
\end{itemize}

\subsubsection{Procedimiento}
\begin{enumerate}
	\item Oprima el botón \btnEliminar del registro que desea eliminar en la pantalla \cdtFigureRef{fig:gestionarActores}{Gestionar Actores}, el sistema mostrará una pantalla emergente de confirmación.
	
	\item Oprima el botón \cdtButton{Aceptar} de la pantalla de confirmación, el sistema mostrará la pantalla \cdtFigureRef{fig:gestionarActores}{Gestionar Actores} con el mensaje de éxito.
\end{enumerate}


\subsection{Eliminar Actor}

En esta subsección se describe el procedimiento a seguir para eliminar un Actor, se podrá eliminar un Actor cuando este no tenga elementos asociados.

\subsubsection{Acciones}
\begin{itemize}
  \item Pantalla emergente \textbf{Confirmación}
  \begin{enumerate}
	\item \cdtButton{Aceptar}: Mediante este botón puede confirmar la eliminación del Actor.
	\item \cdtButton{Cancelar}: Mediante este botón puede cancelar la eliminación del Actor.
  \end{enumerate}
\end{itemize}

\subsubsection{Procedimiento}
\begin{enumerate}
	\item Oprima el botón \btnEliminar del registro que desea eliminar en la pantalla \cdtFigureRef{fig:gestionarActores}{Gestionar Actores}, el sistema mostrará una pantalla emergente de confirmación.
	
	\item Oprima el botón \cdtButton{Aceptar} de la pantalla de confirmación, el sistema mostrará la pantalla \cdtFigureRef{fig:gestionarActores}{Gestionar Actores} con el mensaje de éxito.
\end{enumerate}


\subsection{Eliminar Actor}

En esta subsección se describe el procedimiento a seguir para eliminar un Actor, se podrá eliminar un Actor cuando este no tenga elementos asociados.

\subsubsection{Acciones}
\begin{itemize}
  \item Pantalla emergente \textbf{Confirmación}
  \begin{enumerate}
	\item \cdtButton{Aceptar}: Mediante este botón puede confirmar la eliminación del Actor.
	\item \cdtButton{Cancelar}: Mediante este botón puede cancelar la eliminación del Actor.
  \end{enumerate}
\end{itemize}

\subsubsection{Procedimiento}
\begin{enumerate}
	\item Oprima el botón \btnEliminar del registro que desea eliminar en la pantalla \cdtFigureRef{fig:gestionarActores}{Gestionar Actores}, el sistema mostrará una pantalla emergente de confirmación.
	
	\item Oprima el botón \cdtButton{Aceptar} de la pantalla de confirmación, el sistema mostrará la pantalla \cdtFigureRef{fig:gestionarActores}{Gestionar Actores} con el mensaje de éxito.
\end{enumerate}




