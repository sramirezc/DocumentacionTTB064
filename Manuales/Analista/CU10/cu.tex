
En este capítulo se describe cómo gestionar los Términos del glosario, lo cual corresponde a registrar, modificar, eliminar y
consultar Término del glosario.\\


\begin{objetivos}
	\item Describir cómo registrar un Término del glosario.
	\item Describir cómo modificar un Término del glosario.
	\item Describir cómo eliminar un Término del glosario.
	\item Describir cómo consultar un Término del glosario.
\end{objetivos}

\section{Gestionar Términos del glosario}

Para gestionar los Términos del glosario es necesario que solicite esta operación desde el menú superior una vez que haya entrado a un Proyecto como se indica en la sección \ref{cap:gestionarProyectos}, el sistema mostrará la pantalla \cdtFigureRef{fig:gestionarTerminosGlosario}{Gestionar Términos del glosario}.

\IUfig[.9]{CU10/images/gestionarTerminosGlosario}{fig:gestionarTerminosGlosario}{Gestionar Términos del glosario}

\subsubsection{Acciones}

\begin{enumerate}
	\item \cdtButton{Registrar}: Mediante este botón puede solicitar el registro de un Término del glosario.
	\item \btnEditar[Modificar Término del glosario]: Mediante este botón puede solicitar la modificación de un Término del glosario.
	\item \btnEliminar[Eliminar Término del glosario]: Mediante este botón puede solicitar la eliminación de un Término del glosario.
	\item \btnConsulta[Consultar Término del glosario]: Mediante este botón puede solicitar la consulta de un Término del glosario.
\end{enumerate}

\begin{UseCase}{CU 5.1.3}{Gestionar postcondiciones}
	{
		Este caso de uso permite al analista visualizar las postcondiciones del caso de uso. También permite 
		al actor acceder a las operaciones de registro, modificación y eliminación de las postcondiciones.
	}

	\UCitem{Actor}{\cdtRef{actor:liderAnalisis}{Líder de análisis}, \cdtRef{actor:analista}{Analista}}
	\UCitem{Propósito}{
		Revisar y gestionar las postcondiciones de un caso de uso.
	}
	\UCitem{Entradas}{
		Ninguna
	}
	\UCitem{Salidas}{
		\begin{UClist}
			\UCli \cdtRef{Postcondicion}{Postcondición}: \ioTabla{el \cdtRef{Postcondicion:Numero}{Número} y la \cdtRef{Postcondicion:Redacción}{Redacción}}.
		\end{UClist}
	}
	
	\UCitem{Mensajes}{
		\begin{UClist}
			\UCli \cdtIdRef{MSG2}{No existe información}: Se muestra en la pantalla \cdtIdRef{IU 5.1}{Registrar caso de uso} o \cdtIdRef{IU 5.2}{Modificar caso de uso} cuando no existen postcondiciones registradas.
		\end{UClist}
	}

	\UCitem{Postcondiciones}{
		Ninguna
	}
	
	\UCitem{Postcondiciones}{
		\begin{UClist}
			\UCli Se podrá solicitar el registro de una postcondición por medio del caso de uso \cdtIdRef{CU 5.1.3.1}{Registrar postcondición}.
			\UCli Se podrá solicitar la modificación de una postcondición por medio del caso de uso \cdtIdRef{CU 5.1.3.2}{Modificar postcondición}.
			\UCli Se podrá solicitar la eliminación de una postcondición por medio del caso de uso \cdtIdRef{CU 5.1.3.3}{Eliminar postcondición}.
		\end{UClist}
	}

	\UCitem{Errores}{
		Ninguno
	}

	\UCitem{Tipo}{
		Secundario, extiende del caso de uso \cdtIdRef{CU 5.1}{Registrar caso de uso}.
	}
\end{UseCase}
%-------------------------------------------------------%trayectoria Principal-----------------------------------------------
 \begin{UCtrayectoria}
    \UCpaso[\UCactor] Solicita gestionar las postcondiciones del caso de uso seleccionando la opción ``Postcondiciones'' de la pantalla \cdtIdRef{IU 5.1}{Registrar caso de uso}.
    \UCpaso[\UCsist] Busca información de las postcondiciones del caso de uso. \refTray{A}
    \UCpaso[\UCsist] Muestra la información de las postcondiciones en la pantalla \cdtIdRef{IU 5.1}{Registrar caso de uso} o \cdtIdRef{IU 5.2}{Modificar caso de uso}. 
    \UCpaso[\UCactor] Gestiona las postcondiciones a través de los botones: \btnAgregar, \btnEditar y \btnEliminar. \label{cu5.1.3:gestionaCU}
 \end{UCtrayectoria}
 
 \begin{UCtrayectoriaA}[Fin del caso de uso]{A}{No existen registros de postcondiciones.}
    \UCpaso[\UCsist] Muestra el mensaje \cdtIdRef{MSG2}{No existe información} en pantalla \cdtIdRef{IU 5.1.3}{Gestionar postcondiciones} 
    para indicar que no hay registros de postcondiciones para mostrar.
 \end{UCtrayectoriaA}
 

\subsection{Puntos de extensión}

\UCExtensionPoint{El actor requiere registrar una postcondición}
	{Paso \ref{cu5.1.3:gestionaCU}}
	{\cdtIdRef{CU 5.1.3.1}{Registrar postcondición}}
\UCExtensionPoint{El actor requiere modificar una precondición}
	{Paso \ref{cu5.1.3:gestionaCU}}
	{\cdtIdRef{CU 5.1.3.2}{Modificar postcondición}}
\UCExtensionPoint{El actor requiere eliminar una precondición}
	{Paso \ref{cu5.1.3:gestionaCU}}
	{\cdtIdRef{CU 5.1.3.3}{Eliminar postcondición}}
  
\begin{UseCase}{CU 5.1.3}{Gestionar postcondiciones}
	{
		Este caso de uso permite al analista visualizar las postcondiciones del caso de uso. También permite 
		al actor acceder a las operaciones de registro, modificación y eliminación de las postcondiciones.
	}

	\UCitem{Actor}{\cdtRef{actor:liderAnalisis}{Líder de análisis}, \cdtRef{actor:analista}{Analista}}
	\UCitem{Propósito}{
		Revisar y gestionar las postcondiciones de un caso de uso.
	}
	\UCitem{Entradas}{
		Ninguna
	}
	\UCitem{Salidas}{
		\begin{UClist}
			\UCli \cdtRef{Postcondicion}{Postcondición}: \ioTabla{el \cdtRef{Postcondicion:Numero}{Número} y la \cdtRef{Postcondicion:Redacción}{Redacción}}.
		\end{UClist}
	}
	
	\UCitem{Mensajes}{
		\begin{UClist}
			\UCli \cdtIdRef{MSG2}{No existe información}: Se muestra en la pantalla \cdtIdRef{IU 5.1}{Registrar caso de uso} o \cdtIdRef{IU 5.2}{Modificar caso de uso} cuando no existen postcondiciones registradas.
		\end{UClist}
	}

	\UCitem{Postcondiciones}{
		Ninguna
	}
	
	\UCitem{Postcondiciones}{
		\begin{UClist}
			\UCli Se podrá solicitar el registro de una postcondición por medio del caso de uso \cdtIdRef{CU 5.1.3.1}{Registrar postcondición}.
			\UCli Se podrá solicitar la modificación de una postcondición por medio del caso de uso \cdtIdRef{CU 5.1.3.2}{Modificar postcondición}.
			\UCli Se podrá solicitar la eliminación de una postcondición por medio del caso de uso \cdtIdRef{CU 5.1.3.3}{Eliminar postcondición}.
		\end{UClist}
	}

	\UCitem{Errores}{
		Ninguno
	}

	\UCitem{Tipo}{
		Secundario, extiende del caso de uso \cdtIdRef{CU 5.1}{Registrar caso de uso}.
	}
\end{UseCase}
%-------------------------------------------------------%trayectoria Principal-----------------------------------------------
 \begin{UCtrayectoria}
    \UCpaso[\UCactor] Solicita gestionar las postcondiciones del caso de uso seleccionando la opción ``Postcondiciones'' de la pantalla \cdtIdRef{IU 5.1}{Registrar caso de uso}.
    \UCpaso[\UCsist] Busca información de las postcondiciones del caso de uso. \refTray{A}
    \UCpaso[\UCsist] Muestra la información de las postcondiciones en la pantalla \cdtIdRef{IU 5.1}{Registrar caso de uso} o \cdtIdRef{IU 5.2}{Modificar caso de uso}. 
    \UCpaso[\UCactor] Gestiona las postcondiciones a través de los botones: \btnAgregar, \btnEditar y \btnEliminar. \label{cu5.1.3:gestionaCU}
 \end{UCtrayectoria}
 
 \begin{UCtrayectoriaA}[Fin del caso de uso]{A}{No existen registros de postcondiciones.}
    \UCpaso[\UCsist] Muestra el mensaje \cdtIdRef{MSG2}{No existe información} en pantalla \cdtIdRef{IU 5.1.3}{Gestionar postcondiciones} 
    para indicar que no hay registros de postcondiciones para mostrar.
 \end{UCtrayectoriaA}
 

\subsection{Puntos de extensión}

\UCExtensionPoint{El actor requiere registrar una postcondición}
	{Paso \ref{cu5.1.3:gestionaCU}}
	{\cdtIdRef{CU 5.1.3.1}{Registrar postcondición}}
\UCExtensionPoint{El actor requiere modificar una precondición}
	{Paso \ref{cu5.1.3:gestionaCU}}
	{\cdtIdRef{CU 5.1.3.2}{Modificar postcondición}}
\UCExtensionPoint{El actor requiere eliminar una precondición}
	{Paso \ref{cu5.1.3:gestionaCU}}
	{\cdtIdRef{CU 5.1.3.3}{Eliminar postcondición}}
  
\begin{UseCase}{CU 5.1.3}{Gestionar postcondiciones}
	{
		Este caso de uso permite al analista visualizar las postcondiciones del caso de uso. También permite 
		al actor acceder a las operaciones de registro, modificación y eliminación de las postcondiciones.
	}

	\UCitem{Actor}{\cdtRef{actor:liderAnalisis}{Líder de análisis}, \cdtRef{actor:analista}{Analista}}
	\UCitem{Propósito}{
		Revisar y gestionar las postcondiciones de un caso de uso.
	}
	\UCitem{Entradas}{
		Ninguna
	}
	\UCitem{Salidas}{
		\begin{UClist}
			\UCli \cdtRef{Postcondicion}{Postcondición}: \ioTabla{el \cdtRef{Postcondicion:Numero}{Número} y la \cdtRef{Postcondicion:Redacción}{Redacción}}.
		\end{UClist}
	}
	
	\UCitem{Mensajes}{
		\begin{UClist}
			\UCli \cdtIdRef{MSG2}{No existe información}: Se muestra en la pantalla \cdtIdRef{IU 5.1}{Registrar caso de uso} o \cdtIdRef{IU 5.2}{Modificar caso de uso} cuando no existen postcondiciones registradas.
		\end{UClist}
	}

	\UCitem{Postcondiciones}{
		Ninguna
	}
	
	\UCitem{Postcondiciones}{
		\begin{UClist}
			\UCli Se podrá solicitar el registro de una postcondición por medio del caso de uso \cdtIdRef{CU 5.1.3.1}{Registrar postcondición}.
			\UCli Se podrá solicitar la modificación de una postcondición por medio del caso de uso \cdtIdRef{CU 5.1.3.2}{Modificar postcondición}.
			\UCli Se podrá solicitar la eliminación de una postcondición por medio del caso de uso \cdtIdRef{CU 5.1.3.3}{Eliminar postcondición}.
		\end{UClist}
	}

	\UCitem{Errores}{
		Ninguno
	}

	\UCitem{Tipo}{
		Secundario, extiende del caso de uso \cdtIdRef{CU 5.1}{Registrar caso de uso}.
	}
\end{UseCase}
%-------------------------------------------------------%trayectoria Principal-----------------------------------------------
 \begin{UCtrayectoria}
    \UCpaso[\UCactor] Solicita gestionar las postcondiciones del caso de uso seleccionando la opción ``Postcondiciones'' de la pantalla \cdtIdRef{IU 5.1}{Registrar caso de uso}.
    \UCpaso[\UCsist] Busca información de las postcondiciones del caso de uso. \refTray{A}
    \UCpaso[\UCsist] Muestra la información de las postcondiciones en la pantalla \cdtIdRef{IU 5.1}{Registrar caso de uso} o \cdtIdRef{IU 5.2}{Modificar caso de uso}. 
    \UCpaso[\UCactor] Gestiona las postcondiciones a través de los botones: \btnAgregar, \btnEditar y \btnEliminar. \label{cu5.1.3:gestionaCU}
 \end{UCtrayectoria}
 
 \begin{UCtrayectoriaA}[Fin del caso de uso]{A}{No existen registros de postcondiciones.}
    \UCpaso[\UCsist] Muestra el mensaje \cdtIdRef{MSG2}{No existe información} en pantalla \cdtIdRef{IU 5.1.3}{Gestionar postcondiciones} 
    para indicar que no hay registros de postcondiciones para mostrar.
 \end{UCtrayectoriaA}
 

\subsection{Puntos de extensión}

\UCExtensionPoint{El actor requiere registrar una postcondición}
	{Paso \ref{cu5.1.3:gestionaCU}}
	{\cdtIdRef{CU 5.1.3.1}{Registrar postcondición}}
\UCExtensionPoint{El actor requiere modificar una precondición}
	{Paso \ref{cu5.1.3:gestionaCU}}
	{\cdtIdRef{CU 5.1.3.2}{Modificar postcondición}}
\UCExtensionPoint{El actor requiere eliminar una precondición}
	{Paso \ref{cu5.1.3:gestionaCU}}
	{\cdtIdRef{CU 5.1.3.3}{Eliminar postcondición}}
  


