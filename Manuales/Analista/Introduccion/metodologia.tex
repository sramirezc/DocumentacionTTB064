Para el desarrollo del prototipo se utliza un modelo de proceso incremental, se planea realizar el proyecto en dos incrementos: 
\begin{itemize}
	\item Editor de casos de uso 
	\item Generador de pruebas
\end{itemize}

A continuación se muestran los módulos del {\it Editor de casos de uso} que corresponde al Trabajo Terminal I con sus respectivos objetivos:

\begin{itemize}
	\item Administrador. El objetivo de este módulo es permitir la gestión de proyectos y de personal de la organización. Este módulo representa el {\it 5 \%} del incremento.
	\item Módulos. El objetivo de este módulo es permitir la gestión de módulos en un proyecto. Este módulo representa el {\it 5 \%} del incremento.
	\item Gestión de casos de uso. El objetivo de este módulo es permitir la gestión de casos de uso en un proyecto. Este módulo presenta la parte medular del Trabajo Terminal, ya que conjunta todos los módulos. El porcentaje aproximado de este módulo es del {\it 50 \%} del incremento.
	\item Gestión de reglas de negocio. El objetivo de este módulo es permitir la gestión de reglas de negocio. Este módulo representa el {\it 10 \%} del incremento.
	\item Actores. El objetivo de este módulo es permitir la gestión de actor. Este módulo representa el {\it 5 \%} del incremento.
	\item Mensajes. El objetivo de este módulo es permitir la gestión de mensajes. Este módulo representa el {\it 5 \%} del incremento.
	\item Términos del glosario. El objetivo de este módulo es permitir la gestión de términos del glosario. Este módulo representa el {\it 5 \%} del incremento.
	\item Líder de análisis. El objetivo de este módulo es permitir realizar la elección de participantes para un proyecto, así como la liberación de casos de uso. Este módulo representa el {\it 5 \%} del incremento.
	\item Entidades. El objetivo de este módulo es permitir la gestión de las entidades y atributos. Este módulo representa el {\it 5 \%} del incremento.
	\item Interfaces de usuario. El objetivo de este módulo es permitir la gestión de las pantallas o interfaces de usuario con las que interactúan los casos de uso. Este módulo representa el {\it 5 \%} del incremento.
\end{itemize}

%Debido a lo grande que resultó la curva de aprendizaje y al periodo escolar reducido, se acotó el alcance de este incremento, por lo que se desarrolló principalmente el módulo de gestión de casos de uso por resultar crítico para el prototipo.

