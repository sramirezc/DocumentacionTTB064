\subsection{Revisar Caso de uso}

En esta subsección se describe el procedimiento a seguir para revisar el contenido de un Caso de uso, dejarlo en estado ``Por liberar'' o liberarlo directamente en caso de que sea líder del Proyecto. 
Si alguna de las secciones es incorrecta puede anotar sus observaciones y el Caso de uso cambiará a estado ``Pendiente de corrección'' para que algún analista lo corrija.


\subsubsection{Datos de entrada}
\begin{description}
	\item Para revisar un Caso de uso debe contar con los siguientes datos: \hspace{10pt}
	\begin{description}
	    \item Observaciones del resumen del Caso de uso: Es una descripción de los errores encontrados en la sección de información general del Caso de uso.
	    \item Observaciones de las Trayectorias del Caso de uso: Es una descripción de los errores encontrados en la sección de Trayectorias del Caso de uso.
	    \item Observaciones de los Puntos de extensión del Caso de uso: Es una descripción de los errores encontrados en la sección de Puntos de extensión del Caso de uso.
	 \end{description}
\end{description}

\subsubsection{Acciones}
\begin{itemize}
  \item \cdtButton{Aceptar}: Mediante este botón puede revisar el Caso de uso.
  \item \cdtButton{Cancelar}: Mediante este botón puede cancelar la revisión del Caso de uso.
\end{itemize}


\subsubsection{Procedimiento}
\begin{enumerate}
	\item Oprima el botón \btnRevisar de la pantalla \cdtFigureRef{fig:gestionarCasosUso}{Gestionar Casos de uso}, el sistema mostrará la pantalla \cdtFigureRef{fig:revisarCasoUso}{Revisar Caso de uso}. 

	%--------------------------------------------------------------------------------------------------------
	      \IUfig[.7]{CU5.5/images/revisarCasoUso}{fig:revisarCasoUso}{Revisar Caso de uso}
	%--------------------------------------------------------------------------------------------------------
	 
	\item Seleccione \textbf{Sí} para indicar que la sección de información general del Caso de uso es correcta, seleccione \textbf{No} si considera que la información no es correcta y anote las observaciones correspondientes.
	
	\item Seleccione \textbf{Sí} para indicar que la sección de Trayectorias del Caso de uso es correcta, seleccione \textbf{No} si considera que la información no es correcta y anote las observaciones correspondientes.
	
	\item Seleccione \textbf{Sí} para indicar que la sección de Puntos de extensión del Caso de uso es correcta, seleccione \textbf{No} si considera que la información no es correcta y anote las observaciones correspondientes.
	
	\item Oprima el botón \cdtButton{Aceptar}, el sistema mostrará la pantalla \cdtFigureRef{fig:gestionarCasosUso}{Gestionar Casos de uso} con el mensaje de éxito. Si usted es el líder del Proyecto, el Caso de uso se liberará 
	automáticamente en caso de no tener observaciones, en caso de que usted no sea el líder del Proyecto, entonces el Caso de uso pasará a estado ``Por liberar''.
\end{enumerate}

