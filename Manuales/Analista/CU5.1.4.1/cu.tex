\subsection{Registrar Punto de extensión}
En esta subsección se describe el procedimiento a seguir para registrar un Punto de extensión.

\subsubsection*{Datos de entrada}
\begin{description}
	\item Para registrar un Punto de extensión debe contar con los siguientes datos: \hspace{10pt}
	
	\begin{description}
	    \item \textbf{Información general del Punto de extensión}
	    \begin{itemize}
		  \item Extiende a: Permite elegir el Caso de uso que será extendido.
		  \item Causa: Es la razón que permite que se ejecute el Caso de uso extendido.
		  \item Región de la trayectoria: Es una descripción de los pasos en donde se puede ejecutar el Caso de uso extendido.
	    \end{itemize}
	 \end{description}
\end{description}

\subsubsection*{Acciones}
\begin{itemize}
  \item \cdtButton{Aceptar}: Mediante este botón puede registrar el Punto de extensión.
  \item \cdtButton{Cancelar}: Mediante este botón puede cancelar el registro del Punto de extensión.
\end{itemize}
	
	
\subsubsection*{Procedimiento}
\begin{enumerate}
	\item Oprima el botón \cdtButton{Registrar} de la pantalla \cdtFigureRef{fig:gestionarExtension}{Gestionar Puntos de extensión}, el sistema mostrará la pantalla \cdtFigureRef{fig:registrarExtension}{Registrar Punto de extensión}. 

	%--------------------------------------------------------------------------------------------------------
	      \IUfig[.9]{CU5.1.4.1/images/registrarExtension}{fig:registrarExtension}{Registrar Punto de extensión}
	%--------------------------------------------------------------------------------------------------------
	
	\item Ingrese la \textbf{Información general del Punto de extensión}, en la caja de texto \textbf{Región} puede referenciar Pasos como se indica en la sección \ref{cap:token}.
	
	\item Oprima el botón \cdtButton{Aceptar}, el sistema mostrará la pantalla \cdtFigureRef{fig:gestionarExtension}{Gestionar Puntos de extensión} con el mensaje de éxito.
\end{enumerate}
