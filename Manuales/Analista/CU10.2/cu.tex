\subsection{Modificar Término del glosario}
En esta subsección se describe el procedimiento a seguir para modificar un Término del glosario.

\subsubsection{Datos de entrada}
\begin{description}
	\item Para modificar un Término del glosario debe contar con los siguientes datos: \hspace{10pt}
	\begin{description}
	    \item \textbf{Información general del Término del glosario}
	    \begin{itemize}
		  \item Nombre: Es el nombre del Término del glosario.
		  \item Descripción: Es un texto que detalla la información general del Término del glosario.
	    \end{itemize}
	 \end{description}
\end{description}

\subsubsection{Acciones}
\begin{itemize}
 \item \cdtButton{Aceptar}: Mediante este botón puede modificar el Término del glosario.
 \item \cdtButton{Cancelar}: Mediante este botón puede cancelar la modificación del Término del glosario.
\end{itemize}
	
	
\subsubsection{Procedimiento}
\begin{enumerate}
	\item Oprima el botón \btnEditar del Término del glosario que desea modificar en la pantalla \cdtFigureRef{fig:gestionarTerminosGlosario}{Gestionar Términos del glosario}, el sistema mostrará la pantalla \cdtFigureRef{fig:modificarTerminoGlosario}{Modificar Término del glosario}. 

	%--------------------------------------------------------------------------------------------------------
	      \IUfig[.9]{CU10.2/images/modificarTerminoGlosario}{fig:modificarTerminoGlosario}{Modificar Término del glosario}
	%--------------------------------------------------------------------------------------------------------
	
	\item Modifique la \textbf{Información general del Término del glosario}.
	
	\item Oprima el botón \cdtButton{Aceptar}, el sistema mostrará la pantalla \cdtFigureRef{fig:gestionarTerminosGlosario}{Gestionar Términos del glosario} con el mensaje de éxito.
\end{enumerate}


