\subsection{Registrar Entidad}
En esta subsección se describe el procedimiento a seguir para registrar una Entidad.

\subsubsection{Datos de entrada}
\begin{description}
	\item Para registrar una Entidad debe contar con los siguientes datos: \hspace{10pt}
	\begin{description}
	    \item \textbf{Información general de la Entidad}
	    \begin{itemize}
		  \item Nombre: Es el nombre de la Entidad.
		  \item Descripción: Es un texto que detalla la información general de la Entidad.
	    \end{itemize}
	    \item \textbf{Información general del Atributo}
	    \begin{itemize}
		  \item Nombre: Es el nombre de la Atributo.
		  \item Descripción: Es un texto que detalla la información general de la Atributo.
		  \item Tipo de dato: Sive para definir el tipo de dato del Atributo.
		  \item Obligatorio: Permite indicar si el Atributo es un dato obligatorio u opcional.
		  \item {\bf Tipo de dato ``Archivo''}
		  \begin{itemize}
		   \item Formato de archivo: Es el formato del archivo, se debe indicar la extensión o las extensiones en caso de ser más de un formato.
		   \item Tamaño máximo de archivo: Es un número que indica el tamaño máximo permitido, la unidad se selecciona el la lista desplegable que se encuentra del lado derecho del campo.
		  \end{itemize}
		  \item {\bf Tipo de dato ``Cadena'', ``Entero'' o ``Flotante''}
		  \begin{itemize}
		   \item Longitud máxima: Es el número máximo de caracteres permitidos en el campo.
		  \end{itemize}
		  \item {\bf Tipo de dato ``Otro''}
		  \begin{itemize}
		   \item Especifique: Es la especificación del tipo de dato en caso de no encontrar el tipo de dato en la lista.
		  \end{itemize}
		  
	    \end{itemize}
	 \end{description}
\end{description}

\subsubsection{Atributos}
\begin{itemize}
  \item Sección \textbf{Atributos de la Entidad}
  \begin{enumerate}
	\item \cdtButton{Registrar}: Mediante este botón puede solicitar el registro de un Atributo.
	\item \btnEditar[Modificar Atributo]: Mediante este botón puede solicitar la modificación de un Atributo.
	\item \btnEliminar[Eliminar Atributo]: Mediante este botón puede solicitar la eliminación de un Atributo.
  \end{enumerate}
  \item Pantalla emergente \textbf{Atributo}
  \begin{enumerate}
	\item \cdtButton{Aceptar}: Mediante este botón puede registrar el Atributo.
	\item \cdtButton{Cancelar}: Mediante este botón puede cancelar el registro del Atributo.
  \end{enumerate}
 \item \cdtButton{Aceptar}: Mediante este botón puede registrar la Entidad.
 \item \cdtButton{Cancelar}: Mediante este botón puede cancelar el registro de la Entidad.
\end{itemize}
	
	
\subsubsection{Procedimiento}
\begin{enumerate}
	\item Oprima el botón \cdtButton{Registrar} de la pantalla \cdtFigureRef{fig:gestionarEntidades}{Gestionar Entidades}, el sistema mostrará la pantalla \cdtFigureRef{fig:registrarEntidad}{Registrar Entidad}. 

	%--------------------------------------------------------------------------------------------------------
	      \IUfig[.9]{CU11.1/images/registrarEntidad}{fig:registrarEntidad}{Registrar Entidad}
	%--------------------------------------------------------------------------------------------------------
	
	\item Ingrese la \textbf{Información general de la Entidad}.
	
	\item Oprima el botón \cdtButton{Registrar} de la sección \textbf{Atributos de la Entidad}, el sistema mostrará la pantalla emergente \cdtFigureRef{fig:registrarAtributo}{Registrar Atributo}. \label{cu11.1:paso:registrarAtributo} 

	%--------------------------------------------------------------------------------------------------------
	      \IUfig[.9]{CU11.1/images/registrarAtributo}{fig:registrarAtributo}{Registrar Atributo}
	%--------------------------------------------------------------------------------------------------------

	\item Ingrese la información de la pantalla emergente \textbf{Atributo} y la información referente al tipo de dato seleccionado.
	
	\item Oprima el botón \cdtButton{Aceptar}, el sistema mostrará la pantalla \cdtFigureRef{fig:registrarEntidad}{Registrar Entidad} con el Atributo en la sección \textbf{Atributos de la Entidad}.
	
	\item Registre todos los Atributos repitiendo el procedimiento que se indica a partir del paso \ref{cu11.1:paso:registrarAtributo}.
	
	\item Oprima el botón \cdtButton{Aceptar}, el sistema mostrará la pantalla \cdtFigureRef{fig:gestionarEntidades}{Gestionar Entidades} con el mensaje de éxito.
\end{enumerate}


