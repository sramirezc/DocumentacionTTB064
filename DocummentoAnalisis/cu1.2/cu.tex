\newpage 
\begin{UseCase}{CU 1.2}{Modificar proyecto}
	{
		Este caso de uso permite modificar la información de un proyecto.
	}
	
	\UCitem{Actor}{\cdtRef{actor:administrador}{Administrador}}
	\UCitem{Propósito}{
		Modificar la información de un proyecto.
	}
	\UCitem{Entradas}{
		\begin{UClist}
			\UCli \cdtRef{Proyecto:Clave}{Clave}: \ioEscribir.
			\UCli \cdtRef{Proyecto:Nombre}{Nombre}: \ioEscribir.
			\UCli \cdtRef{Proyecto:FechaDeInicio}{Fecha de inicio}: \ioCalendario.
			\UCli \cdtRef{Proyecto:FechaDeTermino}{Fecha de término}: \ioCalendario.
			\UCli \cdtRef{Proyecto:FechaDeInicioProgramada}{Fecha de inicio programada}: \ioCalendario.
			\UCli \cdtRef{Proyecto:FechaDeTerminoProgramada}{Fecha de término programada}: \ioCalendario.
			\UCli \cdtRef{Proyecto:Lider}{Líder del Proyecto}: \ioSeleccionar.
			\UCli \cdtRef{Proyecto:Descripcion}{Descripción}: \ioEscribir.
			\UCli \cdtRef{Proyecto:Contraparte}{Contraparte}: \ioEscribir.
			\UCli \cdtRef{Proyecto:Presupuesto}{Presupuesto}: \ioEscribir.
			\UCli \cdtRef{gls:EstadoDelProyecto}{Estado del proyecto}: \ioSeleccionar.
		\end{UClist}
	}
	\UCitem{Salidas}{
		\begin{UClist}
			\UCli \cdtRef{Proyecto:Clave}{Clave}: \ioObtener.
			\UCli \cdtRef{Proyecto:Nombre}{Nombre}: \ioObtener.
			\UCli \cdtRef{Proyecto:FechaDeInicio}{Fecha de inicio}: \ioObtener.
			\UCli \cdtRef{Proyecto:FechaDeTermino}{Fecha de término}: \ioObtener.
			\UCli \cdtRef{Proyecto:FechaDeInicioProgramada}{Fecha de Inicio Programada}: \ioObtener.
			\UCli \cdtRef{Proyecto:FechaDeTerminoProgramada}{Fecha de Término Programada}: \ioObtener.
			\UCli \cdtRef{Proyecto:Lider}{Líder del Proyecto}: \ioObtener.
			\UCli \cdtRef{Proyecto:Descripcion}{Descripcion}: \ioObtener.
			\UCli \cdtRef{Proyecto:Contraparte}{Contraparte}: \ioObtener.
			\UCli \cdtRef{Proyecto:Presupuesto}{Presupuesto}: \ioObtener.
			\UCli \cdtRef{gls:EstadoDelProyecto}{Estado del Proyecto}: \ioObtener.
		\end{UClist}
	}
	
	\UCitem{Mensajes}{
		\begin{UClist}
			\UCli \cdtIdRef{MSG1}{Operación exitosa}: Se muestra en la pantalla \cdtIdRef{IU 1}{Gestionar proyectos de Administrador} para indicar que la modificación fue exitosa.
		\end{UClist}
	}

	\UCitem{Precondiciones}{
		\begin{UClist}
			\UCli Que el proyecto se encuentre en estado ``En negociación'' o ``Iniciado''.
		\end{UClist}
	}
	
	\UCitem{Postcondiciones}{
		\begin{UClist}
			\UCli Se modificará la información del proyecto en el sistema.
		\end{UClist}
	}

	\UCitem{Errores}{
		\begin{UClist}
			\UCli \cdtIdRef{MSG4}{Dato obligatorio}: Se muestra en la pantalla \cdtIdRef{IU 1.2}{Modificar proyecto} cuando no se ha ingresado un dato marcado como obligatorio.
			\UCli \cdtIdRef{MSG5}{Dato incorrecto}: Se muestra en la pantalla \cdtIdRef{IU 1.2}{Modificar proyecto} cuando el tipo de dato ingresado no cumple con el tipo de dato solicitado en el campo.
			\UCli \cdtIdRef{MSG6}{Longitud inválida}: Se muestra en la pantalla \cdtIdRef{IU 1.2}{Modificar proyecto} cuando se ha excedido la longitud de alguno de los campos.
			\UCli \cdtIdRef{MSG7}{Registro repetido}: Se muestra en la pantalla \cdtIdRef{IU 1.2}{Modificar proyecto} cuando se registre un proyecto con un nombre o clave que ya exista.
			\UCli \cdtIdRef{MSG13}{Ha ocurrido un error}: Se muestra en la pantalla \cdtIdRef{IU 1}{Gestionar proyectos} cuando no exista información de los estados de un proyecto.
			\UCli \cdtIdRef{MSG22}{Falta información}: Se muestra en la pantalla \cdtIdRef{IU 1}{Gestionar proyectos} cuando no existan colaboradores registrados.
		\end{UClist}
	}

	\UCitem{Tipo}{
		Secundario, extiende del caso de uso \cdtIdRef{CU 1}{Gestionar proyectos de Administrador}.
	}
\end{UseCase}
	%-------------------------------------------------------%trayectoria Principal-----------------------------------------------
	 \begin{UCtrayectoria}
    \UCpaso[\UCactor] Solicita modificar un proyecto oprimiendo el botón \btnEditar del proyecto que desea modificar de la pantalla \cdtIdRef{IU 1}{Gestionar proyectos de Administrador}.
	    \UCpaso[\UCsist] Verifica que exista información referente a los estados de un proyecto, con base en la regla de negocio \cdtIdRef{RN20}{Verificación de catálogos}. \refTray{A}
	    \UCpaso[\UCsist] Verifica que exista al menos un colaborador, con base en la regla de negocio \cdtIdRef{RN20}{Verificación de catálogos}. \refTray{B}
	    \UCpaso[\UCsist] Busca la información del proyecto seleccionado.
		
	    \UCpaso[\UCsist] Muestra la pantalla \cdtIdRef{IU 1.2}{Modificar proyecto} con la información del proyecto.
	    \UCpaso[\UCactor] Ingresa la información solicitada en la pantalla. \label{cu1.2:ingresaDatos}
	    \UCpaso[\UCactor] Solicita guardar el proyecto oprimiendo el botón \cdtButton{Aceptar} de la pantalla \cdtIdRef{IU 1.2}{Modificar proyecto}. \refTray{C} 
    
	    \UCpaso[\UCsist] Verifica que el actor ingrese todos los campos obligatorios con base en la regla de negocio  \cdtIdRef{RN8}{Datos obligatorios}. \refTray{D}
	    \UCpaso[\UCsist] Verifica que el nombre del proyecto no se encuentre registrado en el sistema con base en la regla de negocio  \cdtIdRef{RN6}{Unicidad de nombres}. \refTray{E}
	    \UCpaso[\UCsist] Verifica que la clave del proyecto no se encuentre registrada en el sistema con base en la regla de negocio  \cdtIdRef{RN22}{Unicidad de la clave del Proyecto}. \refTray{F}
	    \UCpaso[\UCsist] Verifica que los datos requeridos sean proporcionados correctamente como se especifica en la regla de negocio \cdtIdRef{RN7}{Información correcta}. \refTray{G} \refTray{H}
	    \UCpaso[\UCsist] Verifica que la fecha de término sea posterior a la fecha de inicio. \refTray{I}
	    \UCpaso[\UCsist] Verifica que la fecha de término programada sea posterior a la fecha de inicio programada. \refTray{J}
	    \UCpaso[\UCsist] Registra la información del proyecto en el sistema.
	    \UCpaso[\UCsist] Muestra el mensaje \cdtIdRef{MSG1}{Operación exitosa} en la pantalla \cdtIdRef{IU 1}{Gestionar proyectos de Administrador}
	    para indicar al actor que el registro se ha realizado exitosamente.
	 \end{UCtrayectoria}

	 %----------------------------------------------------------%trayectoria A---------------------------------------------------- 
	 \begin{UCtrayectoriaA}[Fin del caso de uso]{A}{No hay estados con los que puede iniciar un proyecto.}
	    \UCpaso[\UCsist] Muestra el mensaje \cdtIdRef{MSG13}{Ha ocurrido un error} en la la pantalla \cdtIdRef{IU 1}{Gestionar proyectos de Administrador} para indicar que no es posible realizar la operación debido a la falta de información necesaria para el sistema.
	 \end{UCtrayectoriaA} 
	 %----------------------------------------------------------%trayectoria B---------------------------------------------------- 
	 \begin{UCtrayectoriaA}[Fin del caso de uso]{B}{No hay ningún colaborador registrado.}
	    \UCpaso[\UCsist] Muestra el mensaje \cdtIdRef{MSG22}{Falta información} en la la pantalla \cdtIdRef{IU 1}{Gestionar proyectos de Administrador} indicando que no hay colaboradores registrados.
	 \end{UCtrayectoriaA} 
	 %----------------------------------------------------------%trayectoria C---------------------------------------------------- 
	 \begin{UCtrayectoriaA}[Fin del caso de uso]{C}{El actor desea cancelar la operación.}
	    \UCpaso[\UCactor] Solicita cancelar la operación oprimiendo el botón \cdtButton{Cancelar} de la pantalla \cdtIdRef{IU 1.2}{Modificar proyecto}.
	    \UCpaso[\UCsist] Muestra la pantalla donde se solicitó la operación.
	 \end{UCtrayectoriaA} 
	  %----------------------------------------------------------%trayectoria D---------------------------------------------------- 
	 \begin{UCtrayectoriaA}{D}{El actor no ingresó algún dato marcado como obligatorio.}
	    \UCpaso[\UCsist] Muestra el mensaje \cdtIdRef{MSG4}{Dato obligatorio} y señala el campo que presenta el error en la pantalla 
		    \cdtIdRef{IU 1.2}{Modificar proyecto}, indicando al actor que el dato es obligatorio.
	    \UCpaso[] Continúa con el paso \ref{cu1.2:ingresaDatos} de la trayectoria principal.
	 \end{UCtrayectoriaA}
	 %----------------------------------------------------------%trayectoria E---------------------------------------------------- 
	 \begin{UCtrayectoriaA}{E}{El actor ingresó un nombre de proyecto repetido.}
	    \UCpaso[\UCsist] Muestra el mensaje \cdtIdRef{MSG7}{Registro repetido} y señala el campo que presenta la duplicidad en la pantalla 
		    \cdtIdRef{IU 1.2}{Modificar proyecto}, indicando al actor que existe un proyecto con el mismo nombre.
	    \UCpaso[] Continúa con el paso \ref{cu1.2:ingresaDatos} de la trayectoria principal.
	 \end{UCtrayectoriaA}
	 %----------------------------------------------------------%trayectoria F---------------------------------------------------- 
	 \begin{UCtrayectoriaA}{F}{El actor ingresó una clave de proyecto repetida.}
	    \UCpaso[\UCsist] Muestra el mensaje \cdtIdRef{MSG7}{Registro repetido} y señala el campo que presenta la duplicidad en la pantalla 
		    \cdtIdRef{IU 1.2}{Modificar proyecto}, indicando al actor que existe un proyecto con la misma clave.
	    \UCpaso[] Continúa con el paso \ref{cu1.2:ingresaDatos} de la trayectoria principal.
	 \end{UCtrayectoriaA}
 
	 %----------------------------------------------------------%trayectoria G----------------------------------------------------  
	 \begin{UCtrayectoriaA}{G}{El actor proporciona un dato que excede la longitud máxima.}
	    \UCpaso[\UCsist] Muestra el mensaje \cdtIdRef{MSG5}{Se ha excedido la longitud máxima del campo} y señala el campo que excede la 
	    longitud en la pantalla \cdtIdRef{IU 1.2}{Modificar proyecto}, para indicar que el dato excede el tamaño máximo permitido.
	    \UCpaso[] Continúa con el paso \ref{cu1.2:ingresaDatos} de la trayectoria principal.
	 \end{UCtrayectoriaA}
 
	 %----------------------------------------------------------%trayectoria H---------------------------------------------------- 
	 \begin{UCtrayectoriaA}{H}{El actor ingresó un tipo de dato incorrecto.}
	    \UCpaso[\UCsist] Muestra el mensaje \cdtIdRef{MSG4}{Formato incorrecto} y señala el campo que presenta el dato inválido en la 
	    pantalla \cdtIdRef{IU 1.2}{Modificar proyecto} para indicar que se ha ingresado un tipo de dato inválido.
	    \UCpaso[] Continúa con el paso \ref{cu1.2:ingresaDatos} de la trayectoria principal.
	 \end{UCtrayectoriaA}
	 %----------------------------------------------------------%trayectoria I---------------------------------------------------- 
	 \begin{UCtrayectoriaA}{I}{El actor ingresó una fecha de término que no es posterior a la fecha de inicio.}
	    \UCpaso[\UCsist] Muestra el mensaje \cdtIdRef{MSG35}{Orden de fechas} y señala el campo de la fecha de término en la 
	    pantalla \cdtIdRef{IU 1.2}{Modificar proyecto}.
	    \UCpaso[] Continúa con el paso \ref{cu1.2:ingresaDatos} de la trayectoria principal.
	 \end{UCtrayectoriaA}
	 %----------------------------------------------------------%trayectoria J---------------------------------------------------- 
	 \begin{UCtrayectoriaA}{J}{El actor ingresó una fecha de término programada que no es posterior a la fecha de inicio programada.}
	    \UCpaso[\UCsist] Muestra el mensaje \cdtIdRef{MSG35}{Orden de fechas} y señala el campo de la fecha de término programada en la 
	    pantalla \cdtIdRef{IU 1.2}{Modificar proyecto}.
	    \UCpaso[] Continúa con el paso \ref{cu1.2:ingresaDatos} de la trayectoria principal.
	 \end{UCtrayectoriaA}