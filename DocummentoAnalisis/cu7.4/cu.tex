\begin{UseCase}{CU 7.4}{Consultar actor}
	{
		Este caso de uso permite al analista consultar la información de un actor. 
	}
	
	\UCitem{Actor}{\cdtRef{actor:liderAnalisis}{Líder de análisis}, \cdtRef{actor:analista}{Analista}}
	\UCitem{Propósito}{
		Consultar la información de un actor.
	}
	\UCitem{Entradas}{
		Ninguna
	}
	\UCitem{Salidas}{
		\begin{UClist}
			\UCli \cdtRef{Elemento:Nombre}{Nombre}: \ioObtener.
			\UCli \cdtRef{Elemento:Descripcion}{Descripción}: \ioObtener.
		\end{UClist}
	}
	
	\UCitem{Mensajes}{
		Ninguno
	}

	\UCitem{Precondiciones}{
		Ninguna
	}
	
	\UCitem{Postcondiciones}{
		Ninguna
	}

	\UCitem{Errores}{
		\begin{UClist}
			\UCli \cdtRef{MSG13}{Ha ocurrido un error}: Se muestra en la pantalla donde se solicitó la operación cuando el actor que se desea consultar no existe.
		\end{UClist}
	}

	\UCitem{Tipo}{
		Secundario, extiende del casos de uso \cdtIdRef{CU 7}{Gestionar actores}.
	}
\end{UseCase}
%-------------------------------------------------------%trayectoria Principal-----------------------------------------------
 \begin{UCtrayectoria}
    \UCpaso[\UCactor] Solicita consultar el actor oprimiendo el botón \btnConsulta del registro que desea consultar en la pantalla \cdtIdRef{IU 7}{Gestionar actores} o la liga correspondiente a un actor en la pantalla \cdtIdRef{IU 5.4}{Consultar caso de uso}.
    \UCpaso[\UCsist] Busca la información del actor seleccionado. \refTray{A}
    \UCpaso[\UCsist] Muestra la pantalla \cdtIdRef{IU 7.4}{Consultar actor} en la cual se mostrará toda la información del actor.
    \UCpaso[\UCactor] Solicita finalizar la consulta oprimiendo el botón \cdtButton{Aceptar} de la pantalla \cdtIdRef{IU 7.4}{Consultar actor}.
 \end{UCtrayectoria}
 
 %----------------------------------------------------------%trayectoria A---------------------------------------------------- 
 \begin{UCtrayectoriaA}[Fin del caso de uso]{A}{El actor que se desea consultar no existe.}
    \UCpaso[\UCsist] Muestra la pantalla donde se solicitó la operación con el mensaje \cdtRef{MSG13}{Ha ocurrido un error}.
 \end{UCtrayectoriaA} 
  