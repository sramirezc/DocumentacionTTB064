\begin{UseCase}{CU 8.3}{Eliminar regla de negocio}
	{
		Este caso de uso permite al actor eliminar del sistema una regla de negocio.
	}
	
	\UCitem{Actor}{\cdtRef{actor:liderAnalisis}{Líder de análisis}, \cdtRef{actor:analista}{Analista}}
	\UCitem{Propósito}{
		Eliminar la información de una regla de negocio.
	}
	\UCitem{Entradas}{
		Ninguna
	}
	\UCitem{Salidas}{
		\begin{UClist}
			\UCli \cdtIdRef{MSG1}{Operación exitosa}: Se muestra en la pantalla \cdtIdRef{IU 8}{Gestionar reglas de negocio} para indicar que la eliminación fue exitosa.
		\end{UClist}
	}
	\UCitem{Mensajes}{
		\begin{UClist}
			\UCli \cdtIdRef{MSG11}{Confirmar eliminación}: Se muestra para que el actor confirme la eliminación.
		\end{UClist}
	}

	\UCitem{Precondiciones}{
		Ninguna
	}
	
	\UCitem{Postcondiciones}{
		\begin{UClist}
			\UCli Se eliminará la regla de negocio seleccionada del sistema.
		\end{UClist}
	}

	\UCitem{Errores}{
		\begin{UClist}
			\UCli \cdtIdRef{MSG14}{Eliminación no permitida}: Se muestra en una pantalla emergente cuando no se pueda eliminar la regla de negocio debido a que está siendo referenciada en algún caso de uso.
			\UCli \cdtIdRef{MSG13}{Ha ocurrido un error}: Se muestra en la pantalla \cdtIdRef{IU 8}{Gestionar reglas de negocio} cuando la regla de negocio no se encuentre en un estado que permita la eliminación.
		\end{UClist}
	}

	\UCitem{Tipo}{
		Secundario, extiende del casos de uso \cdtIdRef{CU 8}{Gestionar reglas de negocio}.
	}
\end{UseCase}
%-------------------------------------------------------%trayectoria Principal-----------------------------------------------
 \begin{UCtrayectoria}
    \UCpaso[\UCactor] Solicita eliminar una regla de negocio oprimiendo el botón \btnEliminar del registro que desea eliminar de la pantalla \cdtIdRef{IU 8}{Gestionar reglas de negocio}.
    \UCpaso[\UCsist] Verifica que la regla de negocio se encuentra en un estado que permita la eliminación de acuerdo a la regla de negocio \cdtIdRef{RN18}{Eliminación de elementos}. \refTray{A}
    \UCpaso[\UCsist] Busca los casos de uso que estén referenciando a la regla de negocio.
    \UCpaso[\UCsist] Verifica que ningún caso de uso esté referenciando a la regla de negocio. \refTray{B}
    \UCpaso[\UCsist] Muestra el mensaje \cdtIdRef{MSG11}{Confirmar eliminación} en una pantalla emergente con los botones \cdtButton{Aceptar} y \cdtButton{Cancelar}.
    \UCpaso[\UCactor] Confirma la eliminación de la regla de negocio oprimiendo el botón \cdtButton{Aceptar}. \refTray{C}
    \UCpaso[\UCsist] Verifica que la regla de negocio se encuentra en un estado que permita la eliminación de acuerdo a la regla de negocio \cdtIdRef{RN18}{Eliminación de elementos}. \refTray{A}
    \UCpaso[\UCsist] Elimina la información referente a la regla de negocio del sistema.
    \UCpaso[\UCsist] Muestra el mensaje \cdtIdRef{MSG1}{Operación exitosa} en la pantalla \cdtIdRef{IU 8}{Gestionar reglas de negocio}
    para indicar al actor que se ha eliminado el registro exitosamente.
 \end{UCtrayectoria}
 
 %----------------------------------------------------------%trayectoria A---------------------------------------------------- 
 \begin{UCtrayectoriaA}[Fin del caso de uso]{A}{La regla de negocio está en un estado en que no se permite la eliminación.}
    \UCpaso[\UCsist] Muestra la pantalla \cdtIdRef{IU 8}{Gestionar reglas de negocio} con el mensaje \cdtIdRef{MSG13}{Ha ocurrido un error}.
 \end{UCtrayectoriaA} 
  %----------------------------------------------------------%trayectoria B---------------------------------------------------- 
 \begin{UCtrayectoriaA}[Fin del caso de uso]{B}{La regla de negocio está siendo referenciada en un caso de uso.}
    \UCpaso[\UCsist] Muestra el mensaje \cdtIdRef{MSG14}{Eliminación no permitida} en una pantalla emergente
    con la lista de casos de uso que están referenciando a la regla de negocio.
    \UCpaso[\UCactor] Oprime el botón \cdtButton{Aceptar} de la pantalla emergente.
    \UCpaso[\UCsist] Muestra la pantalla \cdtIdRef{IU 8}{Gestionar reglas de negocio}.
 \end{UCtrayectoriaA}
 %----------------------------------------------------------%trayectoria C---------------------------------------------------- 
 \begin{UCtrayectoriaA}[Fin del caso de uso]{C}{El actor desea cancelar la operación.}
    \UCpaso[\UCactor] Solicita cancelar la operación oprimiendo el botón \cdtButton{Cancelar} de la pantalla emergente.
    \UCpaso[\UCsist] Muestra la pantalla donde se solicitó la operación.
 \end{UCtrayectoriaA} 