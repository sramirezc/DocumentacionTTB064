\newpage 
\begin{UseCase}{CU 5.1.1.1}{Registrar trayectoria}
	{
		Las trayectorias describen los escenarios ideales y alternos de un sistema mediante una serie de pasos.
		Este caso de uso permite al analista registrar la trayectoria principal o una trayectoria alternativa de un caso de uso.
		
	}
	\UCitem{Actor}{\cdtRef{actor:liderAnalisis}{Líder de análisis}, \cdtRef{actor:analista}{Analista}}
	\UCitem{Propósito}{
		Registrar una trayectoria principal o alternativa.
	}
	\UCitem{Entradas}{
		\begin{UClist}
			\UCli \cdtRef{Trayectoria:Identificador}{Clave}: \ioEscribir
			\UCli \cdtRef{Trayectoria:Alternativa}{Tipo}: \ioSeleccionar
			\UCli \cdtRef{Trayectoria:Condicion}{Condición}: \ioEscribir
			\UCli \cdtRef{Trayectoria:FindelCasodeUso}{Fin del caso de uso}: \ioSeleccionar
		\end{UClist}
	}
	\UCitem{Salidas}{
		\begin{UClist}
			\UCli \cdtRef{Elemento:Clave}{Clave}, \cdtRef{Elemento:Numero}{número} y \cdtRef{Elemento:Nombre}{nombre} del caso de uso al que pertenece la trayectoria.
		\end{UClist}
	}
	
	\UCitem{Mensajes}{
		\begin{UClist}
			\UCli \cdtIdRef{MSG1}{Operación exitosa}: Se muestra en la pantalla \cdtIdRef{IU 5.1.1}{Gestionar trayectorias} para indicar que el registro fue exitoso.
		\end{UClist}
	}

	\UCitem{Precondiciones}{
		Ninguna
	}
	
	\UCitem{Postcondiciones}{
		\begin{UClist}
			\UCli Se podrá solicitar el registro de un paso por medio del caso de uso \cdtIdRef{CU 5.1.1.1.1}{Registrar paso}.
			\UCli Se podrá solicitar el registro de un paso por medio del caso de uso \cdtIdRef{CU 5.1.1.1.2}{Modificar paso}.
			\UCli Se podrá solicitar el registro de un paso por medio del caso de uso \cdtIdRef{CU 5.1.1.1.3}{Eliminar paso}.
		\end{UClist}
	}

	\UCitem{Errores}{
		\begin{UClist}
			\UCli \cdtIdRef{MSG4}{Dato obligatorio}: Se muestra en la pantalla \cdtIdRef{CU 5.1.1.1}{Registrar trayectoria} cuando no se ha ingresado un dato marcado como obligatorio.
			\UCli \cdtIdRef{MSG5}{Dato incorrecto}: Se muestra en la pantalla \cdtIdRef{CU 5.1.1.1}{Registrar trayectoria} cuando el tipo de dato ingresado no cumple con el tipo de dato solicitado en el campo.
			\UCli \cdtIdRef{MSG6}{Longitud inválida}: Se muestra en la pantalla \cdtIdRef{CU 5.1.1.1}{Registrar trayectoria} cuando se ha excedido la longitud de alguno de los campos.
			\UCli \cdtIdRef{MSG7}{Registro repetido}: Se muestra en la pantalla \cdtIdRef{CU 5.1.1.1}{Registrar trayectoria} cuando se haya ingresado una clave que ya esté registrada.
			\UCli \cdtIdRef{MSG15}{Dato no registrado}: Se muestra en la pantalla \cdtIdRef{CU 5.1.1.1}{Registrar trayectoria} cuando un elemento referenciado no existe en el sistema.
			\UCli \cdtIdRef{MSG23}{Caracteres inválidos}: Se muestra en la pantalla \cdtIdRef{CU 5.1.1.1}{Registrar trayectoria} cuando el nombre del caso de uso contiene un caracter no válido.
			\UCli \cdtIdRef{MSG27}{Token incorrecto}: Se muestra en la pantalla \cdtIdRef{CU 5.1.1.1}{Registrar trayectoria} cuando el token ingresado está mal formado.
		\end{UClist}
	}

	\UCitem{Tipo}{
		Secundario, extiende del caso de uso \cdtIdRef{CU 5.1.1}{Gestionar trayectorias}.
% 		\clearpage
	}
	
\end{UseCase}

%-------------------------------------------------------%trayectoria Principal-----------------------------------------------
 \begin{UCtrayectoria}
	\UCpaso[\UCactor] Solicita registrar una trayectoria oprimiento el botón \cdtButton{Registrar} de la la pantalla \cdtIdRef{IU 5.1.1}{Gestionar trayectorias}. 
	\UCpaso[\UCsist] Muestra la pantalla \cdtIdRef{IU 5.1.1.1}{Registrar trayectoria} con el mensaje \cdtIdRef{MSG2}{No existe información} debido a que no hay pasos registrados para la trayectoria. 
	\UCpaso[\UCactor] Ingresa el nombre de la trayectoria.
	\UCpaso[\UCactor] Selecciona que es la trayectoria principal. \refTray{A} \label{cu5.1.1.1:ingresaDatos}
	\UCpaso[\UCsist] Marca que en la trayectoria termina el caso de uso.
	
	\UCpaso[\UCactor] Gestiona los pasos de la trayectoria. \label{cu5.1.1.1:gestionaPasos}
	\UCpaso[\UCactor] Solicita registrar la trayectoria oprimiendo el botón \cdtButton{Aceptar} de la pantalla \cdtIdRef{IU 5.1.1.1}{Registrar trayectoria}. \refTray{B} 
	
	\UCpaso[\UCsist] Verifica que el actor haya ingresado todos los campos obligatorios con base en la regla de negocio \cdtIdRef{RN8}{Datos obligatorios}. \refTray{C}
	\UCpaso[\UCsist] Verifica que la trayectoria no se encuentre registrada en el sistema con base en la regla de negocio \cdtIdRef{RN6}{Unicidad de nombres}. \refTray{D}
	\UCpaso[\UCsist] Verifica que los datos requeridos sean proporcionados correctamente como se especifica en la regla de negocio \cdtIdRef{RN7}{Información correcta}. \refTray{E} \refTray{F}
	
	\UCpaso[\UCsist] Verifica que la clave no contenga caracteres inválidos con base en la regla de negocio \cdtIdRef{RN16}{Nombres de las trayectorias}. \refTray{G}
	\UCpaso[\UCsist] Verifica que el usuario haya ingresado al menos un paso. \refTray{H}
	\UCpaso[\UCsist] Verifica que los tokens utilizados en los pasos esten bien formados. \refTray{I}
	\UCpaso[\UCsist] Verifica que los elementos referenciados en los pasos existan. \refTray{J}
	
	\UCpaso[\UCsist] Registra la información de la trayectoria en el sistema.
	
	\UCpaso[\UCsist] Muestra el mensaje \cdtIdRef{MSG1}{Operación exitosa} en la pantalla \cdtIdRef{IU 5.1.1}{Gestionar trayectorias} 
	para indicar al actor que el registro se ha realizado exitosamente.
    
\end{UCtrayectoria}
 %----------------------------------------------------------%trayectoria A---------------------------------------------------- 
 \begin{UCtrayectoriaA}{A}{El actor desea registrar una trayectoria alternativa.}
    \UCpaso[\UCsist] Muestra el campo de condición.
    \UCpaso[\UCactor] Ingresa la condición de la trayectoria.
    \UCpaso[\UCactor] Selecciona si en la trayectoria se termina el caso de uso.
    \UCpaso[] Continúa con el paso \ref{cu5.1.1.1:gestionaPasos} de la trayectoria principal.
 \end{UCtrayectoriaA}
 %----------------------------------------------------------%trayectoria B---------------------------------------------------- 
 \begin{UCtrayectoriaA}[Fin del caso de uso]{B}{El actor desea cancelar la operación.}
    \UCpaso[\UCactor] Solicita cancelar la operación oprimiendo el botón \cdtButton{Cancelar} de la pantalla \cdtIdRef{IU 5.1.1.1}{Registrar trayectoria}.
    \UCpaso[\UCsist] Muestra la pantalla \cdtIdRef{IU 5.1.1}{Gestionar trayectorias}.
 \end{UCtrayectoriaA}
 
 %----------------------------------------------------------%trayectoria C---------------------------------------------------- 
 \begin{UCtrayectoriaA}{C}{El actor no ingresó algún dato marcado como obligatorio.}
    \UCpaso[\UCsist] Muestra el mensaje \cdtIdRef{MSG4}{Dato obligatorio} y señala el campo que presenta el error en la pantalla 
	    \cdtIdRef{CU 5.1.1.1}{Registrar trayectoria}, indicando al actor que el dato es obligatorio.
    \UCpaso[] Continúa con el paso \ref{cu5.1.1.1:ingresaDatos} de la trayectoria principal.
 \end{UCtrayectoriaA}
 %----------------------------------------------------------%trayectoria D---------------------------------------------------- 
 \begin{UCtrayectoriaA}{D}{El actor ingresó un nombre de trayectoria repetido.}
    \UCpaso[\UCsist] Muestra el mensaje \cdtIdRef{MSG7}{Registro repetido} y señala el campo que presenta la duplicidad en la pantalla 
	    \cdtIdRef{CU 5.1.1.1}{Registrar trayectoria}, indicando al actor que existe una trayectoria con el mismo nombre.
    \UCpaso[] Continúa con el paso \ref{cu5.1.1.1:ingresaDatos} de la trayectoria principal.
 \end{UCtrayectoriaA}
 %----------------------------------------------------------%trayectoria E---------------------------------------------------- 
 \begin{UCtrayectoriaA}{E}{El actor ingresó un tipo de dato incorrecto.}
    \UCpaso[\UCsist] Muestra el mensaje \cdtIdRef{MSG4}{Formato incorrecto} y señala el campo que presenta el dato inválido en la 
    pantalla \cdtIdRef{CU 5.1.1.1}{Registrar trayectoria} para indicar que se ha ingresado un tipo de dato inválido.
    \UCpaso[] Continúa con el paso \ref{cu5.1.1.1:ingresaDatos} de la trayectoria principal.
 \end{UCtrayectoriaA}
 %----------------------------------------------------------%trayectoria F--------------------------------------------------  
 \begin{UCtrayectoriaA}{F}{El actor proporciona un dato que excede la longitud máxima.}
    \UCpaso[\UCsist] Muestra el mensaje \cdtIdRef{MSG5}{Se ha excedido la longitud máxima del campo} y señala el campo que excede la 
    longitud en la pantalla \cdtIdRef{CU 5.1.1.1}{Registrar trayectoria}, para indicar que el dato excede el tamaño máximo permitido.
    \UCpaso[] Continúa con el paso \ref{cu5.1.1.1:ingresaDatos} de la trayectoria principal.
 \end{UCtrayectoriaA}
 
 %----------------------------------------------------------%trayectoria G---------------------------------------------------- 
 \begin{UCtrayectoriaA}{G}{El actor ingresó una clave con caracteres inválidos.}
    \UCpaso[\UCsist] Muestra el mensaje \cdtIdRef{MSG23}{Caracteres inválidos} y señala el campo que contiene los caracteres inválidos.
    \UCpaso[] Continúa con el paso \ref{cu5.1.1.1:ingresaDatos} de la trayectoria principal.
 \end{UCtrayectoriaA}
 %----------------------------------------------------------%trayectoria H---------------------------------------------------- 
 \begin{UCtrayectoriaA}{H}{El actor no registró ningún paso.}
    \UCpaso[\UCsist] Muestra el mensaje \cdtIdRef{MSG18}{Registro necesario} en la sección de pasos.
    \UCpaso[] Continúa con el paso \ref{cu5.1.1.1:ingresaDatos} de la trayectoria principal.
 \end{UCtrayectoriaA}
 %----------------------------------------------------------%trayectoria I---------------------------------------------------- 
 \begin{UCtrayectoriaA}{I}{El actor ingresó un token mal formado.}
    \UCpaso[\UCsist] Muestra el mensaje \cdtIdRef{MSG27}{Token incorrecto} mencionando que el token utilizado en alguno de los pasos no es correcto.
    \UCpaso[] Continúa con el paso \ref{cu5.1.1.1:ingresaDatos} de la trayectoria principal.
 \end{UCtrayectoriaA}
 %----------------------------------------------------------%trayectoria J---------------------------------------------------- 
 \begin{UCtrayectoriaA}{J}{Alguno de los elementos referenciados no existe en el sistema.}
    \UCpaso[\UCsist] Muestra el mensaje \cdtIdRef{MSG15}{Dato no registrado} mencionando el elemento que no está registrado en el sistema.
    \UCpaso[] Continúa con el paso \ref{cu5.1.1.1:ingresaDatos} de la trayectoria principal.
 \end{UCtrayectoriaA}

\subsubsection{Puntos de extensión}

\UCExtensionPoint{El actor requiere registrar un paso de la trayectoria}
	{Paso \ref{cu5.1.1.1:gestionaPasos}}
	{\cdtIdRef{CU 5.1.1.1.1}{Registrar paso}}
\UCExtensionPoint{El actor requiere modificar un paso de la trayectoria}
	{Paso \ref{cu5.1.1.1:gestionaPasos}}
	{\cdtIdRef{CU 5.1.1.1.2}{Modificar paso}}	
\UCExtensionPoint{El actor requiere eliminar un paso de la trayectoria}
	{Paso \ref{cu5.1.1.1:gestionaPasos}}
	{\cdtIdRef{CU 5.1.1.1.3}{Eliminar paso}}
  