\section{Reglas de negocio}
%---------------------------------------------------------

% \begin{BusinessRule}{RN1}{Eliminación de elementos}
%     {Restricción de operación}
%     {Controla la operación}
%     \BRitem{Versión}{0.1}
%     \BRitem{Autor}{Natalia Giselle Hernández Sánchez}
%     \BRitem{Estatus}{Edición}
%     \BRitem{Descripción}{Un elemento podrá ser eliminado del sistema cuando no tenga asociado ningún otro elemento.}	
% \end{BusinessRule}
% 
% \begin{BusinessRule}{RN2}{Eliminación de módulos}
%     {Restricción de operación}
%     {Controla la operación}
%     \BRitem{Versión}{0.1}
%     \BRitem{Autor}{Natalia Giselle Hernández Sánchez}
%     \BRitem{Estatus}{Edición}
%     \BRitem{Descripción}{Los módulos del sistema podrán ser eliminados cuando no contengan casos de uso o pantallas asociados.}	
% \end{BusinessRule}
\begin{BusinessRule}{RN1}{Unicidad de números}
    {Restricción de operación}
    {Controla la operación}
    \BRitem{Versión}{1.0}
    \BRitem{Autor}{Natalia Giselle Hernández Sánchez}
    \BRitem{Estatus}{Edición}
    \BRitem{Descripción}{El número de los elementos del mismo tipo no pueden repetirse.}	
\end{BusinessRule}
\begin{BusinessRule}{RN2}{Nombres de los elementos}
    {Restricción de operación}
    {Controla la operación}
    \BRitem{Versión}{1.0}
    \BRitem{Autor}{Natalia Giselle Hernández Sánchez}
    \BRitem{Estatus}{Edición}
    \BRitem{Descripción}{Los nombres de los elementos no pueden contener coma, punto, punto medio, dos puntos o guión bajo.}
\end{BusinessRule}
\begin{BusinessRule}{RN3}{Líder de análisis}
    {Restricción de operación}
    {Controla la operación}
    \BRitem{Versión}{1.0}
    \BRitem{Autor}{Natalia Giselle Hernández Sánchez}
    \BRitem{Estatus}{Edición}
    \BRitem{Descripción}{Los proyectos deben tener asignado solamente un líder de análisis.}	
\end{BusinessRule}

\begin{BusinessRule}{RN4}{Integrantes de un proyecto}
    {Restricción de operación}
    {Controla la operación}
    \BRitem{Versión}{1.0}
    \BRitem{Autor}{Natalia Giselle Hernández Sánchez}
    \BRitem{Estatus}{Edición}
    \BRitem{Descripción}{Los proyectos deben tener al menos un integrante.}	
\end{BusinessRule}

\begin{BusinessRule}{RN5}{Modificación de elementos asociados a casos de uso liberados}
    {Restricción de operación}
    {Controla la operación}
    \BRitem{Versión}{1.0}
    \BRitem{Autor}{Sergio Ramírez Camacho}
    \BRitem{Estatus}{Edición}
    \BRitem{Descripción}{No es posible modificar entidades, reglas de negocio, actores, términos del glosario, pantallas y/o mensajes que se encuentren asociados a casos de uso con estado ``Liberado''.}	
\end{BusinessRule}

%\begin{BusinessRule}{RN5}{Numeración de elementos}
%     {Restricción de operación}
%     {Controla la operación}
%     \BRitem{Versión}{0.1}
%     \BRitem{Autor}{Natalia Giselle Hernández Sánchez}
%     \BRitem{Estatus}{Edición}
%     \BRitem{Descripción}{Para asignar el número de un elemento en el registro, el sistema buscará el último número asociado a la clave y le sumará 1. }
% \end{BusinessRule}

\begin{BusinessRule}{RN6}{Unicidad de nombres}
    {Restricción de operación}
    {Controla la operación}
    \BRitem{Versión}{1.0}
    \BRitem{Autor}{Natalia Giselle Hernández Sánchez}
    \BRitem{Estatus}{Edición}
    \BRitem{Descripción}{El nombre de los elementos del mismo tipo no puede repetirse.}	
\end{BusinessRule}

\begin{BusinessRule}{RN7}{Información correcta}
    {Restricción de operación}
    {Controla la operación}
    \BRitem{Versión}{1.0}
    \BRitem{Autor}{Natalia Giselle Hernández Sánchez}
    \BRitem{Estatus}{Edición}
    \BRitem{Descripción}{La información que el usuario proporcione, debe ser del tipo y longitud definida en el modelo conceptual.}	
\end{BusinessRule}

\begin{BusinessRule}{RN8}{Datos obligatorios}
    {Restricción de operación}
    {Controla la operación}
    \BRitem{Versión}{1.0}
    \BRitem{Autor}{Natalia Giselle Hernández Sánchez}
    \BRitem{Estatus}{Edición}
    \BRitem{Descripción}{El usuario debe ingresar toda la información marcada como obligatoria en el modelo conceptual.}	
\end{BusinessRule}

\begin{BusinessRule}{RN9}{Operaciones disponibles de casos de uso}
    {Restricción de operación}
    {Controla la operación}
    \BRitem{Versión}{1.0}
    \BRitem{Autor}{Natalia Giselle Hernández Sánchez}
    \BRitem{Estatus}{Edición}
    \BRitem{Descripción}{
      Los estados de los casos de uso y el rol del actor determinan las operaciones que pueden solicitarse sobre un caso de uso desde la gestión:
      
      \begin{longtable}{| p{.2\textwidth} | p{.3\textwidth} | p{.3\textwidth} |}%
		\arrayrulecolor{black}%
		\rowcolor{black}%
		\color{white} Estado &  \color{white}Operaciones Analista & \color{white}Operaciones Líder de análisis\\ \hline
		\endhead%
		\arrayrulecolor{black}
		Edición & Consultar, editar, gestionar trayectorias, gestionar puntos de extensión, terminar y eliminar & Consultar, editar, gestionar trayectorias, gestionar puntos de extensión, terminar y eliminar\\ \hline
		
		Revisión & Consultar y revisar & Consultar y revisar\\ \hline
		
		Por liberar & Consultar & Consultar y liberar\\ \hline
		
		Pendiente de corrección & Consultar, editar, gestionar trayectorias, gestionar puntos de extensión, terminar y eliminar & Consultar, editar, gestionar trayectorias, gestionar puntos de extensión, terminar y eliminar\\ \hline
		
		Liberado & Consultar y configurar prueba & Consultar, solicitar correcciones y configurar prueba\\ \hline
		
		Configurado & Consultar, configurar prueba y generar prueba & Consultar, solicitar correcciones, configurar prueba y generar prueba\\ \hline
		
      \end{longtable}
    }	
\end{BusinessRule}

\begin{BusinessRule}{RN10}{Referencias a elementos}
    {Restricción de operación}
    {Controla la operación}
    \BRitem{Versión}{1.0}
    \BRitem{Autor}{Natalia Giselle Hernández Sánchez}
    \BRitem{Estatus}{Edición}
    \BRitem{Descripción}{Los elementos que podrán ser referenciados desde los casos de uso son aquellos que se encuentren registrados en el sistema.}	
    % Se sacrifica la usabilidad por el tiempo reducido de desarrollo
\end{BusinessRule}

\begin{BusinessRule}{RN11}{Registro de trayectorias}
    {Restricción de operación}
    {Controla la operación}
    \BRitem{Versión}{1.0}
    \BRitem{Autor}{Natalia Giselle Hernández Sánchez}
    \BRitem{Estatus}{Edición}
    \BRitem{Descripción}{Al menos una de las trayectorias registradas debe ser marcada como principal.}	
    % Se sacrifica la usabilidad por el tiempo reducido de desarrollo
\end{BusinessRule}

\begin{BusinessRule}{RN12}{Identificador de elemento}
    {Restricción de operación}
    {Controla la operación}
    \BRitem{Versión}{1.0}
    \BRitem{Autor}{Natalia Giselle Hernández Sánchez}
    \BRitem{Estatus}{Edición}
    \BRitem{Descripción}{El identificador de cada elemento se compone de un nombre, número y una clave. Donde el nombre es el que le asigna el usuario, el número 
    es secuencial y la clave define el tipo de elemento: ``ENT'' para las entidades, ``ACT'' para los actores, ``CU'' para los casos de uso, ``IU'' para
    las pantallas, ``MSG'' para los mensajes, ``RN'' para las reglas de negocio y ``GLS'' para los términos del glosario.}	
    % Se sacrifica la usabilidad por el tiempo reducido de desarrollo
\end{BusinessRule}

\begin{BusinessRule}{RN13}{Modificación del identificador}
    {Restricción de operación}
    {Controla la operación}
    \BRitem{Versión}{1.0}
    \BRitem{Autor}{Natalia Giselle Hernández Sánchez}
    \BRitem{Estatus}{Edición}
    \BRitem{Descripción}{Una vez registrado un elemento no se podrá modificar el nombre, el número o la clave del identificador.}
\end{BusinessRule}

\begin{BusinessRule}{RN14}{Salidas del caso de uso}
    {Restricción de operación}
    {Controla la operación}
    \BRitem{Versión}{1.0}
    \BRitem{Autor}{Natalia Giselle Hernández Sánchez}
    \BRitem{Estatus}{Edición}
    \BRitem{Descripción}{En las salidas del caso de uso podrán enlistarse mensajes y atributos de las entidades.}
\end{BusinessRule}

\begin{BusinessRule}{RN15}{Operaciones disponibles}
    {Restricción de operación}
    {Controla la operación}
    \BRitem{Versión}{1.0}
    \BRitem{Autor}{Natalia Giselle Hernández Sánchez}
    \BRitem{Estatus}{Edición}
    \BRitem{Descripción}{Cuando una entidad, regla de negocio, actor, término del glosario, pantalla y/o mensaje, 
    están en estado ``Edición'' es posible solicitar su consulta, modificación y eliminación. Cuando alguno de estos elementos se encuentre asociado 
    a un caso de uso con estado ``Liberado'' solamente estará disponible la operación de consulta.}
\end{BusinessRule}

\begin{BusinessRule}{RN16}{Nombres de las trayecorias}
    {Restricción de operación}
    {Controla la operación}
    \BRitem{Versión}{1.0}
    \BRitem{Autor}{Natalia Giselle Hernández Sánchez}
    \BRitem{Estatus}{Edición}
    \BRitem{Descripción}{Los nombres de las trayectorias no pueden contener espacio, coma, punto, punto medio, dos puntos o guión bajo.}
\end{BusinessRule}

\begin{BusinessRule}{RN17}{Unicidad de puntos de extensión}
    {Restricción de operación}
    {Controla la operación}
    \BRitem{Versión}{1.0}
    \BRitem{Autor}{Natalia Giselle Hernández Sánchez}
    \BRitem{Estatus}{Edición}
    \BRitem{Descripción}{No puede existir más de un punto de extensión con el mismo caso de uso origen y el mismo caso de uso destino.}
\end{BusinessRule}

\begin{BusinessRule}{RN18}{Eliminación de elementos}
    {Restricción de operación}
    {Controla la operación}
    \BRitem{Versión}{1.0}
    \BRitem{Autor}{Natalia Giselle Hernández Sánchez}
    \BRitem{Estatus}{Edición}
    \BRitem{Descripción}{No es posible eliminar entidades, reglas de negocio, actores, términos del glosario, pantallas y/o mensajes que se encuentren asociados a casos de uso con estado ``Liberado''.}
\end{BusinessRule}

\begin{BusinessRule}{RN19}{Formato de correo electrónico}
    {Restricción de operación}
    {Controla la operación}
    \BRitem{Versión}{1.0}
    \BRitem{Autor}{Sergio Ramírez Camacho}
    \BRitem{Estatus}{Edición}
    \BRitem{Descripción}{El correo electrónico debe ser una cadena de caracteres con la siguiente estructura ordenada:
		\begin{enumerate}
			\item Cadena de caracteres
			\item ``@''
			\item Cadena de caracteres
			\item ``.''
			\item Cadena de caracteres
		\end{enumerate}
		}
	\BRitem{Ejemplo}{cadena1@cadena2.cadena3}
\end{BusinessRule}

\begin{BusinessRule}{RN20}{Verificación de catálogos}
    {Restricción de operación}
    {Controla la operación}
    \BRitem{Versión}{1.0}
    \BRitem{Autor}{Natalia Giselle Hernández Sánchez}
    \BRitem{Estatus}{Edición}
    \BRitem{Descripción}{Es necesario que exista información registrada en los catálogos al momento de solicitar una operación que requiera de estos.}
\end{BusinessRule}

\begin{BusinessRule}{RN21}{Estados para iniciar un proyecto}
    {Restricción de operación}
    {Controla la operación}
    \BRitem{Versión}{1.0}
    \BRitem{Autor}{Natalia Giselle Hernández Sánchez}
    \BRitem{Estatus}{Edición}
    \BRitem{Descripción}{Se podrán registrar proyectos con estado ``En Negociación'' o ``Iniciado''.}
\end{BusinessRule}

\begin{BusinessRule}{RN22}{Unicidad de la clave del proyecto}
    {Restricción de operación}
    {Controla la operación}
    \BRitem{Versión}{1.0}
    \BRitem{Autor}{Natalia Giselle Hernández Sánchez}
    \BRitem{Estatus}{Edición}
    \BRitem{Descripción}{La clave de los proyectos debe ser única en todo el sistema.}
\end{BusinessRule}

\begin{BusinessRule}{RN23}{Unicidad de la clave del módulo}
    {Restricción de operación}
    {Controla la operación}
    \BRitem{Versión}{1.0}
    \BRitem{Autor}{Natalia Giselle Hernández Sánchez}
    \BRitem{Estatus}{Edición}
    \BRitem{Descripción}{La clave de los módulos debe ser única en un proyecto.}
\end{BusinessRule}

\begin{BusinessRule}{RN24}{Unicidad de la clave de la trayectoria}
    {Restricción de operación}
    {Controla la operación}
    \BRitem{Versión}{1.0}
    \BRitem{Autor}{Natalia Giselle Hernández Sánchez}
    \BRitem{Estatus}{Edición}
    \BRitem{Descripción}{La clave de las trayectorias debe ser única en un caso de uso.}
\end{BusinessRule}

\begin{BusinessRule}{RN25}{Comparación de fechas del proyecto}
    {Restricción de operación}
    {Controla la operación}
    \BRitem{Versión}{1.0}
    \BRitem{Autor}{Natalia Giselle Hernández Sánchez}
    \BRitem{Estatus}{Edición}
    \BRitem{Descripción}{La fecha de término del proyecto debe ser posterior a la fecha de inicio.}
\end{BusinessRule}

\begin{BusinessRule}{RN26}{Comparación de fechas programadas del proyecto}
    {Restricción de operación}
    {Controla la operación}
    \BRitem{Versión}{1.0}
    \BRitem{Autor}{Natalia Giselle Hernández Sánchez}
    \BRitem{Estatus}{Edición}
    \BRitem{Descripción}{La fecha de término programada del proyecto debe ser posterior a la fecha de inicio programada.}
\end{BusinessRule}


\begin{BusinessRule}{RN27}{Eliminación de personas}
    {Restricción de operación}
    {Controla la operación}
    \BRitem{Versión}{1.0}
    \BRitem{Autor}{Sergio Ramírez Camacho}
    \BRitem{Estatus}{Edición}
    \BRitem{Descripción}{No es posible eliminar una persona, si esta es líder de al menos un proyecto.}
\end{BusinessRule}

\begin{BusinessRule}{RN28}{Eliminación de módulos}
    {Restricción de operación}
    {Controla la operación}
    \BRitem{Versión}{1.0}
    \BRitem{Autor}{Sergio Ramírez Camacho}
    \BRitem{Estatus}{Edición}
    \BRitem{Descripción}{No es posible eliminar un módulo, si algún elemento de otro módulo, tiene referencias a al menos un elemento del módulo que desea eliminarse.}
\end{BusinessRule}

\begin{BusinessRule}{RN29}{Unicidad de casos de uso}
    {Restricción de operación}
    {Controla la operación}
    \BRitem{Versión}{1.0}
    \BRitem{Autor}{Sergio Ramírez Camacho}
    \BRitem{Estatus}{Edición}
    \BRitem{Descripción}{Diferentes casos de uso pueden tener el mismo nombre y/o número, únicamente si cada uno de estos pertenecen a diferentes módulos.}
\end{BusinessRule}

\begin{BusinessRule}{RN30}{Unicidad de pantallas}
    {Restricción de operación}
    {Controla la operación}
    \BRitem{Versión}{1.0}
    \BRitem{Autor}{Sergio Ramírez Camacho}
    \BRitem{Estatus}{Edición}
    \BRitem{Descripción}{Diferentes pantallas pueden tener el mismo nombre y/o número, únicamente si cada una de estas pertenecen a diferentes módulos.}
\end{BusinessRule}

\begin{BusinessRule}{RN31}{Estructura de tokens}
    {Restricción de operación}
    {Controla la operación}
    \BRitem{Versión}{1.0}
    \BRitem{Autor}{Sergio Ramírez Camacho}
    \BRitem{Estatus}{Edición}
    \BRitem{Descripción}{Los tokens utilizados para referenciar elementos deben mantener una estructura determinada de acuerdo al tipo de elemento referenciado:
	\begin{itemize}
		\item Regla de negocio: {\it RN·Número:Nombre}
			\begin{itemize}
				\item ``RN'': cadena que indica que el tipo de elemento referenciado es una regla de negocio.
				\item ``·'': símbolo para separar las partes del token (punto medio).
				\item ``Número'': número de la regla de negocio referenciada.
				\item ``:'': símbolo para separar las partes del token (dos puntos).
				\item ``Nombre'': nombre de la regla de negocio referenciada.
			\end{itemize}
			
			\item Entidad: {\it ENT·Nombre}
				\begin{itemize}
					\item ``ENT'': cadena que indica que el tipo de elemento referenciado es una entidad.
					\item ``·'': símbolo para separar las partes del token (punto medio).
					\item ``Nombre'': nombre de la entidad referenciada.
				\end{itemize}
			\item Caso de uso: {\it CU·ClaveMódulo·Número:Nombre}
				\begin{itemize}
					\item ``CU'': cadena que indica que el tipo de elemento referenciado es un caso de uso.
					\item ``·'': símbolo para separar las partes del token (punto medio).
					\item ``ClaveMódulo'': clave del módulo a la que pertenece el caso de uso referenciado.
					\item ``Número'': número del caso de uso referenciado.
					\item ``:'': símbolo para separar las partes del token (dos puntos).
					\item ``Nombre'': nombre del caso de uso referenciado.
				\end{itemize}
			\item Pantalla: {\it IU·ClaveMódulo·Número:Nombre}
				\begin{itemize}
					\item ``CU'': cadena que indica que el tipo de elemento referenciado es una pantalla.
					\item ``·'': símbolo para separar las partes del token (punto medio).
					\item ``ClaveMódulo'': clave del módulo a la que pertenece la pantalla referenciada.
					\item ``Número'': número de la pantalla referenciada.
					\item ``:'': símbolo para separar las partes del token (dos puntos).
					\item ``Nombre'': nombre de la pantalla referenciada.
				\end{itemize}
			\item Mensaje: {\it MSG·Número:Nombre}
				\begin{itemize}
					\item ``MSG'': cadena que indica que el tipo de elemento referenciado es un mensaje.
					\item ``·'': símbolo para separar las partes del token (punto medio).
					\item ``Número'': número del mensaje referenciado.
					\item ``:'': símbolo para separar las partes del token (dos puntos).
					\item ``Nombre'': nombre del mensaje referenciado.
				\end{itemize}
			\item Actor: {\it ACT·Nombre}
				\begin{itemize}
					\item ``ACT'': cadena que indica que el tipo de elemento referenciado es un actor.
					\item ``·'': símbolo para separar las partes del token (punto medio).
					\item ``Nombre'': nombre del actor referenciado.
				\end{itemize}
			\item Término del glosario: {\it GLS·Nombre}
				\begin{itemize}
					\item ``GLS'': cadena que indica que el tipo de elemento referenciado es un término del glosario.
					\item ``·'': símbolo para separar las partes del token (punto medio).
					\item ``Nombre'': nombre del término del glosario referenciado.
				\end{itemize}
			\item Atributo: {\it ATR·Entidad:Nombre}
				\begin{itemize}
					\item ``ATR'': cadena que indica que el tipo de elemento referenciado es un atributo.
					\item ``·'': símbolo para separar las partes del token (punto medio).
					\item ``Entidad'': nombre de la entidad a la que pertenece el atributo referenciado.
					\item ``:'': símbolo para separar las partes del token (dos puntos).
					\item ``Nombre'': nombre del atributo referenciado.
				\end{itemize}
			\item Paso: {\it P·ClaveCasoUso·NúmeroCasoUso:NombreCasoUso:ClaveTrayectoria·Número}
				\begin{itemize}
					\item ``P'': cadena que indica que el tipo de elemento referenciado es un paso.
					\item ``·'': símbolo para separar las partes del token (punto medio).
					\item ``ClaveCasoUso'': clave del caso de uso al que pertenece el paso referenciado.
					\item ``NúmeroCasoUso'': número del caso de uso al que pertenece el paso referenciado.
					\item ``:'': símbolo para separar las partes del token (dos puntos).
					\item ``NombreCasoUso'': nombre del caso de uso al que pertenece el paso referenciado.
					\item ``ClaveTrayectoria'': clave de la trayectoria a la que pertenece el paso referenciado.
					\item ``Número'': número del paso referenciado.
				\end{itemize}
			\item Trayectoria: {\it TRAY·ClaveCasoUso·NúmeroCasoUso:NombreCasoUso:Clave}
				\begin{itemize}
					\item ``TRAY'': cadena que indica que el tipo de elemento referenciado es una trayectoria.
					\item ``·'': símbolo para separar las partes del token (punto medio).
					\item ``ClaveCasoUso'': clave del caso de uso al que pertenece la trayectoria referenciada.
					\item ``NúmeroCasoUso'': número del caso de uso al que pertenece la trayectoria referenciada.
					\item ``:'': símbolo para separar las partes del token (dos puntos).
					\item ``NombreCasoUso'': nombre del caso de uso al que pertenece la trayectoria referenciada.
					\item ``Clave'': clave de la trayectoria referenciada.									
				\end{itemize}
			\item Acción: {\it ACC·ClavePantalla·NúmeroPantalla:NombrePantalla:Nombre}
				\begin{itemize}
					\item ``ACC'': cadena que indica que el tipo de elemento referenciado es una acción.
					\item ``·'': símbolo para separar las partes del token (punto medio).
					\item ``ClavePantalla'': clave de la pantalla a la que pertenece la acción referenciada.
					\item ``NúmeroPantalla'': número de la pantalla a la que pertenece la acción referenciada.
					\item ``:'': símbolo para separar las partes del token (dos puntos).
					\item ``NombrePantalla'': nombre de la pantalla a la que pertenece la acción referenciada.
					\item ``Nombre'': nombre de la acción referenciada.				
				\end{itemize}
			\item Parámentro (Mensajes): {\it PARAM·Nombre}
				\begin{itemize}
					\item ``PARAM'': cadena que indica que se está realizando una referencia a un parámetro en un mensaje.
					\item ``·'': símbolo para separar las partes del token (punto medio).
					\item ``Nombre'': nombre del parámentro referenciado.
				\end{itemize}
	\end{itemize}		
		}
\end{BusinessRule}

\begin{BusinessRule}{RN32}{Pasos en la trayectoria}
    {Restricción de operación}
    {Controla la operación}
    \BRitem{Versión}{1.0}
    \BRitem{Autor}{Sergio Ramírez Camacho}
    \BRitem{Estatus}{Edición}
    \BRitem{Descripción}{Una trayectoria debe debe contar al menos con un paso.}
\end{BusinessRule}
