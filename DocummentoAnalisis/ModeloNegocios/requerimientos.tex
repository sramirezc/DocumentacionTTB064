\section{Ámbito del sistema}
Los roles de los usuarios del sistema serán: administrador, líder de análisis y analista. El administrador será el encargado 
de registrar los proyectos donde participarán los analistas, así como el líder de análisis de dicho proyecto. El administrador puede
registrar al personal de la organización.\\

El líder de análisis es un analista que además podrá asignar los colaboradores a los proyectos e indicar la fecha de arranque de los mismos. Finalmente
el analista será el encargado de documentar los casos de uso.\\

El editor de casos de uso debe permitir al analista ingresar los elementos que componen un caso de uso: entidades, atributos, 
reglas de negocio, pantallas, mensajes y actores.\\

El editor permitirá documentar casos de uso que contengan referencias a los elementos descritos anteriormente.
Cada elemento debe tener un identificador que facilite la forma de hacerles referencia desde los casos de uso.\\

Los casos de uso deben contener secciones para especificar los actores que participan, las salidas, las entradas, las reglas de negocio,
las precondiciones, las postcondiciones, los errores, la trayectoria principal, las trayectorias alternativas y los puntos de extensión.\\

En el registro de todos los elementos, incluyendo los casos de uso, deberá haber un campo para el nombre, el cual debe ser único.\\

Los casos de uso deben ser clasificados en módulos.\\

Se debe simular el proceso de revisón de los elementos, el cual consiste en que cuando un elemento está siendo editado debe ser notificado a los demás analistas, cuando 
el analista concluya la construcción del elemento el estado de este se considerará terminado, cuando líder de análisis lo está revisando ningún analista
podrá modificar su información, cuando el líder decida enviar correcciones del caso de uso, cualquier analista podrá corregir la información; y al contrario, 
si el líder decide liberar un elemento, este podrá ser liberado.\\

Es importante que por cualquier cambio realizado en un caso de uso se registre en el sistema el responsable y los comentarios referentes.\\

\section{Requerimientos funcionales}
De acuerdo al ámbito del sistema, se enlistan a continuación los requerimientos funcionales.
\begin{enumerate}[{\bf RF1.}]
 \item Existirán 3 roles en el sistema: administrador, líder de análisis y analista.
 \item El administrador podrá gestionar los proyectos y el personal de la organización.
 \item El líder de análisis podrá asignar los colaboradores al proyecto que lidera.
 \item El líder de análisis y el analista podrán registrar entidades, atributos, reglas de negocio, mensajes, actores del sistema e interfaces de usuario.
 \item El líder de análisis y el analista podrán registrar casos de uso que hagan referencia a las entidades, atributos, reglas de negocio, mensajes, actores del sistema e interfaces de usuario.
 \item Los casos podrán ser organizados en módulos para evitar tener listas muy grandes de casos de uso.
 \item El analista podrá crear, modificar y eliminar módulos.
 \item Cuando alguno de los elementos ha sido modificado el sistema debe notificar al usuario.
 \item El registro de los elementos se puede quedar pendiente.
 \item Cualquier analista, incluyendo el líder de análisis podrá indicar cuando ha terminado la captura del caso de uso para revisión.
 \item Cualquier analista puede revisar un caso de uso.
 \item El encargado de liberar los casos de uso es el líder de análisis. Los casos de uso que se pueden liberar son aquellos que ya están revisados.
%  \item El líder de análisis podrá indicar la fecha de arranque del proyecto.
 \item El usuario podrá elegir entre una serie de reglas de negocio definidas:
	\begin{itemize}
		\item Verificación de catálogos
		\item Operaciones aritméticas
		\item Unicidad de parámetros
		\item Datos obligatorios
		\item Longitud correcta
		\item Tipo de dato correcto
		\item Formato de archivos
		\item Tamaño de archivos
		\item Intervalo de fechas correctas
		\item Formato correcto
	\end{itemize}
	Las reglas de negocio serán referenciadas desde los casos de uso y utilizadas en las trayectorias.
%  \item Cuando el administrador registre un proyecto podrá definir si este se encuentra en etapa de negociación o ya se ha iniciado, en caso de haberse iniciado tendrá
%  que registrar la fecha de inicio del proyecto. 
    \RCitem{PC1}{\TODO{Falta indicar si el líder o el administrador deciden la fecha de inicio y término de un proyecto.}}{}
\end{enumerate}



