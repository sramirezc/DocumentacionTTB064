\newpage 
\begin{UseCase}{CU 5.1.4.1}{Registrar punto de extensión}
	{
		Los puntos de extensión describen una región de la trayectoria en la que se puede extender el funcionamiento a través de otro caso de uso.
		Este caso de uso permite al analista registrar un punto de extensión.
		
	}

	\UCitem{Actor}{\cdtRef{actor:liderAnalisis}{Líder de análisis}, \cdtRef{actor:analista}{Analista}}
	\UCitem{Propósito}{
		Registrar los puntos de extensión de un caso de uso.
	}
	\UCitem{Entradas}{
		\begin{UClist}
 			\UCli \cdtRef{Extension:Causa}{Causa}: \ioEscribir.
 			\UCli \cdtRef{Trayectoria}{Región de la trayectoria}: \ioEscribir.
 			\UCli \cdtRef{CasodeUso}{Caso de uso al que extiende}: \ioSeleccionar.
		\end{UClist}
	}
	\UCitem{Salidas}{
		Ninguna
	}
	
	\UCitem{Mensajes}{
		\begin{UClist}
			\UCli \cdtIdRef{MSG1}{Operación exitosa}: Se muestra en la pantalla \cdtIdRef{IU 5.1.4}{Gestionar puntos de extensión}
			para indicar que el registro del punto de extensión fue exitoso.
		\end{UClist}
	}

	\UCitem{Precondiciones}{
		Ninguna
	}
	
	\UCitem{Postcondiciones}{
		Ninguna
	}

	\UCitem{Errores}{
		\begin{UClist}
			\UCli \cdtIdRef{MSG4}{Dato obligatorio}: Se muestra en la pantalla \cdtIdRef{CU 5.1.4.1}{Registrar punto de extensión} cuando no se ha ingresado un dato marcado como obligatorio.
			\UCli \cdtIdRef{MSG6}{Longitud inválida}: Se muestra en la pantalla \cdtIdRef{CU 5.1.4.1}{Registrar punto de extensión} cuando se ha excedido la longitud de alguno de los campos.
			\UCli \cdtIdRef{MSG7}{Registro repetido}: Se muestra en la pantalla \cdtIdRef{CU 5.1.4.1}{Registrar punto de extensión} cuando se ha ingresado un punto de extensión que ya existe.
			\UCli \cdtIdRef{MSG15}{Dato no registrado}: Se muestra en la pantalla \cdtIdRef{CU 5.1.4.1}{Registrar punto de extensión} cuando un elemento referenciado no existe en el sistema.
			\UCli \cdtIdRef{MSG27}{Token incorrecto}: Se muestra en la pantalla \cdtIdRef{CU 5.1.4.1}{Registrar punto de extensión} cuando el token ingresado está mal formado.
		\end{UClist}
	}

	\UCitem{Tipo}{
		Secundario, extiende del caso de uso \cdtIdRef{CU 5.1.4}{Gestionar puntos de extensión}.
	}
\end{UseCase}



%-------------------------------------------------------%trayectoria Principal-----------------------------------------------

 \begin{UCtrayectoria}
	\UCpaso[\UCactor] Solicita registrar un punto de extensión oprimiendo el botón \cdtButton{Registrar} de la pantalla \cdtIdRef{IU 5.1.4}{Gestionar puntos de extensión}. 
	\UCpaso[\UCsist] Muestra la pantalla \cdtIdRef{CU 5.1.4.1}{Registrar punto de extensión}. 
	\UCpaso[\UCactor] Selecciona el caso de uso que extiende. \label{cu5.1.4.1:ingresaDatos}
	\UCpaso[\UCactor] Ingresa la causa del punto de extensión.
	\UCpaso[\UCactor] Ingresa la región de la trayectoria. \label{cu5.1.4.1:ingresaRedaccion} \refTray{A}
	\UCpaso[\UCactor] Solicita registrar el punto de extensión oprimiendo el botón \cdtButton{Aceptar} de la pantalla \cdtIdRef{CU 5.1.4.1}{Registrar punto de extensión}. \refTray{B} 
	\UCpaso[\UCsist] Verifica que el actor haya ingresado todos los campos obligatorios con base en la regla de negocio \cdtIdRef{RN8}{Datos obligatorios}. \refTray{C}
	\UCpaso[\UCsist] Verifica que los datos requeridos sean proporcionados correctamente como se especifica en la regla de negocio \cdtIdRef{RN7}{Información correcta}. \refTray{D} 
	\UCpaso[\UCsist] Verifica que el punto de extensión no esté registrado de acuerdo a la regla de negocio \cdtIdRef{RN17}{Unicidad de puntos de extensión}. \refTray{E}
	\UCpaso[\UCsist] Verifica que los tokens utilizados se encuentren bien formados. \refTray{F}
	\UCpaso[\UCsist] Verifica que los elementos referenciados existan. \refTray{G}
	\UCpaso[\UCsist] Registra la información del punto de extensión en el sistema.
	\UCpaso[\UCsist] Muestra el mensaje \cdtIdRef{MSG1}{Operación exitosa} en la pantalla donde se solicitó la operación
	para indicar al actor que el registro se ha realizado exitosamente. 
\end{UCtrayectoria}

%----------------------------------------------------------%trayectoria A---------------------------------------------------- 
 \begin{UCtrayectoriaA}{A}{El actor desea ingresar un paso.}
 	\UCpaso[\UCactor] Ingresa el token ``P·''.
  	\UCpaso[\UCsist] Busca los pasos de las trayectorias del caso de uso.
  	\UCpaso[\UCsist] Muestra la lista de pasos encontrados.
 	\UCpaso[\UCactor] Selecciona un paso de la lista.
  	\UCpaso[\UCsist] Oculta la lista y agrega el número del paso al texto.
    \UCpaso[] Continúa con el paso \ref{cu5.1.4.1:ingresaRedaccion} de la trayectoria principal.
 \end{UCtrayectoriaA}

 %----------------------------------------------------------%trayectoria B---------------------------------------------------- 
 \begin{UCtrayectoriaA}[Fin del caso de uso]{B}{El actor desea cancelar la operación.}
    \UCpaso[\UCactor] Solicita cancelar la operación oprimiendo el botón \cdtButton{Cancelar} de la pantalla \cdtIdRef{CU 5.1.4.1}{Registrar punto de extensión}.
    \UCpaso[\UCsist] Muestra la pantalla \cdtIdRef{IU 5.1}{Registrar caso de uso} o \cdtIdRef{IU 5.2}{Modificar caso de uso} según corresponda.
 \end{UCtrayectoriaA}
 
 %----------------------------------------------------------%trayectoria C---------------------------------------------------- 
 \begin{UCtrayectoriaA}{C}{El actor no ingresó algún dato marcado como obligatorio.}
    \UCpaso[\UCsist] Muestra el mensaje \cdtIdRef{MSG4}{Dato obligatorio} y señala el campo que presenta el error en la pantalla \cdtIdRef{CU 5.1.4.1}{Registrar punto de extensión}, indicando al actor que el dato es obligatorio.
    \UCpaso[] Continúa con el paso \ref{cu5.1.4.1:ingresaDatos} de la trayectoria principal.
 \end{UCtrayectoriaA}
 
 %----------------------------------------------------------%trayectoria D --------------------------------------------------  
 \begin{UCtrayectoriaA}{D}{El actor proporciona un dato que excede la longitud máxima.}
    \UCpaso[\UCsist] Muestra el mensaje \cdtIdRef{MSG5}{Se ha excedido la longitud máxima del campo} y señala el campo que excede la 
    longitud en la pantalla \cdtIdRef{CU 5.1.4.1}{Registrar punto de extensión}, para indicar que el dato excede el tamaño máximo permitido.
    \UCpaso[] Continúa con el paso \ref{cu5.1.4.1:ingresaDatos} de la trayectoria principal.
 \end{UCtrayectoriaA}
%----------------------------------------------------------%trayectoria E---------------------------------------------------- 
 \begin{UCtrayectoriaA}{E}{El actor ingresó un punto de extensión que ya existe.}
    \UCpaso[\UCsist] Muestra el mensaje \cdtIdRef{MSG7}{Registro repetido} señalando que el punto de extensión ya existe.
    \UCpaso[] Continúa con el paso \ref{cu5.1.4.1:ingresaDatos} de la trayectoria principal.
 \end{UCtrayectoriaA}
%----------------------------------------------------------%trayectoria F---------------------------------------------------- 
 \begin{UCtrayectoriaA}{F}{El actor ingresó un token mal formado.}
    \UCpaso[\UCsist] Muestra el mensaje \cdtIdRef{MSG27}{Token incorrecto} mencionando que el token utilizado no es correcto.
    \UCpaso[] Continúa con el paso \ref{cu5.1.4.1:ingresaDatos} de la trayectoria principal.
 \end{UCtrayectoriaA}
 %----------------------------------------------------------%trayectoria G---------------------------------------------------- 
 \begin{UCtrayectoriaA}{G}{Alguno de los elementos referenciados no existe en el sistema.}
    \UCpaso[\UCsist] Muestra el mensaje \cdtIdRef{MSG15}{Dato no registrado} mencionando el elemento que no está registrado en el sistema.
    \UCpaso[] Continúa con el paso \ref{cu5.1.4.1:ingresaDatos} de la trayectoria principal.
 \end{UCtrayectoriaA}
  