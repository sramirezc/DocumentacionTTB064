\section{Implementación}

La implementación del sistema se realizó utilizando diferentes tecnologías:

\begin{itemize}
 \item {\bf Struts2.}  Se utilizó este framework para simplificar el desarrollo del sistema, ya que se ahorra gran parte del trabajo al brindar mecanismos para la construcción de formularios.
 \item {\bf Hibernate.}  Se utilizó este framework para facilitar el acceso ala base de datos, ya que proporciona mecanismos para la inserción y extracción de información de la base de datos.
 \item {\bf JSP.}  Se utilizaron para construir las vistas con las que el usuario manipulará el sistema.
 \item {\bf CSS.} Se utilzaron para brindar estilo a los JSP's.
 \item {\bf Javascript.} Se utilizó principalmente para realizar validaciones en la capa de presentación y para otorgar comportamiento que brindara mayor usabilidad a la aplicación.
\end{itemize}


\newpage
Hasta este momento se han implementado un total de 34 casos de uso, correspondientes al módulo del editor y se distribuyen de la siguiente manera:

 \begin{longtable}{| p{.3\textwidth} | p{.2\textwidth} | p{.2\textwidth} |}%
		\arrayrulecolor{black}%
		\rowcolor{black}%
		\color{white}Módulo  &  \color{white}Total de casos de uso & \color{white}Casos de uso implementados\\ \hline
		\endhead%
		\arrayrulecolor{black}
		Casos de uso & 16 & 26\\ \hline 
		Pantallas & 0 & 9\\ \hline
		Actores & 3 & 5\\ \hline
		Términos del glosario & 3 & 5\\ \hline
		Entidades & 6 & 9\\ \hline
		Reglas de negocio & 3 & 5\\ \hline
		Mensaje & 3 & 5\\ \hline


\end{longtable}
