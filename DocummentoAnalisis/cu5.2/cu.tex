\newpage 
\begin{UseCase}{CU 5.2}{Modificar caso de uso}
	{
		Este caso de uso permite al actor modificar un caso de uso. Al ingresar los datos, el actor podrá utilizar un token que le desplegará una lista de elementos disponibles para su utilización. 
		
	}
	\UCitem{Actor}{\cdtRef{actor:liderAnalisis}{Líder de análisis}, \cdtRef{actor:analista}{Analista}}
	\UCitem{Propósito}{
		Modificar la información de un caso de uso.
	}
	\UCitem{Entradas}{
		\begin{UClist} 
			\UCli De la sección {\bf Información general del caso de uso}: 
				\begin{Citemize}
					\item \cdtRef{CasodeUso:Numero}{Número}: \ioEscribir.
					\item \cdtRef{CasodeUso:Nombre}{Nombre}: \ioEscribir.
					\item \cdtRef{CasodeUso:Resumen}{Resumen}: \ioEscribir.
				\end{Citemize}
			\UCli De la sección {\bf Descripción del caso de uso}:
				\begin{Citemize}
				        \item \cdtRef{Actor}{Actores}: \ioEscribirSeleccionar.
				        \item \cdtRef{Entrada}{Entradas}: \ioEscribirSeleccionar.
				        \item \cdtRef{Salida}{Salidas}: \ioEscribirSeleccionar.
					\item \cdtRef{RegladeNegocio}{Reglas de negocio}: \ioEscribirSeleccionar.
				\end{Citemize}
			
		\end{UClist}
	}	
	
	\UCitem{Salidas}{	
		\begin{UClist}
			\UCli \cdtRef{Proyecto:Clave}{Clave} del proyecto. \ioObtener.
			\UCli \cdtRef{Proyecto:Nombre}{Nombre} del proyecto. \ioObtener.
			\UCli \cdtRef{Modulo:Clave}{Clave} del módulo: \ioObtener.
			\UCli \cdtRef{Modulo:Nombre}{Nombre} del módulo: \ioObtener.
			\UCli \cdtRef{CasodeUso:Clave}{Clave}: \ioCalcular{\cdtIdRef{RN12}{Identificador de elemento}}.
		\end{UClist}
	}
	
	\UCitem{Mensajes}{	
		\begin{UClist}
			\UCli \cdtIdRef{MSG1}{Operación exitosa}: Se muestra en la pantalla \cdtIdRef{IU 5}{Gestionar casos de uso} para indicar que la modificación fue exitosa.
		\end{UClist}
	}
	
	\UCitem{Precondiciones}{ 
		\begin{UClist}
			\UCli Que el caso de uso se encuentre en estado ``Edición''.
	 	\end{UClist}
	}			
	\UCitem{Postcondiciones}{ 
		\begin{UClist}
			\UCli Se modificará la información del caso de uso en el sistema.	
	 	\end{UClist}
	}
	\UCitem{Errores}{
		\begin{UClist}
			\UCli \cdtIdRef{MSG4}{Dato obligatorio}: Se muestra en la pantalla \cdtIdRef{IU 5.2}{Modificar caso de uso} cuando no se ha ingresado un dato marcado como obligatorio.
			\UCli \cdtIdRef{MSG5}{Dato incorrecto}: Se muestra en la pantalla \cdtIdRef{IU 5.2}{Modificar caso de uso} cuando el tipo de dato ingresado no cumple con el tipo de dato solicitado en el campo.
			\UCli \cdtIdRef{MSG6}{Longitud inválida}: Se muestra en la pantalla \cdtIdRef{IU 5.2}{Modificar caso de uso} cuando se ha excedido la longitud de alguno de los campos.
			\UCli \cdtIdRef{MSG7}{Registro repetido}: Se muestra en la pantalla \cdtIdRef{IU 5.2}{Modificar caso de uso} cuando se registre un caso de uso con un nombre o número que ya este registrado.
			\UCli \cdtIdRef{MSG13}{Ha ocurrido un error}: Se muestra en la pantalla \cdtIdRef{IU 5}{Gestionar casos de uso} cuando el estado del caso de uso no sea ``Edición''.
			\UCli \cdtIdRef{MSG15}{Dato no registrado}: Se muestra en la pantalla \cdtIdRef{IU 5.2}{Modificar caso de uso} cuando un elemento referenciado no existe en el sistema.
			\UCli \cdtIdRef{MSG23}{Caracteres inválidos}: Se muestra en la pantalla \cdtIdRef{IU 5.2}{Modificar caso de uso} cuando el nombre del caso de uso contiene un caracter no válido.
			\UCli \cdtIdRef{MSG27}{Token incorrecto}: Se muestra en la pantalla \cdtIdRef{IU 5.2}{Modificar caso de uso} cuando el token ingresado está mal formado.
		\end{UClist}
	}
	
	\UCitem{Tipo}{
		Secundario, extiende del caso de uso \cdtIdRef{CU 5}{Gestionar casos de uso}.
	}

	
\end{UseCase}
 %-------------------------------------------------------%trayectoria Principal-----------------------------------------------
 \begin{UCtrayectoria}
    \UCpaso[\UCactor] Solicita modificar un caso de uso oprimiendo el botón \btnEditar del registro que desee de la pantalla \cdtIdRef{IU 5}{Gestionar casos de uso}.
    \UCpaso[\UCsist] Busca los términos del glosario del proyecto actual registrados en el sistema. 
    \UCpaso[\UCsist] Busca las entidades del proyecto actual registradas en el sistema. 
    \UCpaso[\UCsist] Busca las reglas de negocio del proyecto actual registradas en el sistema. 
    \UCpaso[\UCsist] Busca los mensajes del proyecto actual registrados en el sistema. 
    \UCpaso[\UCsist] Busca los actores del proyecto actual registrados en el sistema. 
    \UCpaso[\UCsist] Busca las pantallas del proyecto actual registradas en el sistema. 
    \UCpaso[\UCsist] Busca los casos de uso del proyecto actual registrados en el sistema. 
	\UCpaso[\UCsist] Busca la información del caso de uso.
	\UCpaso[\UCsist] Verifica que el estado del caso de uso sea ``Edición''. \refTray{Q}
    \UCpaso[\UCsist] Muestra la pantalla \cdtIdRef{IU 5.2}{Modificar caso de uso} en la cual se realizará la modificación del caso de uso. 
    \UCpaso[\UCactor] Modifica la información general del caso de uso. \label{cu5.2:ingresaDatos}
    \UCpaso[\UCactor] Modifica los actores del caso de uso. \refTray{A} \label{cu5.2:ingresaActores}
    \UCpaso[\UCactor] Modifica las entradas del caso de uso. \refTray{B} \refTray{C} \label{cu5.2:ingresaEntradas}
    \UCpaso[\UCactor] Modifica las salidas en la pantalla \cdtIdRef{IU 5.2}{Modificar caso de uso}. \refTray{B} \refTray{C} \refTray{D} \label{cu5.2:ingresaSalidas}
    \UCpaso[\UCactor] Modifica las reglas de negocio en la pantalla \cdtIdRef{IU 5.2}{Modificar caso de uso}. \refTray{E} \label{cu5.2:ingresaReglasNegocio}
	\UCpaso[\UCactor] Gestiona las precondiciones.\label{cu5.2:ingresaPrecond}
    \UCpaso[\UCactor] Gestiona las postcondiciones.\label{cu5.2:ingresaPostcond}
    \UCpaso[\UCactor] Solicita modificar el caso de uso oprimiendo el botón \cdtButton{Aceptar} de la pantalla \cdtIdRef{IU 5.2}{Modificar caso de uso}. \refTray{F}
	\UCpaso[\UCsist] Verifica que el estado del caso de uso sea ``Edición''. \refTray{Q}
    \UCpaso[\UCsist] Verifica que el actor ingrese todos los campos obligatorios con base en la regla de negocio  \cdtIdRef{RN8}{Datos obligatorios}. \refTray{G}
	\UCpaso[\UCsist] Verifica que el nombre del caso de uso no se encuentre registrado, con base en la regla de negocio  \cdtIdRef{RN29}{Unicidad de casos de uso}. \refTray{H}
    \UCpaso[\UCsist] Verifica que el nombre no contenga caracteres inválidos con base en la regla de negocio \cdtIdRef{RN2}{Nombres de los elementos}. \refTray{I}
	\UCpaso[\UCsist] Verifica que el número del caso de uso no se encuentre registrado, con base en la regla de negocio  \cdtIdRef{RN29}{Unicidad de casos de uso}. \refTray{J}
	    \UCpaso[\UCsist] Verifica que los datos sean proporcionados correctamente, con base en la regla de negocio \cdtIdRef{RN7}{Información correcta}. \refTray{K} \refTray{L}
    \UCpaso[\UCsist] Verifica que los tokens utilizados se encuentren correctamente estructurados, con base en la regla de negocio \cdtIdRef{RN31}{Estructura de tokens}. \refTray{M}
    \UCpaso[\UCsist] Verifica que los elementos referenciados existan en el sistema, con base en la regla de negocio \cdtIdRef{RN10}{Referencias a elementos}. \refTray{N}
    
    \UCpaso[\UCsist] Modifica el caso de uso en el sistema.
    \UCpaso[\UCsist] Muestra el mensaje \cdtIdRef{MSG1}{Operación exitosa} en la pantalla \cdtIdRef{IU 5}{Gestionar casos de uso} 
    para indicar al actor que la modificación se ha realizado exitosamente.
 \end{UCtrayectoria}
 
  %----------------------------------------------------------%trayectoria A---------------------------------------------------- 
 \begin{UCtrayectoriaA}{A}{El actor desea seleccionar un actor.}
 	\UCpaso[\UCactor] Ingresa el token {\it ACT·}.
 	\UCpaso[\UCsist] Muestra una lista con los actores encontrados.
 	\UCpaso[\UCactor] Selecciona un actor de la lista.
  	\UCpaso[\UCsist] Verifica que el nombre del actor seleccionado no contenga espacios. \refTray{Z}
  	\UCpaso[\UCsist] Agrega el nombre del actor al texto.
    \UCpaso[] Continúa con el paso \ref{cu5.2:ingresaActores} de la trayectoria principal.
 \end{UCtrayectoriaA}
  %----------------------------------------------------------%trayectoria B---------------------------------------------------- 
 \begin{UCtrayectoriaA}{B}{El actor desea seleccionar un término del glosario.}
 	\UCpaso[\UCactor] Ingresa el token {\it GLS·}.	
 	\UCpaso[\UCsist] Muestra una lista con los términos del glosario encontrados.
 	\UCpaso[\UCactor] Selecciona un término del glosario de la lista.
  	\UCpaso[\UCsist] Verifica que el nombre del término del glosario seleccionado no contenga espacios. \refTray{Z}
  	\UCpaso[\UCsist] Agrega el nombre del término del glosario al texto.
    \UCpaso[] Continúa con el paso \ref{cu5.2:ingresaEntradas} o el paso \ref{cu5.2:ingresaSalidas} de la trayectoria principal, según corresponda.
 \end{UCtrayectoriaA}

  %----------------------------------------------------------%trayectoria C---------------------------------------------------- 
 \begin{UCtrayectoriaA}{C}{El actor desea seleccionar un atributo.}
 	\UCpaso[\UCactor] Ingresa el token {\it ATR·}.
  	\UCpaso[\UCsist] Busca los atributos de las entidades encontradas.
  	\UCpaso[\UCsist] Muestra una lista de los atributos encontrados.
 	\UCpaso[\UCactor] Selecciona un atributo de la lista.
  	\UCpaso[\UCsist] Verifica que el nombre de la entidad a la que pertenece el atributo seleccionado no contenga espacios. \refTray{Z}
  	\UCpaso[\UCsist] Verifica que el nombre del atributo seleccionado no contenga espacios. \refTray{Z}
  	\UCpaso[\UCsist] Agrega el nombre de la entidad a la que pertenece el atributo al texto, seguido del signo ``:''.
  	\UCpaso[\UCsist] Agrega el nombre del atributo al texto.
    \UCpaso[] Continúa con el paso \ref{cu5.2:ingresaEntradas} o el paso \ref{cu5.2:ingresaSalidas} de la trayectoria principal, según corresponda.
 \end{UCtrayectoriaA}
%----------------------------------------------------------%trayectoria D---------------------------------------------------- 
 \begin{UCtrayectoriaA}{D}{El actor desea seleccionar un mensaje.}
 	 \UCpaso[\UCactor] Ingresa el token {\it MSG·}.	
 	\UCpaso[\UCsist] Muestra una lista con los mensajes encontrados.
 	\UCpaso[\UCactor] Selecciona un mensaje de la lista.
  	\UCpaso[\UCsist] Verifica que el nombre del mensaje seleccionado no contenga espacios. \refTray{Z}
  	\UCpaso[\UCsist] Agrega el número del mensaje al texto, seguido del signo ``:''.
  	\UCpaso[\UCsist] Agrega el nombre del mensaje al texto.
    \UCpaso[] Continúa con el paso \ref{cu5.2:ingresaSalidas} de la trayectoria principal.
 \end{UCtrayectoriaA}
 %----------------------------------------------------------%trayectoria E---------------------------------------------------- 
 \begin{UCtrayectoriaA}{E}{El actor desea seleccionar una regla de negocio.}
 	\UCpaso[\UCactor] Ingresa el token {\it RN·}.	
 	\UCpaso[\UCsist] Muestra una lista con las reglas de negocio encontradas.
 	\UCpaso[\UCactor] Selecciona una regla de negocio de la lista.
  	\UCpaso[\UCsist] Verifica que el nombre de la regla de negocio seleccionada no contenga espacios. \refTray{Z}
  	\UCpaso[\UCsist] Agrega el número de la regla de negocio al texto, seguido del signo ``:''.
  	\UCpaso[\UCsist] Agrega el nombre de la regla de negocio al texto.
    \UCpaso[] Continúa con el paso \ref{cu5.2:ingresaReglasNegocio} de la trayectoria principal.
 \end{UCtrayectoriaA}
 %----------------------------------------------------------%trayectoria F---------------------------------------------------- 
 \begin{UCtrayectoriaA}[Fin del caso de uso]{F}{El actor desea cancelar la operación.}
    \UCpaso[\UCactor] Solicita cancelar la operación oprimiendo el botón \cdtButton{Cancelar} de la pantalla \cdtIdRef{IU 5.2}{Modificar caso de uso}.
    \UCpaso[\UCsist] Muestra la pantalla \cdtIdRef{IU 5}{Gestionar casos de uso}.
 \end{UCtrayectoriaA}
  %----------------------------------------------------------%trayectoria G---------------------------------------------------- 
 \begin{UCtrayectoriaA}{G}{El actor no ingresó algún dato marcado como obligatorio.}
    \UCpaso[\UCsist] Muestra el mensaje \cdtIdRef{MSG4}{Dato obligatorio} y señala el campo que presenta el error en la pantalla 
	    \cdtIdRef{CU 5.2}{Modificar caso de uso}, indicando al actor que el dato es obligatorio.
    \UCpaso[] Continúa con el paso \ref{cu5.2:ingresaDatos} de la trayectoria principal.
 \end{UCtrayectoriaA}
 %----------------------------------------------------------%trayectoria H---------------------------------------------------- 
 \begin{UCtrayectoriaA}{H}{El actor ingresó un nombre de caso de uso repetido.}
    \UCpaso[\UCsist] Muestra el mensaje \cdtIdRef{MSG7}{Registro repetido} y señala el campo que presenta la duplicidad en la pantalla 
	    \cdtIdRef{CU 5.2}{Modificar caso de uso}, indicando al actor que existe un caso de uso con el mismo nombre.
    \UCpaso[] Continúa con el paso \ref{cu5.2:ingresaDatos} de la trayectoria principal.
 \end{UCtrayectoriaA}
%----------------------------------------------------------%trayectoria I---------------------------------------------------- 
 \begin{UCtrayectoriaA}{I}{El actor ingresó un nombre con caracteres inválidos.}
    \UCpaso[\UCsist] Muestra el mensaje \cdtIdRef{MSG23}{Caracteres inválidos} y señala el campo que contiene los caracteres inválidos.
    \UCpaso[] Continúa con el paso \ref{cu5.2:ingresaDatos} de la trayectoria principal.
 \end{UCtrayectoriaA}
 %----------------------------------------------------------%trayectoria J---------------------------------------------------- 
 \begin{UCtrayectoriaA}{J}{El actor ingresó un número de caso de uso repetido.}
    \UCpaso[\UCsist] Muestra el mensaje \cdtIdRef{MSG7}{Registro repetido} y señala el campo que presenta la duplicidad en la pantalla 
	    \cdtIdRef{CU 5.2}{Modificar caso de uso}, indicando al actor que existe un caso de uso con el mismo número.
    \UCpaso[] Continúa con el paso \ref{cu5.2:ingresaDatos} de la trayectoria principal.
 \end{UCtrayectoriaA}
 %----------------------------------------------------------%trayectoria K---------------------------------------------------- 
 \begin{UCtrayectoriaA}{K}{El actor ingresó un tipo de dato incorrecto.}
    \UCpaso[\UCsist] Muestra el mensaje \cdtIdRef{MSG4}{Formato incorrecto} y señala el campo que presenta el dato inválido en la 
    pantalla \cdtIdRef{IU 5.2}{Modificar caso de uso} para indicar que se ha ingresado un tipo de dato inválido.
    \UCpaso[] Continúa con el paso \ref{cu5.2:ingresaDatos} de la trayectoria principal.
 \end{UCtrayectoriaA}
 %----------------------------------------------------------%trayectoria L----------------------------------------------------  
 \begin{UCtrayectoriaA}{L}{El actor proporciona un dato que excede la longitud máxima.}
    \UCpaso[\UCsist] Muestra el mensaje \cdtIdRef{MSG5}{Se ha excedido la longitud máxima del campo} y señala el campo que excede la 
    longitud en la pantalla \cdtIdRef{IU 5.2}{Modificar caso de uso}, para indicar que el dato excede el tamaño máximo permitido.
    \UCpaso[] Continúa con el paso \ref{cu5.2:ingresaDatos} de la trayectoria principal.
 \end{UCtrayectoriaA}
 %----------------------------------------------------------%trayectoria M---------------------------------------------------- 
 \begin{UCtrayectoriaA}{M}{El actor ingresó un token mal formado.}
    \UCpaso[\UCsist] Muestra el mensaje \cdtIdRef{MSG27}{Token incorrecto} mencionando que el token utilizado no es correcto.
    \UCpaso[] Continúa con el paso \ref{cu5.2:ingresaDatos} de la trayectoria principal.
 \end{UCtrayectoriaA}
 %----------------------------------------------------------%trayectoria N---------------------------------------------------- 
 \begin{UCtrayectoriaA}{N}{Alguno de los elementos referenciados no existe en el sistema.}
    \UCpaso[\UCsist] Muestra el mensaje \cdtIdRef{MSG15}{Dato no registrado} mencionando el elemento que no está registrado en el sistema.
    \UCpaso[] Continúa con el paso \ref{cu5.2:ingresaDatos} de la trayectoria principal.
 \end{UCtrayectoriaA}
 %----------------------------------------------------------%trayectoria N---------------------------------------------------- 
 \begin{UCtrayectoriaA}{Q}{El caso de uso no se encuentra en estado ``Edición''.}
    \UCpaso[\UCsist] Muestra el mensaje \cdtIdRef{MSG13}{Ha ocurrido un error} en la pantalla \cdtIdRef{IU 5}{Gestionar casos de uso}, para indicar que no es posible realizar la operación debido a que el caso de uso no cuenta con el estado adecuado.
 \end{UCtrayectoriaA}
 

\subsubsection{Puntos de extensión}	
\UCExtensionPoint{El actor requiere registrar una precondición}
	{Paso \ref{cu5.2:ingresaPrecond} de la trayectoria principal}
	{\cdtIdRef{CU 5.2.2}{Registrar precondición}}
	
\UCExtensionPoint{El actor requiere registrar una postcondición}
	{Paso \ref{cu5.2:ingresaPostcond} de la trayectoria principal}
	{\cdtIdRef{CU 5.2.3}{Registrar postcondición}}