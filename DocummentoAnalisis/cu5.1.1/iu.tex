\newpage 
\subsection{IU 5.1.1 Gestionar trayectorias}
\subsubsection{Objetivo}
	
	En esta pantalla el actor puede visualizar algunos atributos de las trayectorias y las operaciones para registrar, modificar y eliminar las mismas.

\subsubsection{Diseño}

    En la figura ~\ref{IU 5.1.1} se muestra la pantalla ``Gestionar trayectorias'', por medio de la cual 
    se podrán gestionar las trayectorias a través de una tabla.
    El actor podrá podrá solicitar el registro, la modificación y la eliminación de una trayectoria mediante los botones
    \cdtButton{Registrar}, \btnEditar y \btnEliminar respectivamente. \\
	
	En la parte superior derecha, el sistema muestra el proyecto, el módulo y el caso de uso en el que actualmente se encuentra trabajando.
	

    \IUfig[.9]{cu5.1.1/images/iu.png}{IU 5.1.1}{Gestionar trayectorias}


\subsubsection{Comandos}
\begin{itemize}
	\item \cdtButton{Regresar}: Permite al actor regresar a la gestión de casos de uso, dirige a la pantalla \cdtIdRef{IU 5}{Gestionar casos de uso}.
	\item \cdtButton{Registrar}: Permite al actor solicitar el registro de una trayectoria, dirige a la pantalla \cdtIdRef{IU 5.1.1.1}{Registrar trayectoria}.
	\item \btnEditar[Modificar]: Permite al actor solicitar la modificación de una trayectoria, dirige a la pantalla \cdtIdRef{IU 5.1.1.2}{Modificar trayectoria}.
	\item \btnEliminar[Eliminar]: Permite al actor solicitar la eliminación de una trayectoria, dirige a una pantalla emergente.
\end{itemize}

\subsubsection{Mensajes}

	
\begin{description}
	\item[\cdtIdRef{MSG2}{No existe información}:] Se muestra en la pantalla \cdtIdRef{IU 5.1.1}{Gestionar trayectoria} cuando no existen trayectorias registradas.
	\item[\cdtIdRef{MSG13}{Ha ocurrido un error}:] Se muestra en la pantalla  \cdtIdRef{IU 5}{Gestionar casos de uso} cuando el estado del caso de uso sea inválido.
\end{description}
