\begin{UseCase}{CU 5.1.1}{Gestionar trayectorias}
	{
		Este caso de uso permite al analista visualizar la trayectoria principal y las trayectorias alternativas del caso de uso. También permite 
		al actor acceder a las operaciones de registro, modificación y eliminación de las trayectorias.
	}
	\UCitem{Actor}{\cdtRef{actor:liderAnalisis}{Líder de análisis}, \cdtRef{actor:analista}{Analista}}
	\UCitem{Propósito}{
		Revisar y gestionar las trayectorias de un caso de uso.
	}
	\UCitem{Entradas}{
		Ninguna
	}
	\UCitem{Salidas}{
		\begin{UClist}
			\UCli \cdtRef{Elemento:Clave}{Clave}, \cdtRef{Elemento:Numero}{número} y \cdtRef{Elemento:Nombre}{nombre} del caso de uso al que pertenece la trayectoria.
			\UCli \cdtRef{Trayectoria}{Trayectoria}: \ioTabla{\cdtRef{Elemento:Clave}{Clave}, \cdtRef{Elemento:Nombre}{Nombre}, y la \cdtRef{}{condición}}{de las trayectorias}.
		\end{UClist}
	}
	
	\UCitem{Mensajes}{
		\begin{UClist}
			\UCli \cdtIdRef{MSG2}{No existe información}: Se muestra en la pantalla \cdtIdRef{IU 5.1.1}{Gestionar trayectorias} cuando no existen trayectorias registradas.
		\end{UClist}
	}

	\UCitem{Precondiciones}{
		Ninguna
	}
	
	\UCitem{Postcondiciones}{
		\begin{UClist}
			\UCli Se podrá solicitar el registro de una trayectoria por medio del caso de uso \cdtIdRef{CU 5.1.1.1}{Registrar trayectoria}.
			\UCli Se podrá solicitar la modificación de una trayectoria por medio del caso de uso \cdtIdRef{CU 5.1.1.2}{Modificar trayectoria}.
			\UCli Se podrá solicitar la eliminación de una trayectoria por medio del caso de uso \cdtIdRef{CU 5.1.1.3}{Eliminar trayectoria}.
			\UCli Se podrá regresar a la gestión de casos de uso.
		\end{UClist}
	}

	\UCitem{Errores}{
		Ninguno
	}

	\UCitem{Tipo}{
		Secundario, extiende del caso de uso \cdtIdRef{CU 5}{Gestionar casos de uso}.
	}
\end{UseCase}
%-------------------------------------------------------%trayectoria Principal-----------------------------------------------
 \begin{UCtrayectoria}
    \UCpaso[\UCactor] Solicita gestionar las trayectorias del caso de uso presionando el botón \btnTray de la pantalla \cdtIdRef{IU 5}{Gestionar casos de uso}.
    \UCpaso[\UCsist] Busca la información de las trayectorias del caso de uso. \refTray{A}
    \UCpaso[\UCsist] Muestra la información de las trayectorias en la pantalla \cdtIdRef{IU 5.1.1}{Gestionar trayectorias}. 
    \UCpaso[\UCactor] Gestiona las trayectorias a través de los botones: \cdtButton{Registrar}, \btnEditar y \btnEliminar. \refTray{B} \label{cu5.1.1:gestionaCU}
 \end{UCtrayectoria}
 
 \begin{UCtrayectoriaA}[Fin del caso de uso]{A}{No existen registros de trayectorias.}
    \UCpaso[\UCsist] Muestra el mensaje \cdtIdRef{MSG2}{No existe información} en pantalla \cdtIdRef{IU 5.1.1}{Gestionar trayectorias} 
    para indicar que no hay registros de trayectorias para mostrar.
 \end{UCtrayectoriaA}
 
 \begin{UCtrayectoriaA}[Fin del caso de uso]{B}{El actor desea regresar a la gestión de casos de uso.}
    \UCpaso[\UCactor] Presiona el botón \cdtButton{Regresar}.
    \UCpaso[\UCsist] Muestra la pantalla \cdtIdRef{IU 5}{Gestionar casos de uso}.
 \end{UCtrayectoriaA}
 
\subsubsection{Puntos de extensión}

\UCExtensionPoint{El actor requiere registrar una trayectoria}
	{Paso \ref{cu5.1.1:gestionaCU}}
	{\cdtIdRef{CU 5.1.1.1}{Registrar trayectoria}}
\UCExtensionPoint{El actor requiere modificar una trayectoria}
	{Paso \ref{cu5.1.1:gestionaCU}}
	{\cdtIdRef{CU 5.1.1.2}{Modificar trayectoria}}
\UCExtensionPoint{El actor requiere eliminar una trayectoria}
	{Paso \ref{cu5.1.1:gestionaCU}}
	{\cdtIdRef{CU 5.1.1.3}{Eliminar trayectoria}}
  