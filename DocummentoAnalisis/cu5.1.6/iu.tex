\newpage 
\subsection{IU 5.1.6 Gestionar puntos de extensión}
\subsubsection{Objetivo}
	
	En esta pantalla el actor puede visualizar los puntos de extensión registrados y puede solicitar las operaciones de registrar, modificar y eliminar puntos de extensión.

\subsubsection{Diseño}

    En la figura ~\ref{IU 5.1.6} se muestra la pantalla ``Gestionar puntos de extensión'', por medio de la cual 
    se podrán gestionar los puntos de extensión a través de una tabla.
    El actor podrá podrá solicitar el registro, la modificación y la eliminación de un punto de extensión mediante los botones
    \cdtButton{Registrar}, \btnEditar y \btnEliminar respectivamente. \\
    

    \IUfig[.9]{cu5.1.6/images/iu.png}{IU 5.1.6}{Gestionar puntos de extensión}


\subsubsection{Comandos}
\begin{itemize}
	\item \cdtButton{Regresar}: Permite al actor regresar a la gestión de casos de uso, dirige a la pantalla \cdtIdRef{IU 5}{Gestionar casos de uso}.
	\item \cdtButton{Registrar}: Permite al actor solicitar el registro de un punto de extensión, dirige a la pantalla \cdtIdRef{IU 5.1.6.1}{Registrar punto de extensión}.
	\item \btnEditar[Modificar]: Permite al actor solicitar la modificación de un punto de extensión, dirige a la pantalla \cdtIdRef{IU 5.1.6.2}{Modificar punto de extensión}.
	\item \btnEliminar[Eliminar]: Permite al actor solicitar la eliminación de un punto de extensión, dirige a la pantalla \cdtIdRef{IU 5.1.6.3}{Eliminar punto de extensión}.
\end{itemize}

\subsubsection{Mensajes}

	
\begin{description}
	\item[\cdtIdRef{MSG2}{No existe información}:] Se muestra en la pantalla \cdtIdRef{IU 5.1.6}{Gestionar puntos de extensión} cuando no existen puntos de extensión registrados.
\end{description}
