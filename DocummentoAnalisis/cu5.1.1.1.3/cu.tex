\begin{UseCase}{CU 5.1.1.1.3}{Eliminar paso}
	{
		Este caso de uso permite al actor eliminar un registro de la tabla de pasos de la pantalla \cdtIdRef{IU 5.1.1.1}{Registrar trayectoria} o de la pantalla \cdtIdRef{IU 5.1.1.2}{Modificar trayectoria}.
	}
	
	\UCitem{Actor}{\cdtRef{actor:liderAnalisis}{Líder de análisis}, \cdtRef{actor:analista}{Analista}}
	\UCitem{Propósito}{
		Eliminar la información de un paso de la tabla de pasos.
	}
	\UCitem{Entradas}{
		Ninguna
	}
	\UCitem{Salidas}{
		Ninguna
	}
	\UCitem{Mensajes}{
		Ninguno
	}
	\UCitem{Precondiciones}{
		Ninguna
	}
	
	\UCitem{Postcondiciones}{
		\begin{UClist}
			\UCli Se eliminará el atributo de la tabla de atributos correspondiente.
		\end{UClist}
	}

	\UCitem{Errores}{
		\begin{UClist}
			\UCli \cdtIdRef{MSG14}{Eliminación no permitida}: Se muestra en una pantalla emergente cuando no se pueda eliminar el paso debido a que está siendo utilizado por otro elemento.
		\end{UClist}
	}

	\UCitem{Tipo}{
		Secundario, extiende de los casos de uso \cdtIdRef{CU 5.1.1.1}{Registrar trayectoria} y \cdtIdRef{CU 5.1.1.2}{Modificar trayectoria}.
	}
\end{UseCase}
%-------------------------------------------------------%trayectoria Principal-----------------------------------------------
 \begin{UCtrayectoria}
    \UCpaso[\UCactor] Solicita eliminar un paso oprimiendo el botón \btnEliminar del registro que desea eliminar de la pantalla \cdtIdRef{IU 5.1.1.1}{Registrar trayectoria} o de la pantalla \cdtIdRef{IU 5.1.1.2}{Modificar trayectoria}.
    \UCpaso[\UCsist] Busca los elementos que estén referenciando al paso.
    \UCpaso[\UCsist] Verifica que ningún elemento esté referenciando al paso. \refTray{A}
    \UCpaso[\UCsist] Elimina el paso de la tabla correspondiente.
 \end{UCtrayectoria}
  %----------------------------------------------------------%trayectoria A---------------------------------------------------- 
 \begin{UCtrayectoriaA}[Fin del caso de uso]{A}{El paso está siendo referenciado en un elemento.}
    \UCpaso[\UCsist] Muestra el mensaje \cdtIdRef{MSG14}{Eliminación no permitida} en una pantalla emergente
    con la lista de elementos que están referenciando al paso.
    \UCpaso[\UCactor] Oprime el botón \cdtButton{Aceptar} de la pantalla emergente.
    \UCpaso[\UCsist] Muestra la pantalla donde se solicitó la operación.
 \end{UCtrayectoriaA}