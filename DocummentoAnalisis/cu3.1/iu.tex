\subsection{IU 3.1 Registrar módulo}

\subsubsection{Objetivo}
	
    Esta pantalla permite al actor registrar la información de un módulo.

\subsubsection{Diseño}

    En la figura ~\ref{IU 3.1} se muestra la pantalla ``Registrar módulo'' que permite al actor registrar un módulo. \\
    
    Una vez ingresada la información solicitada, el actor deberá oprimir el botón \cdtButton{Aceptar}. El sistema validará y registrará la 
    información solo si se han cumplido todas las reglas de negocio establecidas.  \\
    
    Finalmente se mostrará el mensaje \cdtIdRef{MSG1}{Operación exitosa} en la pantalla \cdtIdRef{IU 3}{Gestionar módulos}, 
    para indicar que la información del módulo
    se ha registrado correctamente.        


    \IUfig[.9]{cu3.1/images/iu.png}{IU 3.1}{Registrar módulo}

\subsubsection{Comandos}
\begin{itemize}
	\item \cdtButton{Aceptar}: Permite al actor guardar el registro del módulo, dirige a la pantalla \cdtIdRef{IU 3}{Gestionar módulos}.
	\item \cdtButton{Cancelar}: Permite al actor cancelar el registro del módulo, dirige a la pantalla \cdtIdRef{IU 3}{Gestionar módulos}.
\end{itemize}

\subsubsection{Mensajes}
	
\begin{description}
	\item[\cdtIdRef{MSG1}{Operación exitosa}:] Se muestra en la pantalla \cdtIdRef{IU 3}{Gestionar módulos} para indicar que el registro fue exitoso.
	\item[\cdtIdRef{MSG4}{Dato obligatorio}:] Se muestra en la pantalla \cdtIdRef{IU 3.1}{Registrar módulo} cuando no se ha ingresado un dato marcado como obligatorio.
	\item[\cdtIdRef{MSG5}{Dato incorrecto}:] Se muestra en la pantalla \cdtIdRef{IU 3.1}{Registrar módulo} cuando el tipo de dato ingresado no cumple con el tipo de dato solicitado en el campo.
	\item[\cdtIdRef{MSG6}{Longitud inválida}:] Se muestra en la pantalla \cdtIdRef{IU 3.1}{Registrar módulo} cuando se ha excedido la longitud de alguno de los campos.
	\item[\cdtIdRef{MSG7}{Registro repetido}:] Se muestra en la pantalla \cdtIdRef{IU 3.1}{Registrar módulo} cuando se registre un módulo con un nombre o clave repetido en el proyecto.
\end{description}
