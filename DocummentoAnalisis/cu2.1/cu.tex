\begin{UseCase}{CU 2.1}{Registrar persona}
	{
		Este caso de uso permite registrar la información de una persona que podrá ser elegida como colaborador de un proyecto.
	}
	
	\UCitem{Actor}{\cdtRef{actor:administrador}{Administrador}}
	\UCitem{Propósito}{
		Registrar la información de una persona.
	}
	\UCitem{Entradas}{
		\begin{UClist}
			\UCli \cdtRef{Persona:CURP}{CURP}: \ioEscribir.
			\UCli \cdtRef{Persona:Nombre}{Nombre}: \ioEscribir.
			\UCli \cdtRef{Persona:PrimerApellido}{Primer Apellido}: \ioEscribir.
			\UCli \cdtRef{Persona:SegundoApellido}{Segundo Apellido}: \ioEscribir.
			\UCli \cdtRef{Persona:CorreoElectronico}{Correo Electrónico}: \ioEscribir.
			\UCli \cdtRef{Persona:Contrasena}{Contraseña}: \ioEscribir.
			\UCli \cdtRef{Persona:Telefono}{Teléfono}: \ioEscribir.
		\end{UClist}
	}
	\UCitem{Salidas}{
		Ninguna
	}
	
	\UCitem{Mensajes}{
		\begin{UClist}
			\UCli \cdtIdRef{MSG1}{Operación exitosa}: Se muestra en la pantalla \cdtIdRef{IU 2}{Gestionar personal} para indicar que el registro fue exitoso.
		\end{UClist}
	}

	\UCitem{Precondiciones}{
		Ninguna
	}
	
	\UCitem{Postcondiciones}{
		\begin{UClist}
			\UCli Se registrará una persona en el sistema.
		\end{UClist}
	}

	\UCitem{Errores}{
		\begin{UClist}
			\UCli \cdtIdRef{MSG4}{Dato obligatorio}: Se muestra en la pantalla \cdtIdRef{IU 2.1}{Registrar persona} cuando no se ha ingresado un dato marcado como obligatorio.
			\UCli \cdtIdRef{MSG5}{Dato incorrecto}: Se muestra en la pantalla \cdtIdRef{IU 2.1}{Registrar persona} cuando el tipo de dato ingresado no cumple con el tipo de dato solicitado en el campo.
			\UCli \cdtIdRef{MSG6}{Longitud inválida}: Se muestra en la pantalla \cdtIdRef{IU 2.1}{Registrar persona} cuando se ha excedido la longitud de alguno de los campos.
			\UCli \cdtIdRef{MSG7}{Registro repetido}: Se muestra en la pantalla \cdtIdRef{IU 2.1}{Registrar persona} cuando se registre una persona con una CURP que ya exista.
		\end{UClist}
	}

	\UCitem{Tipo}{
		Secundario, extiende del casos de uso \cdtIdRef{CU 2}{Gestionar personal}.
	}
\end{UseCase}
%-------------------------------------------------------%trayectoria Principal-----------------------------------------------
 \begin{UCtrayectoria}
    \UCpaso[\UCactor] Solicita registrar una persona oprimiendo el botón \cdtButton{Registrar} de la pantalla \cdtIdRef{IU 2}{Gestionar personal}.
    \UCpaso[\UCsist] Muestra la pantalla \cdtIdRef{IU 2.1}{Registrar persona} en la cual se realizará el registro. 
    \UCpaso[\UCactor] Ingresa la información solicitada en la pantalla. \label{cu2.1:ingresaDatos}
    \UCpaso[\UCactor] Solicita guardar la información de la persona oprimiendo el botón \cdtButton{Aceptar} de la pantalla \cdtIdRef{IU 2.1}{Registrar persona}. \refTray{A} 
    \UCpaso[\UCsist] Verifica que el actor ingrese todos los campos obligatorios con base en la regla de negocio  \cdtIdRef{RN8}{Datos obligatorios}. \refTray{B}
    \UCpaso[\UCsist] Verifica que la CURP no se encuentre registrada en el sistema con base en la regla de negocio  \cdtIdRef{RN6}{Unicidad de nombres}. \refTray{C}
    \UCpaso[\UCsist] Verifica que los datos requeridos sean proporcionados correctamente como se especifica en la regla de negocio \cdtIdRef{RN7}{Información correcta}. \refTray{D} \refTray{E}
    \UCpaso[\UCsist] Registra la información de la persona en el sistema.
    \UCpaso[\UCsist] Envía un correo con el mensaje \cdtIdRef{MSG34}{Datos de sesión} a la cuenta de correo electrónico ingresado con la contraseña ingresada.
    \UCpaso[\UCsist] Muestra el mensaje \cdtIdRef{MSG1}{Operación exitosa} en la pantalla \cdtIdRef{IU 2}{Gestionar personal}
    para indicar al actor que el registro se ha realizado exitosamente.
 \end{UCtrayectoria}
 %----------------------------------------------------------%trayectoria A---------------------------------------------------- 
 \begin{UCtrayectoriaA}[Fin del caso de uso]{A}{El actor desea cancelar la operación.}
    \UCpaso[\UCactor] Solicita cancelar la operación oprimiendo el botón \cdtButton{Cancelar} de la pantalla \cdtIdRef{IU 2.1}{Registrar persona}.
    \UCpaso[\UCsist] Muestra la pantalla donde se solicitó la operación.
 \end{UCtrayectoriaA} 
  %----------------------------------------------------------%trayectoria B---------------------------------------------------- 
 \begin{UCtrayectoriaA}{B}{El actor no ingresó algún dato marcado como obligatorio.}
    \UCpaso[\UCsist] Muestra el mensaje \cdtIdRef{MSG4}{Dato obligatorio} y señala el campo que presenta el error en la pantalla 
	    \cdtIdRef{IU 2.1}{Registrar persona}, indicando al actor que el dato es obligatorio.
    \UCpaso[] Continúa con el paso \ref{cu2.1:ingresaDatos} de la trayectoria principal.
 \end{UCtrayectoriaA}
 %----------------------------------------------------------%trayectoria C---------------------------------------------------- 
 \begin{UCtrayectoriaA}{C}{El actor ingresó una CURP repetida.}
    \UCpaso[\UCsist] Muestra el mensaje \cdtIdRef{MSG7}{Registro repetido} y señala el campo que presenta la duplicidad en la pantalla 
	    \cdtIdRef{IU 2.1}{Registrar persona}, indicando al actor que existe una persona con la misma CURP.
    \UCpaso[] Continúa con el paso \ref{cu2.1:ingresaDatos} de la trayectoria principal.
 \end{UCtrayectoriaA}
 
 %----------------------------------------------------------%trayectoria D----------------------------------------------------  
 \begin{UCtrayectoriaA}{D}{El actor proporciona un dato que excede la longitud máxima.}
    \UCpaso[\UCsist] Muestra el mensaje \cdtIdRef{MSG5}{Se ha excedido la longitud máxima del campo} y señala el campo que excede la 
    longitud en la pantalla \cdtIdRef{IU 2.1}{Registrar persona}, para indicar que el dato excede el tamaño máximo permitido.
    \UCpaso[] Continúa con el paso \ref{cu2.1:ingresaDatos} de la trayectoria principal.
 \end{UCtrayectoriaA}
 
 %----------------------------------------------------------%trayectoria E---------------------------------------------------- 
 \begin{UCtrayectoriaA}{E}{El actor ingresó un tipo de dato incorrecto.}
    \UCpaso[\UCsist] Muestra el mensaje \cdtIdRef{MSG4}{Formato incorrecto} y señala el campo que presenta el dato inválido en la 
    pantalla \cdtIdRef{IU 2.1}{Registrar persona} para indicar que se ha ingresado un tipo de dato inválido.
    \UCpaso[] Continúa con el paso \ref{cu2.1:ingresaDatos} de la trayectoria principal.
 \end{UCtrayectoriaA}
 