\begin{UseCase}{CU 13}{Iniciar sesión}
	{
		Este caso de uso permite al actor iniciar sesión en el sistema.
		
	}
	\UCitem{Actor}{\cdtRef{actor:liderAnalisis}{Líder de análisis}, \cdtRef{actor:analista}{Analista}, \cdtRef{actor:administrador}{Administrador}}
	\UCitem{Propósito}{
		Iniciar sesión en el sistema.
	}
	\UCitem{Entradas}{
		\begin{UClist} 
			\UCli \cdtRef{Colaborador:CorreoElectronico}{Correo electrónico}: \ioEscribir
			\UCli \cdtRef{Colaborador:Contrasena}{Contraseña}: \ioEscribir
		\end{UClist}
	}	
	
	\UCitem{Salidas}{	
		Ninguna
	}
	
	\UCitem{Mensajes}{	
		Ninguno
	}
	
	\UCitem{Precondiciones}{ 
		\begin{UClist}
			\UCli Que el actor se encuentre registrado en el sistema.
		\end{UClist}
	}			
	\UCitem{Postcondiciones}{ 
		\begin{UClist}
			\UCli El actor podrá hacer uso del sistema.
	 	\end{UClist}
	}
	\UCitem{Errores}{
		\begin{UClist}
			\UCli \cdtIdRef{MSG4}{Dato obligatorio}: Se muestra en la pantalla \cdtIdRef{IU 13}{Iniciar sesión} cuando no se ha ingresado un dato marcado como obligatorio.
			\UCli \cdtIdRef{MSG6}{Longitud inválida}: Se muestra en la pantalla \cdtIdRef{IU 13}{Iniciar sesión} cuando se ha excedido la longitud de alguno de los campos.
			\UCli \cdtIdRef{MSG31}{Correo electrónico y/o contraseña incorrectos}: Se muestra en la pantalla \cdtIdRef{IU 13}{Iniciar sesión} cuando el correo electrónico y/o la contraseña ingresada son incorrectos.
		\end{UClist}
	}
	
	\UCitem{Tipo}{
		Primario
	}

	
\end{UseCase}
 %-------------------------------------------------------%trayectoria Principal-----------------------------------------------
 \begin{UCtrayectoria}
    \UCpaso[\UCactor] Ingresa al sistema a través de la URL.
    \UCpaso[\UCsist] Muestra la pantalla \cdtIdRef{IU 13}{Iniciar sesión}.
    \UCpaso[\UCactor] Ingresa los datos solicitados en la pantalla \cdtIdRef{IU 13}{Iniciar sesión}. \label{cu13:ingresaDatos}
	\UCpaso[\UCactor] Oprime el botón \cdtButton{Aceptar} de la pantalla \cdtIdRef{IU 13}{Iniciar sesión}.
    \UCpaso[\UCsist] Verifica que el actor ingrese todos los campos obligatorios con base en la regla de negocio  \cdtIdRef{RN8}{Datos obligatorios}. \refTray{A}
    \UCpaso[\UCsist] Verifica que los datos requeridos sean proporcionados correctamente como se especifica en la regla de negocio \cdtIdRef{RN7}{Información correcta}. \refTray{B}
    \UCpaso[\UCsist] Verifica que el actor se encuentre registrado. \refTray{C}
    \UCpaso[\UCsist] Verifica que la contraseña ingresada sea correcta. \refTray{C}
    \UCpaso[\UCsist] Muestra la pantalla \cdtIdRef{IU 4}{Gestionar proyectos}.
 \end{UCtrayectoria}
 
  
  %----------------------------------------------------------%trayectoria A---------------------------------------------------- 
 \begin{UCtrayectoriaA}{A}{El actor no ingresó algún dato marcado como obligatorio.}
    \UCpaso[\UCsist] Muestra el mensaje \cdtIdRef{MSG4}{Dato obligatorio} y señala el campo que presenta el error en la pantalla 
	    \cdtIdRef{IU 13}{Iniciar sesión}, indicando al actor que el dato es obligatorio.
    \UCpaso[] Continúa con el paso \ref{cu13:ingresaDatos} de la trayectoria principal.
 \end{UCtrayectoriaA}

 %----------------------------------------------------------%trayectoria B----------------------------------------------------  
 \begin{UCtrayectoriaA}{B}{El actor proporciona un dato que excede la longitud máxima.}
    \UCpaso[\UCsist] Muestra el mensaje \cdtIdRef{MSG6}{Longitud inválida} y señala el campo que excede la 
    longitud en la pantalla \cdtIdRef{IU 13}{Iniciar sesión}, para indicar que el dato excede el tamaño máximo permitido.
    \UCpaso[] Continúa con el paso \ref{cu13:ingresaDatos} de la trayectoria principal.
 \end{UCtrayectoriaA}
 %----------------------------------------------------------%trayectoria c----------------------------------------------------  
 \begin{UCtrayectoriaA}{C}{Existe un error en los datos.}
    \UCpaso[\UCsist] Muestra el mensaje \cdtIdRef{MSG31}{Correo electrónico y/o contraseña incorrectos}, para indicar que los datos ingresados son incorrectos.
    \UCpaso[] Continúa con el paso \ref{cu13:ingresaDatos} de la trayectoria principal.
 \end{UCtrayectoriaA}