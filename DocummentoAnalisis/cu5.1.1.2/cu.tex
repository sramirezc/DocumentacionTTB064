\begin{UseCase}{CU 5.1.1.2}{Modificar trayectoria}
	{
		Las trayectorias describen los escenarios ideales y alternos de un sistema mediante una serie de pasos.
		Este caso de uso permite al analista modificar la información de una trayectoria, así como gestionar sus pasos.
		
	}
	\UCitem{Actor}{\cdtRef{actor:liderAnalisis}{Líder de análisis}, \cdtRef{actor:analista}{Analista}}
	\UCitem{Propósito}{
		Modificar la información de una trayectoria, así como gestionar sus pasos.
	}
	\UCitem{Entradas}{
		\begin{UClist}
			\UCli \cdtRef{Trayectoria:Identificador}{Clave}: \ioEscribir
			\UCli \cdtRef{Trayectoria:Alternativa}{Tipo}: \ioSeleccionar
			\UCli \cdtRef{Trayectoria:Condicion}{Condición}: \ioEscribir
			\UCli \cdtRef{Trayectoria:FindelCasodeUso}{Fin del caso de uso}: \ioSeleccionar
			\UCli \cdtRef{Actualizacion:Comentario}{Comentario de la actualización}: \ioEscribir.
		\end{UClist}
	}
	\UCitem{Salidas}{
		\begin{UClist}
			\UCli \cdtRef{Revision:Observaciones}{Observaciones}: \ioObtener.
			\UCli \cdtRef{Elemento:Clave}{Clave}, \cdtRef{Elemento:Numero}{número} y \cdtRef{Elemento:Nombre}{nombre} del caso de uso al que pertenece la trayectoria.
			\UCli \cdtRef{Trayectoria:Identificador}{Clave}: \ioObtener.
			\UCli \cdtRef{Trayectoria:Alternativa}{Tipo}: \ioObtener.
			\UCli \cdtRef{Trayectoria:Condicion}{Condición}: \ioObtener.
			\UCli \cdtRef{Trayectoria:FindelCasodeUso}{Fin del caso de uso}: \ioObtener.
			\UCli \cdtRef{Paso}{Paso}: \ioTabla{\cdtRef{Paso:Numero}{Número} y la \cdtRef{Paso:Redaccion}{Redacción}}{de pasos de la trayectoria}.
		\end{UClist}
	}
	
	\UCitem{Mensajes}{
		\begin{UClist}
			\UCli \cdtIdRef{MSG1}{Operación exitosa}: Se muestra en la pantalla \cdtIdRef{IU 5.1.1}{Gestionar trayectorias} para indicar que la modificación fue exitosa.
		\end{UClist}
	}

	\UCitem{Precondiciones}{
		\begin{UClist}
			\UCli Que el caso de uso al que pertenece la trayectoria se encuentre en estado ``Edición'' o en estado ``Pendiente de corrección''.
			\UCli Que existan verbos registrados en el sistema.
		\end{UClist}
	}
	
	\UCitem{Postcondiciones}{
		\UCli El caso de uso cambiará a estado ``Edición''.
	}

	\UCitem{Errores}{
		\begin{UClist}
			\UCli \cdtIdRef{MSG4}{Dato obligatorio}: Se muestra en la pantalla \cdtIdRef{CU 5.1.1.2}{Modificar trayectoria} o en la pantalla emergente \cdtIdRef{IU 5.1.1.2A}{Modificar trayectoria: Comentario} cuando no se ha ingresado un dato marcado como obligatorio.
			\UCli \cdtIdRef{MSG5}{Dato incorrecto}: Se muestra en la pantalla \cdtIdRef{CU 5.1.1.2}{Modificar trayectoria} cuando el tipo de dato ingresado no cumple con el tipo de dato solicitado en el campo.
			\UCli \cdtIdRef{MSG6}{Longitud inválida}: Se muestra en la pantalla \cdtIdRef{CU 5.1.1.2}{Modificar trayectoria} o en la pantalla emergente \cdtIdRef{IU 5.1.1.2A}{Modificar trayectoria: Comentario}  cuando se ha excedido la longitud de alguno de los campos.
			\UCli \cdtIdRef{MSG7}{Registro repetido}: Se muestra en la pantalla \cdtIdRef{CU 5.1.1.2}{Modificar trayectoria} cuando se haya ingresado una clave que ya esté registrada.
			\UCli \cdtIdRef{MSG13}{Ha ocurrido un error}: Se muestra en la pantalla \cdtIdRef{IU 5.1.1}{Gestionar trayectorias} cuando no existan verbos registrados en el sistema o cuando el estado del caso de uso al que pertece la trayectoria sea inválido.
			\UCli \cdtIdRef{MSG15}{Dato no registrado}: Se muestra en la pantalla \cdtIdRef{CU 5.1.1.2}{Modificar trayectoria} cuando un elemento referenciado no existe en el sistema.
			\UCli \cdtIdRef{MSG23}{Caracteres inválidos}: Se muestra en la pantalla \cdtIdRef{CU 5.1.1.2}{Modificar trayectoria} cuando la clave de la trayectoria contiene un carácter no válido.
			\UCli \cdtIdRef{MSG27}{Token incorrecto}: Se muestra en la pantalla \cdtIdRef{CU 5.1.1.2}{Modificar trayectoria} cuando el token ingresado está mal formado.
			
		\end{UClist}
	}

	\UCitem{Tipo}{
		Secundario, extiende del caso de uso \cdtIdRef{CU 5.1.1}{Gestionar trayectorias}.
% 		\clearpage
	}
	
\end{UseCase}

%-------------------------------------------------------%trayectoria Principal-----------------------------------------------
 \begin{UCtrayectoria}
	\UCpaso[\UCactor] Oprime el botón \btnEditar de la trayectoria que desea modificar en la pantalla \cdtIdRef{IU 5.1.1}{Gestionar trayectorias}. 
	\UCpaso[\UCsist] Busca información referente a los verbos en el sistema. \refTray{K}
	\UCpaso[\UCsist] Busca la información de la trayectoria.	
	\UCpaso[\UCsist] Verifica que el caso de uso al que pertenece la trayectoria se encuentre en estado ``Edición'' o en estado ``Pendiente de corrección''.\refTray{L}
	\UCpaso[\UCsist] Busca entidades, actores, casos de uso, pantallas, mensajes, reglas de negocio y términos del glosario asociados al proyecto.
	\UCpaso[\UCsist] Busca las trayectorias y los pasos del caso de uso al que pertenece la trayectoria.
	\UCpaso[\UCsist] Verifica que la sección trayectorias del caso de uso al que pertenece la trayectoria no cuente con observaciones. \refTray{M}
	\UCpaso[\UCsist] Verifica que no exista una trayectoria principal asociada al caso de uso. \refTray{N} 
	\UCpaso[\UCsist] Muestra la información encontrada en la pantalla \cdtIdRef{IU 5.1.1.2}{Modificar trayectoria}. \label{cu5.1.1.2:muestra}
	\UCpaso[\UCactor] Modifica la clave de la trayectoria.
	\UCpaso[\UCactor] Selecciona que es la trayectoria principal. \refTray{A} \label{cu5.1.1.2:ingresaDatos}
	\UCpaso[\UCsist] Marca que en la trayectoria termina el caso de uso.
	\UCpaso[\UCactor] Gestiona los pasos de la trayectoria. \label{cu5.1.1.2:gestionaPasos}
	\UCpaso[\UCactor] Solicita modificar la trayectoria oprimiendo el botón \cdtButton{Aceptar} de la pantalla \cdtIdRef{IU 5.1.1.2}{Modificar trayectoria}. \refTray{B} 
	\UCpaso[\UCsist] Verifica que el actor haya ingresado todos los campos obligatorios con base en la regla de negocio \cdtIdRef{RN8}{Datos obligatorios}. \refTray{C}
	\UCpaso[\UCsist] Verifica que la trayectoria no se encuentre registrada en el sistema con base en la regla de negocio \cdtIdRef{RN6}{Unicidad de nombres}. \refTray{D}
	\UCpaso[\UCsist] Verifica que los datos requeridos sean proporcionados correctamente como se especifica en la regla de negocio \cdtIdRef{RN7}{Información correcta}. \refTray{E} \refTray{F}
	\UCpaso[\UCsist] Verifica que la clave no contenga caracteres inválidos con base en la regla de negocio \cdtIdRef{RN16}{Nombres de las trayectorias}. \refTray{G}	
	\UCpaso[\UCsist] Verifica que el usuario haya ingresado al menos un paso. \refTray{H}
	\UCpaso[\UCsist] Verifica que los tokens utilizados en los pasos estén bien formados. \refTray{I}
	\UCpaso[\UCsist] Verifica que los elementos referenciados en los pasos existan. \refTray{J}
	\UCpaso[\UCsist] Muestra la pantalla emergente \cdtIdRef{IU 5.1.1.2A}{Modificar trayectoria: Comentarios}.
	\UCpaso[\UCsist] Ingresa el comentario referente a la modificación realizada en la pantalla emergente \cdtIdRef{IU 5.1.1.2A}{Modificar trayectoria: Comentarios}. \label{cu5.1.1.2:ingresaComentario}
    \UCpaso[\UCsist] Verifica que el actor ingrese todos los campos obligatorios con base en la regla de negocio  \cdtIdRef{RN8}{Datos obligatorios}. \refTray{C}
    \UCpaso[\UCsist] Verifica que los datos requeridos sean proporcionados correctamente como se especifica en la regla de negocio \cdtIdRef{RN7}{Información correcta}. \refTray{F}
	\UCpaso[\UCsist] Verifica que el caso de uso al que pertenece la trayectoria se encuentre en estado ``Edición'' o en estado ``Pendiente de corrección''.\refTray{L}
	\UCpaso[\UCsist] Modifica la información de la trayectoria en el sistema.
    \UCpaso[\UCsist] Cambia el estado del caso de uso a ``Edición''.
	\UCpaso[\UCsist] Muestra el mensaje \cdtIdRef{MSG1}{Operación exitosa} en la pantalla \cdtIdRef{IU 5.1.1}{Gestionar trayectorias} 
	para indicar al actor que la modificación se ha realizado exitosamente.
    
\end{UCtrayectoria}
 %----------------------------------------------------------%trayectoria A---------------------------------------------------- 
 \begin{UCtrayectoriaA}{A}{El actor selecciona que la trayectoria es alternativa.}
    \UCpaso[\UCsist] Muestra el campo de condición.
    \UCpaso[\UCactor] Ingresa la condición de la trayectoria.
    \UCpaso[\UCactor] Selecciona si en la trayectoria se termina el caso de uso.
    \UCpaso[] Continúa con el paso \ref{cu5.1.1.2:gestionaPasos} de la trayectoria principal.
 \end{UCtrayectoriaA}
 %----------------------------------------------------------%trayectoria B---------------------------------------------------- 
 \begin{UCtrayectoriaA}[Fin del caso de uso]{B}{El actor desea cancelar la operación.}
    \UCpaso[\UCactor] Solicita cancelar la operación oprimiendo el botón \cdtButton{Cancelar} de la pantalla \cdtIdRef{IU 5.1.1.2}{Modificar trayectoria}.
    \UCpaso[\UCsist] Muestra la pantalla \cdtIdRef{IU 5.1.1}{Gestionar trayectorias}.
 \end{UCtrayectoriaA}
 
 %----------------------------------------------------------%trayectoria C---------------------------------------------------- 
 \begin{UCtrayectoriaA}{C}{El actor no ingresó algún dato marcado como obligatorio.}
    \UCpaso[\UCsist] Muestra el mensaje \cdtIdRef{MSG4}{Dato obligatorio} y señala el campo que presenta el error en la pantalla 
	    \cdtIdRef{CU 5.1.1.2}{Modificar trayectoria} o en la pantalla emergente \cdtIdRef{IU 5.1.1.2A}{Modificar trayectoria: Comentarios}, según corresponda, indicando al actor que el dato es obligatorio.
    \UCpaso[] Continúa con el paso \ref{cu5.1.1.2:ingresaDatos} o con el paso \ref{cu5.1.1.2:ingresaComentario} de la trayectoria principal, según corresponda.
 \end{UCtrayectoriaA}
 %----------------------------------------------------------%trayectoria D---------------------------------------------------- 
 \begin{UCtrayectoriaA}{D}{El actor ingresó un nombre de trayectoria repetido.}
    \UCpaso[\UCsist] Muestra el mensaje \cdtIdRef{MSG7}{Registro repetido} y señala el campo que presenta la duplicidad en la pantalla 
	    \cdtIdRef{CU 5.1.1.2}{Modificar trayectoria}, indicando al actor que existe una trayectoria con el mismo nombre.
    \UCpaso[] Continúa con el paso \ref{cu5.1.1.2:ingresaDatos} de la trayectoria principal.
 \end{UCtrayectoriaA}
 %----------------------------------------------------------%trayectoria E---------------------------------------------------- 
 \begin{UCtrayectoriaA}{E}{El actor ingresó un tipo de dato incorrecto.}
    \UCpaso[\UCsist] Muestra el mensaje \cdtIdRef{MSG4}{Formato incorrecto} y señala el campo que presenta el dato inválido en la 
    pantalla \cdtIdRef{CU 5.1.1.2}{Modificar trayectoria} para indicar que se ha ingresado un tipo de dato inválido.
    \UCpaso[] Continúa con el paso \ref{cu5.1.1.2:ingresaDatos} de la trayectoria principal.
 \end{UCtrayectoriaA}
 %----------------------------------------------------------%trayectoria F--------------------------------------------------  
 \begin{UCtrayectoriaA}{F}{El actor proporciona un dato que excede la longitud máxima.}
    \UCpaso[\UCsist] Muestra el mensaje \cdtIdRef{MSG5}{Se ha excedido la longitud máxima del campo} y señala el campo que excede la 
    longitud en la pantalla \cdtIdRef{CU 5.1.1.2}{Modificar trayectoria} o en la pantalla emergente \cdtIdRef{IU 5.1.1.2A}{Modificar trayectoria: Comentarios}, según corresponda, para indicar que el dato excede el tamaño máximo permitido.
    \UCpaso[] Continúa con el paso \ref{cu5.1.1.2:ingresaDatos} de la trayectoria principal.
 \end{UCtrayectoriaA}
 
 %----------------------------------------------------------%trayectoria G---------------------------------------------------- 
 \begin{UCtrayectoriaA}{G}{El actor ingresó una clave con caracteres inválidos.}
    \UCpaso[\UCsist] Muestra el mensaje \cdtIdRef{MSG23}{Caracteres inválidos} y señala el campo que contiene los caracteres inválidos.
    \UCpaso[] Continúa con el paso \ref{cu5.1.1.2:ingresaDatos} de la trayectoria principal.
 \end{UCtrayectoriaA}
 %----------------------------------------------------------%trayectoria H---------------------------------------------------- 
 \begin{UCtrayectoriaA}{H}{El actor no registró ningún paso.}
    \UCpaso[\UCsist] Muestra el mensaje \cdtIdRef{MSG18}{Registro necesario} en la sección de pasos.
    \UCpaso[] Continúa con el paso \ref{cu5.1.1.2:ingresaDatos} de la trayectoria principal.
 \end{UCtrayectoriaA}
 %----------------------------------------------------------%trayectoria I---------------------------------------------------- 
 \begin{UCtrayectoriaA}{I}{El actor ingresó un token mal formado.}
    \UCpaso[\UCsist] Muestra el mensaje \cdtIdRef{MSG27}{Token incorrecto} mencionando que el token utilizado en alguno de los pasos no es correcto.
    \UCpaso[] Continúa con el paso \ref{cu5.1.1.2:ingresaDatos} de la trayectoria principal.
 \end{UCtrayectoriaA}
 %----------------------------------------------------------%trayectoria J---------------------------------------------------- 
 \begin{UCtrayectoriaA}{J}{Alguno de los elementos referenciados no existe en el sistema.}
    \UCpaso[\UCsist] Muestra el mensaje \cdtIdRef{MSG15}{Dato no registrado} mencionando el elemento que no está registrado en el sistema.
    \UCpaso[] Continúa con el paso \ref{cu5.1.1.2:ingresaDatos} de la trayectoria principal.
 \end{UCtrayectoriaA}
 %----------------------------------------------------------%trayectoria K---------------------------------------------------- 
 \begin{UCtrayectoriaA}[Fin del caso de uso]{K}{No existe información referente a verbos en el sistema.}
    \UCpaso[\UCsist] Muestra el mensaje \cdtIdRef{MSG13}{Ha ocurrido un error} en la pantalla \cdtIdRef{IU 5.1.1}{Gestionar trayectorias}, indicando que no es posible
	continuar con la operación debido a la falta de verbos en el sistema.
 \end{UCtrayectoriaA}
 %----------------------------------------------------------%trayectoria L---------------------------------------------------- 
 \begin{UCtrayectoriaA}[Fin del caso de uso]{L}{El estado del caso de uso al que pertenece la trayectoria no es ``Edición'' o ``Pendiente de corrección''.}
    \UCpaso[\UCsist] Muestra el mensaje \cdtIdRef{MSG13}{Ha ocurrido un error} en la pantalla \cdtIdRef{IU 5}{Gestionar casos de uso}, indicando que no es posible
	continuar con la operación debido a que el caso de uso no cuenta con el estado adecuado.
 \end{UCtrayectoriaA}
 %----------------------------------------------------------%trayectoria M---------------------------------------------------- 
 \begin{UCtrayectoriaA}{M}{La sección trayectorias del caso de uso al que pertenece la trayectoria cuenta con observaciones.}
    \UCpaso[\UCsist] Muestra la sección ``Observaciones'' con las observaciones encontradas.
	\UCpaso[] Continúa con el paso \ref{cu5.1.1.2:muestra} de la trayectoria principal.
 \end{UCtrayectoriaA}
 %----------------------------------------------------------%trayectoria N---------------------------------------------------- 
 \begin{UCtrayectoriaA}{N}{Existe una trayectoria principal asociada al caso de uso.}
    \UCpaso[\UCsist] Verifica que la trayecoria principal encontrada no sea la trayectoria a modificar. \refTray{O}
	\UCpaso[\UCsist] Bloquea el campo tipo.	
	\UCpaso[] Continúa con el paso \ref{cu5.1.1.2:muestra} de la trayectoria principal.
 \end{UCtrayectoriaA}
 %----------------------------------------------------------%trayectoria O---------------------------------------------------- 
 \begin{UCtrayectoriaA}{O}{La trayectoria a modificar es la trayectoria principal.}
	\UCpaso[] Continúa con el paso \ref{cu5.1.1.2:muestra} de la trayectoria principal.
 \end{UCtrayectoriaA}

\subsection{Puntos de extensión}

\UCExtensionPoint{El actor requiere registrar un paso de la trayectoria}
	{Paso \ref{cu5.1.1.2:gestionaPasos}}
	{\cdtIdRef{CU 5.1.1.2.1}{Registrar paso}}
\UCExtensionPoint{El actor requiere modificar un paso de la trayectoria}
	{Paso \ref{cu5.1.1.2:gestionaPasos}}
	{\cdtIdRef{CU 5.1.1.2.2}{Modificar paso}}	
\UCExtensionPoint{El actor requiere eliminar un paso de la trayectoria}
	{Paso \ref{cu5.1.1.2:gestionaPasos}}
	{\cdtIdRef{CU 5.1.1.2.3}{Eliminar paso}}
  