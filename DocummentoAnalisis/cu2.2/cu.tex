\newpage 
\begin{UseCase}{CU 2.2}{Modificar persona}
	{
		Este caso de uso permite modificar la información de una persona que podrá ser elegida como colaborador de un proyecto.
	}
	
	\UCitem{Actor}{\cdtRef{actor:administrador}{Administrador}}
	\UCitem{Propósito}{
		Modificar la información de una persona.
	}
	\UCitem{Entradas}{
		\begin{UClist}
			\UCli \cdtRef{Colaborador:Nombre}{Nombre}: \ioEscribir.
			\UCli \cdtRef{Colaborador:PrimerApellido}{Primer Apellido}: \ioEscribir.
			\UCli \cdtRef{Colaborador:SegundoApellido}{Segundo Apellido}: \ioEscribir.
			\UCli \cdtRef{Colaborador:CorreoElectronico}{Correo electrónico}: \ioEscribir.
			\UCli \cdtRef{Colaborador:Contrasena}{Contraseña}: \ioEscribir.
		\end{UClist}
	}
	\UCitem{Salidas}{
		\begin{UClist}
			\UCli \cdtRef{Colaborador:CURP}{CURP}: \ioObtener.
			\UCli \cdtRef{Colaborador:Nombre}{Nombre}: \ioObtener.
			\UCli \cdtRef{Colaborador:PrimerApellido}{Primer Apellido}: \ioObtener.
			\UCli \cdtRef{Colaborador:SegundoApellido}{Segundo Apellido}: \ioObtener.
			\UCli \cdtRef{Colaborador:CorreoElectronico}{Correo electrónico}: \ioObtener.
			\UCli \cdtRef{Colaborador:Contrasena}{Contraseña}: \ioObtener.
		\end{UClist}
	}
	
	\UCitem{Mensajes}{
		\begin{UClist}
			\UCli \cdtIdRef{MSG1}{Operación exitosa}: Se muestra en la pantalla \cdtIdRef{IU 2}{Gestionar personal} para indicar que la modificación fue exitosa.
		\end{UClist}
	}

	\UCitem{Precondiciones}{
		Ninguna
	}
	
	\UCitem{Postcondiciones}{
		\begin{UClist}
			\UCli Se actualizará la información de una persona en el sistema.
		\end{UClist}
	}

	\UCitem{Errores}{
		\begin{UClist}
			\UCli \cdtIdRef{MSG4}{Dato obligatorio}: Se muestra en la pantalla \cdtIdRef{IU 2.2}{Modificar persona} cuando no se ha ingresado un dato marcado como obligatorio.
			\UCli \cdtIdRef{MSG5}{Dato incorrecto}: Se muestra en la pantalla \cdtIdRef{IU 2.2}{Modificar persona} cuando el tipo de dato ingresado no cumple con el tipo de dato solicitado en el campo.
			\UCli \cdtIdRef{MSG6}{Longitud inválida}: Se muestra en la pantalla \cdtIdRef{IU 2.2}{Modificar persona} cuando se ha excedido la longitud de alguno de los campos.
		\end{UClist}
	}

	\UCitem{Tipo}{
		Secundario, extiende del caso de uso \cdtIdRef{CU 2}{Gestionar personal}.
	}
\end{UseCase}
%-------------------------------------------------------%trayectoria Principal-----------------------------------------------
 \begin{UCtrayectoria}
    \UCpaso[\UCactor] Solicita modificar una persona oprimiendo el botón \btnEditar de la pantalla \cdtIdRef{IU 2}{Gestionar personal}.
    \UCpaso[\UCsist] Busca la información de la persona seleccionada.
    \UCpaso[\UCsist] Muestra la pantalla \cdtIdRef{IU 2.2}{Modificar persona} con la información encontrada. 
    \UCpaso[\UCactor] Modifica la información en la pantalla \cdtIdRef{IU 2.2}{Modificar persona}. \label{cu2.2:ingresaDatos}
    \UCpaso[\UCactor] Solicita guardar la información de la persona oprimiendo el botón \cdtButton{Aceptar} de la pantalla \cdtIdRef{IU 2.2}{Modificar persona}. \refTray{A} 
    \UCpaso[\UCsist] Verifica que el actor ingrese todos los campos obligatorios con base en la regla de negocio  \cdtIdRef{RN8}{Datos obligatorios}. \refTray{B}
    \UCpaso[\UCsist] Verifica que los datos requeridos sean proporcionados correctamente como se especifica en la regla de negocio \cdtIdRef{RN7}{Información correcta}. \refTray{C} \refTray{D}
    \UCpaso[\UCsist] Registra la información de la persona en el sistema.
    \UCpaso[\UCsist] Verifica que el correo electrónico y la contraseña no hayan cambiado. \refTray{E}
    \UCpaso[\UCsist] Muestra el mensaje \cdtIdRef{MSG1}{Operación exitosa} en la pantalla \cdtIdRef{IU 2}{Gestionar personal}
    para indicar al actor que el registro se ha realizado exitosamente. \label{cu2.2:muestraMensaje}
 \end{UCtrayectoria}
 %----------------------------------------------------------%trayectoria A---------------------------------------------------- 
 \begin{UCtrayectoriaA}[Fin del caso de uso]{A}{El actor desea cancelar la operación.}
    \UCpaso[\UCactor] Solicita cancelar la operación oprimiendo el botón \cdtButton{Cancelar} de la pantalla \cdtIdRef{IU 2.2}{Modificar persona}.
    \UCpaso[\UCsist] Muestra la pantalla donde se solicitó la operación.
 \end{UCtrayectoriaA} 
  %----------------------------------------------------------%trayectoria B---------------------------------------------------- 
 \begin{UCtrayectoriaA}{B}{El actor no ingresó algún dato marcado como obligatorio.}
    \UCpaso[\UCsist] Muestra el mensaje \cdtIdRef{MSG4}{Dato obligatorio} y señala el campo que presenta el error en la pantalla 
	    \cdtIdRef{IU 2.2}{Modificar persona}, indicando al actor que el dato es obligatorio.
    \UCpaso[] Continúa con el paso \ref{cu2.2:ingresaDatos} de la trayectoria principal.
 \end{UCtrayectoriaA}
 %----------------------------------------------------------%trayectoria C----------------------------------------------------  
 \begin{UCtrayectoriaA}{C}{El actor proporciona un dato que excede la longitud máxima.}
    \UCpaso[\UCsist] Muestra el mensaje \cdtIdRef{MSG6}{Longitud inválida} y señala el campo que excede la 
    longitud en la pantalla \cdtIdRef{IU 2.2}{Modificar persona}, para indicar que el dato excede el tamaño máximo permitido.
    \UCpaso[] Continúa con el paso \ref{cu2.2:ingresaDatos} de la trayectoria principal.
 \end{UCtrayectoriaA}
 
 %----------------------------------------------------------%trayectoria D---------------------------------------------------- 
 \begin{UCtrayectoriaA}{D}{El actor ingresó un tipo de dato incorrecto.}
    \UCpaso[\UCsist] Muestra el mensaje \cdtIdRef{MSG5}{Dato incorrecto} y señala el campo que presenta el dato inválido en la 
    pantalla \cdtIdRef{IU 2.2}{Modificar persona} para indicar que se ha ingresado un tipo de dato inválido.
    \UCpaso[] Continúa con el paso \ref{cu2.2:ingresaDatos} de la trayectoria principal.
 \end{UCtrayectoriaA}
 %----------------------------------------------------------%trayectoria E---------------------------------------------------- 
 \begin{UCtrayectoriaA}{E}{El actor modificó el correo electrónico o la contraseña.}
    \UCpaso[\UCsist] Envía un correo con el mensaje \cdtIdRef{MSG34}{Datos de sesión} a la cuenta de correo electrónico ingresado con la contraseña ingresada.
    \UCpaso[] Continúa con el paso \ref{cu2.2:muestraMensaje} de la trayectoria principal.
 \end{UCtrayectoriaA}
 