\newpage 
\subsection{IU 3 Gestionar módulos}
\subsubsection{Objetivo}
	
	En esta pantalla el actor puede visualizar algunos atributos de los módulos registrados y las operaciones disponibles de gestión.

\subsubsection{Diseño}

    En la figura ~\ref{IU 3} se muestra la pantalla ``Gestionar módulos'', por medio de la cual 
    se podrán gestionar los módulos a través de una tabla. El actor podrá podrá solicitar el registro, la modificación, la eliminación, la gestión de casos de uso y la gestión de pantallas
    de un módulo mediante los botones
    \cdtButton{Registrar}, \btnEditar, \btnEliminar, \btnCU y \btnIU respectivamente. \\
	
	En la esquina superior derecha de la pantalla, el sistema mostrará la clave y el nombre del proyecto del cual se están gestionando los módulos.

    \IUfig[.9]{cu3/images/iu.png}{IU 3}{Gestionar módulos}


\subsubsection{Comandos}
\begin{itemize}
	\item \cdtButton{Registrar}: Permite al actor solicitar el registro de un módulo, dirige a la pantalla \cdtIdRef{IU 3.1}{Registrar módulo}.
	\item \btnEditar[Modificar]: Permite al actor solicitar la modificación de un módulo, dirige a la pantalla \cdtIdRef{IU 3.2}{Modificar módulo}.
	\item \btnEliminar[Eliminar]: Permite al actor solicitar la eliminación de un módulo, dirige a una pantalla emergente.
	\item \btnCU[Gestionar casos de uso]: Permite al actor solicitar la gestión de los casos de uso del módulo, dirige a la pantalla \cdtIdRef{IU 5}{Gestionar casos de uso}.
	\item \btnIU[Gestionar pantallas]: Permite al actor solicitar la gestión de las pantallas del módulo, dirige a la pantalla \cdtIdRef{IU 6}{Gestionar pantallas}.
\end{itemize}
\subsubsection{Mensajes}
	
\begin{description}
	\item[\cdtIdRef{MSG2}{No existe información}:] Se muestra en la pantalla \cdtIdRef{IU 3}{Gestionar módulos} cuando no existe módulol registrado.
\end{description}
