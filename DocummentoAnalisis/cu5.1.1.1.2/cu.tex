\newpage 
\begin{UseCase}{CU 5.1.1.1.2}{Modificar paso}
	{
		Los pasos que describen una trayectoria deben indicar las acciones que realiza el sistema o el actor.
		Este caso de uso permite al analista modificar un paso de alguna trayectoria.
		
	}
	\UCitem{Actor}{\cdtRef{actor:liderAnalisisx}{Líder de análisis}, \cdtRef{actor:analista}{Analista}}
	\UCitem{Propósito}{
		Modificar los pasos de la trayectoria principal o de alguna trayectoria alternativa.
	}
	\UCitem{Entradas}{
		\begin{UClist}
 			\UCli \cdtRef{Paso:RealizaActor}{Quien realiza el paso}: \ioSeleccionar.
 			\UCli \cdtRef{Verbo}{Verbo}: \ioSeleccionar.
 			\UCli \cdtRef{Paso:Redaccion}{Redacción del paso}: \ioEscribir.
		\end{UClist}
	}
	\UCitem{Salidas}{
		\begin{UClist}
 			\UCli \cdtRef{Paso:RealizaActor}{Quien realiza el paso}: \ioSeleccionar.
 			\UCli \cdtRef{Verbo}{Verbo}: \ioSeleccionar.
 			\UCli \cdtRef{Paso:Redaccion}{Redacción del paso}: \ioEscribir.
		\end{UClist}
	}
	
	\UCitem{Mensajes}{
		Ninguno
	}

	\UCitem{Precondiciones}{
		Ninguna
	}
	
	\UCitem{Postcondiciones}{
		Ninguna
	}

	\UCitem{Errores}{
		\begin{UClist}
			\UCli \cdtIdRef{MSG4}{Dato obligatorio}: Se muestra en la pantalla \cdtIdRef{IU 5.1.1.1.2}{Modificar paso} cuando no se ha ingresado un dato marcado como obligatorio.
			\UCli \cdtIdRef{MSG6}{Longitud inválida}: Se muestra en la pantalla \cdtIdRef{IU 5.1.1.1.2}{Modificar paso} cuando se ha excedido la longitud de alguno de los campos.
		\end{UClist}
	}

	\UCitem{Tipo}{
		Secundario, extiende de los casos de uso \cdtIdRef{CU 5.1.1.1}{Registrar trayectoria} y \cdtIdRef{CU 5.1.1.2}{Modificar trayectoria}.
	}
\end{UseCase}



%-------------------------------------------------------%trayectoria Principal-----------------------------------------------

 \begin{UCtrayectoria}
	\UCpaso[\UCactor] Solicita modificar un paso oprimiendo el botón \btnEditar del registro que desee de la pantalla \cdtIdRef{IU 5.1.1.1}{Registrar trayectoria} o de la pantalla \cdtIdRef{IU 5.1.1.2}{Modificar trayectoria}. 
	\UCpaso[\UCsist] Muestra la pantalla emergente \cdtIdRef{IU 5.1.1.1.2}{Modificar paso} en la cual se realizará la modificación del paso. 
	\UCpaso[\UCactor] Modifica quien realiza el paso.\label{cu5.1.1.1.2:ingresaDatos}
	\UCpaso[\UCactor] Modifica el verbo.
	\UCpaso[\UCactor] Modifica la redacción del paso. \refTray{A} \refTray{B} \refTray{C} \refTray{D} \refTray{E} \refTray{F} \refTray{G} \refTray{H} \refTray{I} \refTray{J} \refTray{K}\label{cu5.1.1.1.2:ingresaRedaccion}
	\UCpaso[\UCactor] Solicita modificar el paso oprimiendo el botón \cdtButton{Aceptar} de la pantalla \cdtIdRef{IU 5.1.1.1.2}{Modificar paso}. \refTray{L} 
	\UCpaso[\UCsist] Verifica que el actor haya ingresado todos los campos obligatorios con base en la regla de negocio \cdtIdRef{RN8}{Datos obligatorios}. \refTray{M}
	\UCpaso[\UCsist] Verifica que los datos requeridos sean proporcionados correctamente como se especifica en la regla de negocio \cdtIdRef{RN7}{Información correcta}. \refTray{N} 
	\UCpaso[\UCsist] Modifica el paso en la tabla de la pantalla \cdtIdRef{IU 5.1.1.1}{Registrar trayectoria} o de la pantalla \cdtIdRef{CU 5.1.1.2}{Modificar trayectoria}.
\end{UCtrayectoria}

    

  %----------------------------------------------------------%trayectoria A---------------------------------------------------- 
 \begin{UCtrayectoriaA}{A}{El actor desea seleccionar un actor.}
 	\UCpaso[\UCactor] Ingresa el token {\it ACT·}.
 	\UCpaso[\UCsist] Muestra una lista con los actores encontrados.
 	\UCpaso[\UCactor] Selecciona un actor de la lista.
  	\UCpaso[\UCsist] Verifica que el nombre del actor seleccionado no contenga espacios. \refTray{Z}
  	\UCpaso[\UCsist] Agrega el nombre del actor al texto.
    \UCpaso[] Continúa con el paso \ref{cu5.1.1.1.2:ingresaRedaccion} de la trayectoria principal.
 \end{UCtrayectoriaA}
  %----------------------------------------------------------%trayectoria B---------------------------------------------------- 
 \begin{UCtrayectoriaA}{B}{El actor desea seleccionar un término del glosario.}
 	\UCpaso[\UCactor] Ingresa el token {\it GLS·}.	
 	\UCpaso[\UCsist] Muestra una lista con los términos del glosario encontrados.
 	\UCpaso[\UCactor] Selecciona un término del glosario de la lista.
  	\UCpaso[\UCsist] Verifica que el nombre del término del glosario seleccionado no contenga espacios. \refTray{Z}
  	\UCpaso[\UCsist] Agrega el nombre del término del glosario al texto.
    \UCpaso[] Continúa con el paso \ref{cu5.1.1.1.2:ingresaRedaccion} de la trayectoria principal.
 \end{UCtrayectoriaA}

  %----------------------------------------------------------%trayectoria C---------------------------------------------------- 
 \begin{UCtrayectoriaA}{C}{El actor desea seleccionar un atributo.}
 	\UCpaso[\UCactor] Ingresa el token {\it ATR·}.
  	\UCpaso[\UCsist] Busca los atributos de las entidades encontradas.
  	\UCpaso[\UCsist] Muestra una lista de los atributos encontrados.
 	\UCpaso[\UCactor] Selecciona un atributo de la lista.
  	\UCpaso[\UCsist] Verifica que el nombre de la entidad a la que pertenece el atributo seleccionado no contenga espacios. \refTray{Z}
  	\UCpaso[\UCsist] Verifica que el nombre del atributo seleccionado no contenga espacios. \refTray{Z}
  	\UCpaso[\UCsist] Agrega el nombre de la entidad a la que pertenece el atributo al texto, seguido del signo ``:''.
  	\UCpaso[\UCsist] Agrega el nombre del atributo al texto.
    \UCpaso[] Continúa con el paso \ref{cu5.1.1.1.2:ingresaRedaccion} de la trayectoria principal.
 \end{UCtrayectoriaA}
%----------------------------------------------------------%trayectoria D---------------------------------------------------- 
 \begin{UCtrayectoriaA}{D}{El actor desea seleccionar un mensaje.}
 	 \UCpaso[\UCactor] Ingresa el token {\it MSG·}.	
 	\UCpaso[\UCsist] Muestra una lista con los mensajes encontrados.
 	\UCpaso[\UCactor] Selecciona un mensaje de la lista.
  	\UCpaso[\UCsist] Verifica que el nombre del mensaje seleccionado no contenga espacios. \refTray{Z}
  	\UCpaso[\UCsist] Agrega el número del mensaje al texto, seguido del signo ``:''.
  	\UCpaso[\UCsist] Agrega el nombre del mensaje al texto.
    \UCpaso[] Continúa con el paso \ref{cu5.1.1.1.2:ingresaRedaccion} de la trayectoria principal.
 \end{UCtrayectoriaA}
 %----------------------------------------------------------%trayectoria E---------------------------------------------------- 
 \begin{UCtrayectoriaA}{E}{El actor desea seleccionar una regla de negocio.}
 	\UCpaso[\UCactor] Ingresa el token {\it RN·}.	
 	\UCpaso[\UCsist] Muestra una lista con las reglas de negocio encontradas.
 	\UCpaso[\UCactor] Selecciona una regla de negocio de la lista.
  	\UCpaso[\UCsist] Verifica que el nombre de la regla de negocio seleccionada no contenga espacios. \refTray{Z}
  	\UCpaso[\UCsist] Agrega el número de la regla de negocio al texto, seguido del signo ``:''.
  	\UCpaso[\UCsist] Agrega el nombre de la regla de negocio al texto.
    \UCpaso[] Continúa con el paso \ref{cu5.1.1.1.2:ingresaRedaccion} de la trayectoria principal.
 \end{UCtrayectoriaA}
  %----------------------------------------------------------%trayectoria F---------------------------------------------------- 
 \begin{UCtrayectoriaA}{F}{El actor desea seleccionar una entidad.}
 	\UCpaso[\UCactor] Ingresa el token {\it ENT·}.	
 	\UCpaso[\UCsist] Muestra una lista con las entidades encontradas.
 	\UCpaso[\UCactor] Selecciona una entidad de la lista.
  	\UCpaso[\UCsist] Verifica que el nombre de la entidad seleccionada no contenga espacios. \refTray{Z}
  	\UCpaso[\UCsist] Agrega el nombre de la entidad al texto.
    \UCpaso[] Continúa con el paso \ref{cu5.1.1.1.2:ingresaRedaccion} de la trayectoria principal.
 \end{UCtrayectoriaA}
  %----------------------------------------------------------%trayectoria G---------------------------------------------------- 
 \begin{UCtrayectoriaA}{G}{El actor desea seleccionar un caso de uso.}
 	\UCpaso[\UCactor] Ingresa el token {\it CU·}.
  	\UCpaso[\UCsist] Muestra una lista de los casos de uso encontrados.
 	\UCpaso[\UCactor] Selecciona un caso de uso de la lista.
  	\UCpaso[\UCsist] Verifica que el nombre del caso de uso seleccionado no contenga espacios. \refTray{Z}
  	\UCpaso[\UCsist] Agrega la clave del módulo al que pertenece el caso de uso al texto, seguido del signo ``·''.
  	\UCpaso[\UCsist] Agrega el número del caso de uso al texto, seguido del signo ``:''.
  	\UCpaso[\UCsist] Agrega el nombre del caso de uso al texto.
    \UCpaso[] Continúa con el paso \ref{cu5.1.1.1.2:ingresaRedaccion} de la trayectoria principal.
 \end{UCtrayectoriaA}
 
  %----------------------------------------------------------%trayectoria H---------------------------------------------------- 
 \begin{UCtrayectoriaA}{H}{El actor desea seleccionar una pantalla.}
 	\UCpaso[\UCactor] Ingresa el token {\it CU·}.
  	\UCpaso[\UCsist] Muestra una lista de los casos de uso encontrados.
 	\UCpaso[\UCactor] Selecciona un caso de uso de la lista.
  	\UCpaso[\UCsist] Verifica que el nombre del caso de uso seleccionado no contenga espacios. \refTray{Z}
  	\UCpaso[\UCsist] Agrega la clave del módulo al que pertenece el caso de uso al texto, seguido del signo ``·''.
  	\UCpaso[\UCsist] Agrega el número del caso de uso al texto, seguido del signo ``:''.
  	\UCpaso[\UCsist] Agrega el nombre del caso de uso al texto.
    \UCpaso[] Continúa con el paso \ref{cu5.1.1.1.2:ingresaRedaccion} de la trayectoria principal.
 \end{UCtrayectoriaA}
  %----------------------------------------------------------%trayectoria I---------------------------------------------------- 
 \begin{UCtrayectoriaA}{I}{El actor desea seleccionar un paso.}
 	\UCpaso[\UCactor] Ingresa el token {\it P·}.
	\UCpaso[\UCsist] Busca los pasos del caso de uso.
  	\UCpaso[\UCsist] Muestra una lista de los pasos encontrados.
 	\UCpaso[\UCactor] Selecciona un paso de la lista.
  	\UCpaso[\UCsist] Verifica que el nombre del caso de uso al que pertenece el paso no contenga espacios. \refTray{Z}
  	\UCpaso[\UCsist] Agrega la clave del caso de uso al que pertenece el paso al texto, seguido del signo ``·''.
  	\UCpaso[\UCsist] Agrega el número del caso de uso al texto, seguido del signo ``:''.
  	\UCpaso[\UCsist] Agrega el nombre del caso de uso al texto, seguido del signo ``:''.
  	\UCpaso[\UCsist] Agrega el nombre del caso de uso al texto, seguido del signo ``:''.
  	\UCpaso[\UCsist] Agrega la clave de la trayectoria a la que petenece el paso al texto, seguido del signo ``·''.
  	\UCpaso[\UCsist] Agrega el número del paso seleccionado al texto.
    \UCpaso[] Continúa con el paso \ref{cu5.1.1.1.2:ingresaRedaccion} de la trayectoria principal.
 \end{UCtrayectoriaA}
  %----------------------------------------------------------%trayectoria J---------------------------------------------------- 
 \begin{UCtrayectoriaA}{J}{El actor desea seleccionar una trayectoria.}
 	\UCpaso[\UCactor] Ingresa el token {\it TRAY·}.
	\UCpaso[\UCsist] Busca las trayectorias del caso de uso.
  	\UCpaso[\UCsist] Muestra una lista de las trayectorias encontradas.
 	\UCpaso[\UCactor] Selecciona una trayectoria de la lista.
  	\UCpaso[\UCsist] Verifica que el nombre del caso de uso al que pertenece la trayectoria no contenga espacios. \refTray{Z}
  	\UCpaso[\UCsist] Agrega la clave del caso de uso al que pertenece el paso al texto, seguido del signo ``·''.
  	\UCpaso[\UCsist] Agrega el número del caso de uso al texto, seguido del signo ``:''.
  	\UCpaso[\UCsist] Agrega el nombre del caso de uso al texto, seguido del signo ``:''.
  	\UCpaso[\UCsist] Agrega el nombre del caso de uso al texto, seguido del signo ``:''.
  	\UCpaso[\UCsist] Agrega la clave de la trayectoria seleccionada.
    \UCpaso[] Continúa con el paso \ref{cu5.1.1.1.2:ingresaRedaccion} de la trayectoria principal.
 \end{UCtrayectoriaA}
  %----------------------------------------------------------%trayectoria K---------------------------------------------------- 
 \begin{UCtrayectoriaA}{K}{El actor desea seleccionar una acción.}
 	\UCpaso[\UCactor] Ingresa el token {\it ACC·}.
	\UCpaso[\UCsist] Busca las acciones de las pantallas.
  	\UCpaso[\UCsist] Muestra una lista de las acciones encontradas.
 	\UCpaso[\UCactor] Selecciona una acción de la lista.
  	\UCpaso[\UCsist] Verifica que el nombre de la pantalla a la que pertenece la acción no contenga espacios. \refTray{Z}
  	\UCpaso[\UCsist] Agrega la clave de la pantalla a la que pertenece la acción al texto, seguido del signo ``·''.
  	\UCpaso[\UCsist] Agrega el número de la pantalla al texto, seguido del signo ``:''.
  	\UCpaso[\UCsist] Agrega el nombre de la pantalla al texto, seguido del signo ``:''.
  	\UCpaso[\UCsist] Agrega el nombre de la acción seleccionada al texto.
    \UCpaso[] Continúa con el paso \ref{cu5.1.1.1.2:ingresaRedaccion} de la trayectoria principal.
 \end{UCtrayectoriaA}

 %----------------------------------------------------------%trayectoria L---------------------------------------------------- 
 \begin{UCtrayectoriaA}[Fin del caso de uso]{L}{El actor desea cancelar la operación.}
    \UCpaso[\UCactor] Solicita cancelar la operación oprimiendo el botón \cdtButton{Cancelar} de la pantalla \cdtIdRef{IU 5.1.1.1.2}{Modificar paso}.
    \UCpaso[\UCsist] Muestra la pantalla \cdtIdRef{IU 5.1.1.1}{Registrar trayectoria} o \cdtIdRef{CU 5.1.1.2}{Modificar trayectoria} según corresponda.
 \end{UCtrayectoriaA}
 
 %----------------------------------------------------------%trayectoria M---------------------------------------------------- 
 \begin{UCtrayectoriaA}{M}{El actor no ingresó algún dato marcado como obligatorio.}
    \UCpaso[\UCsist] Muestra el mensaje \cdtIdRef{MSG4}{Dato obligatorio} en una pantalla emergente.
    \UCpaso[] Continúa con el paso \ref{cu5.1.1.1.2:ingresaDatos} de la trayectoria principal.
 \end{UCtrayectoriaA}
 
 %----------------------------------------------------------%trayectoria N --------------------------------------------------  
 \begin{UCtrayectoriaA}{N}{El actor proporciona un dato que excede la longitud máxima.}
    \UCpaso[\UCsist] Muestra el mensaje \cdtIdRef{MSG5}{Se ha excedido la longitud máxima del campo} en una pantalla emergente.
    \UCpaso[] Continúa con el paso \ref{cu5.1.1.1.2:ingresaDatos} de la trayectoria principal.
 \end{UCtrayectoriaA}

  %----------------------------------------------------------%trayectoria Z---------------------------------------------------- 
  \begin{UCtrayectoriaA}{Z}{El texto contiene espacios.}
     \UCpaso[\UCsist] Sustituye los espacios por guiones bajos.
  \end{UCtrayectoriaA}