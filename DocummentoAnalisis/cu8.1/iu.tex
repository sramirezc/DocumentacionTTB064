\newpage 
\subsection{IU 8.1 Registrar regla de negocio}

\subsubsection{Objetivo}
	
	Esta pantalla permite al actor registrar la información de una regla de negocio nueva.

\subsubsection{Diseño}

    En la figura ~\ref{IU 8.1} se muestra la pantalla ``Registrar regla de negocio'' que permite registrar una regla de negocio. El actor deberá 
    seleccionar el tipo de regla de negocio y el sistema mostrará los campos de los parámetros de la regla de negocio. \\
    
    Cuando el actor seleccione el tipo ``Comparación de atributos'', el sistema mostrará la pantalla ~\ref{IU 8.1a}; si el actor selecciona el tipo
    ``Unicidad de parámetros'' el sistema mostrará la pantalla ~\ref{IU 8.1b}; si el actor selecciona el tipo ``Formato correcto'' el sistema
    mostrará la pantalla ~\ref{IU 8.1c}. Para los demás tipos de reglas de negocio el sistema no mostrará más campos.\\
    
    Una vez ingresada la información solicitada, el actor deberá oprimir el botón \cdtButton{Aceptar}. El sistema validará y registrará la 
    información sólo si se han cumplido todas las reglas de negocio establecidas.  \\
    
    Finalmente se mostrará el mensaje \cdtIdRef{MSG1}{Operación exitosa} en la pantalla \cdtIdRef{IU 8}{Gestionar reglas de negocio}, 
    para indicar que la información de la regla de negocio
    se ha registrado correctamente.        


    \IUfig[.9]{cu8.1/images/iu.png}{IU 8.1}{Registrar regla de negocio}
     \IUfig[.9]{cu8.1/images/iua.png}{IU 8.1a}{Registrar regla de negocio: Comparación de atributos}
     \IUfig[.9]{cu8.1/images/iub.png}{IU 8.1b}{Registrar regla de negocio: Unicidad de parámetros}
     \IUfig[.9]{cu8.1/images/iuc.png}{IU 8.1c}{Registrar regla de negocio: Formato correcto}

\subsubsection{Comandos}
\begin{itemize}
	\item \cdtButton{Aceptar}: Permite al actor guardar el registro de la regla de negocio, dirige a la pantalla \cdtIdRef{IU 8}{Gestionar reglas de negocio}.
	\item \cdtButton{Cancelar}: Permite al actor cancelar el registro de la regla de negocio, dirige a la pantalla \cdtIdRef{IU 8}{Gestionar reglas de negocio}.
\end{itemize}

\subsubsection{Mensajes}
	
\begin{description}
	\item[\cdtIdRef{MSG1}{Operación exitosa}:] Se muestra en la pantalla \cdtIdRef{IU 8}{Gestionar reglas de negocio} para indicar que el registro fue exitoso.
	\item[\cdtIdRef{MSG4}{Dato obligatorio}:] Se muestra en la pantalla \cdtIdRef{IU 8.1}{Registrar regla de negocio} cuando no se ha ingresado un dato marcado como obligatorio.
	\item[\cdtIdRef{MSG5}{Dato incorrecto}:] Se muestra en la pantalla \cdtIdRef{IU 8.1}{Registrar regla de negocio} cuando el tipo de dato ingresado no cumple con el tipo de dato solicitado en el campo.
	\item[\cdtIdRef{MSG6}{Longitud inválida}:] Se muestra en la pantalla \cdtIdRef{IU 8.1}{Registrar regla de negocio} cuando se ha excedido la longitud de alguno de los campos.
	\item[\cdtIdRef{MSG7}{Registro repetido}:] Se muestra en la pantalla \cdtIdRef{IU 8.1}{Registrar regla de negocio} cuando se registre una regla de negocio con un nombre o número que ya este registrado.
	\item[\cdtIdRef{MSG23}{Caracteres inválidos}:] Se muestra en la pantalla \cdtIdRef{IU 8.1}{Registrar regla de negocio} cuando el nombre de la regla de negocio contiene un caracter no válido.
\end{description}
