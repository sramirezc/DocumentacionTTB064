\newpage 
\begin{UseCase}{CU 14}{Recuperar contraseña}
	{
		Este caso de uso permite al actor recuperar su contraseña.
		
	}
	\UCitem{Actor}{\cdtRef{actor:liderAnalisis}{Líder de análisis}, \cdtRef{actor:analista}{Analista}}
	\UCitem{Propósito}{
		Recuperar la contraseña del actor.
	}
	\UCitem{Entradas}{
		\begin{UClist} 
			\UCli \cdtRef{Colaborador:CorreoElectronico}{Correo electrónico}: \ioEscribir
		\end{UClist}
	}	
	
	\UCitem{Salidas}{
		Ninguna
	}
	
	\UCitem{Mensajes}{
	\begin{UClist}	
		\UCli \cdtIdRef{MSG32}{Recuperar contraseña}: Se muestra en la pantalla \cdtIdRef{IU 13}{Iniciar sesión} indicando que se ha enviado el correo electrónico con la contraseña exitosamente.
		\end{UClist}
	}
	
	\UCitem{Precondiciones}{ 
		\begin{UClist}
			\UCli Que el actor se encuentre registrado en el sistema.
		\end{UClist}
	}			
	\UCitem{Postcondiciones}{ 
		\begin{UClist}
			\UCli Se enviará un correo electrónico al actor con su contraseña.
	 	\end{UClist}
	}
	\UCitem{Errores}{
		\begin{UClist}
			\UCli \cdtIdRef{MSG4}{Dato obligatorio}: Se muestra en la pantalla \cdtIdRef{IU 14}{Recuperar contraseña} cuando no se ha ingresado un dato marcado como obligatorio.
			\UCli \cdtIdRef{MSG6}{Longitud inválida}: Se muestra en la pantalla \cdtIdRef{IU 14}{Iniciar sesión} cuando se ha excedido la longitud de alguno de los campos.
			\UCli \cdtIdRef{MSG33}{Correo electrónico inválido}: Se muestra en la pantalla \cdtIdRef{IU 14}{Recuperar contraseña} cuando el correo ingresado es incorrecto.
			\end{UClist}
}
	
	\UCitem{Tipo}{
		Primario
	}

	
\end{UseCase}
 %-------------------------------------------------------%trayectoria Principal-----------------------------------------------
 \begin{UCtrayectoria}
    \UCpaso[\UCactor] Oprime la liga ``Recuperar contraseña'' de la pantalla \cdtIdRef{IU 13}{Iniciar sesión}.
    \UCpaso[\UCsist] Muestra la pantalla \cdtIdRef{IU 14}{Recuperar contraseña}.
    \UCpaso[\UCactor] Ingresa los datos solicitados en la pantalla  \cdtIdRef{IU 14}{Recuperar contraseña}. \label{cu14:ingresaDatos}
	\UCpaso[\UCactor] Oprime el botón \cdtButton{Aceptar} de la pantalla \cdtIdRef{IU 14}{Recuperar contraseña}.
    \UCpaso[\UCsist] Verifica que el actor ingrese todos los campos obligatorios con base en la regla de negocio  \cdtIdRef{RN8}{Datos obligatorios}. \refTray{A}
    \UCpaso[\UCsist] Verifica que los datos requeridos sean proporcionados correctamente como se especifica en la regla de negocio \cdtIdRef{RN7}{Información correcta}. \refTray{B}
    \UCpaso[\UCsist] Verifica que el correo electrónico ingresado tenga el formato correcto con base en la regla de negocio \cdtIdRef{RN19}{Formato de correo electrónico}. \refTray{C}
    \UCpaso[\UCsist] Verifica que el correo electrónico se encuentre registrado. \refTray{C}
    \UCpaso[\UCsist] Envía un correo electrónico con la contraseña.
    \UCpaso[\UCactor] Muestra el mensaje \cdtIdRef{MSG32}{Recuperar contraseña} en la pantalla \cdtIdRef{IU 13}{Iniciar sesión}, para indicar que se ha solicitado la recuperación de contraseña exitosamente.
 \end{UCtrayectoria}
 
  
  %----------------------------------------------------------%trayectoria A---------------------------------------------------- 
 \begin{UCtrayectoriaA}{A}{El actor no ingresó algún dato marcado como obligatorio.}
    \UCpaso[\UCsist] Muestra el mensaje \cdtIdRef{MSG4}{Dato obligatorio} y señala el campo que presenta el error en la pantalla 
	    \cdtIdRef{IU 14}{Recuperar contraseña}, indicando al actor que el dato es obligatorio.
    \UCpaso[] Continúa con el paso \ref{cu14:ingresaDatos} de la trayectoria principal.
 \end{UCtrayectoriaA}
 %----------------------------------------------------------%trayectoria B----------------------------------------------------  
 \begin{UCtrayectoriaA}{B}{El actor proporciona un dato que excede la longitud máxima.}
    \UCpaso[\UCsist] Muestra el mensaje \cdtIdRef{MSG6}{Longitud inválida} y señala el campo que excede la 
    longitud en la pantalla \cdtIdRef{IU 14}{Recuperar contraseña}, para indicar que el dato excede el tamaño máximo permitido.
    \UCpaso[] Continúa con el paso \ref{cu14:ingresaDatos} de la trayectoria principal.
 \end{UCtrayectoriaA}
 %----------------------------------------------------------%trayectoria B----------------------------------------------------  
 \begin{UCtrayectoriaA}{C}{El actor proporciona un correo electrónico incorrecto.}
    \UCpaso[\UCsist] Muestra el mensaje \cdtIdRef{MSG33}{Correo electrónico inválido} en la pantalla \cdtIdRef{IU 14}{Recuperar contraseña}, para indicar que el correo electrónico es inválido.
    \UCpaso[] Continúa con el paso \ref{cu14:ingresaDatos} de la trayectoria principal.
 \end{UCtrayectoriaA}
 