\begin{UseCase}{CU 8.4}{Consultar regla de negocio}
	{
		Este caso de uso permite al analista consultar la información de una regla de negocio mostrando la información según su tipo.
	}
	
	\UCitem{Actor}{\cdtRef{actor:liderAnalisis}{Líder de análisis}, \cdtRef{actor:analista}{Analista}}
	\UCitem{Propósito}{
		Consultar la información de una regla de negocio.
	}
	\UCitem{Entradas}{
		Ninguna
	}
	\UCitem{Salidas}{
		\begin{UClist}
			\UCli \cdtRef{Elemento:Clave}{Clave}: \ioCalcular{\cdtIdRef{RN12}{Identificador de elemento}}.
			\UCli \cdtRef{Elemento:Numero}{Número}: \ioObtener.
			\UCli \cdtRef{Elemento:Nombre}{Nombre}: \ioObtener.
			\UCli \cdtRef{Elemento:Descripcion}{Descripción}: \ioObtener.
 			\UCli \cdtRef{ReglaDeNegocio:Redaccion}{Redacción}: \ioObtener.
 			\UCli \cdtRef{gls:TipoDeReglaDeNegocio}{Tipo}: \ioObtener.
 			\UCli Para el tipo ``Comparación de atributos'':
 			\begin{itemize}
 			 \item \cdtRef{ValorDelParametroEnReglaDeNegocio}{Entidad 1}: \ioObtener.
 			 \item \cdtRef{ValorDelParametroEnReglaDeNegocio}{Atributo 1}: \ioObtener.
 			 \item \cdtRef{Operador}{Operador}: \ioObtener.
 			 \item \cdtRef{ValorDelParametroEnReglaDeNegocio}{Entidad 2}: \ioObtener.
 			 \item \cdtRef{ValorDelParametroEnReglaDeNegocio}{Atributo 2}: \ioObtener.
 			\end{itemize}
 			
 			\UCli Para el tipo ``Unicidad de parámetros'':
 			\begin{itemize}
 			 \item \cdtRef{ValorDelParametroEnReglaDeNegocio}{Entidad}: \ioObtener.
 			 \item \cdtRef{ValorDelParametroEnReglaDeNegocio}{Atributo único}: \ioObtener. 
 			\end{itemize}

 			\UCli Para el tipo ``Formato correcto'':
 			\begin{itemize}
 			 \item \cdtRef{ValorDelParametroEnReglaDeNegocio}{Entidad que contiene el atributo para verificar el formato}: \ioObtener.
 			 \item \cdtRef{ValorDelParametroEnReglaDeNegocio}{Atributo que se verificará con la expresión regular}: \ioObtener.
 			 \item \cdtRef{}{Expresión regular}: \ioObtener.
 			\end{itemize}
		\end{UClist}
	}
	
	\UCitem{Mensajes}{
		Ninguno
	}

	\UCitem{Precondiciones}{
		Ninguna
	}
	
	\UCitem{Postcondiciones}{
		Ninguna
	}

	\UCitem{Errores}{
		\begin{UClist}
			\UCli \cdtRef{MSG13}{Ha ocurrido un error}: Se muestra en la pantalla donde se solicitó la operación cuando la pantalla que se desea consultar no existe.
		\end{UClist}
	}

	\UCitem{Tipo}{
		Secundario, extiende del casos de uso \cdtIdRef{CU 8}{Gestionar reglas de negocio}.
	}
\end{UseCase}
%-------------------------------------------------------%trayectoria Principal-----------------------------------------------
 \begin{UCtrayectoria}
    \UCpaso[\UCactor] Solicita consultar una regla de negocio oprimiendo el botón \btnConsulta del registro que desea consultar en la pantalla \cdtIdRef{IU 8}{Gestionar reglas de negocio} o la liga correspondiente a una regla de negocio en la pantalla \cdtIdRef{IU 5.4}{Consultar caso de uso}.
    \UCpaso[\UCsist] Busca la información de la regla de negocio seleccionada. \refTray{A} %No existe información
    \UCpaso[\UCsist] Verifica que el tipo de regla de negocio no tenga parámetros. \refTray{B} \refTray{C} \refTray{D}
    \UCpaso[\UCsist] Muestra la pantalla \cdtIdRef{IU 8.4}{Consultar regla de negocio}.
    \UCpaso[\UCactor] Consulta la información de la regla de negocio. \label{cu8.4:consultarElemento} 
    \UCpaso[\UCactor] Solicita finalizar la consulta oprimiendo el botón \cdtButton{Aceptar} de la pantalla \cdtIdRef{IU 8.4}{Consultar regla de negocio}. \label{cu8.4:finalizar}
 \end{UCtrayectoria}
 
 %----------------------------------------------------------%trayectoria A---------------------------------------------------- 
 \begin{UCtrayectoriaA}[Fin del caso de uso]{A}{La regla de negocio que se desea consultar no existe.}
    \UCpaso[\UCsist] Muestra la pantalla donde se solicitó la operación con el mensaje \cdtRef{MSG13}{Ha ocurrido un error}.
 \end{UCtrayectoriaA} 
 %----------------------------------------------------------%trayectoria B---------------------------------------------------- 
 \begin{UCtrayectoriaA}{B}{La regla de negocio es de tipo ``Comparación de atributos''.}
    \UCpaso[\UCsist] Muestra la pantalla \cdtIdRef{IU 8.4a}{Consultar regla de negocio: Comparación de atributos}.
    \UCpaso[] Continúa con el paso \ref{cu8.4:finalizar} de la trayectoria principal.
 \end{UCtrayectoriaA}
 %----------------------------------------------------------%trayectoria C---------------------------------------------------- 
 \begin{UCtrayectoriaA}{C}{La regla de negocio es de tipo ``Unicidad de parámetros''.}
    \UCpaso[\UCsist] Muestra la pantalla \cdtIdRef{IU 8.4b}{Consultar regla de negocio: Unicidad de parámetros}.
    \UCpaso[] Continúa con el paso \ref{cu8.4:finalizar} de la trayectoria principal.
 \end{UCtrayectoriaA} 
 %----------------------------------------------------------%trayectoria D---------------------------------------------------- 
 \begin{UCtrayectoriaA}{D}{La regla de negocio es de tipo ``Formato correcto''.}
    \UCpaso[\UCsist] Muestra la pantalla \cdtIdRef{IU 8.4c}{Consultar regla de negocio: Formato correcto}.
    \UCpaso[] Continúa con el paso \ref{cu8.4:finalizar} de la trayectoria principal.
 \end{UCtrayectoriaA} 
  
 \subsubsection{Puntos de extensión}
 
 \UCExtensionPoint{El actor desea consultar alguno de los atributos mostrados en la pantalla.}
 	{Paso \ref{cu8.4:consultarElemento} de la trayectoria principal}
 	{\cdtIdRef{CU 11.4}{Consultar entidad}}