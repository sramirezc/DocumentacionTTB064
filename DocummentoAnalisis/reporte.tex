 \documentclass[10pt]{book}
\usepackage{cdt/cdtBusiness}
\usepackage{prisma}
\usepackage{enumerate}
\usepackage{subfigure}
\usepackage{cite}
\usepackage{ragged2e}
\pagenumbering{roman}
\addtocontents{toc}{\protect\setcounter{tocdepth}{2}}





\organizacion[IPN]{Directores: M. en C. Idalia Maldonado Castillo y M. en C. José Jaime López Rabadán}
\proyecto[TT]{Prototipo de editor de casos de uso para la construcción asistida de casos de prueba}
\documento{2014}{}{\RELEASE{1}} %\DRAFT{\today}
\entregable{B064}{\Large{Documento de análisis}}
\title{Instituto Politécnico Nacional \\           Escuela Superior de Cómputo}
\author{Natalia Giselle Hernández Sánchez, Sergio Ramírez Camacho}
\fecha{7 de diciembre de 2015}

\begin{document}
\bibliographystyle{IEEEtran}

\thispagestyle{empty}
\maketitle

%=========================================================
% Indices del documento
\frontmatter
\tableofcontents
\listoffigures
\mainmatter

% Para esconder la información del documentador se descomenta el \hideControlVersion
 \hideControlVersion

%=========================================================
\chapter{Introducción}\label{chp:introduccion}
\hypertarget{chp:introduccion}{}
\cfinput{Introduccion/introduccion}
%=========================================================

\chapter{Marco teórico}\label{chp:marcoTeorico} 
\hypertarget{chp:marcoTeorico}{}
\cfinput{MarcoTeorico/marcoTeorico}
%=========================================================
%\chapter{Requerimientos del sistema}\label{chp:requerimientos}
%\hypertarget{chp:requerimientos}{}
%\cfinput{ModeloNegocios/requerimientos}
%=========================================================
%\chapter{Análisis de riesgos}\label{chp:riesgos}
%\hypertarget{chp:riesgos}{}
%\cfinput{AnalisisRiesgos/analisisRiesgos}

%=========================================================
\chapter{Modelo de negocio}\label{chp:modeloNegocios}
\hypertarget{chp:modeloNegocios}{}
En este capítulo se mostrará aquella información que define el negocio de la herramienta: el glosario, los diagramas de estado, el modelo conceptual 
y las reglas de negocio.\\

	\cfinput{ModeloNegocios/glosario}
	\cfinput{ModeloNegocios/estados}
	\cfinput{ModeloNegocios/conceptual}
	\cfinput{ModeloNegocios/conceptualProyecto}
	\cfinput{ModeloNegocios/reglas}
%===========================================================
 \chapter{Modelo de comportamiento}\label{chp:modeloComportamiento}
 \hypertarget{chp:modeloComportamiento}{}
	\cfinput{ModeloComportamiento/comportamiento.tex}
	\cfinput{cu1/cu} 		%
	\cfinput{cu1.1/cu}		%
	\cfinput{cu1.2/cu}		%
	\cfinput{cu1.3/cu}		%
	\cfinput{cu2/cu}		%
	\cfinput{cu2.1/cu}		%
	\cfinput{cu2.2/cu}		%	
	\cfinput{cu2.3/cu}		%
	\cfinput{cu3/cu}		%
	\cfinput{cu3.1/cu}		%
	\cfinput{cu3.2/cu}		%
	\cfinput{cu3.3/cu}		%
	\cfinput{cu4/cu}		%
	\cfinput{cu4.1/cu}		%
	\cfinput{cu4.2/cu}		%
	\cfinput{cu5/cu} 		%
	\cfinput{cu5.1/cu}		%
	\cfinput{cu5.1.1/cu}	%	Gestionar trayectorias
	\cfinput{cu5.1.1.1/cu}	%	Registrar t
	\cfinput{cu5.1.1.1.1/cu}%	Registrar p
 	\cfinput{cu5.1.1.1.2/cu}%	Modificar p
 	\cfinput{cu5.1.1.1.3/cu}%	Eliminar p
	\cfinput{cu5.1.1.2/cu} 	% 	Modificar t
	\cfinput{cu5.1.1.3/cu} 	%	Eliminar t
	
	\cfinput{cu5.1.2/cu}	%	Registrar precondición
	\cfinput{cu5.1.3/cu}	% 	Eliminar precondición
	\cfinput{cu5.1.4/cu}	% 	Registrar post
	\cfinput{cu5.1.5/cu} 	% 	Eliminar post
	
	\cfinput{cu5.1.6/cu}	% Gestionar ptos
	\cfinput{cu5.1.6.1/cu} 	% Registrar pto
	\cfinput{cu5.1.6.2/cu} 	% Eliminar pto
		
		
	\cfinput{cu5.2/cu}		% Modificar cu
	\cfinput{cu5.3/cu}		% Eliminar cu
	\cfinput{cu5.4/cu}		% Consultar cu
	\cfinput{cu5.5/cu}		% Revisar cu
	\cfinput{cu5.6/cu}		% Terminar cu
	
	\cfinput{cu5.7/cu}		% Configurar prueba
	\cfinput{cu5.7.1/cu}	% Configurar información general
	\cfinput{cu5.7.2/cu}	% Gestionar casos de uso previos
	\cfinput{cu5.7.2.1/cu}	% Configurar caso de uso previo
	\cfinput{cu5.7.3/cu}	% Configurar caso de uso
	
	\cfinput{cu5.8/cu}	% Generar prueba
	
	\cfinput{cu5.9/cu}	% Liberar caso de uso
	\cfinput{cu5.10/cu}% Desbloquear caso de uso
	
	
	\cfinput{cu6/cu}
	\cfinput{cu6.1/cu}
	\cfinput{cu6.1.1/cu}
	\cfinput{cu6.1.2/cu}	
	\cfinput{cu6.1.3/cu}
	\cfinput{cu6.2/cu}
	\cfinput{cu6.3/cu}
	\cfinput{cu6.4/cu}
	\cfinput{cu7/cu}
	\cfinput{cu7.1/cu}
	\cfinput{cu7.2/cu}
	\cfinput{cu7.3/cu}
	\cfinput{cu7.4/cu}
	\cfinput{cu8/cu}
	\cfinput{cu8.1/cu}
	\cfinput{cu8.2/cu}
	\cfinput{cu8.3/cu}
	\cfinput{cu8.4/cu}
	\cfinput{cu9/cu}
	\cfinput{cu9.1/cu}
	\cfinput{cu9.2/cu}
	\cfinput{cu9.3/cu}
	\cfinput{cu9.4/cu}
	\cfinput{cu10/cu}
	\cfinput{cu10.1/cu}
	\cfinput{cu10.2/cu}
	\cfinput{cu10.3/cu}
	\cfinput{cu10.4/cu}
	\cfinput{cu11/cu}
	\cfinput{cu11.1/cu}
	\cfinput{cu11.1.1/cu}
	\cfinput{cu11.1.2/cu}
	\cfinput{cu11.1.3/cu}
	\cfinput{cu11.2/cu}
	\cfinput{cu11.3/cu}
	\cfinput{cu11.4/cu}
	\cfinput{cu13/cu}
	\cfinput{cu14/cu}
	\cfinput{cu15/cu}	
% 	\cfinput{cu/cu}

\chapter{Modelo de interacción con el usuario}\label{chp:modeloInteraccionUsuario}
\hypertarget{chp:modeloInteraccionUsuario}{}
\cfinput{ModeloInteraccion/interaccion}

%---------------------------------------------------------------------

	\cfinput{cu1/iu}
	\cfinput{cu1.1/iu}
	\cfinput{cu1.2/iu}
	\cfinput{cu2/iu}
	\cfinput{cu2.1/iu}
	\cfinput{cu2.2/iu}
	\cfinput{cu3/iu}
	\cfinput{cu3.1/iu}
	\cfinput{cu3.2/iu}
	\cfinput{cu4/iu}
	\cfinput{cu4.1/iu}
	\cfinput{cu5/iu}
	\cfinput{cu5.1/iu}
	\cfinput{cu5.1.1/iu}
	\cfinput{cu5.1.1.1/iu}
	\cfinput{cu5.1.1.2/iu}
	\cfinput{cu5.1.1.1.1/iu}
 	\cfinput{cu5.1.1.1.2/iu}
	\cfinput{cu5.1.2/iu}
	\cfinput{cu5.1.4/iu}
	\cfinput{cu5.1.6/iu}
	\cfinput{cu5.1.6.1/iu}
	\cfinput{cu5.2/iu}
	\cfinput{cu5.4/iu}
	\cfinput{cu5.5/iu}
	
	\cfinput{cu5.7.1/iu}
	\cfinput{cu5.7.2/iu}
	\cfinput{cu5.7.2.1/iu}	
	\cfinput{cu5.7.3/iu}
	
	\cfinput{cu5.9/iu}	
		
	\cfinput{cu6/iu}
	\cfinput{cu6.1/iu}
	\cfinput{cu6.1.1/iu}
	\cfinput{cu6.1.2/iu}
	\cfinput{cu6.2/iu}
	\cfinput{cu6.4/iu}
	\cfinput{cu7/iu}
	\cfinput{cu7.1/iu}
	\cfinput{cu7.2/iu}
	\cfinput{cu7.4/iu}
	\cfinput{cu8/iu}
	\cfinput{cu8.1/iu}
	\cfinput{cu8.2/iu}
	\cfinput{cu8.4/iu}
	\cfinput{cu9/iu}
	\cfinput{cu9.1/iu}
	\cfinput{cu9.2/iu}
	\cfinput{cu9.4/iu}
	\cfinput{cu10/iu}
	\cfinput{cu10.1/iu}
	\cfinput{cu10.2/iu}
	\cfinput{cu10.4/iu}
	\cfinput{cu11/iu}
	\cfinput{cu11.1/iu}
	\cfinput{cu11.1.1/iu}
	\cfinput{cu11.1.2/iu}
	\cfinput{cu11.2/iu}	
	\cfinput{cu11.4/iu}	
	\cfinput{cu13/iu}
	\cfinput{cu14/iu}
	
%---------------------------------------------------------------------
\section{Mensajes}
	\cfinput{ModeloInteraccion/mensajes}

% \clossing

%---------------------------------------------------------------------

\bibliography{Referencias}

\end{document}


