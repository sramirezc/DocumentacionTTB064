\newpage 
\subsection{IU 5.1.4 Registrar postcondición}

\subsubsection{Objetivo}
	
	Esta pantalla permite al actor registrar una postcondición.

\subsubsection{Diseño}

    En la figura ~\ref{IU 5.1.4} se muestra la pantalla ``Registrar postcondición'' que permite registrar una postcondición.  
    Una vez ingresada la redacción solicitada en la pantalla, el actor deberá oprimir el botón 
    \cdtButton{Aceptar}, el sistema validará y agregará la postcondición a la tabla de ``Postcondición'' solo si se han cumplido todas las reglas de negocio establecidas.  \\


    \IUfig[.9]{cu5.1.4/images/iu.png}{IU 5.1.4}{Registrar postcondición}


\subsubsection{Comandos}
\begin{itemize}
	\item \cdtButton{Aceptar}: Permite al actor guardar el registro de la postcondición, dirige a la pantalla \cdtIdRef{IU 5.1}{Registrar caso de uso} o a la pantalla \cdtIdRef{IU 5.2}{Modificar caso de uso}, según corresponda.
	\item \cdtButton{Cancelar}: Permite al actor cancelar el registro de la postcondición, dirige a la pantalla \cdtIdRef{IU 5.1}{Registrar caso de uso} o a la pantalla \cdtIdRef{IU 5.2}{Modificar caso de uso}, según corresponda.
\end{itemize}

\subsubsection{Mensajes}

	
\begin{description}
	\item[ \cdtIdRef{MSG4}{Dato obligatorio}:] Se muestra en la pantalla \cdtIdRef{IU 5.1.4}{Registrar postcondición} cuando no se ha ingresado un dato marcado como obligatorio.
	\item[ \cdtIdRef{MSG6}{Longitud inválida}:] Se muestra en la pantalla \cdtIdRef{IU 5.1.4}{Registrar postcondición} cuando se ha excedido la longitud de alguno de los campos.
\end{description}
