\newpage 
\begin{UseCase}{CU 3.3}{Eliminar módulo}
	{
		Este caso de uso permite al actor eliminar del sistema un módulo.
	}
	
	\UCitem{Actor}{\cdtRef{actor:liderAnalisis}{Líder de análisis}, \cdtRef{actor:analista}{Analista}}
	\UCitem{Propósito}{
		Eliminar la información de un módulo.
	}
	\UCitem{Entradas}{
		Ninguna
	}
	\UCitem{Salidas}{
		\begin{UClist}
			\UCli \cdtIdRef{MSG1}{Operación exitosa}: Se muestra en la pantalla \cdtIdRef{IU 3}{Gestionar módulos} para indicar que la eliminación fue exitosa.
		\end{UClist}
	}
	\UCitem{Mensajes}{
		\begin{UClist}
			\UCli \cdtIdRef{MSG11}{Confirmar eliminación}: Se muestra para que el actor confirme la eliminación.
		\end{UClist}
	}

	\UCitem{Precondiciones}{
		\begin{UClist}
			\UCli Que no se estén haciendo referencias al contenido del módulo a eliminar desde elementos de otro módulo.
		\end{UClist}
	}
	
	\UCitem{Postcondiciones}{
		\begin{UClist}
			\UCli Se eliminará del sistema el módulo.
		\end{UClist}
	}

	\UCitem{Errores}{
		\begin{UClist}
				\UCli \cdtIdRef{MSG14}{Eliminación no permitida}: Se muestra en la pantalla \cdtIdRef{IU 3}{Gestionar módulos} cuando no sea posible eliminar el módulo, con base en la regla de negocio \cdtIdRef{RN28}{Eliminación de módulos}.
		\end{UClist}	
	}

	\UCitem{Tipo}{
		Secundario, extiende del caso de uso \cdtIdRef{CU 3}{Gestionar módulos}.
	}
\end{UseCase}
%-------------------------------------------------------%trayectoria Principal-----------------------------------------------
 \begin{UCtrayectoria}
    \UCpaso[\UCactor] Solicita eliminar un módulo oprimiendo el botón \btnEliminar del registro que desea eliminar de la pantalla \cdtIdRef{IU 3}{Gestionar módulos}.
    \UCpaso[\UCsist] Busca la información del módulo seleccionado.
    \UCpaso[\UCsist] Verifica que el módulo pueda eliminarse, con base en la regla de negocio \cdtIdRef{RN28}{Eliminación de módulos}. \refTray{A}
    \UCpaso[\UCsist] Muestra el mensaje \cdtIdRef{MSG11}{Confirmar eliminación} en una pantalla emergente con los botones \cdtButton{Aceptar} y \cdtButton{Cancelar}.
    \UCpaso[\UCactor] Confirma la eliminación del módulo oprimiendo el botón \cdtButton{Aceptar} de la pantalla emergente. \refTray{B}
    \UCpaso[\UCsist] Verifica que el módulo pueda eliminarse, con base en la regla de negocio \cdtIdRef{RN28}{Eliminación de módulos}. \refTray{A}
    \UCpaso[\UCsist] Elimina la información referente al módulo.
    \UCpaso[\UCsist] Muestra el mensaje \cdtIdRef{MSG1}{Operación exitosa} en la pantalla \cdtIdRef{IU 3}{Gestionar módulos}
    para indicar al actor que se ha eliminado el registro exitosamente.
 \end{UCtrayectoria}
 
 %----------------------------------------------------------%trayectoria A---------------------------------------------------- 
 \begin{UCtrayectoriaA}[Fin del caso de uso]{A}{No es posible eliminar el módulo.}
    \UCpaso[\UCsist] Muestra la pantalla \cdtIdRef{IU 3}{Gestionar módulos} con el mensaje \cdtIdRef{MSG14}{Eliminación no permitida}.
 \end{UCtrayectoriaA}
 %----------------------------------------------------------%trayectoria B---------------------------------------------------- 
 \begin{UCtrayectoriaA}[Fin del caso de uso]{B}{El actor desea cancelar la operación.}
    \UCpaso[\UCactor] Solicita cancelar la operación oprimiendo el botón \cdtButton{Cancelar} de la pantalla emergente.
    \UCpaso[\UCsist] Muestra la pantalla donde se solicitó la operación.
 \end{UCtrayectoriaA} 