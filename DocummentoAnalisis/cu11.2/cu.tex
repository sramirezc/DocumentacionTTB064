\begin{UseCase}{CU 11.2}{Modificar entidad}
	{
		Este caso de uso permite al analista modificar la información de una entidad, así como gestionar sus atributos.
	}
	
	\UCitem{Actor}{\cdtRef{actor:liderAnalisis}{Líder de análisis}, \cdtRef{actor:analista}{Analista}}
	\UCitem{Propósito}{
		Modificar una entidad y gestionar sus atributos.
	}
	\UCitem{Entradas}{
		\begin{UClist}
			\UCli \cdtRef{Elemento:Nombre}{Nombre}: \ioEscribir.
			\UCli \cdtRef{Elemento:Descripcion}{Descripción}: \ioEscribir.
			\UCli \cdtRef{Actualizacion:Comentario}{Comentario de la actualización}
		\end{UClist}
	}
	\UCitem{Salidas}{
		\begin{UClist}
			\UCli \cdtRef{Elemento:Nombre}{Nombre}: \ioObtener.
			\UCli \cdtRef{Elemento:Descripcion}{Descripción}: \ioObtener.
			\UCli \cdtRef{Atributo}{Atributos}: \ioTabla{\cdtRef{Elemento:Nombre}{Nombre}, 
			\cdtRef{Atributo:Obligatorio}{Obligatorio (sí o no)} y \cdtRef{gls:TipoDeDato}{Tipo de Dato}}{de los atributos}.
		\end{UClist}
	}
	
	\UCitem{Mensajes}{
		\begin{UClist}
			\UCli \cdtIdRef{MSG1}{Operación exitosa}: Se muestra en la pantalla \cdtIdRef{IU 11}{Gestionar entidades} para indicar que la modificación fue exitosa.
		\end{UClist}
	}

	\UCitem{Precondiciones}{
		\begin{UClist}
			\UCli Que la entidad no se encuentre asociada a un caso de uso en estado ``Liberado''.
		\end{UClist}
	}
	
	\UCitem{Postcondiciones}{
		Ninguna
	}

	\UCitem{Errores}{
		\begin{UClist}
			\UCli \cdtIdRef{MSG4}{Dato obligatorio}: Se muestra en la pantalla \cdtIdRef{IU 11.2}{Modificar entidad} o en la pantalla emergente \cdtIdRef{IU 11.2A}{Modificar entidad: Comentarios} cuando no se ha ingresado un dato marcado como obligatorio.
			\UCli \cdtIdRef{MSG5}{Dato incorrecto}: Se muestra en la pantalla \cdtIdRef{IU 11.2}{Modificar entidad} cuando el tipo de dato ingresado no cumple con el tipo de dato solicitado en el campo.
			\UCli \cdtIdRef{MSG6}{Longitud inválida}: Se muestra en la pantalla \cdtIdRef{IU 11.2}{Modificar entidad} o en la pantalla emergente \cdtIdRef{IU 11.2A}{Modificar entidad: Comentarios} cuando se ha excedido la longitud de alguno de los campos.
			\UCli \cdtIdRef{MSG7}{Registro repetido}: Se muestra en la pantalla \cdtIdRef{IU 11.2}{Modificar entidad} cuando se registre una entidad con un nombre que ya este registrado.
			\UCli \cdtIdRef{MSG18}{Registro necesario}: Se muestra en la pantalla \cdtIdRef{IU 11.2}{Modificar entidad} cuando el actor no ingrese ningún atributo.
			\UCli \cdtIdRef{MSG23}{Caracteres inválidos}: Se muestra en la pantalla \cdtIdRef{IU 11.2}{Modificar entidad} cuando el nombre de la entidad contiene un carácter no válido.
			\UCli \cdtIdRef{MSG30}{Modificación no permitida}: Se muestra en la pantalla \cdtIdRef{IU 11}{Gestionar entidades} cuando la entidad que se desea modificar se encuentra asociada a casos de uso liberados.
		\end{UClist}
	}

	\UCitem{Tipo}{
		Secundario, extiende del caso de uso \cdtIdRef{CU 11}{Gestionar entidades}.
% 		\clearpage
	}
\end{UseCase}
%-------------------------------------------------------%trayectoria Principal-----------------------------------------------
 \begin{UCtrayectoria}
    \UCpaso[\UCactor] Oprime el botón \btnEditar de la entidad que desea modificar de la pantalla \cdtIdRef{IU 11}{Gestionar entidades}.
    \UCpaso[\UCsist] Busca la información de la entidad.
	\UCpaso[\UCsist] Verifica que la entidad pueda modificarse con base en la regla de negocio \cdtIdRef{RN05}{Modificación de elementos asociados a casos de uso liberados}. \refTray{G}
    \UCpaso[\UCsist] Muestra la información encontrada en la pantalla \cdtIdRef{IU 11.2}{Modificar entidad}. \label{cu11.2:muestra}
    \UCpaso[\UCactor] Modifica el nombre y la descripción de la entidad en la pantalla \cdtIdRef{IU 11.2}{Modificar entidad}. \label{cu11.2:ingresaDatos}
    \UCpaso[\UCactor] Gestiona los atributos a través de los botones: \cdtButton{Registrar}, \btnEditar y \btnEliminar. \label{cu11.2:gestionar}
    \UCpaso[\UCactor] Solicita guardar la entidad oprimiendo el botón \cdtButton{Aceptar} de la pantalla \cdtIdRef{IU 11.2}{Modificar entidad}. \refTray{A}
    \UCpaso[\UCsist] Muestra la pantalla \cdtIdRef{IU 11.2A}{Modificar entidad: Comentarios}.
    \UCpaso[\UCactor] Ingresa el comentario referente a la modificación realizada en la pantalla emergente \cdtIdRef{IU 11.2A}{Modificar entidad: Comentarios}. \label{cu11.2:ingresaComentario}
    \UCpaso[\UCactor] Oprime el botón \cdtButton{Aceptar} de la pantalla emergente \cdtIdRef{IU 11.2A}{Modificar entidad: Comentarios}. \refTray{H}	
    \UCpaso[\UCsist] Verifica que el actor ingrese todos los campos obligatorios con base en la regla de negocio  \cdtIdRef{RN8}{Datos obligatorios}. \refTray{B}
    \UCpaso[\UCsist] Verifica que el nombre de la entidad no se encuentre registrado en el sistema con base en la regla de negocio  \cdtIdRef{RN6}{Unicidad de nombres}. \refTray{C}
    \UCpaso[\UCsist] Verifica que los datos requeridos sean proporcionados correctamente como se especifica en la regla de negocio \cdtIdRef{RN7}{Información correcta}. \refTray{D}
    \UCpaso[\UCsist] Verifica que el nombre no contenga caracteres inválidos con base en la regla de negocio \cdtIdRef{RN2}{Nombres de los elementos}. \refTray{E}
    \UCpaso[\UCsist] Verifica que el actor haya ingresado al menos un atributo. \refTray{F}
    \UCpaso[\UCsist] Modifica la información de la entidad en el sistema.
    \UCpaso[\UCsist] Muestra el mensaje \cdtIdRef{MSG1}{Operación exitosa} en la pantalla \cdtIdRef{IU 11}{Gestionar entidades} 
    para indicar al actor que el registro se ha realizado exitosamente.
 \end{UCtrayectoria}
 
 %----------------------------------------------------------%trayectoria A---------------------------------------------------- 
 \begin{UCtrayectoriaA}[Fin del caso de uso]{A}{El actor desea cancelar la operación.}
    \UCpaso[\UCactor] Solicita cancelar la operación oprimiendo el botón \cdtButton{Cancelar} de la pantalla \cdtIdRef{IU 11.2}{Modificar entidad}.
    \UCpaso[\UCsist] Muestra la pantalla \cdtIdRef{IU 11}{Gestionar entidades}.
 \end{UCtrayectoriaA}
  %----------------------------------------------------------%trayectoria B---------------------------------------------------- 
 \begin{UCtrayectoriaA}{B}{El actor no ingresó algún dato marcado como obligatorio.}
    \UCpaso[\UCsist] Muestra el mensaje \cdtIdRef{MSG4}{Dato obligatorio} y señala el campo que presenta el error en la pantalla 
	    \cdtIdRef{IU 11.2}{Modificar entidad} o en la pantalla emergente \cdtIdRef{IU 11.2A}{Modificar entidad: Comentarios}, indicando al actor que el dato es obligatorio.
    \UCpaso[] Continúa con el paso \ref{cu11.2:ingresaDatos} o con el paso \label{cu11.2:ingresaComentario} de la trayectoria principal.
 \end{UCtrayectoriaA}
 %----------------------------------------------------------%trayectoria C---------------------------------------------------- 
 \begin{UCtrayectoriaA}{C}{El actor ingresó un nombre de entidad repetido.}
    \UCpaso[\UCsist] Muestra el mensaje \cdtIdRef{MSG7}{Registro repetido} y señala el campo que presenta la duplicidad en la pantalla 
	    \cdtIdRef{IU 11.2}{Modificar entidad}, indicando al actor que existe una entidad con el mismo nombre.
    \UCpaso[] Continúa con el paso \ref{cu11.2:ingresaDatos} de la trayectoria principal.
 \end{UCtrayectoriaA}
 %----------------------------------------------------------%trayectoria D----------------------------------------------------  
 \begin{UCtrayectoriaA}{D}{El actor proporciona un dato que excede la longitud máxima.}
    \UCpaso[\UCsist] Muestra el mensaje \cdtIdRef{MSG5}{Se ha excedido la longitud máxima del campo} y señala el campo que excede la 
    longitud en la pantalla \cdtIdRef{IU 11.2}{Modificar entidad} o en la pantalla emergente \cdtIdRef{IU 11.2A}{Modificar entidad: Comentarios}, para indicar que el dato excede el tamaño máximo permitido.
    \UCpaso[] Continúa con el paso \ref{cu11.2:ingresaDatos} o con el paso \label{cu11.2:ingresaComentario} de la trayectoria principal.
 \end{UCtrayectoriaA}
 %----------------------------------------------------------%trayectoria E---------------------------------------------------- 
 \begin{UCtrayectoriaA}{E}{El actor ingresó un nombre con caracteres inválidos.}
    \UCpaso[\UCsist] Muestra el mensaje \cdtIdRef{MSG23}{Caracteres inválidos} y señala el campo que contiene los caracteres inválidos.
    \UCpaso[] Continúa con el paso \ref{cu11.2:ingresaDatos} de la trayectoria principal.
 \end{UCtrayectoriaA}
 %----------------------------------------------------------%trayectoria F---------------------------------------------------- 
 \begin{UCtrayectoriaA}{F}{El actor no registró ningún atributo.}
    \UCpaso[\UCsist] Muestra el mensaje \cdtIdRef{MSG18}{Registro necesario} en la sección de atributos.
    \UCpaso[] Continúa con el paso \ref{cu11.2:ingresaDatos} de la trayectoria principal.
 \end{UCtrayectoriaA}
 %----------------------------------------------------------%trayectoria G---------------------------------------------------- 
 \begin{UCtrayectoriaA}[Fin del caso de uso]{G}{La entidad no puede modificarse debido a que se encuentra asociado a casos de uso liberados.}
    \UCpaso[\UCsist] Muestra la pantalla emergente con el mensaje \cdtIdRef{MSG30}{Modificación no permitida} en la pantalla \cdtIdRef{IU 11}{Gestionar entidades}.
 \end{UCtrayectoriaA}
 %----------------------------------------------------------%trayectoria H---------------------------------------------------- 
 \begin{UCtrayectoriaA}[Fin del caso de uso]{H}{El actor no desea confirmar la modificación.}
    \UCpaso[\UCactor] Solicita cancelar la operación oprimiendo el botón \cdtButton{Cancelar} de la pantalla \cdtIdRef{IU 11.2A}{Modificar entidad: Comentarios}.
    \UCpaso[] Continúa con el paso \ref{cu11.2:muestra} de la trayectoria principal.
 \end{UCtrayectoriaA}
\subsection{Puntos de extensión}
 
 \UCExtensionPoint{El actor requiere registrar un atributo}
	{Paso \ref{cu11.2:gestionar}}
	{\cdtIdRef{CU 11.2}{Registrar atributo}}
\UCExtensionPoint{El actor requiere modificar un atributo}
	{Paso \ref{cu11.2:gestionar}}
	{\cdtIdRef{CU 11.2}{Modificar atributo}}	
\UCExtensionPoint{El actor requiere eliminar un atributo}
	{Paso \ref{cu11.2:gestionar}}
	{\cdtIdRef{CU 11.3}{Eliminar atributo}}
