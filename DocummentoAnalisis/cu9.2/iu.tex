\newpage 
\subsection{IU 9.2 Modificar mensaje}

\subsubsection{Objetivo}
	
	Esta pantalla permite al actor modificar la información de un mensaje nuevo.

\subsubsection{Diseño}

    En la figura ~\ref{IU 9.2} se muestra la pantalla ``Modificar mensaje: Parametrizado'' que permite modificar un mensaje, si el mensaje no es parametrizado se mostrará la pantalla de la figura ~\ref{IU 9.2A}. El actor deberá modificar la información mostrada y oprimir el botón \cdtButton{Aceptar}. El sistema validará y modificará la información solo si se han cumplido todas las reglas de negocio establecidas. \\
    
    Finalmente se mostrará el mensaje \cdtIdRef{MSG1}{Operación exitosa} en la pantalla \cdtIdRef{IU 9}{Gestionar mensajes}, para indicar que la información del mensaje
    se ha modificado correctamente.        

    \IUfig[.9]{cu9.2/images/iu.png}{IU 9.2}{Modificar mensaje: Parametrizado}
    \IUfig[.9]{cu9.2/images/iua.png}{IU 9.2A}{Modificar mensaje: No parametrizado}
	

\subsubsection{Comandos}
\begin{itemize}
	\item \cdtButton{Aceptar}: Permite al actor modificar la información del mensaje, dirige a la pantalla \cdtIdRef{IU 9}{Gestionar mensajes}.
	\item \cdtButton{Cancelar}: Permite al actor cancelar la modificación del mensaje, dirige a la pantalla \cdtIdRef{IU 9}{Gestionar mensajes}.
	\item \btnEditar: Permite habilitar la edición del campo de redacción.
	
\end{itemize}

\subsubsection{Mensajes}

	
\begin{description}
	\item[\cdtIdRef{MSG1}{Operación exitosa}:] Se muestra en la pantalla \cdtIdRef{IU 9}{Gestionar mensajes} para indicar que la modificación fue exitosa.
	\item[\cdtIdRef{MSG4}{Dato obligatorio}:] Se muestra en la pantalla \cdtIdRef{IU 9.2}{Modificar mensaje} o en la pantalla emergente \cdtIdRef{IU 9.2B}{Modificar mensaje: Comentario} cuando no se ha ingresado un dato marcado como obligatorio.
	\item[\cdtIdRef{MSG5}{Dato incorrecto}:] Se muestra en la pantalla \cdtIdRef{IU 9.2}{Modificar mensaje} cuando el tipo de dato ingresado no cumple con el tipo de dato solicitado en el campo.
	\item[\cdtIdRef{MSG6}{Longitud inválida}:] Se muestra en la pantalla \cdtIdRef{IU 9.2}{Modificar mensaje} o en la pantalla emergente \cdtIdRef{IU 9.2B}{Modificar mensaje: Comentario} cuando se ha excedido la longitud de alguno de los campos.
	\item[\cdtIdRef{MSG7}{Registro repetido}:] Se muestra en la pantalla \cdtIdRef{IU 9.2}{Modificar mensaje} cuando se registre un mensaje con un nombre o número que ya está registrado.
	\item[\cdtIdRef{MSG23}{Caracteres inválidos}:] Se muestra en la pantalla \cdtIdRef{IU 9.2}{Modificar mensaje} cuando el nombre del mensaje contenga un carácter no válido.
	\item[\cdtIdRef{MSG30}{Modificación no permitida}:] Se muestra en la pantalla \cdtIdRef{IU 9}{Gestionar mensajes} cuando el mensaje que se desea modificar se encuentra asociado a casos de uso liberados.
\end{description}
