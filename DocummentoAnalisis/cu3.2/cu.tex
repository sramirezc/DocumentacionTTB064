\newpage 
\begin{UseCase}{CU 3.2}{Modificar módulo}
	{
		Este caso de uso permite modificar la información de un módulo.
	}
	
	\UCitem{Actor}{\cdtRef{actor:liderAnalisis}{Líder de análisis}, \cdtRef{actor:analista}{Analista}}
	\UCitem{Propósito}{
		Modificar la información de un módulo.
	}
	\UCitem{Entradas}{
		\begin{UClist}
			\UCli \cdtRef{Modulo:Nombre}{Nombre}: \ioEscribir.
			\UCli \cdtRef{Modulo:Descripcion}{Descripción}: \ioEscribir.
		\end{UClist}
	}
	\UCitem{Salidas}{
		\begin{UClist}
			\UCli \cdtRef{Proyecto:Clave}{Clave} del proyecto. \ioObtener.
			\UCli \cdtRef{Proyecto:Nombre}{Nombre} del proyecto. \ioObtener.
			\UCli \cdtRef{Modulo:Clave}{Clave}: \ioObtener.
			\UCli \cdtRef{Modulo:Nombre}{Nombre}: \ioObtener.
			\UCli \cdtRef{Modulo:Descripcion}{Descripción}: \ioObtener.
		\end{UClist}
	}
	
	\UCitem{Mensajes}{
		\begin{UClist}
			\UCli \cdtIdRef{MSG1}{Operación exitosa}: Se muestra en la pantalla \cdtIdRef{IU 3}{Gestionar módulos} para indicar que la modificación fue exitosa.
		\end{UClist}
	}

	\UCitem{Precondiciones}{
		Ninguna
	}
	
	\UCitem{Postcondiciones}{
		\begin{UClist}
			\UCli Se actualizará la información de un módulo en el sistema.
		\end{UClist}
	}

	\UCitem{Errores}{
		\begin{UClist}
			\UCli \cdtIdRef{MSG4}{Dato obligatorio}: Se muestra en la pantalla \cdtIdRef{IU 3.2}{Modificar módulo} cuando no se ha ingresado un dato marcado como obligatorio.
			\UCli \cdtIdRef{MSG5}{Dato incorrecto}: Se muestra en la pantalla \cdtIdRef{IU 3.2}{Modificar módulo} cuando el tipo de dato ingresado no cumple con el tipo de dato solicitado en el campo.
			\UCli \cdtIdRef{MSG6}{Longitud inválida}: Se muestra en la pantalla \cdtIdRef{IU 3.2}{Modificar módulo} cuando se ha excedido la longitud de alguno de los campos.
			\UCli \cdtIdRef{MSG7}{Registro repetido}: Se muestra en la pantalla \cdtIdRef{IU 3.2}{Modificar módulo} cuando se ingrese un nombre o clave repetido en el proyecto.
			\UCli \cdtIdRef{MSG23}{Caracteres inválidos}: Se muestra en la pantalla \cdtIdRef{IU 3.2}{Modificar módulo} cuando el actor ingrese una clave inválida, con base en la regla de negocio \cdtIdRef{RN2}{Nombres de los elementos}.
			
		\end{UClist}
	}

	\UCitem{Tipo}{
		Secundario, extiende del caso de uso \cdtIdRef{CU 3}{Gestionar módulos}.
	}
\end{UseCase}
%-------------------------------------------------------%trayectoria Principal-----------------------------------------------
 \begin{UCtrayectoria}
    \UCpaso[\UCactor] Solicita modificar un módulo oprimiendo el botón \btnEditar del módulo que desea modificar de la pantalla \cdtIdRef{IU 3}{Gestionar módulos}.
    \UCpaso[\UCsist] Busca la información del múdulo seleccionado.
    \UCpaso[\UCsist] Muestra la pantalla \cdtIdRef{IU 3.2}{Modificar módulo} con la información del módulo. 
    \UCpaso[\UCactor] Modifica la información en la pantalla \cdtIdRef{IU 3.2}{Modificar módulo}. \label{cu3.2:ingresaDatos}
    \UCpaso[\UCactor] Solicita guardar la información del módulo oprimiendo el botón \cdtButton{Aceptar} de la pantalla \cdtIdRef{IU 3.2}{Modificar módulo}. \refTray{A} 
    \UCpaso[\UCsist] Verifica que el actor ingrese todos los campos obligatorios con base en la regla de negocio \cdtIdRef{RN8}{Datos obligatorios}. \refTray{B}
    \UCpaso[\UCsist] Verifica que la clave y el nombre no se encuentren registrados en el sistema con base en la regla de negocio  \cdtIdRef{RN6}{Unicidad de nombres}. \refTray{C}
    \UCpaso[\UCsist] Verifica que los datos requeridos sean proporcionados correctamente como se especifica en la regla de negocio \cdtIdRef{RN7}{Información correcta}. \refTray{D} \refTray{E}
    \UCpaso[\UCsist] Verifica que la clave del módulo sea válida como se especifica en la regla de negocio \cdtIdRef{RN2}{Nombres de los elementos}. \refTray{F}
    \UCpaso[\UCsist] Actualiza la información del módulo en el sistema.
    \UCpaso[\UCsist] Muestra el mensaje \cdtIdRef{MSG1}{Operación exitosa} en la pantalla \cdtIdRef{IU 3}{Gestionar módulos}
    para indicar al actor que el registro se ha realizado exitosamente. 
 \end{UCtrayectoria}
 %----------------------------------------------------------%trayectoria A---------------------------------------------------- 
 \begin{UCtrayectoriaA}[Fin del caso de uso]{A}{El actor desea cancelar la operación.}
    \UCpaso[\UCactor] Solicita cancelar la operación oprimiendo el botón \cdtButton{Cancelar} de la pantalla \cdtIdRef{IU 3.2}{Modificar módulo}.
    \UCpaso[\UCsist] Muestra la pantalla donde se solicitó la operación.
 \end{UCtrayectoriaA} 
  %----------------------------------------------------------%trayectoria B---------------------------------------------------- 
 \begin{UCtrayectoriaA}{B}{El actor no ingresó algún dato marcado como obligatorio.}
    \UCpaso[\UCsist] Muestra el mensaje \cdtIdRef{MSG4}{Dato obligatorio} y señala el campo que presenta el error en la pantalla 
	    \cdtIdRef{IU 3.2}{Modificar módulo}, indicando al actor que el dato es obligatorio.
    \UCpaso[] Continúa con el paso \ref{cu3.2:ingresaDatos} de la trayectoria principal.
 \end{UCtrayectoriaA}
 %----------------------------------------------------------%trayectoria C---------------------------------------------------- 
 \begin{UCtrayectoriaA}{C}{El actor ingresó un nombre o clave repetido.}
    \UCpaso[\UCsist] Muestra el mensaje \cdtIdRef{MSG7}{Registro repetido} y señala el campo que presenta la duplicidad en la pantalla 
	    \cdtIdRef{IU 3.2}{Modificar módulo}, indicando al actor que existe un módulo con el mismo nombre o clave.
    \UCpaso[] Continúa con el paso \ref{cu3.2:ingresaDatos} de la trayectoria principal.
 \end{UCtrayectoriaA}
 
 %----------------------------------------------------------%trayectoria D----------------------------------------------------  
 \begin{UCtrayectoriaA}{D}{El actor proporciona un dato que excede la longitud máxima.}
    \UCpaso[\UCsist] Muestra el mensaje \cdtIdRef{MSG5}{Se ha excedido la longitud máxima del campo} y señala el campo que excede la 
    longitud en la pantalla \cdtIdRef{IU 3.2}{Modificar módulo}, para indicar que el dato excede el tamaño máximo permitido.
    \UCpaso[] Continúa con el paso \ref{cu3.2:ingresaDatos} de la trayectoria principal.
 \end{UCtrayectoriaA}
 
 %----------------------------------------------------------%trayectoria E---------------------------------------------------- 
 \begin{UCtrayectoriaA}{E}{El actor ingresó un tipo de dato incorrecto.}
    \UCpaso[\UCsist] Muestra el mensaje \cdtIdRef{MSG4}{Formato incorrecto} y señala el campo que presenta el dato inválido en la 
    pantalla \cdtIdRef{IU 3.2}{Modificar módulo} para indicar que se ha ingresado un tipo de dato inválido.
    \UCpaso[] Continúa con el paso \ref{cu3.2:ingresaDatos} de la trayectoria principal.
 \end{UCtrayectoriaA}
 %----------------------------------------------------------%trayectoria F---------------------------------------------------- 
 \begin{UCtrayectoriaA}{F}{El actor ingresó una clave inválida.}
    \UCpaso[\UCsist] Muestra el mensaje \cdtIdRef{MSG23}{Caracteres inválidos} y señala el campo en la 
    pantalla \cdtIdRef{IU 3.2}{Modificar módulo} para indicar que se ha ingresado una clave inválida.
    \UCpaso[] Continúa con el paso \ref{cu3.2:ingresaDatos} de la trayectoria principal.
 \end{UCtrayectoriaA}