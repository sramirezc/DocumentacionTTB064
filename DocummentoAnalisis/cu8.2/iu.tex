\newpage 
\subsection{IU 8.2 Modificar regla de negocio}

\subsubsection{Objetivo}
	
	Esta pantalla permite al actor modificar la información de una regla de negocio registrada en el sistema.

\subsubsection{Diseño}

    En la figura ~\ref{IU 8.2} se muestra la pantalla ``Modificar regla de negocio'' que permite modificar una regla de negocio. El actor deberá 
    seleccionar el tipo de regla de negocio y el sistema mostrará los campos de los parámetros de la regla de negocio. \\
    
    Cuando el actor seleccione el tipo ``Comparación de atributos'', el sistema mostrará la pantalla ~\ref{IU 8.2a}; si el actor selecciona el tipo
    ``Unicidad de parámetros'' el sistema mostrará la pantalla ~\ref{IU 8.2b}; si el actor selecciona el tipo ``Formato correcto'' el sistema
    mostrará la pantalla ~\ref{IU 8.2c}. Para los demás tipos de reglas de negocio el sistema no mostrará más campos.\\
    
    Una vez ingresada la información solicitada, el actor deberá oprimir el botón \cdtButton{Aceptar}, el sistema validará y registrará la información sólo si se han cumplido todas las reglas de negocio establecidas.  \\
    
    Finalmente se mostrará el mensaje \cdtIdRef{MSG1}{Operación exitosa} en la pantalla \cdtIdRef{IU 8}{Gestionar reglas de negocio}, 
    para indicar que la información de la regla de negocio
    se ha registrado correctamente.        


    \IUfig[.9]{cu8.2/images/iu.png}{IU 8.2}{Modificar regla de negocio}
     \IUfig[.9]{cu8.2/images/iua.png}{IU 8.2a}{Modificar regla de negocio: Comparación de atributos}
     \IUfig[.9]{cu8.2/images/iub.png}{IU 8.2b}{Modificar regla de negocio: Unicidad de parámetros}
     \IUfig[.9]{cu8.2/images/iuc.png}{IU 8.2c}{Modificar regla de negocio: Formato correcto}
\subsubsection{Comandos}
\begin{itemize}
	\item \cdtButton{Aceptar}: Permite al actor guardar los cambios de la regla de negocio, dirige a la pantalla \cdtIdRef{IU 8}{Gestionar reglas de negocio}.
	\item \cdtButton{Cancelar}: Permite al actor cancelar los cambios de la regla de negocio, dirige a la pantalla \cdtIdRef{IU 8}{Gestionar reglas de negocio}.
\end{itemize}

\subsubsection{Mensajes}
	
\begin{description}
	\item[\cdtIdRef{MSG1}{Operación exitosa}:] Se muestra en la pantalla \cdtIdRef{IU 8}{Gestionar reglas de negocio} para indicar que la modificación fue exitosa.
	\item[\cdtIdRef{MSG4}{Dato obligatorio}:] Se muestra en la pantalla \cdtIdRef{IU 8.2}{Modificar regla de negocio} cuando no se ha ingresado un dato marcado como obligatorio.
	\item[\cdtIdRef{MSG5}{Dato incorrecto}:] Se muestra en la pantalla \cdtIdRef{IU 8.2}{Modificar regla de negocio} cuando el tipo de dato ingresado no cumple con el tipo de dato solicitado en el campo.
	\item[\cdtIdRef{MSG6}{Longitud inválida}:] Se muestra en la pantalla \cdtIdRef{IU 8.2}{Modificar regla de negocio} cuando se ha excedido la longitud de alguno de los campos.
	\item[\cdtIdRef{MSG7}{Registro repetido}:] Se muestra en la pantalla \cdtIdRef{IU 8.2}{Modificar regla de negocio} cuando se registre una regla de negocio con un nombre o número que ya se encuentre registrado.
	\item[\cdtIdRef{MSG23}{Caracteres inválidos}:] Se muestra en la pantalla \cdtIdRef{IU 8.2}{Modificar regla de negocio} cuando el nombre de la regla de negocio contiene un caracter no válido.
	\item[\cdtIdRef{MSG30}{Modificación no permitida}:] Se muestra en la pantalla \cdtIdRef{IU 8}{Gestionar reglas de negocio} cuando la regla de negocio que se desea modificar se encuentra asociada a casos de uso liberados.
\end{description}
