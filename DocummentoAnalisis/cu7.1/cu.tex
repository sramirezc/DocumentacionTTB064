\newpage 
\begin{UseCase}{CU 7.1}{Registrar actor}
	{
		Un actor es un rol que va a interactuar con el sistema. Este caso de uso permite al analista registrar 
		un actor.
		
	}
	\UCitem{Actor}{\cdtRef{actor:liderAnalisis}{Líder de análisis}, \cdtRef{actor:analista}{Analista}}
	\UCitem{Propósito}{
		Registrar la información de un actor.
	}
	\UCitem{Entradas}{
		\begin{UClist} 
			\UCli De la sección {\bf Información general del actor}: 
				\begin{Citemize}
				        \item \cdtRef{Actor:Nombre}{Nombre}: \ioEscribir.
					\item \cdtRef{Actor:Descripcion}{Descripción}: \ioEscribir.
					\item \cdtRef{gls:Cardinalidad}{Cardinalidad}: \ioSeleccionar.
					\item \cdtRef{Actor:OtraCardinalidad}{Otra cardinalidad}: \ioEscribir.
				\end{Citemize}
		\end{UClist}
	}	
	
	\UCitem{Salidas}{	
		\begin{UClist}
			\UCli \cdtRef{Actor:Clave}{Clave}: \ioCalcular{\cdtIdRef{RN12}{Identificador de elemento}}.
		\end{UClist}
	}
	
	\UCitem{Mensajes}{	
		\begin{UClist}
			\UCli \cdtIdRef{MSG1}{Operación exitosa}: Se muestra en la pantalla \cdtIdRef{IU 7}{Gestionar actores} para indicar que el registro fue exitoso.
		\end{UClist}
	}
	
	\UCitem{Precondiciones}{ 
		Ninguna
	}			
	\UCitem{Postcondiciones}{ 
		Ninguna
	}
	\UCitem{Errores}{
		\begin{UClist}
			\UCli \cdtIdRef{MSG4}{Dato obligatorio}: Se muestra en la pantalla \cdtIdRef{IU 7.1}{Registrar actor} cuando no se ha ingresado un dato marcado como obligatorio.
			\UCli \cdtIdRef{MSG5}{Dato incorrecto}: Se muestra en la pantalla \cdtIdRef{IU 7.1}{Registrar actor} cuando el tipo de dato ingresado no cumple con el tipo de dato solicitado en el campo.
			\UCli \cdtIdRef{MSG6}{Longitud inválida}: Se muestra en la pantalla \cdtIdRef{IU 7.1}{Registrar actor} cuando se ha excedido la longitud de alguno de los campos.
			\UCli \cdtIdRef{MSG7}{Registro repetido}: Se muestra en la pantalla \cdtIdRef{IU 7.1}{Registrar actor} cuando se registre un actor con un nombre que ya este registrado.
			\UCli \cdtIdRef{MSG13}{Ha ocurrido un error}: Se muestra en la pantalla \cdtIdRef{IU 7}{Gestionar actores} cuando no existe información base para el sistema.
			\UCli \cdtIdRef{MSG23}{Caracteres inválidos}: Se muestra en la pantalla \cdtIdRef{IU 7.1}{Registrar actor} cuando el nombre del actor contiene un caracter no válido.
		\end{UClist}
	}
	
	\UCitem{Tipo}{
		Secundario, extiende del caso de uso \cdtIdRef{CU 7}{Gestionar actores}.
	}

	
\end{UseCase}
 %-------------------------------------------------------%trayectoria Principal-----------------------------------------------
 \begin{UCtrayectoria}
    \UCpaso[\UCactor] Solicita registrar un actor oprimiendo el botón \cdtButton{Registrar} de la pantalla \cdtIdRef{IU 7}{Gestionar actores}.
    \UCpaso[\UCsist] Verifica que exista información referente a la Cardindalidad con base en la regla de negocio \cdtIdRef{RN20}{Verificación de catálogos}. \refTray{A}
    \UCpaso[\UCsist] Muestra la pantalla \cdtIdRef{IU 7.1}{Registrar actor} en la cual se realizará el registro del actor. 
    \UCpaso[\UCactor] Ingresa la información del actor en la pantalla \cdtIdRef{IU 7.1}{Registrar actor}. \label{cu7.1:ingresaDatos}
    \UCpaso[\UCactor] Solicita registrar al actor oprimiendo el botón \cdtButton{Aceptar} de la pantalla \cdtIdRef{IU 7.1}{Registrar actor}. \refTray{B}
    
    \UCpaso[\UCsist] Verifica que el actor ingrese todos los campos obligatorios con base en la regla de negocio  \cdtIdRef{RN8}{Datos obligatorios}. \refTray{C}
    \UCpaso[\UCsist] Verifica que el nombre del actor no se encuentre registrado en el sistema con base en la regla de negocio  \cdtIdRef{RN6}{Unicidad de nombres}. \refTray{D}
    \UCpaso[\UCsist] Verifica que el nombre no contenga caracteres inválidos con base en la regla de negocio \cdtIdRef{RN2}{Nombres de los elementos}. \refTray{E}
    \UCpaso[\UCsist] Verifica que los datos requeridos sean proporcionados correctamente como se especifica en la regla de negocio \cdtIdRef{RN7}{Información correcta}. \refTray{F} 
    \UCpaso[\UCsist] Verifica que la longitud de los datos sea correcta como se especifica en la regla de negocio \cdtIdRef{RN7}{Información correcta}. \refTray{G}
    \UCpaso[\UCsist] Registra la información del actor en el sistema.
    \UCpaso[\UCsist] Muestra el mensaje \cdtIdRef{MSG1}{Operación exitosa} en la pantalla \cdtIdRef{IU 7}{Gestionar actores} 
    para indicar al actor que el registro se ha realizado exitosamente.
 \end{UCtrayectoria}
 %----------------------------------------------------------%trayectoria A---------------------------------------------------- 
 \begin{UCtrayectoriaA}[Fin del caso de uso]{A}{No existe información base para el sistema.}
    \UCpaso[\UCsist] Muestra el mensaje \cdtRef{MSG13}{Ha ocurrido un error} en la pantalla \cdtIdRef{IU 7}{Gestionar actores}.
 \end{UCtrayectoriaA}
 %----------------------------------------------------------%trayectoria B---------------------------------------------------- 
 \begin{UCtrayectoriaA}[Fin del caso de uso]{B}{El actor desea cancelar la operación.}
    \UCpaso[\UCactor] Solicita cancelar la operación oprimiendo el botón \cdtButton{Cancelar} de la pantalla \cdtIdRef{IU 7.1}{Registrar actor}.
    \UCpaso[\UCsist] Muestra la pantalla \cdtIdRef{IU 7}{Gestionar actores}.
 \end{UCtrayectoriaA}
  %----------------------------------------------------------%trayectoria C---------------------------------------------------- 
 \begin{UCtrayectoriaA}{C}{El actor no ingresó algún dato marcado como obligatorio.}
    \UCpaso[\UCsist] Muestra el mensaje \cdtIdRef{MSG4}{Dato obligatorio} y señala el campo que presenta el error en la pantalla 
	    \cdtIdRef{CU 7.1}{Registrar actor}, indicando al actor que el dato es obligatorio.
    \UCpaso[] Continúa con el paso \ref{cu7.1:ingresaDatos} de la trayectoria principal.
 \end{UCtrayectoriaA}
 %----------------------------------------------------------%trayectoria D---------------------------------------------------- 
 \begin{UCtrayectoriaA}{D}{El actor ingresó un nombre de actor repetido.}
    \UCpaso[\UCsist] Muestra el mensaje \cdtIdRef{MSG7}{Registro repetido} y señala el campo que presenta la duplicidad en la pantalla 
	    \cdtIdRef{CU 7.1}{Registrar actor}, indicando al actor que existe un actor con el mismo nombre.
    \UCpaso[] Continúa con el paso \ref{cu7.1:ingresaDatos} de la trayectoria principal.
 \end{UCtrayectoriaA}
%----------------------------------------------------------%trayectoria E---------------------------------------------------- 
 \begin{UCtrayectoriaA}{E}{El actor ingresó un nombre con caracteres inválidos.}
    \UCpaso[\UCsist] Muestra el mensaje \cdtIdRef{MSG23}{Caracteres inválidos} y señala el campo que contiene los caracteres inválidos.
    \UCpaso[] Continúa con el paso \ref{cu7.1:ingresaDatos} de la trayectoria principal.
 \end{UCtrayectoriaA}
 %----------------------------------------------------------%trayectoria F---------------------------------------------------- 
 \begin{UCtrayectoriaA}{F}{El actor ingresó un tipo de dato incorrecto.}
    \UCpaso[\UCsist] Muestra el mensaje \cdtIdRef{MSG4}{Formato incorrecto} y señala el campo que presenta el dato inválido en la 
    pantalla \cdtIdRef{IU 7.1}{Registrar actor} para indicar que se ha ingresado un tipo de dato inválido.
    \UCpaso[] Continúa con el paso \ref{cu7.1:ingresaDatos} de la trayectoria principal.
 \end{UCtrayectoriaA}
 %----------------------------------------------------------%trayectoria G----------------------------------------------------  
 \begin{UCtrayectoriaA}{G}{El actor proporciona un dato que excede la longitud máxima.}
    \UCpaso[\UCsist] Muestra el mensaje \cdtIdRef{MSG7}{Se ha excedido la longitud máxima del campo} y señala el campo que excede la 
    longitud en la pantalla \cdtIdRef{IU 7.1}{Registrar actor}, para indicar que el dato excede el tamaño máximo permitido.
    \UCpaso[] Continúa con el paso \ref{cu7.1:ingresaDatos} de la trayectoria principal.
 \end{UCtrayectoriaA}
 
 