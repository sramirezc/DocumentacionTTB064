\newpage 
\begin{UseCase}{CU 5.1.5}{Eliminar postcondición}
	{
		Este caso de uso permite al actor eliminar un registro de la tabla de 
		postcondiciones de la pantalla \cdtIdRef{IU 5.1}{Registrar caso de uso} o de la pantalla \cdtIdRef{IU 5.2}{Modificar caso de uso}.
	}
	
	\UCitem{Actor}{\cdtRef{actor:liderAnalisis}{Líder de análisis}, \cdtRef{actor:analista}{Analista}}
	\UCitem{Propósito}{
		Eliminar una postcondición.
	}
	\UCitem{Entradas}{
		Ninguna
	}
	\UCitem{Salidas}{
		Ninguna
	}
	\UCitem{Mensajes}{
		Ninguno
	}
	\UCitem{Precondiciones}{
		Ninguna
	}
	
	\UCitem{Postcondiciones}{
		\begin{UClist}
			\UCli Se eliminará la postcondición de la tabla.
		\end{UClist}
	}

	\UCitem{Errores}{
		Ninguno
	}

	\UCitem{Tipo}{
		Secundario, extiende de los casos de uso \cdtIdRef{CU 5.1}{Registrar caso de uso} y \cdtIdRef{CU 5.2}{Modificar caso de uso}.
	}
\end{UseCase}
%-------------------------------------------------------%trayectoria Principal-----------------------------------------------
 \begin{UCtrayectoria}
    \UCpaso[\UCactor] Solicita eliminar una postcondición oprimiendo el botón \btnEliminar del registro que desea eliminar de la pantalla \cdtIdRef{IU 5.1}{Registrar caso de uso} o de la pantalla \cdtIdRef{IU 5.2}{Modificar caso de uso}, de la sección ``Postcondiciones''.
    \UCpaso[\UCsist] Elimina la postcondición de la tabla correspondiente.
 \end{UCtrayectoria}
