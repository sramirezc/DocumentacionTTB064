\begin{UseCase}{CU 9.4}{Consultar mensaje}
	{
		Este caso de uso permite al analista consultar la información de un mensaje.
	}
	
	\UCitem{Actor}{\cdtRef{actor:liderAnalisis}{Líder de análisis}, \cdtRef{actor:analista}{Analista}}
	\UCitem{Propósito}{
		Consultar la información de un mensaje.
	}
	\UCitem{Entradas}{
		Ninguna
	}
	\UCitem{Salidas}{
			\UCli \cdtRef{Elemento:Clave}{Clave}: \ioCalcular{\cdtIdRef{RN12}{Identificador de elemento}}.
			\UCli \cdtRef{Elemento:Numero}{Número}: \ioObtener.
			\UCli \cdtRef{Elemento:Nombre}{Nombre}: \ioObtener.
			\UCli \cdtRef{Elemento:Descripcion}{Descripción}: \ioObtener.
 			\UCli \cdtRef{Mensaje:Redaccion}{Redacción}: \ioObtener.
 			\UCli \cdtRef{Mensaje:Parametrizado}{Parametrizado}: \ioCheckBox.
 			\UCli \cdtRef{ValorDelParametroEnMensaje}{Parámetros}: \ioObtener.
	}	
	
	\UCitem{Mensajes}{
		Ninguno
	}

	\UCitem{Precondiciones}{
		Ninguna
	}
	
	\UCitem{Postcondiciones}{
		Ninguna
	}

	\UCitem{Errores}{
		\begin{UClist}
			\UCli \cdtIdRef{MSG13}{Ha ocurrido un error}: Se muestra en la pantalla donde se solicitó la operación cuando el mensaje que se desea consultar no existe.
		\end{UClist}
	}

	\UCitem{Tipo}{
		Secundario, extiende del caso de uso \cdtIdRef{CU 9}{Gestionar mensajes}.
	}
\end{UseCase}
%-------------------------------------------------------%trayectoria Principal-----------------------------------------------
 \begin{UCtrayectoria}
	\UCpaso[\UCactor] Oprime el botón \btnConsulta del mensaje que desea consultar en la pantalla \cdtIdRef{IU 9}{Gestionar mensajes}.
    \UCpaso[\UCsist] Busca la información del mensaje. \refTray{A} %No existe información
    \UCpaso[\UCsist] Muestra la información encontrada en la pantalla \cdtIdRef{IU 9.4}{Consultar mensaje}.
    \UCpaso[\UCactor] Consulta la información del mensaje. \label{cu9.4:consultarElemento} 
    \UCpaso[\UCactor] Solicita finalizar la consulta oprimiendo el botón \cdtButton{Aceptar} de la pantalla \cdtIdRef{IU 9.4}{Consultar mensaje}.
 \end{UCtrayectoria}
 
 %----------------------------------------------------------%trayectoria A---------------------------------------------------- 
 \begin{UCtrayectoriaA}[Fin del caso de uso]{A}{El mensaje que se desea consultar no existe.}
    \UCpaso[\UCsist] Muestra el mensaje \cdtIdRef{MSG13}{Ha ocurrido un error} en la pantalla donde se solicitó la consulta.
 \end{UCtrayectoriaA} 
