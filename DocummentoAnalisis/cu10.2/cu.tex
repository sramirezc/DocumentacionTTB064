\newpage 
\begin{UseCase}{CU 10.2}{Modificar término}
	{
		Este caso de uso permite al modificar la información de un término del glosario. 
	}
	
	\UCitem{Actor}{\cdtRef{actor:liderAnalisis}{Líder de análisis}, \cdtRef{actor:analista}{Analista}}
	\UCitem{Propósito}{
		Registrar la información de un término del glosario.
	}
	\UCitem{Entradas}{
		\begin{UClist}
			\UCli \cdtRef{Elemento:Nombre}{Nombre}: \ioEscribir.
			\UCli \cdtRef{Elemento:Descripcion}{Descripción}: \ioEscribir.
			\UCli \cdtRef{Actualizacion:Comentario}{Comentario de la actualización}: \ioEscribir
		\end{UClist}
	}
	\UCitem{Salidas}{
		\begin{UClist}
			\UCli \cdtRef{Elemento:Clave}{Clave}: \ioObtener
			\UCli \cdtRef{Elemento:Nombre}{Nombre}: \ioObtener.
			\UCli \cdtRef{Elemento:Descripcion}{Descripción}: \ioObtener.
		\end{UClist}
	}
	
	\UCitem{Mensajes}{
		\begin{UClist}
			\UCli \cdtIdRef{MSG1}{Operación exitosa}: Se muestra en la pantalla \cdtIdRef{IU 10}{Gestionar términos} para indicar que la modificación fue exitosa.
		\end{UClist}
	}

	\UCitem{Precondiciones}{
		\begin{UClist}
			\UCli Que el término no se encuentre asociado a un caso de uso en estado ``Liberado''.
		\end{UClist}
	}
	
	\UCitem{Postcondiciones}{
		Ninguna
	}

	\UCitem{Errores}{
		\begin{UClist}
			\UCli \cdtIdRef{MSG4}{Dato obligatorio}: Se muestra en la pantalla \cdtIdRef{IU 10.2}{Modificar término} o en la pantalla emergente \cdtIdRef{IU 10.2A}{Modificar término: Comentario} cuando no se ha ingresado un dato marcado como obligatorio.
			\UCli \cdtIdRef{MSG5}{Dato incorrecto}: Se muestra en la pantalla \cdtIdRef{IU 10.2}{Modificar término} cuando el tipo de dato ingresado no cumple con el tipo de dato solicitado en el campo.
			\UCli \cdtIdRef{MSG6}{Longitud inválida}: Se muestra en la pantalla \cdtIdRef{IU 10.2}{Modificar término} o en la pantalla emergente \cdtIdRef{IU 10.2A}{Modificar término: Comentario} cuando se ha excedido la longitud de alguno de los campos.
			\UCli \cdtIdRef{MSG7}{Registro repetido}: Se muestra en la pantalla \cdtIdRef{IU 10.2}{Modificar término} cuando se registre un término con un nombre que ya está registrado.
			\UCli \cdtIdRef{MSG23}{Caracteres inválidos}: Se muestra en la pantalla \cdtIdRef{IU 10.2}{Modificar término} cuando el nombre del término contenga un carácter no válido.
			\UCli \cdtIdRef{MSG30}{Modificación no permitida}: Se muestra en la pantalla \cdtIdRef{IU 10}{Gestionar términos} cuando el término que se desea modificar se encuentra asociado a casos de uso liberados.			
		\end{UClist}
	}

	\UCitem{Tipo}{
		Secundario, extiende del caso de uso \cdtIdRef{CU 10}{Gestionar términos}.
	}
\end{UseCase}
%-------------------------------------------------------%trayectoria Principal-----------------------------------------------
 \begin{UCtrayectoria}
	\UCpaso[\UCactor] Oprime el botón \btnEditar del término que desea modificar en la pantalla \cdtIdRef{IU 10}{Gestionar términos}.
	\UCpaso[\UCsist] Busca la información del término.
	\UCpaso[\UCsist] Verifica que el término pueda modificarse con base en la regla de negocio \cdtIdRef{RN05}{Modificación de elementos asociados a casos de uso liberados}. \refTray{G}
    \UCpaso[\UCsist] Muestra la información encontrada en la pantalla \cdtIdRef{IU 10.2}{Modificar término}.
    \UCpaso[\UCactor] Modifica el nombre y la descripción del término. \label{cu10.2:ingresaDatos}
    \UCpaso[\UCactor] Solicita modificar el término oprimiendo el botón \cdtButton{Aceptar} de la pantalla \cdtIdRef{IU 10.2}{Modificar término}. \refTray{A} 
    \UCpaso[\UCsist] Verifica que el actor ingrese todos los campos obligatorios con base en la regla de negocio  \cdtIdRef{RN8}{Datos obligatorios}. \refTray{B}
    \UCpaso[\UCsist] Verifica que el nombre del término no se encuentre registrado en el sistema con base en la regla de negocio  \cdtIdRef{RN6}{Unicidad de nombres}. \refTray{C}
    \UCpaso[\UCsist] Verifica que los datos requeridos sean proporcionados correctamente como se especifica en la regla de negocio \cdtIdRef{RN7}{Información correcta}. \refTray{D} \refTray{E}
    \UCpaso[\UCsist] Verifica que el nombre no contenga caracteres inválidos con base en la regla de negocio \cdtIdRef{RN2}{Nombres de los elementos}. \refTray{F}
	\UCpaso[\UCsist] Verifica que el término pueda modificarse con base en la regla de negocio \cdtIdRef{RN05}{Modificación de elementos asociados a casos de uso liberados}. \refTray{G}
	\UCpaso[\UCsist] Muestra la pantalla emergente \cdtIdRef{IU 10.2A}{Modificar término: Comentario}.
	\UCpaso[\UCactor] Ingresa el comentario referente a la modificación realizada en la pantalla emergente \cdtIdRef{IU 10.2A}{Modificar término: Comentario}. \label{cu10.2:ingresaComentario}
    \UCpaso[\UCsist] Verifica que el actor ingrese todos los campos obligatorios con base en la regla de negocio  \cdtIdRef{RN8}{Datos obligatorios}. \refTray{B}
    \UCpaso[\UCsist] Verifica que los datos requeridos sean proporcionados correctamente como se especifica en la regla de negocio \cdtIdRef{RN7}{Información correcta}. \refTray{E}
    \UCpaso[\UCsist] Modifica la información del término en el sistema.
    \UCpaso[\UCsist] Muestra el mensaje \cdtIdRef{MSG1}{Operación exitosa} en la pantalla \cdtIdRef{IU 10}{Gestionar términos}
    para indicar al actor que la modificación se ha realizado exitosamente.
 \end{UCtrayectoria}
 
 %----------------------------------------------------------%trayectoria A---------------------------------------------------- 
 \begin{UCtrayectoriaA}[Fin del caso de uso]{A}{El actor desea cancelar la operación.}
    \UCpaso[\UCactor] Solicita cancelar la operación oprimiendo el botón \cdtButton{Cancelar} de la pantalla \cdtIdRef{IU 10.2}{Modificar término}.
    \UCpaso[\UCsist] Muestra la pantalla \cdtIdRef{IU 10}{Gestionar términos}.
 \end{UCtrayectoriaA} 
  %----------------------------------------------------------%trayectoria B---------------------------------------------------- 
 \begin{UCtrayectoriaA}{B}{El actor no ingresó algún dato marcado como obligatorio.}
    \UCpaso[\UCsist] Muestra el mensaje \cdtIdRef{MSG4}{Dato obligatorio} y señala el campo que presenta el error en la pantalla 
	    \cdtIdRef{IU 10.2}{Modificar término} o la pantalla emergente \cdtIdRef{IU 10.2A}{Modificar término: Comentario}, indicando al actor que el dato es obligatorio.
    \UCpaso[] Continúa con el paso \ref{cu10.2:ingresaDatos} o con el paso \ref{cu10.2:ingresaComentario} de la trayectoria principal, según corresponda.
 \end{UCtrayectoriaA}
 %----------------------------------------------------------%trayectoria C---------------------------------------------------- 
 \begin{UCtrayectoriaA}{C}{El actor ingresó un nombre de término repetido.}
    \UCpaso[\UCsist] Muestra el mensaje \cdtIdRef{MSG7}{Registro repetido} y señala el campo que presenta la duplicidad en la pantalla 
	    \cdtIdRef{IU 10.2}{Modificar término}, indicando al actor que existe un término con el mismo nombre.
    \UCpaso[] Continúa con el paso \ref{cu10.2:ingresaDatos} de la trayectoria principal.
 \end{UCtrayectoriaA}
 
 %----------------------------------------------------------%trayectoria D----------------------------------------------------  
 \begin{UCtrayectoriaA}{D}{El actor proporciona un dato que excede la longitud máxima.}
    \UCpaso[\UCsist] Muestra el mensaje \cdtIdRef{MSG5}{Se ha excedido la longitud máxima del campo} y señala el campo que excede la 
    longitud en la pantalla \cdtIdRef{IU 10.2}{Modificar término}, para indicar que el dato excede el tamaño máximo permitido.
    \UCpaso[] Continúa con el paso \ref{cu10.2:ingresaDatos} de la trayectoria principal.
 \end{UCtrayectoriaA}
 
 %----------------------------------------------------------%trayectoria E---------------------------------------------------- 
 \begin{UCtrayectoriaA}{E}{El actor ingresó un tipo de dato incorrecto.}
    \UCpaso[\UCsist] Muestra el mensaje \cdtIdRef{MSG4}{Formato incorrecto} y señala el campo que presenta el dato inválido en la 
    pantalla \cdtIdRef{IU 10.2}{Modificar término} o la pantalla emergente \cdtIdRef{IU 10.2A}{Modificar término: Comentario} para indicar que se ha ingresado un tipo de dato inválido.
    \UCpaso[] Continúa con el paso \ref{cu10.2:ingresaDatos} o con el paso \ref{cu10.2:ingresaComentario} de la trayectoria principal, según corresponda.
 \end{UCtrayectoriaA}
 
 %----------------------------------------------------------%trayectoria F---------------------------------------------------- 
 \begin{UCtrayectoriaA}{F}{El actor ingresó un nombre con caracteres inválidos.}
    \UCpaso[\UCsist] Muestra el mensaje \cdtIdRef{MSG23}{Caracteres inválidos} y señala el campo que contiene los caracteres inválidos.
    \UCpaso[] Continúa con el paso \ref{cu10.2:ingresaDatos} de la trayectoria principal.
 \end{UCtrayectoriaA}
 %----------------------------------------------------------%trayectoria G---------------------------------------------------- 
 \begin{UCtrayectoriaA}[Fin del caso de uso]{G}{El término no puede modificarse debido a que se encuentra asociado a casos de uso liberados.}
    \UCpaso[\UCsist] Muestra la pantalla emergente con el mensaje \cdtIdRef{MSG30}{Modificación no permitida} en la pantalla \cdtIdRef{IU 10}{Gestionar términos}.
 \end{UCtrayectoriaA}
