\newpage 
\subsection{IU 5.1.2.1 Registrar precondición}

\subsubsection{Objetivo}
	
	Esta pantalla permite al actor registrar la información de una precondición.

\subsubsection{Diseño}

    En la figura ~\ref{IU 5.1.2.1} se muestra la pantalla ``Registrar precondición'' que permite registrar una precondición. El actor deberá ingresar la redacción de la precondición en la pantalla.
    
    Una vez ingresada la información solicitada, el actor deberá oprimir el botón \cdtButton{Aceptar}, el sistema validará y agregará la precondición al caso de uso
    solo si se han cumplido todas las reglas de negocio establecidas.  \\

    \IUfig[.9]{cu5.1.2.1/images/iu.png}{IU 5.1.2.1}{Registrar precondición}


\subsubsection{Comandos}
\begin{itemize}
	\item \cdtButton{Aceptar}: Permite al actor guardar el registro de la precondición, dirige a la pantalla \cdtIdRef{IU 5.1}{Registrar caso de uso} o \cdtIdRef{IU 5.2}{Modificar caso de uso}.
	\item \cdtButton{Cancelar}: Permite al actor cancelar el registro de la precondición, dirige a la pantalla \cdtIdRef{IU 5.1}{Registrar caso de uso} o \cdtIdRef{IU 5.2}{Modificar caso de uso}.
\end{itemize}

\subsubsection{Mensajes}

	
\begin{description}
	\item[ \cdtIdRef{MSG4}{Dato obligatorio}:] Se muestra en la pantalla \cdtIdRef{IU 5.1.2.1}{Registrar precondición} cuando no se ha ingresado un dato marcado como obligatorio.
	\item[ \cdtIdRef{MSG6}{Longitud inválida}:] Se muestra en la pantalla \cdtIdRef{IU 5.1.2.1}{Registrar precondición} cuando se ha excedido la longitud de alguno de los campos.
\end{description}
