\begin{UseCase}{CU 5.1.2.1}{Registrar precondición}
	{
		Las precondiciones describen un conjunto de condiciones que deben cumplirse para poder ejecutar el caso de uso satisfactoriamente.
		Este caso de uso permite al analista registrar una precondición.
		
	}

	\UCitem{Actor}{\cdtRef{actor:liderAnalisis}{Líder de análisis}, \cdtRef{actor:analista}{Analista}}
	\UCitem{Propósito}{
		Registrar las precondiciones de un caso de uso.
	}
	\UCitem{Entradas}{
		\begin{UClist}
 			\UCli \cdtRef{Precondicion:Redaccion}{Redacción de la precondición}: \ioEscribir.
		\end{UClist}
	}
	\UCitem{Salidas}{
		Ninguna
	}
	
	\UCitem{Mensajes}{
		Ninguno
	}

	\UCitem{Precondiciones}{
		Ninguna
	}
	
	\UCitem{Postcondiciones}{
		Ninguna
	}

	\UCitem{Errores}{
		\begin{UClist}
			\UCli \cdtIdRef{MSG4}{Dato obligatorio}: Se muestra en la pantalla \cdtIdRef{CU 5.1.2.1}{Registrar precondición} cuando no se ha ingresado un dato marcado como obligatorio.
			\UCli \cdtIdRef{MSG6}{Longitud inválida}: Se muestra en la pantalla \cdtIdRef{CU 5.1.2.1}{Registrar precondición} cuando se ha excedido la longitud de alguno de los campos.
		\end{UClist}
	}

	\UCitem{Tipo}{
		Secundario, extiende del caso de uso \cdtIdRef{CU 5.1.2}{Gestionar precondiciones}.
	}
\end{UseCase}



%-------------------------------------------------------%trayectoria Principal-----------------------------------------------

 \begin{UCtrayectoria}
	\UCpaso[\UCactor] Solicita registrar una precondición oprimiento el botón \cdtButton{Registrar} de la sección ``Precondiciones'' de la pantalla \cdtIdRef{IU 5.1}{Registrar caso de uso} o de la pantalla \cdtIdRef{IU 5.2}{Modificar caso de uso}. 
	\UCpaso[\UCsist] Muestra la pantalla \cdtIdRef{CU 5.1.2.1}{Registrar precondición}. 
	\UCpaso[\UCactor] Ingresa la redacción de la precondición. \refTray{A} \refTray{B} \refTray{C} \refTray{D} \refTray{E} \refTray{F} \refTray{G} \refTray{H} \refTray{I} \label{cu5.1.2.1:ingresaRedaccion}
	\UCpaso[\UCactor] Solicita registrar la precondición oprimiendo el botón \cdtButton{Aceptar} de la pantalla \cdtIdRef{CU 5.1.2.1}{Registrar precondición}. \refTray{J} 
	\UCpaso[\UCsist] Verifica que el actor haya ingresado todos los campos obligatorios con base en la regla de negocio \cdtIdRef{RN8}{Datos obligatorios}. \refTray{K}
	\UCpaso[\UCsist] Verifica que los datos requeridos sean proporcionados correctamente como se especifica en la regla de negocio \cdtIdRef{RN7}{Información correcta}. \refTray{L} 
	\UCpaso[\UCsist] Muestra la precondición en la tabla de precondiciones de la pantalla \cdtIdRef{IU 5.1}{Registrar caso de uso} o de la pantalla \cdtIdRef{IU 5.2}{Modificar caso de uso}.
\end{UCtrayectoria}

    

 %----------------------------------------------------------%trayectoria A---------------------------------------------------- 
 \begin{UCtrayectoriaA}{A}{El actor desea ingresar una entidad.}
 	
 	 \UCpaso[\UCactor] Ingresa el token ``ENT·''.
 	\UCpaso[\UCsist] Busca las entidades registradas en el sistema. 
 	\UCpaso[\UCsist] Muestra la lista de entidades encontradas.
 	\UCpaso[\UCactor] Selecciona una entidad de la lista.
  	\UCpaso[\UCsist] Oculta la lista y agrega el nombre de la entidad al texto.
    \UCpaso[] Continúa con el paso \ref{cu5.1.2.1:ingresaRedaccion} de la trayectoria principal.
 \end{UCtrayectoriaA}
 
  %----------------------------------------------------------%trayectoria B---------------------------------------------------- 
 \begin{UCtrayectoriaA}{B}{El actor desea ingresar un actor.}
 	\UCpaso[\UCactor] Ingresa el token ``ACT·''.
 	\UCpaso[\UCsist] Busca los actores registrados en el sistema. 
 	\UCpaso[\UCsist] Muestra la lista de actores encontrados.
 	\UCpaso[\UCactor] Selecciona un actor de la lista.
  	\UCpaso[\UCsist] Oculta la lista y agrega el nombre del actor al texto.
    \UCpaso[] Continúa con el paso \ref{cu5.1.2.1:ingresaRedaccion} de la trayectoria principal.
 \end{UCtrayectoriaA}

  %----------------------------------------------------------%trayectoria C---------------------------------------------------- 
 \begin{UCtrayectoriaA}{C}{El actor desea ingresar un caso de uso.}
  	\UCpaso[\UCactor] Ingresa el token ``CU·''.	
 	\UCpaso[\UCsist] Busca los casos de uso registrados en el sistema. 
 	\UCpaso[\UCsist] Muestra la lista de casos de uso encontrados.
 	\UCpaso[\UCactor] Selecciona un caso de uso de la lista.
  	\UCpaso[\UCsist] Oculta la lista y agrega el nombre del caso de uso al texto.
    \UCpaso[] Continúa con el paso \ref{cu5.1.2.1:ingresaRedaccion} de la trayectoria principal.
 \end{UCtrayectoriaA}

  %----------------------------------------------------------%trayectoria D---------------------------------------------------- 
 \begin{UCtrayectoriaA}{D}{El actor desea ingresar una pantalla.}
 	\UCpaso[\UCactor] Ingresa el token ``IU·''.	
 	\UCpaso[\UCsist] Busca las pantallas registradas en el sistema. 
 	\UCpaso[\UCsist] Muestra la lista de pantallas encontradas.
 	\UCpaso[\UCactor] Selecciona una pantalla de la lista.
  	\UCpaso[\UCsist] Oculta la lista y agrega el nombre de la pantalla al texto.
    \UCpaso[] Continúa con el paso \ref{cu5.1.2.1:ingresaRedaccion} de la trayectoria principal.
 \end{UCtrayectoriaA}

 %----------------------------------------------------------%trayectoria E---------------------------------------------------- 
 \begin{UCtrayectoriaA}{E}{El actor desea ingresar un mensaje.}
 	 \UCpaso[\UCactor] Ingresa el token ``MSG·''.	
 	\UCpaso[\UCsist] Busca los mensajes registrados en el sistema. 
 	\UCpaso[\UCsist] Muestra la lista de mensajes.
 	\UCpaso[\UCactor] Selecciona un mensaje de la lista.
  	\UCpaso[\UCsist] Oculta la lista y agrega el nombre del mensaje al texto.
    \UCpaso[] Continúa con el paso \ref{cu5.1.2.1:ingresaRedaccion} de la trayectoria principal.
 \end{UCtrayectoriaA}

  %----------------------------------------------------------%trayectoria F---------------------------------------------------- 
 \begin{UCtrayectoriaA}{F}{El actor desea ingresar una regla de negocio.}
 	\UCpaso[\UCactor] Ingresa el token ``RN·''.	
 	\UCpaso[\UCsist] Busca las reglas de negocio registradas en el sistema. 
 	\UCpaso[\UCsist] Muestra la lista de reglas de negocio.
 	\UCpaso[\UCactor] Selecciona una regla de negocio de la lista.
  	\UCpaso[\UCsist] Oculta la lista y agrega el nombre de la regla de negocio al texto.
    \UCpaso[] Continúa con el paso \ref{cu5.1.2.1:ingresaRedaccion} de la trayectoria principal.
 \end{UCtrayectoriaA}

  %----------------------------------------------------------%trayectoria G---------------------------------------------------- 
 \begin{UCtrayectoriaA}{G}{El actor desea ingresar un término del glosario.}
 	\UCpaso[\UCactor] Ingresa el token ``GLS·''.	
 	\UCpaso[\UCsist] Busca los términos del glosario registrados en el sistema. 
 	\UCpaso[\UCsist] Muestra la lista de términos del glosario.
 	\UCpaso[\UCactor] Selecciona un término del glosario de la lista.
  	\UCpaso[\UCsist] Oculta la lista y agrega el nombre del término del glosario al texto.
    \UCpaso[] Continúa con el paso \ref{cu5.1.2.1:ingresaRedaccion} de la trayectoria principal.
 \end{UCtrayectoriaA}

  %----------------------------------------------------------%trayectoria H---------------------------------------------------- 
 \begin{UCtrayectoriaA}{H}{El actor desea ingresar un atributo de una entidad.}
 	\UCpaso[\UCactor] Ingresa el token ``ATR·''.
  	\UCpaso[\UCsist] Busca los atributos de las entidades registradas.
  	\UCpaso[\UCsist] Muestra la lista de atributos encontrados.
 	\UCpaso[\UCactor] Selecciona un atributo de la lista.
  	\UCpaso[\UCsist] Oculta la lista y agrega el nombre del atributo al texto.
    \UCpaso[] Continúa con el paso \ref{cu5.1.2.1:ingresaRedaccion} de la trayectoria principal.
 \end{UCtrayectoriaA}

 %----------------------------------------------------------%trayectoria I---------------------------------------------------- 
 \begin{UCtrayectoriaA}{I}{El actor desea ingresar un paso.}
 	\UCpaso[\UCactor] Ingresa el token ``P·''.
  	\UCpaso[\UCsist] Busca los pasos de las trayectorias del caso de uso.
  	\UCpaso[\UCsist] Muestra la lista de pasos encontrados.
 	\UCpaso[\UCactor] Selecciona un paso de la lista.
  	\UCpaso[\UCsist] Oculta la lista y agrega el número del paso al texto.
    \UCpaso[] Continúa con el paso \ref{cu5.1.2.1:ingresaRedaccion} de la trayectoria principal.
 \end{UCtrayectoriaA}

 %----------------------------------------------------------%trayectoria J---------------------------------------------------- 
 \begin{UCtrayectoriaA}[Fin del caso de uso]{J}{El actor desea cancelar la operación.}
    \UCpaso[\UCactor] Solicita cancelar la operación oprimiendo el botón \cdtButton{Cancelar} de la pantalla \cdtIdRef{CU 5.1.2.1}{Registrar precondición}.
    \UCpaso[\UCsist] Muestra la pantalla \cdtIdRef{IU 5.1}{Registrar caso de uso} o \cdtIdRef{IU 5.2}{Modificar caso de uso} según corresponda.
 \end{UCtrayectoriaA}
 
 %----------------------------------------------------------%trayectoria K---------------------------------------------------- 
 \begin{UCtrayectoriaA}{K}{El actor no ingresó algún dato marcado como obligatorio.}
    \UCpaso[\UCsist] Muestra el mensaje \cdtIdRef{MSG4}{Dato obligatorio} y señala el campo que presenta el error en la pantalla \cdtIdRef{CU 5.1.2.1}{Registrar precondición}, indicando al actor que el dato es obligatorio.
    \UCpaso[] Continúa con el paso \ref{cu5.1.2.1:ingresaRedaccion} de la trayectoria principal.
 \end{UCtrayectoriaA}
 
 %----------------------------------------------------------%trayectoria L --------------------------------------------------  
 \begin{UCtrayectoriaA}{L}{El actor proporciona un dato que excede la longitud máxima.}
    \UCpaso[\UCsist] Muestra el mensaje \cdtIdRef{MSG5}{Se ha excedido la longitud máxima del campo} y señala el campo que excede la 
    longitud en la pantalla \cdtIdRef{CU 5.1.2.1}{Registrar precondición}, para indicar que el dato excede el tamaño máximo permitido.
    \UCpaso[] Continúa con el paso \ref{cu5.1.2.1:ingresaRedaccion} de la trayectoria principal.
 \end{UCtrayectoriaA}

  