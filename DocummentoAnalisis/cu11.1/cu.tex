\newpage 
\begin{UseCase}{CU 11.1}{Registrar entidad}
	{
		Este caso de uso permite al analista registrar la información de una entidad, así como 
		gestionar sus atributos. También permite 
		al actor acceder a las operaciones de registro, modificación y eliminación de un atributo.
	}
	
	\UCitem{Actor}{\cdtRef{actor:liderAnalisis}{Líder de análisis}, \cdtRef{actor:analista}{Analista}}
	\UCitem{Propósito}{
		Registrar una entidad y gestionar sus atributos.
	}
	\UCitem{Entradas}{
		\begin{UClist}
			\UCli \cdtRef{Elemento:Nombre}{Nombre}: \ioEscribir.
			\UCli \cdtRef{Elemento:Descripcion}{Descripción}: \ioEscribir.
		\end{UClist}
	}
	\UCitem{Salidas}{
		\begin{UClist}
			\UCli \cdtRef{Atributo}{Atributo}: \ioTabla{\cdtRef{Elemento:Nombre}{Nombre}, 
			\cdtRef{Atributo:Obligatorio}{Obligatorio (sí o no)} y \cdtRef{gls:TipoDeDato}{Tipo de Dato}}{de los atributos}.
			\UCli \cdtRef{Elemento:Clave}{Clave}: \ioCalcular{\cdtIdRef{RN12}{Identificador de elemento}}.
		\end{UClist}
	}
	
	\UCitem{Mensajes}{
		\begin{UClist}
			\UCli \cdtIdRef{MSG1}{Operación exitosa}: Se muestra en la pantalla \cdtIdRef{IU 11}{Gestionar entidades} para indicar que el registro fue exitoso.
		\end{UClist}
	}

	\UCitem{Precondiciones}{
		Ninguna
	}
	
	\UCitem{Postcondiciones}{
		\begin{UClist}
			\UCli Se podrá solicitar el registro de un atributo por medio del caso de uso \cdtIdRef{CU 11.1}{Registrar atributo}.
			\UCli Se podrá solicitar la modificación de un atributo por medio del caso de uso \cdtIdRef{CU 11.2}{Modificar atributo}.
			\UCli Se podrá solicitar la eliminación de un atributo por medio del caso de uso \cdtIdRef{CU 11.3}{Eliminar atributo}.
		\end{UClist}
	}

	\UCitem{Errores}{
		\begin{UClist}
			\UCli \cdtIdRef{MSG4}{Dato obligatorio}: Se muestra en la pantalla \cdtIdRef{IU 11.1}{Registrar entidad} cuando no se ha ingresado un dato marcado como obligatorio.
			\UCli \cdtIdRef{MSG5}{Dato incorrecto}: Se muestra en la pantalla \cdtIdRef{IU 11.1}{Registrar entidad} cuando el tipo de dato ingresado no cumple con el tipo de dato solicitado en el campo.
			\UCli \cdtIdRef{MSG6}{Longitud inválida}: Se muestra en la pantalla \cdtIdRef{IU 11.1}{Registrar entidad} cuando se ha excedido la longitud de alguno de los campos.
			\UCli \cdtIdRef{MSG7}{Registro repetido}: Se muestra en la pantalla \cdtIdRef{IU 11.1}{Registrar entidad} cuando se registre una entidad con un nombre que ya se encuentre registrado.
			\UCli \cdtIdRef{MSG18}{Registro necesario}: Se muestra en la pantalla \cdtIdRef{IU 11.1}{Registrar entidad} cuando el actor no ingrese ningún atributo.
			\UCli \cdtIdRef{MSG23}{Caracteres inválidos}: Se muestra en la pantalla \cdtIdRef{IU 11.1}{Registrar entidad} cuando el nombre de la entidad contiene un caracter no válido.
		\end{UClist}
	}

	\UCitem{Tipo}{
		Secundario, extiende del caso de uso \cdtIdRef{CU 11}{Gestionar entidades}.
% 		\clearpage
	}
\end{UseCase}
%-------------------------------------------------------%trayectoria Principal-----------------------------------------------
 \begin{UCtrayectoria}
    \UCpaso[\UCactor] Solicita registrar una entidad oprimiendo el botón \cdtButton{Registrar} de la pantalla \cdtIdRef{IU 11}{Gestionar entidades}.
    \UCpaso[\UCsist] Muestra la pantalla \cdtIdRef{IU 11.1}{Registrar entidad} en la cual se realizará el registro de la entidad. 
    \UCpaso[\UCactor] Ingresa el nombre y la descripción de la entidad en la pantalla \cdtIdRef{IU 11.1}{Registrar entidad}. \label{cu11.1:ingresaDatos}
    
    \UCpaso[\UCactor] Gestiona los atributos a través de los botones: \cdtButton{Registrar}, \btnEditar y \btnEliminar. \label{cu11.1:gestionar}
    
    \UCpaso[\UCactor] Solicita guardar la entidad oprimiendo el botón \cdtButton{Aceptar} de la pantalla \cdtIdRef{IU 11.1}{Registrar entidad}. \refTray{A}
    \UCpaso[\UCsist] Verifica que el actor ingrese todos los campos obligatorios con base en la regla de negocio  \cdtIdRef{RN8}{Datos obligatorios}. \refTray{B}
    \UCpaso[\UCsist] Verifica que el nombre de la entidad no se encuentre registrado en el sistema con base en la regla de negocio  \cdtIdRef{RN6}{Unicidad de nombres}. \refTray{C}
    \UCpaso[\UCsist] Verifica que los datos requeridos sean proporcionados correctamente como se especifica en la regla de negocio \cdtIdRef{RN7}{Información correcta}. \refTray{D}
    
    \UCpaso[\UCsist] Verifica que el nombre no contenga caracteres inválidos con base en la regla de negocio \cdtIdRef{RN2}{Nombres de los elementos}. \refTray{E}
    \UCpaso[\UCsist] Verifica que el actor haya ingresado al menos un atributo. \refTray{F}

    \UCpaso[\UCsist] Registra la información de la entidad en el sistema.
    \UCpaso[\UCsist] Muestra el mensaje \cdtIdRef{MSG1}{Operación exitosa} en la pantalla \cdtIdRef{IU 11}{Gestionar entidades} 
    para indicar al actor que el registro se ha realizado exitosamente.
 \end{UCtrayectoria}
 
 %----------------------------------------------------------%trayectoria A---------------------------------------------------- 
 \begin{UCtrayectoriaA}[Fin del caso de uso]{A}{El actor desea cancelar la operación.}
    \UCpaso[\UCactor] Solicita cancelar la operación oprimiendo el botón \cdtButton{Cancelar} de la pantalla \cdtIdRef{IU 11.1}{Registrar entidad}.
    \UCpaso[\UCsist] Muestra la pantalla \cdtIdRef{IU 11}{Gestionar entidades}.
 \end{UCtrayectoriaA}
  %----------------------------------------------------------%trayectoria B---------------------------------------------------- 
 \begin{UCtrayectoriaA}{B}{El actor no ingresó algún dato marcado como obligatorio.}
    \UCpaso[\UCsist] Muestra el mensaje \cdtIdRef{MSG4}{Dato obligatorio} y señala el campo que presenta el error en la pantalla 
	    \cdtIdRef{IU 11.1}{Registrar entidad}, indicando al actor que el dato es obligatorio.
    \UCpaso[] Continúa con el paso \ref{cu11.1:ingresaDatos} de la trayectoria principal.
 \end{UCtrayectoriaA}
 %----------------------------------------------------------%trayectoria C---------------------------------------------------- 
 \begin{UCtrayectoriaA}{C}{El actor ingresó un nombre de entidad repetido.}
    \UCpaso[\UCsist] Muestra el mensaje \cdtIdRef{MSG7}{Registro repetido} y señala el campo que presenta la duplicidad en la pantalla 
	    \cdtIdRef{IU 11.1}{Registrar entidad}, indicando al actor que existe una entidad con el mismo nombre.
    \UCpaso[] Continúa con el paso \ref{cu11.1:ingresaDatos} de la trayectoria principal.
 \end{UCtrayectoriaA}
 %----------------------------------------------------------%trayectoria D----------------------------------------------------  
 \begin{UCtrayectoriaA}{D}{El actor proporciona un dato que excede la longitud máxima.}
    \UCpaso[\UCsist] Muestra el mensaje \cdtIdRef{MSG6}{Longitud inválida} y señala el campo que excede la 
    longitud en la pantalla \cdtIdRef{IU 11.1}{Registrar entidad}, para indicar que el dato excede el tamaño máximo permitido.
    \UCpaso[] Continúa con el paso \ref{cu11.1:ingresaDatos} de la trayectoria principal.
 \end{UCtrayectoriaA}
 %----------------------------------------------------------%trayectoria E---------------------------------------------------- 
 \begin{UCtrayectoriaA}{E}{El actor ingresó un nombre con caracteres inválidos.}
    \UCpaso[\UCsist] Muestra el mensaje \cdtIdRef{MSG23}{Caracteres inválidos} y señala el campo que contiene los caracteres inválidos.
    \UCpaso[] Continúa con el paso \ref{cu11.1:ingresaDatos} de la trayectoria principal.
 \end{UCtrayectoriaA}
 %----------------------------------------------------------%trayectoria F---------------------------------------------------- 
 \begin{UCtrayectoriaA}{F}{El actor no registró ningún atributo.}
    \UCpaso[\UCsist] Muestra el mensaje \cdtIdRef{MSG18}{Registro necesario} en la sección de atributos.
    \UCpaso[] Continúa con el paso \ref{cu11.1:ingresaDatos} de la trayectoria principal.
 \end{UCtrayectoriaA}
 
\subsubsection{Puntos de extensión}
 
 \UCExtensionPoint{El actor requiere registrar un atributo}
	{Paso \ref{cu11.1:gestionar}}
	{\cdtIdRef{CU 11.1}{Registrar atributo}}
\UCExtensionPoint{El actor requiere modificar un atributo}
	{Paso \ref{cu11.1:gestionar}}
	{\cdtIdRef{CU 11.1.2}{Modificar atributo}}	
\UCExtensionPoint{El actor requiere eliminar un atributo}
	{Paso \ref{cu11.1:gestionar}}
	{\cdtIdRef{CU 11.1.3}{Eliminar atributo}}
