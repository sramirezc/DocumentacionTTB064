En la etapa de Análisis se definen elementos muy importantes para el desarrollo del sistema. Toda la información debe estar organizada para permitir
a los colaboradores del proyecto entender su contenido sin dejar espacio a las ambigüedades, para ello se construye un documento de análisis donde se define la información 
con la que trabajará el sistema, la estructura donde se organizará y las relaciones entre todos los elementos definidos.\\

En este documento también se busca detallar todos los posibles comportamientos que tendrá el sistema, lo que se vuelve complicado debido a 
la cantidad de validaciones y a todas sus combinaciones.\\

Las herramientas que existen actualmente para generar un documento de análisis están limitadas debido a que no permiten realizar las siguientes tareas en conjunto:
\begin{itemize}
 \item Organizar y relacionar las entidades de un sistema con los casos de uso.
 \item Proveer un mecanismo para organizar y relacionar las reglas de negocio utilizadas en la documentación de casos de uso.
 \item Proveer un mecanismo para organizar y relacionar los mensajes utilizados en la documentación de casos de uso.
 \item Entrelazar los casos de uso con las reglas de negocio, mensajes y entidades. 
 \item Documentar el comportamiento que tendrá el sistema a través de las trayectorias principales y alternativas.
\end{itemize}

Otro problema en el desarrollo del los sistemas se presenta en la etapa de Pruebas, donde es muy costoso en tiempo y recursos desarrollar las pruebas funcionales
de un sistema debido a la cantidad de escenarios y validaciones que se deben de probar. Además las personas responsables de realizar las pruebas funcionales 
de un sistema requieren forzosamente conocer el negocio, las validaciones y el comportamiento del sistema; esto lo hacen a través del documento de análisis o del estudio del 
sistema implementado, y no existen mecanismos automatizados para utilizar el conocimiento vertido en el documento de análisis para la creación de pruebas
funcionales. Algunas de las desventajas de las herramientas de pruebas que existen se deben a que:

\begin{itemize}
 \item No reutilizan la información de los datos de entrada que se definieron en el documento de análisis.
 \item No reutilizan la información de las salidas esperadas que se describieron en el documento de análisis.
 \item No permiten reutilizar todas las validaciones descritas en las trayectorias de los casos de uso.
 \item No proveen un mecanismo que transforme las trayectorias de los casos de uso en casos de prueba.
\end{itemize}