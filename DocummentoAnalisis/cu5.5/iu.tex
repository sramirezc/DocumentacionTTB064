\newpage 
\subsection{IU 5.5 Revisar caso de uso}

\subsubsection{Objetivo}
	
	Esta pantalla permite al actor revisar la información de un caso de uso y determinar si es correcto o no.

\subsubsection{Diseño}

    En la figura ~\ref{IU 5.5} se muestra la pantalla ``Revisar caso de uso'' en la cual el actor podrá revisar el caso de uso y decidir si es correcto o no. Cada uno de los elementos utilizados contará con un enlace a su respectiva consulta. En caso de que alguna de las secciones no cuente con información se mostrará la leyenda ``Sin información''. \\
	
	Una vez que el actor ha consultado la información mostrada, deberá seleccionar para cada sección si esta es correcta o no. Para las secciones marcadas como incorrectas el sistema solicitará que se ingresen las correspondientes observaciones. Finalmente el actor deberá oprimir el botón \cdtButton{Aceptar}, el sistema realizará las validaciones correspondientes y determinará el nuevo estado del caso de uso.

    \IUfig[.8]{cu5.5/images/iu.png}{IU 5.5}{Revisar caso de uso}

\subsubsection{Comandos}
\begin{itemize}
	\item \cdtButton{Aceptar}: Permite al actor concluir la revisión, dirige a la pantalla \cdtRef{IU 5}{Gestionar casos de uso}
\end{itemize}

\subsubsection{Mensajes}
	
\begin{description}
	
	\item \cdtIdRef{MSG1}{Operación exitosa}: Se muestra en la pantalla \cdtIdRef{IU 5}{Gestionar casos de uso} para indicar que la revisión se ha realizado exitosamente.
	
	\item \cdtIdRef{MSG4}{Dato obligatorio}: Se muestra en la pantalla \cdtIdRef{CU 5.5}{Revisar caso de uso} cuando no se ha ingresado un dato marcado como obligatorio.
	
	\item \cdtIdRef{MSG6}{Longitud inválida}: Se muestra en la pantalla \cdtIdRef{CU 5.5}{Revisar caso de uso} cuando se ha excedido la longitud de alguno de los campos.
	
	\item \cdtIdRef{MSG13}{Ha ocurrido un error}: Se muestra en la pantalla donde se solicitó la operación cuando el caso de uso que se desea revisar no se encuentra en estado ``Revisión''.
	
\end{description}
